% Template for IGARSS-2020 paper; to be used with:
%          spconf.sty  - LaTeX style file, and
%          IEEEbib.bst - IEEE bibliography style file.
% --------------------------------------------------------------------------
\documentclass{article}
\usepackage{spconf,amsmath}
\usepackage{bm,bbm}                                        % AAB inserted
\usepackage[boxed]{algorithm2e}                            % AAB inserted
\usepackage[caption=false,font=footnotesize]{subfig}       % AAB inserted
\usepackage[binary-units]{siunitx}                         % AAB inserted
\usepackage{booktabs}                                      % AAB inserted
\usepackage[hidelinks]{hyperref}                                           % AAB inserted
\usepackage{cite}
\usepackage{tikz}                                          % AAB inserted 
\usetikzlibrary{shapes,arrows,shadows}                     % AAB inserted
\usepackage{enumitem}                                      % AAB inserted
\usepackage{calrsfs}                                       % AAB inserted
\usepackage{amsfonts}                                      % AAB inserted
\usepackage[caption=false,font=footnotesize]{subfig}       % AAB inserted
%
\graphicspath{{../igarss_2022/figures_igarss_2022/}}        % AAB inserted
%
\DeclareMathOperator{\traco}{tr}                           % AAB inserted
% Example definitions.
% --------------------
\def\x{{\mathbf x}}
\def\L{{\cal L}}
% Title.
% ------
\title{Comparing the Gambini's algorithm and a CFAR method to edge detection for PolSAR images}
%
% Single address.
% ---------------
\name{Anderson A.\ de Borba$^a$, Maurício Marengoni$^b$, and Alejandro C.\ Frery$^c$.\thanks{e-mail:$^a$anderson.aborba@professores.ibmec.edu.br, $^b$mmarengoni@albion.edu, $^c$alejandro.frery@vuw.ac.nz }}
\address{$^a$IBMEC-SP, Alameda Santos, 2356 - Jardim Paulista, SP -- Brazil, \\
$^b$Albion College, Albion, MI -- USA,           \\
$^c$School of Mathematics and Statistics, Victoria University of Wellington, 6140, New Zealand.}
%
% For example:
% ------------
%\address{School\\
%	Department\\
%	Address}
%
% Two addresses (uncomment and modify for two-address case).
% ----------------------------------------------------------
%\twoauthors
%  {A. Author-one, B. Author-two\sthanks{Thanks to XYZ agency for funding.}}
%	{School A-B\\
%	Department A-B\\
%	Address A-B}
%  {C. Author-three, D. Author-four\sthanks{The fourth author performed the work
%	while at ...}}
%	{School C-D\\
%	Department C-D\\
%	Address C-D}
%
\begin{document}
%\ninept
%
\maketitle
%
\begin{abstract}
In this work we compare two different method to detect edges for PolSAR images. The Gambini's algorithm and a simple CFAR method (The Coefficient of Variation Detector). Both methods were able to detect edges in the simulated images used in this work. The Gambini's algorithm has the advantage of finding a single pixel as an edge while the CFAR method detects multiple edges, mainly when the edge is a ramp edge. In the other hand, the CFAR method can be directly applied to the images while the Gambini's algorithm needs to have as input a region of interest and lines where the transition point between distributions (edges) are detected.
\end{abstract}
%
\begin{keywords}
Gambine's Algorithm, CFAR coefficient of variation, edge detection.
\end{keywords}
%
\section{Introduction}
\label{sec:intro}

The present research aims to compare two different methods to find edges in PolSAR images. The first method is the Gambini's algorithm (GA) (see Refs.~\cite{fbgm, gmbf}) which is not an edge detector but a method to find the transition point between two Gamma distributions, as different areas in the ground will respond differently for a PolSAR signal the transition point between these two areas can be seen as an edge. The second method is a Constant False Alarm Rate (CFAR) - Coefficient of Variation Detector (see Ref.~\cite{tlb}) where a coefficient of variation is used to find edges in PolSAR images. Simulated images were used in this research, a simple rectangle area in the middle of the image. Nine different images were used, three with a sharp edge and six with a ramp edge.

In Refs~\cite{fbgm, gmbf}, Gambini et al. used the $\mathcal{G}^0$ polarimetric distributions (see Refs~\cite{fmcs, ffc}). In Refs.~\cite{bmf_2019,bmf_2020}, Borba et al. modified the polarimetric $\mathcal{G}^0$ distribution to the Wishart distribution applying it to PolSAR images. The simulated images were drawn from the works of~\cite{EDbook,gamf}.  

Refs.~\cite{tlb} and~\cite{ssnc}  presents some CFAR-type methods. This work presented here implements the idea proposed in~\cite{tlb} for the CFAR method coefficient of variation detector. 

It is essential to highlight the difference between the Gambini algorithm and the CFAR methods. It is known that PolSAR images have multiplicative noise known as speckle. GA uses the Speckle noise as an ally; that is, no treatment such as filtering is performed. On the other hand, in CFAR-like methods, the speckle noise is treated with filtering processes.

The article is structured as follows.
Section~\ref{sec:SimImag} describes the simulated image tests.
Section~\ref{sec:edge_detc} highlight the Gambini Algorithm (GA) to edge detection and describe CFAR coefficient of variation detect methods .
Section~\ref{sec:results} presents the results.
Finally, in Section~\ref{sec:conc_disc} we discuss the results.



\section{Simulated Images}
\label{sec:SimImag}
Based on Refs~\cite{EDbook,gamf}, we get the simulated images with dimension $800 \times 800 \times 3$ with two classes, the third dimension is a channel number. To achieve this goal, we built a reference image as follows:
\begin{itemize}
\item Set $0$ to all pixels in the image.
\item Insert a rectangle $[x_l, x_u]\times[y_l,y_u]$ centered at the pixel $(x_c, y_c) = (400, 400)$, for all channels, where $x_l=x_c - 100,$ is the lower horizontal coordinate, $x_u=x_c + 100,$ is the upper horizontal coordinate,  $y_l=y_c - 50,$ is the lower vertical coordinate, $y_u=y_c + 50,$ is the upper vertical coordinate, and $\epsilon,$ is the ramp constant.
	
\item Set 1 for pixels inside the rectangle and its boundary, 

\end{itemize}
	
We define the function \eqref{eq01} in the horizontal ramps $[x_l+\epsilon, x_u-\epsilon]\times[y_l-\epsilon,y_l+\epsilon]$, and $[x_l+\epsilon, x_u -\epsilon]\times[y_u-\epsilon,y_u+\epsilon]$. And, the same way in the vertical ramps $[x_l-\epsilon, x_l+\epsilon]\times[y_l+\epsilon,y_u-\epsilon]$, and $[x_u-\epsilon, x_u+\epsilon]\times[y_l+\epsilon,y_u-\epsilon]$.
	
We use function \eqref{eq03}, on the corners. 

%\begin{enumerate}[label=(\roman*)]
%	\item \label{item:obj_i}  Set $0$ to all pixels in the image $800 \times 800$, and insert a rectangle $[x_l+\epsilon, x_u-\epsilon]\times[y_l+\epsilon,y_u-\epsilon]$ centered at the pixel $(x_c, y_c) = (400, 400)$, for all channels.
%	\item \label{item:obj_iii} Set 1 for the pixels inside the rectangle and its boundary, where, $x_l=x_c - 100,$ is the lower horizontal coordinate, $x_u=x_c + 100,$ is the upper horizontal coordinate,  $y_l=y_c - 50,$ is the lower vertical coordinate, $y_u=y_c + 50,$ is the upper vertical coordinate, and $\epsilon,$ is the ramp constant. 
%	\item \label{item:obj_iv}  We define the function \eqref{eq01} in the horizontal ramps $[x_l+\epsilon, x_u-\epsilon]\times[y_l-\epsilon,y_l+\epsilon]$, and $[x_l+\epsilon, x_u -\epsilon]\times[y_u-\epsilon,y_u+\epsilon]$. And, the same way in the vertical ramps $[x_l-\epsilon, x_l+\epsilon]\times[y_l+\epsilon,y_u-\epsilon]$, and $[x_u-\epsilon, x_u+\epsilon]\times[y_l+\epsilon,y_u-\epsilon]$.
%	\item  \label{item:obj_v} We use function \eqref{eq03}, on the corners. %
%\end{enumerate} 

The outer region in the simulated image is the  complementary region of the ramp, defined around the rectangle, plus the internal region of the rectangle. 

The function for the horizontal ramps is:
\begin{equation}f_x(x)=
  \begin{cases}
    0                 & \textrm{if } x < 0 \\
    6x^5 -15x^4 + 10x^3 & \textrm{if } 0 \leq x \leq 1 \\
    1                 & \textrm{if } x > 1.
  \end{cases}\label{eq01}
\end{equation}
%The function for the vertical ramps is similar
% 
%\begin{equation}f_y(y)=
%  \begin{cases}
%    0                 & \textrm{if } y < 0 \\
%    6y^5 -15y^4 + 10y^3 & \textrm{if } 0 \leq y \leq 1 \\
%    1                 & \textrm{if } y > 1.
%  \end{cases}\label{eq02}
%\end{equation}

The function for the corners is:
\begin{eqnarray}
f(x,y) = f_x(x) \cdot f_y(y),& \textrm{with }(x,y)\in \mathbb{R}^2  
\label{eq03}
\end{eqnarray} 

The reference image is defined as $\text{R}_1$. In addition to this image, we define two more images, $\text{R}_2$, which we call the negative image of $\text{R}_1$. It consists of switching the values of 0 and 1 in the image. Similarly, in $\text{R}_3$, we switched the values of 0 by 5 and the values of 1 by 10. In both cases, the functions from \ref{eq01} to \ref{eq03} are modified accordingly.

Three images tests based on the procedure above were defined. First, the image test $\text{T}_1$, based on $\text{R}_1$ , was defined using Wishart distribution with $\mu1$ added in the outer region, $\mu2$ added in the internal rectangle and $\mu_{av}= (\mu_1+\mu_2)/2$ added in the ramps. The parameter $\text{L}=4$ is the same for all regions.

Second, the image test $\text{T}_2$ and $\text{T}_3$ were built similarly to image test T1.

The parameters used for the Wishart distribution in the HH channel are $\text{HH}$  are $\mu_1=0.042811$, and $\mu_2=0.014380$:  
%

Fig.~\ref{sim_image_mu1_greater_mu2_with_ramp_eps_10_basica} shows the red channel of the Pauli decomposition of image $T_1$ and the slice of the function at coordinate 400 showing the ramp between the two areas.
\begin{figure}[hbt]
	\centering
     \subfloat[Pauli red channel\label{sim_image_red_pauli_mu1_greater_mu2_with_ramp_eps_10_basica}]{%
       \includegraphics[width=.4\linewidth]{fig_Pauli_c_1_smooth_ramp_mu1_greater_mu2_eps_10}
     }
     \subfloat[Slice function \label{sim_image_perfil_mu1_greater_mu2_with_ramp_eps_10_basica}]{%
       \includegraphics[width=.4\linewidth]{fig_perfil_x_400_c_2_smooth_ramp_mu1_greater_mu2_eps_10}
     }
     \caption{Image $\text{T}_1$ and the slice function for $\epsilon=10$}
     \label{sim_image_mu1_greater_mu2_with_ramp_eps_10_basica}
\end{figure}

Fig.~\ref{sim_image_mu1_greater_mu2_with_ramp_eps_10_basica_step_1_0} shows the red channel of the Pauli decomposition of image $\text{T}_2$ and the slice of the function is similar with Fig.~\ref{sim_image_mu1_greater_mu2_with_ramp_eps_10_basica}



%Fig.~\ref{sim_image_mu1_greater_mu2_with_ramp_eps_10_basica_step_1_0} shows the red channel of the Pauli decomposition image and its slice function \ref{sim_image_mu1_greater_mu2_with_ramp_eps_10_basica_step_1_0}\subref{sim_image_perfil_mu1_greater_mu2_with_ramp_eps_10_basica_step_1_0} to test image $\text{T}_2$. Its slice function with the smooth ramp was performed at the same conditions as shown in Fig.~\ref{sim_image_mu1_greater_mu2_with_ramp_eps_10_basica}~.
\begin{figure}[hbt]
	\centering
     \subfloat[Pauli red channel\label{sim_image_red_pauli_mu1_greater_mu2_with_ramp_eps_10_basica_step_1_0}]{%
       \includegraphics[width=.4\linewidth]{fig_Pauli_c_1_smooth_ramp_mu1_greater_mu2_eps_10_step_1_0}
     }
     \subfloat[Slice function \label{sim_image_perfil_mu1_greater_mu2_with_ramp_eps_10_basica_step_1_0}]{%
       \includegraphics[width=.4\linewidth]{fig_perfil_x_400_c_2_smooth_ramp_mu1_greater_mu2_eps_10_step_1_0}
     }
     \caption{Image $\text{T}_2$ and its slice function with  $\epsilon=10$}
     \label{sim_image_mu1_greater_mu2_with_ramp_eps_10_basica_step_1_0}
\end{figure}

Fig.~\ref{sim_image_mu1_greater_mu2_with_ramp_eps_10_basica_step_5_10} shows the red channel of the Pauli decomposition of image $\text{T}_3$ and the slice of the function is similar with Fig.~\ref{sim_image_mu1_greater_mu2_with_ramp_eps_10_basica}

%Fig.~\ref{sim_image_mu1_greater_mu2_with_ramp_eps_10_basica_step_5_10} shows the red channel of the Pauli decomposition image \ref{sim_image_mu1_greater_mu2_with_ramp_eps_10_basica_step_5_10}\subref{sim_image_red_pauli_mu1_greater_mu2_with_ramp_eps_10_basica_step_5_10} and its slice function \ref{sim_image_mu1_greater_mu2_with_ramp_eps_10_basica_step_5_10}\subref{sim_image_perfil_mu1_greater_mu2_with_ramp_eps_10_basica_step_5_10} to test image $\text{T}_2$. Its slice function with the smooth ramp was performed at the same conditions as before. 
\begin{figure}[hbt]
	\centering
     \subfloat[Pauli red channel\label{sim_image_red_pauli_mu1_greater_mu2_with_ramp_eps_10_basica_step_5_10}]{%
       \includegraphics[width=.4\linewidth]{fig_Pauli_c_1_smooth_ramp_mu1_greater_mu2_eps_10_step_5_10}
     }
     \subfloat[Slice function \label{sim_image_perfil_mu1_greater_mu2_with_ramp_eps_10_basica_step_5_10}]{%
       \includegraphics[width=.4\linewidth]{fig_perfil_x_400_c_2_smooth_ramp_mu1_greater_mu2_eps_10_step_5_10}
     }
     \caption{Image $\text{T}_3$ and its slice function with $\epsilon=10$}
     \label{sim_image_mu1_greater_mu2_with_ramp_eps_10_basica_step_5_10}
\end{figure}
%



\section{Edge detection}\label{sec:edge_detc}
In this work, we compare two popular methods to find edges in PolSAR images, the GA and a Constant False Alarm Rate (CFAR) that uses the coefficient of variation detector. 
\subsection{Gambini's algorithm -- GA}
The GA is actually not an edge detector. It's main purpose is to find the transition point between two different Gamma distributions in neighbor areas. This transition point can be seen, and have been used as an edge detector. The algorithm is presented in Refs.~\cite{gmbf, fbgm}, and they are based on the density distributions \textbf{$\mathcal{G}^0$}, that can be found in Refs.~\cite{fmcs,ffc}. We modified the algorithm to apply in Wishart distributions from PolSAR images. The method is described in Refs.~\cite{bmf_2020,bmf_2019}. 
\subsection{CFAR Coefficient of Variation Detector}
One of the methods shown in Refs [9] is called CFAR Coefficient of Variation Detector. It is a CFAR type method known for edge detection in radar images implemented as follows:
\begin{itemize}
\item Calculate the mean $M$ for the center pixel defined by the sliding window with dimension $N = n \times n$ pixels, where $n$ is the width and height of the window.  $M=\big(\sum_{i=1}^N x_i\big)/ N$. 

\item Calculate the standard deviation $S$ for the center pixel using the same window before. $S=\sum_{i=1}^N \big(x_i-M\big)^2/ N - 1$.

\item For each pixel, calculate the ratio  $r=S/M$.

\item Compare $r$ to the threshold $\ell_1=1/\sqrt{L} + \tau$, where L is the multi-look parameter. Large values of r are considered edge evidence, then if $r> \ell_1$, set the pixel as an edge. 

\end{itemize}
%\begin{enumerate}[label=(\roman*)]
%	\item \label{item:cfar_i} Calculate the mean for the center pixel defined by the sliding window with dimension $N = n \times n$ pixels, where $n$ is the width and height of the window. Moreover, calculate the average using the equation $M=\big(\sum_{i=1}^N x_i\big)/ N$. 
%	\item \label{item:cfar_ii} Calculate the standard deviation for the center pixel using the same window before, with the formula $S=\sum_{i=1}^N \big(x_i-M\big)^2/ N - 1$.
%    \item \label{item:cfar_ii} For each pixel, calculate the ratio  $r=S/M$.
%    \item \label{item:cfar_iii} Compare $r$ to the threshold $\ell_1=1/\sqrt{L} + \tau$, where L is the multi-look parameter. Large values of r are considered edges evidence, then if $r> \ell_1$, it is defined as an edge evidence. 
%\end{enumerate}



\section{Results}\label{sec:results}
We applied the GA in test images $\text{T}_1$, $\text{T}_2$, and $\text{T}_3$. Tests were performed for $\epsilon$ equals $0$ and $10$. The results were different for these two cases.  

Fig.~\ref{sim_image_mu1_greater_mu2_with_ramp_eps_0} shows edges evidences detected for image test $\text{T}_1$ and  $\text{T}_2$. We can see that GA presents similar results for both images and no outliers. 
\begin{figure}[hbt]
	\centering
	 \subfloat[Image test $\text{T}_1$ \label{fig_c_1_mu1_greater_mu2_eps_10_radial}]{%
       \includegraphics[width=.4\linewidth]{fig_PI_c_1_smooth_ramp_mu1_greater_mu2_eps_0_radial}
     }                                          
     \subfloat[Image test $\text{T}_2$ \label{fig_c_1_mu1_greater_mu2_eps_10_negative_radial}]{%
       \includegraphics[width=.4\linewidth]{fig_PI_c_1_smooth_ramp_mu1_greater_mu2_eps_0_negative_radial}
     }%     
     \caption{Edge evidences to images test $\text{T}_1$ and $\text{T}_2$ with $\epsilon=0$}
     \label{sim_image_mu1_greater_mu2_with_ramp_eps_0}
\end{figure}

Fig.~\ref{sim_image_mu1_greater_mu2_with_ramp_eps_10} shows edges evidences detected when we added a ramp around the internal rectangle. Fig.~\ref{sim_image_mu1_greater_mu2_with_ramp_eps_10}\subref{fig_c_1_mu1_greater_mu2_eps_10_radial} shows edge evidence near the outer region. Fig~\ref{sim_image_mu1_greater_mu2_with_ramp_eps_10}\subref{fig_c_1_mu1_greater_mu2_eps_10_negative_radial} shows edge evidence internal to the rectangle. Therefore, we can note that edges evidences are detected when the sample value goes to zero in the image test $\text{T}_1$ and $\text{T}_2$  .
% \begin{figure}[hbt]
%	\centering
%     \subfloat[Image test $\text{T}_1$\label{fig_c_1_mu1_greater_mu2_eps_8_radial}]{%
%       \includegraphics[width=.5\linewidth]{fig_PI_c_1_smooth_ramp_mu1_greater_mu2_eps_08_radial}
%     }
%     \subfloat[Image test $\text{T}_2$\label{fig_c_1_mu1_greater_mu2_eps_8_negative_radial}]{%
%       \includegraphics[width=.5\linewidth]{fig_PI_c_1_smooth_ramp_mu1_greater_mu2_eps_08_negative_radial}
%     }%
%     \caption{Edge evidences to images test $\text{T}_1$ and $\text{T}_2$ with $\epsilon=8$}
%     \label{sim_image_mu1_greater_mu2_with_ramp_eps_8}
%\end{figure} 
%%
% \begin{figure}[hbt]
%	\centering
%     \subfloat[Image test $\text{T}_1$\label{fig_c_1_mu1_greater_mu2_eps_9_radial}]{%
%       \includegraphics[width=.5\linewidth]{fig_PI_c_1_smooth_ramp_mu1_greater_mu2_eps_09_radial}
%     }                                          
%     \subfloat[Image test $\text{T}_2$\label{fig_c_1_mu1_greater_mu2_eps_9_negative_radial}]{%
%       \includegraphics[width=.5\linewidth]{fig_PI_c_1_smooth_ramp_mu1_greater_mu2_eps_09_negative_radial}
%     }%
%     \caption{Edge evidences to images test $\text{T}_1$ and $\text{T}_2$ with $\epsilon=9$}
%     \label{sim_image_mu1_greater_mu2_with_ramp_eps_9}
%\end{figure} 
%
 \begin{figure}[hbt]
	\centering
     \subfloat[Image test $\text{T}_1$ \label{fig_c_1_mu1_greater_mu2_eps_10_radial}]{%
       \includegraphics[width=.4\linewidth]{fig_PI_c_1_smooth_ramp_mu1_greater_mu2_eps_10_radial}
     }                                          
     \subfloat[Image test $\text{T}_2$ \label{fig_c_1_mu1_greater_mu2_eps_10_negative_radial}]{%
       \includegraphics[width=.4\linewidth]{fig_PI_c_1_smooth_ramp_mu1_greater_mu2_eps_10_negative_radial}
     }%
     \caption{Edge evidences to images test $\text{T}_1$ and $\text{T}_2$ with $\epsilon=10$}
     \label{sim_image_mu1_greater_mu2_with_ramp_eps_10}
\end{figure}

Looking at the results presented in Fig.~\ref{sim_image_mu1_greater_mu2_with_ramp_eps_10} we thought that the area with the 0 value could be a problem, so we decided to create image tests $\text{T}_3$ and $\text{T}_4$ (not shown here) with $\epsilon=10$ and applied the GA to these images. In both cases the result was the same as presented in Fig.~\ref{fig_c_1_mu1_greater_mu2_eps_10 step_5_10_radial} with the edge at about the center of the ramp.
 \begin{figure}[hbt]
	\centering
     {%
       \includegraphics[width=.4\linewidth]{fig_PI_c_1_smooth_ramp_mu1_greater_mu2_eps_10_radial_step_5_10}
     }%
     \caption{Edge evidences to image test $\text{T}_3$ with $\epsilon=10$}
     \label{fig_c_1_mu1_greater_mu2_eps_10 step_5_10_radial}
\end{figure}
%
% Cfar results.
%

The Coefficient of Variation Detector was applied to image test $\text{T}_1$ and $\text{T}_2$, and the results are shown in Figs.~\ref{fig_PI_c_1_smooth_ramp_mu1_greater_mu2_eps_0_cfar_coef_var_step_0_1_kernel_3}. We configured the method with a kernel value equals to $3\times 3$, and $\tau=0.3$, to control the threshold value and reduce outliers.
 \begin{figure}[hbt]
	\centering
     \subfloat[Image test $\text{T}_1$ \label{fig_PI_c_1_smooth_ramp_mu1_greater_mu2_eps_0_cfar_coef_var_step_0_1_kernel_3_positive}]{%
       \includegraphics[width=.4\linewidth]{fig_PI_c_1_smooth_ramp_mu1_greater_mu2_eps_0_cfar_coef_var_step_0_1_kernel_3}
     }
     \subfloat[Image test $\text{T}_2$\label{fig_PI_c_1_smooth_ramp_mu1_greater_mu2_eps_0_cfar_coef_var_step_0_1_kernel_3_negative}]{%
       \includegraphics[width=.4\linewidth]{fig_PI_c_1_smooth_ramp_mu1_greater_mu2_eps_0_negative_cfar_coef_var_step_0_1_kernel_3}
     }%
     \caption{Edge evidences to images test $\text{T}_1$ and $\text{T}_2$ with $\epsilon=0$}
     \label{fig_PI_c_1_smooth_ramp_mu1_greater_mu2_eps_0_cfar_coef_var_step_0_1_kernel_3}
\end{figure} 
%

Because of the outliers shown in Fig~\ref{fig_PI_c_1_smooth_ramp_mu1_greater_mu2_eps_0_cfar_coef_var_step_0_1_kernel_3}. we changed the CFAR parameters using a $5 \times 5$ kernel and $\tau$ equals 0.5. The result is presented in Fig.~\ref{fig_PI_c_1_smooth_ramp_mu1_greater_mu2_eps_10_cfar_coef_var_step_0_1_kernel_5}. In Figs.~\ref{fig_PI_c_1_smooth_ramp_mu1_greater_mu2_eps_10_cfar_coef_var_step_0_1_kernel_5}\subref{fig_PI_c_1_smooth_ramp_mu1_greater_mu2_eps_10_positive_cfar_coef_var_step_0_1_kernal_5} and~\ref{fig_PI_c_1_smooth_ramp_mu1_greater_mu2_eps_10_cfar_coef_var_step_0_1_kernel_5}\subref{fig_PI_c_1_smooth_ramp_mu1_greater_mu2_eps_10_negative_cfar_coef_var_step_0_1_kernal_5}, we observe a similar behavior to GA. Evidence of edges was detected near the zero-based sample.
 \begin{figure}[hbt]
	\centering
     \subfloat[Image test $\text{T}_1$ \label{fig_PI_c_1_smooth_ramp_mu1_greater_mu2_eps_10_positive_cfar_coef_var_step_0_1_kernal_5}]{%
       \includegraphics[width=.4\linewidth]{fig_PI_c_1_smooth_ramp_mu1_greater_mu2_eps_10_cfar_coef_var_step_0_1_kernal_5}
     }
     \subfloat[Image test $\text{T}_2$ \label{fig_PI_c_1_smooth_ramp_mu1_greater_mu2_eps_10_negative_cfar_coef_var_step_0_1_kernal_5}]{%
       \includegraphics[width=.4\linewidth]{fig_PI_c_1_smooth_ramp_mu1_greater_mu2_eps_10_negative_cfar_coef_var_step_0_1_kernal_5}
     }%
     \caption{Edge evidences to images test $\text{T}_1$ and $\text{T}_2$ with $\epsilon=10$}
     \label{fig_PI_c_1_smooth_ramp_mu1_greater_mu2_eps_10_cfar_coef_var_step_0_1_kernel_5}
\end{figure}
%
%

The Coefficient of Variation Detector was applied to image test $\text{T}_3$, and the results are shown in Figs.~\ref{fig_PI_c_1_smooth_ramp_mu1_greater_mu2_eps_10_cfar_coef_var_step_5_10_kernal_5}. We configured the method with a kernel value equals to $5 \times 5$, and defined $\ell_2$ as $\frac{1}{5} \sqrt{L} + \tau$. This threshold was defined empirically. The threshold original defined in Ref.\cite{tlb} ($\ell_1$) did not work well to test image $\text{T}_3$.  Fig.~\ref{fig_PI_c_1_smooth_ramp_mu1_greater_mu2_eps_10_cfar_coef_var_step_5_10_kernal_5}\subref{fig_PI_c_1_smooth_ramp_mu1_greater_mu2_eps_10_without_cfar_coef_var_step_5_10_kernal_5} shows that no edge evidence was detected with the $\ell_1$ threshold.

Fig.\ref{fig_PI_c_1_smooth_ramp_mu1_greater_mu2_eps_10_cfar_coef_var_step_5_10_kernal_5}\subref{fig_PI_c_1_smooth_ramp_mu1_greater_mu2_eps_10_positive_cfar_coef_var_step_5_10_kernal_5} shows the edge evidence using the threshold  $\ell_2$ for the Coefficient of Variation Detector.
 \begin{figure}[hbt]
	\centering
     \subfloat[ Image test $\text{T}_3$ with $\ell_2$ \label{fig_PI_c_1_smooth_ramp_mu1_greater_mu2_eps_10_positive_cfar_coef_var_step_5_10_kernal_5}]{%
        \includegraphics[width=0.4\linewidth]{fig_PI_c_1_smooth_ramp_mu1_greater_mu2_eps_10_cfar_coef_var_step_5_10_kernal_5}
        }
     \subfloat[Image test $\text{T}_3$ with $\ell_1$ \label{fig_PI_c_1_smooth_ramp_mu1_greater_mu2_eps_10_without_cfar_coef_var_step_5_10_kernal_5}]{%
       \includegraphics[width=0.4\linewidth]{fig_PI_c_1_smooth_ramp_mu1_greater_mu2_eps_10_without_edge_cfar_coef_var_step_5_10_kernal_5}
       }%
     \caption{Edge evidences to images test $\text{T}_3$ with $\epsilon=10$}
     \label{fig_PI_c_1_smooth_ramp_mu1_greater_mu2_eps_10_cfar_coef_var_step_5_10_kernal_5}
\end{figure}

We emphasize that the tests performed to detect evidence of edges do not depend on the order  between $\mu_1$ and $\mu_2$. Also, all PolSAR channels show similar results to the $\text{HH}$ channel used to generate the pictures presented in this paper.

 We can highlight the following results:
 
% When the tests are performed using  $\text{T}_1$ and $\text{T}_2$, the edge was detected near sample based on 0. For both detectors. The CFAR coefficient of variation detects multiple edges when there is a ramp edge in the image and needs further processing to refine its result and the GA finds a single edge in both cases, sharp or ramp edges.  

%When the tests were performed in $\text{T}_3$, the GA method finds edge evidence within the ramp.

%In the tests performed with the negative image of $\text{T}_3$, the results for GA were similar to the tests performed on the test image $\text{T}_3$.

%The CFAR coefficient of variation detector method has a high sensitivity to the choice of the $\tau$ parameter.

%The CFAR coefficient of variation detector is applied in $\text{T}_1$, $\text{T}_2$ and $\text{T}_3$, and there are arising outliers. We can note that the method has  sensitivity in the choice of the $\tau$ parameter.

%The CFAR coefficient of variation detector is applied in $\text{T}_3$. In this case, the threshold that does not respect the rule proposed in the article \cite{tlb}. We propose empirically the threshold $\ell_2=\frac{1}{5}\sqrt{L}+\tau$ which showed good results visually.

%In the CFAR coefficient of variation detector with test image $\text{T}_3$ almost all strip region was detected. 
 
\begin{enumerate}[label=(\roman*)]
	\item \label{item:conc_i} When the tests are performed using  $\text{T}_1$ and $\text{T}_2$, the edge was detected near sample based on 0 for both detectors. The Coefficient of Variation Detector detects multiple edges when there is a ramp edge in the image and needs further processing to refine its result. In this case the GA finds a single edge.   
	\item \label{item:conc_i} When the tests were performed in $\text{T}_3$ and its negative, the GA method finds edge evidence within the ramp.

	\item  \label{item:conc_v} The Coefficient of Variation Detector is applied in $\text{T}_1$, $\text{T}_2$ and $\text{T}_3$, and there are arising outliers. We can note that the method has  sensitivity in the choice of the $\tau$ parameter.
	\item  \label{item:conc_vi} The Coefficient of Variation Detector is applied in $\text{T}_3$. In this case, the threshold does not respect the rule proposed in the article \cite{tlb}. We propose empirically the threshold $\ell_2=\frac{1}{5}\sqrt{L}+\tau$ which showed good results visually.
	\item  \label{item:conc_vii} In the Coefficient of Variation Detector with test image $\text{T}_3$ almost all ramp region was detected. 
\end{enumerate} 

\section{Conclusions and discussions}\label{sec:conc_disc}
This work investigates how the GA and the CFAR Coefficient of Variation Detector method detect edge evidence. The two detectors worked similarly when applied to the simulated images tests $\text{T}_1$ and $\text{T}_2$. However, the Coefficient of Variation Detector finds multiple edges in for these images which does not happen when using GA.

We can point out that the detectors work differently on the $\text{T}_3$ test, while for GA, the evidence of edges was shifted to the center on the ramp. The Coefficient of Variation Detector finds almost the entire ramp region as edges and would need further processing to narrow the edge in that area. 

In all test performed the GA showed an advantage over the Coefficient of Variation Detector  as it detects only one pixel as edge evidence.

On the other hand, GA compared with Variable Coefficients Detector presents the limitation of defining the region of interest for edge detection and the lines for detecting the transition point between Wishart distributions. At the same time, the Coefficient of Variation Detector method can be applied to the entire image.

As a future work we plan to apply these methods to real PolSAR images and also investigate the values of $\mu_1$ and $\mu_2$ related to the values added to the Wishart distribution.

\bibliographystyle{IEEEtran}
\bibliography{strings,refs}
\end{document}
