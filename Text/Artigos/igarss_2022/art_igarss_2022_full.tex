% Template for IGARSS-2020 paper; to be used with:
%          spconf.sty  - LaTeX style file, and
%          IEEEbib.bst - IEEE bibliography style file.
% --------------------------------------------------------------------------
\documentclass{article}
\usepackage{spconf,amsmath}
\usepackage{bm,bbm}                                        % AAB inserted
\usepackage[boxed]{algorithm2e}                            % AAB inserted
\usepackage[caption=false,font=footnotesize]{subfig}       % AAB inserted
\usepackage[binary-units]{siunitx}                         % AAB inserted
\usepackage{booktabs}                                      % AAB inserted
\usepackage[hidelinks]{hyperref}                                           % AAB inserted
\usepackage{cite}
\usepackage{tikz}                                          % AAB inserted 
\usetikzlibrary{shapes,arrows,shadows}                     % AAB inserted
\usepackage{enumitem}                                      % AAB inserted
\usepackage{calrsfs}                                       % AAB inserted
\usepackage{amsfonts}                                      % AAB inserted
%
\graphicspath{{../igarss_2022/figures_igarss_2022/}}        % AAB inserted
%
\DeclareMathOperator{\traco}{tr}                           % AAB inserted
% Example definitions.
% --------------------
\def\x{{\mathbf x}}
\def\L{{\cal L}}
% Title.
% ------
\title{Title}
%
% Single address.
% ---------------
\name{Anderson A.\ de Borba$^a$, Maurício Marengoni$^b$, and Alejandro C.\ Frery$^c$.\thanks{e-mail:$^a$anderson.aborba@professores.ibmec.edu.br, $^b$mmarengoni@hotmail.com, $^c$alejandro.frery@vuw.ac.nz }}
\address{$^a$IBMEC-SP, Alameda Santos, 2356 - Jardim Paulista, SP -- Brazil, \\
$^b$Departamento de Ciência da Computação, Universidade Federal de Minas Gerais, MG -- Brazil,           \\
$^c$School of Mathematics and Statistics, Victoria University of Wellington, 6140, New Zealand.}


%\title{Fusion of Evidences in Intensity Channels for Edge Detection in PolSAR Images}
%\author{Anderson A.\ de Borba, Maurício Marengoni, and Alejandro C.\ Frery,~\IEEEmembership{Senior Member,~IEEE}%
%\thanks{This study was financed in part by the Coordenação de Aperfeiçoamento de Pessoal de Nível Superior - Brasil (CAPES) - Finance Code 001}
%\thanks{A.\ A.\ de Borba is with the Dept.\ Engenharia Elétrica e Computação, Universidade Presbiteriana Mackenzie (UPM), and with IBMEC-SP, São Paulo, Brazil. anderson.aborba@professores.ibmec.edu.br}
%\thanks{M.\ Marengoni is with the Dept.\ Engenharia Elétrica e Computação,
%UPM, São Paulo, Brazil. mauricio.marengoni@mackenzie.br}
%\thanks{A.\ C.\ Frery is with the School of Mathematics and Statistics,
%Victoria University of Wellington, 6140, New Zealand. alejandro.frery@vuw.ac.nz}}

%
% For example:
% ------------
%\address{School\\
%	Department\\
%	Address}
%
% Two addresses (uncomment and modify for two-address case).
% ----------------------------------------------------------
%\twoauthors
%  {A. Author-one, B. Author-two\sthanks{Thanks to XYZ agency for funding.}}
%	{School A-B\\
%	Department A-B\\
%	Address A-B}
%  {C. Author-three, D. Author-four\sthanks{The fourth author performed the work
%	while at ...}}
%	{School C-D\\
%	Department C-D\\
%	Address C-D}
%
\begin{document}
%\ninept
%
\maketitle
%
\begin{abstract}

\end{abstract}
%
\begin{keywords}

\end{keywords}
%
\section{Introduction}
\label{sec:intro}
\section{Simulated Images}
\label{sec:SimImag}
We build the simulated images $800 \times 800 \times 3$ two-classes phantom images, the third dimensions is a channel number. Into the image, we insert a rectangle centered in the pixel $(x_c, y_c) = (400, 400)$ and defined below. One class we define like outside of the rectangle, and another class is inside. The classes are Wishart distributions with $L=4$, $\mu_1$ and $\mu_2$.

In the edge of the rectangle, we define a variable $\epsilon$ to both sides. This manner built a strip around the rectangle edges, inside this strip we using the same $L$ and the $\mu$ is arithmetic average between $\mu_1$ and $\mu_2$.

We insert the samples under a base image built like:

\begin{enumerate}[label=(\roman*)]
	\item \label{item:obj_i}  To put $0$ in the image $800 \times 800$, for all channels.
	\item \label{item:obj_ii} Into to the rectangle $[x_i+\epsilon, x_u-\epsilon]\times[y_i+\epsilon,y_u-\epsilon]$ set the value pixel $1$. Where:
	\begin{enumerate}[label=(\alph*)]
		\item $x_i=x_c - 100,$ lower horizontal coordinate.
		\item $x_u=x_c + 100,$ upper horizontal coordinate.
		\item $y_i=y_c - 50,$ lower vertical coordinate.
		\item $y_u=y_c + 50,$ upper vertical coordinate.
		\item $\epsilon,$ adjustment constant.
	\end{enumerate} 
	\item \label{item:obj_iii} In the horizontal strips $[x_i-\epsilon, x_u+\epsilon]\times[y_i-\epsilon,y_i+\epsilon]$, and $[x_i-\epsilon, x_u +\epsilon]\times[y_u-\epsilon,y_u+\epsilon]$, we define the function \eqref{eq01}.
	\item  \label{item:obj_iv} In the 4vertical strips $[x_i-\epsilon, x_i+\epsilon]\times[y_i-\epsilon,y_u+\epsilon]$, and $[x_u-\epsilon, x_u+\epsilon]\times[y_i-\epsilon,y_u+\epsilon]$, we define the function \eqref{eq02}.
	\item  \label{item:obj_v} In the corners, we using the function \eqref{eq03}.
\end{enumerate} 

The function for the horizontal strips is:
\begin{equation}f_x(x)=
  \begin{cases}
    0                 & \textrm{if } x < 0 \\
    6x^5 -15x^4 + 10x^3 & \textrm{if } 0 \leq x \leq 1 \\
    1                 & \textrm{if } x > 1.
  \end{cases}\label{eq01}
\end{equation}

The function for the vertical strips is:
\begin{equation}f_y(y)=
  \begin{cases}
    0                 & \textrm{if } y < 0 \\
    6y^5 -15y^4 + 10y^3 & \textrm{if } 0 \leq y \leq 1 \\
    1                 & \textrm{if } y > 1.
  \end{cases}\label{eq02}
\end{equation}

The function for the corners is:
\begin{eqnarray}
f(x,y) = f_x(x) \cdot f_y(y),& \textrm{with }(x,y)\in \mathbb{R}^2  
\label{eq03}
\end{eqnarray} 

The idea to built the simulated image was based in \cite{gamf}.

We use two images tests based in the procedure above and \cite{gamf}.
Firstly, the image test ($\text{T}_1$) where was defined the samples distributed as Wishart, where $L$, $\mu_1^{\text{c}_i}$, $\mu_2^{\text{c}_i}$, and $\mu_{av}^{\text{c}_1}= {(\mu_1^{\text{c}_i}+\mu_2^{\text{c}_i})}/{2}$ using to strips in the image. The index $c_i$ defines the channels, that is, $c_1$, $c_2$, and $c_3$ are respectively the intensities channels $\text{HH}$, $\text{HV}$, and $\text{VV}$.

Secondly, in the test ($\text{T}_2$) was done a negative image, that is, in the item \ref{item:obj_i} we define 1 to the pixels and the item \ref{item:obj_ii} put 0 to the pixel. The items from \ref{item:obj_iii} to \ref{item:obj_v} were modified as needed. 

To each channel in both images, we use $\mu_1^{\text{c}_i} > \mu_2^{\text{c}_i}$ and $\mu_2^{\text{c}_i} < \mu_1^{\text{c}_i}$. The parameters are using  in this article: 
\begin{enumerate}[label=(\roman*)]
		\item $\mu_1^{\text{c}_1}=0.042811$, $\mu_2^{\text{c}_1}=0.014380$;
		\item $\mu_1^{\text{c}_2}=0.035977$, $\mu_2^{\text{c}_2}=0.002789$;
		\item $\mu_1^{\text{c}_3}=0.066498$, $\mu_2^{\text{c}_3}=0.015387$.
\end{enumerate}
%
\begin{figure}[hbt]
	\centering
     \subfloat[Red channel\label{red_pauli_mu1_great_mu2}]{%
       \includegraphics[width=.32\linewidth]{fig_PI_c_1_smooth_ramp_mu1_greater_mu2}
     }
     \subfloat[Green channel\label{green_pauli_mu1_great_mu2}]{%
       \includegraphics[width=.32\linewidth]{fig_PI_c_2_smooth_ramp_mu1_greater_mu2}
     }
     \subfloat[Blue channel\label{blue_pauli_mu1_great_mu2}]{%
       %\includegraphics[width=0.2\textwidth]{example-image-a}
       \includegraphics[width=.32\linewidth]{fig_PI_c_3_smooth_ramp_mu1_greater_mu2}       
     }
     \caption{Pauli decomposition to simulated image with $\mu_1$ greater than $\mu_2$ and smooth ramp}
     \label{sim_image_mu1_greater_mu2_with_ramp}
\end{figure}
%
\begin{figure}[hbt]
	\centering
     \subfloat[Red channel\label{red_pauli_mu1_less_mu2}]{%
       \includegraphics[width=.32\linewidth]{fig_PI_c_1_smooth_ramp_mu1_less_mu2}
     }
     \subfloat[Green channel\label{green_pauli_mu1_less_mu2}]{%
       \includegraphics[width=.32\linewidth]{fig_PI_c_2_smooth_ramp_mu1_less_mu2}
     }
     \subfloat[Blue channel\label{blue_pauli_mu1_less_mu2}]{%
       %\includegraphics[width=0.2\textwidth]{example-image-a}
       \includegraphics[width=.32\linewidth]{fig_PI_c_3_smooth_ramp_mu1_less_mu2}       
     }
     \caption{Pauli decomposition to simulated image with $\mu_1$ less than $\mu_2$ and smooth ramp}
     \label{sim_image_mu1_less_mu2_with_ramp}
\end{figure}
%
\begin{figure}[hbt]
	\centering
     \subfloat[Red channel\label{red_pauli_mu1_great_mu2}]{%
       \includegraphics[width=.32\linewidth]{fig_PI_c_1_mu1_greater_mu2}
     }
     \subfloat[Green channel\label{green_pauli_mu1_great_mu2}]{%
       \includegraphics[width=.32\linewidth]{fig_PI_c_2_mu1_greater_mu2}
     }
     \subfloat[Blue channel\label{blue_pauli_mu1_great_mu2}]{%
       %\includegraphics[width=0.2\textwidth]{example-image-a}
       \includegraphics[width=.32\linewidth]{fig_PI_c_3_mu1_greater_mu2}       
     }
     \caption{Pauli decomposition to simulated image with $\mu_1$ greater than $\mu_2$}
     \label{sim_image_mu1_greater_mu2}
\end{figure}
%
\begin{figure}[hbt]
	\centering
     \subfloat[Red channel\label{red_pauli_mu1_great_mu2}]{%
       \includegraphics[width=.32\linewidth]{fig_PI_c_1_mu1_less_mu2}
     }
     \subfloat[Green channel\label{green_pauli_mu1_great_mu2}]{%
       \includegraphics[width=.32\linewidth]{fig_PI_c_2_mu1_less_mu2}
     }
     \subfloat[Blue channel\label{blue_pauli_mu1_great_mu2}]{%
       %\includegraphics[width=0.2\textwidth]{example-image-a}
       \includegraphics[width=.32\linewidth]{fig_PI_c_3_mu1_less_mu2}       
     }
     \caption{Pauli decomposition to simulated image with $\mu_1$ less than $\mu_2$}
     \label{sim_image_mu1_less_mu2}
\end{figure}
%
%
%\begin{figure}[hbt]
%	\centering
%     \subfloat[Channel hh\label{hh_mu1_great_mu2_with_smooth_ramp}]{%
%       \includegraphics[width=0.8\linewidth]{img_hh_mu1_greater_mu2_with_smooth_ramp}
%     }\\
%     \subfloat[Channel hv\label{hv_mu1_great_mu2_with_smooth_ramp}]{%
%       \includegraphics[width=0.8\linewidth]{img_hv_mu1_greater_mu2_with_smooth_ramp}
%     }\\
%     \subfloat[Channel vv\label{vv_mu1_great_mu2_with_smooth_ramp}]{%
%       %\includegraphics[width=0.2\textwidth]{example-image-a}
%       \includegraphics[width=0.8\linewidth]{img_vv_mu1_greater_mu2_with_smooth_ramp}       
%     }
%     \caption{Evidences in hh channel to simulated image with $\mu_1$ greater than $\mu_2$ and smooth ramp}
%     \label{sim_image_mu1_greater_mu2_with_ramp}
%\end{figure}
%
%\begin{figure}[hbt]
%	\centering
%     \subfloat[Channel hh\label{hh_mu1_less_mu2_with_smooth_ramp}]{%
%       \includegraphics[width=0.8\linewidth]{img_hh_mu1_less_mu2_with_smooth_ramp}
%     }\\
%     \subfloat[Channel hv\label{hv_mu1_less_mu2_with_smooth_ramp}]{%
%       \includegraphics[width=0.8\linewidth]{img_hv_mu1_less_mu2_with_smooth_ramp}
%     }\\
%     \subfloat[Channel vv\label{vv_mu1_less_mu2_with_smooth_ramp}]{%
%       %\includegraphics[width=0.2\textwidth]{example-image-a}
%       \includegraphics[width=0.8\linewidth]{img_vv_mu1_less_mu2_with_smooth_ramp}       
%     }
%     \caption{Evidences in hh channel to simulated image with $\mu_1$ less than $\mu_2$ and smooth ramp}
%     \label{sim_image_mu1_greater_mu2_with_ramp}
%\end{figure}
%
%\begin{figure}[hbt]
%	\centering
%     \subfloat[Channel hh\label{hh_mu1_great_mu2}]{%
%       \includegraphics[width=0.8\linewidth]{img_hh_mu1_greater_mu2}
%     }\\
%     \subfloat[Channel hv\label{hv_mu1_great_mu2}]{%
%       \includegraphics[width=0.8\linewidth]{img_hv_mu1_greater_mu2}
%     }\\
%     \subfloat[Channel vv\label{vv_mu1_great_mu2}]{%
%       %\includegraphics[width=0.2\textwidth]{example-image-a}
%       \includegraphics[width=0.8\linewidth]{img_vv_mu1_greater_mu2}       
%     }
%     \caption{Evidences in hh channel to simulated image with $\mu_1$ greater than $\mu_2$}
%     \label{sim_image_mu1_greater_mu2_with_ramp}
%\end{figure}
%
%\begin{figure}[hbt]
%	\centering
%     \subfloat[Channel hh\label{hh_mu1_less_mu2}]{%
%       \includegraphics[width=0.8\linewidth]{img_hh_mu1_less_mu2}
%     }\\
%     \subfloat[Channel hv\label{hv_mu1_less_mu2}]{%
%       \includegraphics[width=0.8\linewidth]{img_hv_mu1_less_mu2}
%     }\\
%     \subfloat[Channel vv\label{vv_mu1_less_mu2}]{%
%       %\includegraphics[width=0.2\textwidth]{example-image-a}
%       \includegraphics[width=0.8\linewidth]{img_vv_mu1_less_mu2}       
%     }
%     \caption{Evidences in hh channel to simulated image with $\mu_1$ less than $\mu_2$}
%     \label{sim_image_mu1_greater_mu2_with_ramp}
%\end{figure}
%
\section{Edge detection}
In this work, we compare two edge detector. The Gambini algorithm with the edge detector based in the constant false alarm rate (CFAR). 
\subsection{Gambini algorithm}
The methods were presented in \cite{gmbf} and \cite{fbgm}, and they are based in density distributions \textbf{$\mathcal{G}^0$} that can be founded in \cite{fmcs} or \cite{ffc}. We use the procedure with the Wishart distribution \cite{bmf_2020,bmf_2019}. The unique difference between 
\subsection{CFAR}
\section{Results}
 Applying the Gambini algorithm from Fig.~\ref{sim_image_mu1_greater_mu2_with_ramp_eps_8} until Fig.~\ref{sim_image_mu1_greater_mu2_with_ramp_eps_10} show $\text{T}_1$ and $t{T}_2$ with $\mu_1^{\text{c}_i}> \mu_2^{\text{c}_i}$ with $\epsilon=8$, $\epsilon=9$ and $\epsilon=10$. We can noted that edges evidences are detected when sample is based in zero. 

Similar results happen in the Fig.~\ref{sim_image_mu1_less_mu2_with_ramp_eps_8} until Fig.~\ref{sim_image_mu1_less_mu2_with_ramp_eps_10} that show $\text{T}_3$ and $\text{T}_4$ with $\mu_2^{\text{c}_i} < \mu_1^{\text{c}_i}$ with $\epsilon=8$, $\epsilon=9$ and $\epsilon=10$. 

The figures show the channels $\text{c}_1$ in the figures (a) and (b), the channel $\text{c}_2$ in the figures (c) and (d), and the channel $\text{c}_3$ in the figures (e) and (f).   
 \begin{figure}[hbt]
	\centering
     \subfloat[T1 with $\mu_1^{\text{c}_1} > \mu_2^{\text{c}_1}$ and $\epsilon = 8$ \label{fig_c_1_mu1_greater_mu2_eps_8}]{%
       \includegraphics[width=.5\linewidth]{fig_PI_c_1_smooth_ramp_mu1_greater_mu2_eps_08_radial}
     }
     \subfloat[T2 with $\mu_1^{\text{c}_1} > \mu_2^{\text{c}_1}$\label{fig_c_1_mu1_greater_mu2_eps_8_negative}]{%
       \includegraphics[width=.5\linewidth]{fig_PI_c_1_smooth_ramp_mu1_greater_mu2_eps_08_negative_radial}
     }\\
     \subfloat[T1 with $\mu_1^{\text{c}_2} > \mu_2^{\text{c}_2}$ \label{fig_c_2_mu1_greater_mu2_eps_8}]{%
       \includegraphics[width=.5\linewidth]{fig_PI_c_2_smooth_ramp_mu1_greater_mu2_eps_08_radial}
     }
     \subfloat[T2 with $\mu_1^{\text{c}_2} > \mu_2^{\text{c}_2}$\label{fig_c_2_mu1_greater_mu2_eps_8_negative}]{%
       \includegraphics[width=.5\linewidth]{fig_PI_c_2_smooth_ramp_mu1_greater_mu2_eps_08_negative_radial}
     }\\
     \subfloat[T1 with $\mu_1^{\text{c}_3} > \mu_2^{\text{c}_3}$ \label{fig_c_2_mu1_greater_mu2_eps_8}]{%
       \includegraphics[width=.5\linewidth]{fig_PI_c_3_smooth_ramp_mu1_greater_mu2_eps_08_radial}
     }
     \subfloat[T2 with $\mu_1^{\text{c}_3} > \mu_2^{\text{c}_3}$\label{fig_c_2_mu1_greater_mu2_eps_8_negative}]{%
       \includegraphics[width=.5\linewidth]{fig_PI_c_3_smooth_ramp_mu1_greater_mu2_eps_08_negative_radial}
     }
     \caption{The intensities channels to simulated image with $\mu_1^{\text{c}_i} > \mu_2^{\text{c}_i}$ and smooth ramp $\epsilon=8$}
     \label{sim_image_mu1_greater_mu2_with_ramp_eps_8}
\end{figure} 
%
 \begin{figure}[hbt]
	\centering
     \subfloat[T1 with $\mu_1^{\text{c}_1} > \mu_2^{\text{c}_1}$ and $\epsilon = 9$ \label{fig_c_1_mu1_greater_mu2_eps_9}]{%
       \includegraphics[width=.5\linewidth]{fig_PI_c_1_smooth_ramp_mu1_greater_mu2_eps_09_radial}
     }                                          
     \subfloat[T2 with $\mu_1^{\text{c}_1} > \mu_2^{\text{c}_1}$ and $\epsilon = 9$\label{fig_c_1_mu1_greater_mu2_eps_9_negative}]{%
       \includegraphics[width=.5\linewidth]{fig_PI_c_1_smooth_ramp_mu1_greater_mu2_eps_09_negative_radial}
     }\\                                   
     \subfloat[T1 with $\mu_1^{\text{c}_2} > \mu_2^{\text{c}_2}$ and $\epsilon = 9$ \label{fig_c_2_mu1_greater_mu2_eps_9}]{%
       \includegraphics[width=.5\linewidth]{fig_PI_c_2_smooth_ramp_mu1_greater_mu2_eps_09_radial}
     }
     \subfloat[T2 with $\mu_1^{\text{c}_2} > \mu_2^{\text{c}_2}$ and $\epsilon = 9$\label{fig_c_2_mu1_greater_mu2_eps_9_negative}]{%
       \includegraphics[width=.5\linewidth]{fig_PI_c_2_smooth_ramp_mu1_greater_mu2_eps_09_negative_radial}
     }\\
     \subfloat[T1 with $\mu_1^{\text{c}_3} > \mu_2^{\text{c}_3}$ and $\epsilon = 9$ \label{fig_c_2_mu1_greater_mu2_eps_9}]{%
       \includegraphics[width=.5\linewidth]{fig_PI_c_3_smooth_ramp_mu1_greater_mu2_eps_09_radial}
     }
     \subfloat[T2 with $\mu_1^{\text{c}_3} > \mu_2^{\text{c}_3}$ and $\epsilon = 9$\label{fig_c_2_mu1_greater_mu2_eps_9_negative}]{%
       \includegraphics[width=.5\linewidth]{fig_PI_c_3_smooth_ramp_mu1_greater_mu2_eps_09_negative_radial}
     }
     \caption{The intensities channels to simulated image with $\mu_1^{\text{c}_i} > \mu_2^{\text{c}_i}$ and smooth ramp $\epsilon=9$}
     \label{sim_image_mu1_greater_mu2_with_ramp_eps_9}
\end{figure} 
%
 \begin{figure}[hbt]
	\centering
     \subfloat[T1 with $\mu_1^{\text{c}_1} > \mu_2^{\text{c}_1}$ and $\epsilon = 10$ \label{fig_c_1_mu1_greater_mu2_eps_10}]{%
       \includegraphics[width=.5\linewidth]{fig_PI_c_1_smooth_ramp_mu1_greater_mu2_eps_10_radial}
     }                                          
     \subfloat[T2 with $\mu_1^{\text{c}_1} > \mu_2^{\text{c}_1}$ and $\epsilon = 10$\label{fig_c_1_mu1_greater_mu2_eps_10_negative}]{%
       \includegraphics[width=.5\linewidth]{fig_PI_c_1_smooth_ramp_mu1_greater_mu2_eps_10_negative_radial}
     }\\                                   
     \subfloat[T1 with $\mu_1^{\text{c}_2} > \mu_2^{\text{c}_2}$ and $\epsilon = 10$ \label{fig_c_2_mu1_greater_mu2_eps_10}]{%
       \includegraphics[width=.5\linewidth]{fig_PI_c_2_smooth_ramp_mu1_greater_mu2_eps_10_radial}
     }
     \subfloat[T2 with $\mu_1^{\text{c}_2} > \mu_2^{\text{c}_2}$ and $\epsilon = 10$\label{fig_c_2_mu1_greater_mu2_eps_10_negative}]{%
       \includegraphics[width=.5\linewidth]{fig_PI_c_2_smooth_ramp_mu1_greater_mu2_eps_10_negative_radial}
     }\\
     \subfloat[T1 with $\mu_1^{\text{c}_3} > \mu_2^{\text{c}_3}$ and $\epsilon = 10$ \label{fig_c_2_mu1_greater_mu2_eps_10}]{%
       \includegraphics[width=.5\linewidth]{fig_PI_c_3_smooth_ramp_mu1_greater_mu2_eps_10_radial}
     }
     \subfloat[T2 with $\mu_1^{\text{c}_3} > \mu_2^{\text{c}_3}$ and $\epsilon = 10$\label{fig_c_2_mu1_greater_mu2_eps_10_negative}]{%
       \includegraphics[width=.5\linewidth]{fig_PI_c_3_smooth_ramp_mu1_greater_mu2_eps_10_negative_radial}
     }
     \caption{The intensities channels to simulated image with $\mu_1^{\text{c}_i} > \mu_2^{\text{c}_i}$ and smooth ramp $\epsilon=10$}
     \label{sim_image_mu1_greater_mu2_with_ramp_eps_10}
\end{figure}
%
 \begin{figure}[hbt]
	\centering
     \subfloat[T1 with $\mu_2^{\text{c}_1} < \mu_1^{\text{c}_1}$ and $\epsilon = 8$ \label{fig_c_1_mu1_less_mu2_eps_08}]{%
       \includegraphics[width=.5\linewidth]{fig_PI_c_1_smooth_ramp_mu1_less_mu2_eps_08_radial}
     }                                          
     \subfloat[T2 with $\mu_2^{\text{c}_1} < \mu_1^{\text{c}_1}$ and $\epsilon = 8$\label{fig_c_1_mu1_less_mu2_eps_08_negative}]{%
       \includegraphics[width=.5\linewidth]{fig_PI_c_1_smooth_ramp_mu1_less_mu2_eps_08_negative_radial}
     }\\                                   
     \subfloat[T1 with $\mu_2^{\text{c}_2} < \mu_1^{\text{c}_2}$ and $\epsilon = 8$ \label{fig_c_2_mu1_less_mu2_eps_08}]{%
       \includegraphics[width=.5\linewidth]{fig_PI_c_2_smooth_ramp_mu1_less_mu2_eps_08_radial}
     }
     \subfloat[T2 with $\mu_2^{\text{c}_2} < \mu_1^{\text{c}_2}$ and $\epsilon = 8$\label{fig_c_2_mu1_less_mu2_eps_08_negative}]{%
       \includegraphics[width=.5\linewidth]{fig_PI_c_2_smooth_ramp_mu1_less_mu2_eps_08_negative_radial}
     }\\
     \subfloat[T1 with $\mu_2^{\text{c}_3} < \mu_1^{\text{c}_3}$ and $\epsilon = 8$ \label{fig_c_2_mu1_less_mu2_eps_08}]{%
       \includegraphics[width=.5\linewidth]{fig_PI_c_3_smooth_ramp_mu1_less_mu2_eps_08_radial}
     }
     \subfloat[T2 with $\mu_2^{\text{c}_3} < \mu_1^{\text{c}_3}$ and $\epsilon = 8$\label{fig_c_2_mu1_less_mu2_eps_08_negative}]{%
       \includegraphics[width=.5\linewidth]{fig_PI_c_3_smooth_ramp_mu1_less_mu2_eps_08_negative_radial}
     }
     \caption{The intensities channels to simulated image with $\mu_2^{\text{c}_i} < \mu_1^{\text{c}_i}$ and smooth ramp $\epsilon=8$}
     \label{sim_image_mu1_less_mu2_with_ramp_eps_8}
\end{figure} 
%
 \begin{figure}[hbt]
	\centering
     \subfloat[T1 with $\mu_2^{\text{c}_1} < \mu_1^{\text{c}_1}$ and $\epsilon = 9$ \label{fig_c_1_mu1_less_mu2_eps_09}]{%
       \includegraphics[width=.5\linewidth]{fig_PI_c_1_smooth_ramp_mu1_less_mu2_eps_09_radial}
     }                                          
     \subfloat[T2 with $\mu_2^{\text{c}_1} < \mu_1^{\text{c}_1}$ and $\epsilon = 9$\label{fig_c_1_mu1_less_mu2_eps_09_negative}]{%
       \includegraphics[width=.5\linewidth]{fig_PI_c_1_smooth_ramp_mu1_less_mu2_eps_09_negative_radial}
     }\\                                   
     \subfloat[T1 with $\mu_2^{\text{c}_2} < \mu_1^{\text{c}_2}$ and $\epsilon = 9$ \label{fig_c_2_mu1_less_mu2_eps_09}]{%
       \includegraphics[width=.5\linewidth]{fig_PI_c_2_smooth_ramp_mu1_less_mu2_eps_09_radial}
     }
     \subfloat[T2 with $\mu_2^{\text{c}_2} < \mu_1^{\text{c}_2}$ and $\epsilon = 9$\label{fig_c_2_mu1_less_mu2_eps_09_negative}]{%
       \includegraphics[width=.5\linewidth]{fig_PI_c_2_smooth_ramp_mu1_less_mu2_eps_09_negative_radial}
     }\\
     \subfloat[T1 with $\mu_2^{\text{c}_3} < \mu_1^{\text{c}_3}$ and $\epsilon = 9$ \label{fig_c_2_mu1_less_mu2_eps_09}]{%
       \includegraphics[width=.5\linewidth]{fig_PI_c_3_smooth_ramp_mu1_less_mu2_eps_09_radial}
     }
     \subfloat[T2 with $\mu_2^{\text{c}_3} < \mu_1^{\text{c}_3}$ and $\epsilon = 9$\label{fig_c_2_mu1_less_mu2_eps_09_negative}]{%
       \includegraphics[width=.5\linewidth]{fig_PI_c_3_smooth_ramp_mu1_less_mu2_eps_09_negative_radial}
     }
     \caption{The intensities channels to simulated image with $\mu_2^{\text{c}_i} < \mu_1^{\text{c}_i}$ and smooth ramp $\epsilon=9$}
     \label{sim_image_mu1_less_mu2_with_ramp_eps_9}
\end{figure}
%
 \begin{figure}[hbt]
	\centering
     \subfloat[T1 with $\mu_2^{\text{c}_1} < \mu_1^{\text{c}_1}$ and $\epsilon = 10$ \label{fig_c_1_mu1_less_mu2_eps_10}]{%
       \includegraphics[width=.5\linewidth]{fig_PI_c_1_smooth_ramp_mu1_less_mu2_eps_10_radial}
     }                                          
     \subfloat[T2 with $\mu_2^{\text{c}_1} < \mu_1^{\text{c}_1}$ and $\epsilon = 10$\label{fig_c_1_mu1_less_mu2_eps_10_negative}]{%
       \includegraphics[width=.5\linewidth]{fig_PI_c_1_smooth_ramp_mu1_less_mu2_eps_10_negative_radial}
     }\\                                   
     \subfloat[T1 with $\mu_2^{\text{c}_1} < \mu_1^{\text{c}_1}$ and $\epsilon = 10$ \label{fig_c_2_mu1_less_mu2_eps_10}]{%
       \includegraphics[width=.5\linewidth]{fig_PI_c_2_smooth_ramp_mu1_less_mu2_eps_10_radial}
     }
     \subfloat[T2 with $\mu_2^{\text{c}_1} < \mu_1^{\text{c}_1}$ and $\epsilon = 10$\label{fig_c_2_mu1_less_mu2_eps_10_negative}]{%
       \includegraphics[width=.5\linewidth]{fig_PI_c_2_smooth_ramp_mu1_less_mu2_eps_10_negative_radial}
     }\\
     \subfloat[T1 with $\mu_2^{\text{c}_1} < \mu_1^{\text{c}_1}$ and $\epsilon = 10$ \label{fig_c_2_mu1_less_mu2_eps_10}]{%
       \includegraphics[width=.5\linewidth]{fig_PI_c_3_smooth_ramp_mu1_less_mu2_eps_10_radial}
     }
     \subfloat[T2 with $\mu_2^{\text{c}_1} < \mu_1^{\text{c}_1}$ and $\epsilon = 10$\label{fig_c_2_mu1_less_mu2_eps_10_negative}]{%
       \includegraphics[width=.5\linewidth]{fig_PI_c_3_smooth_ramp_mu1_less_mu2_eps_10_negative_radial}
     }
     \caption{The intensities channels to simulated image with $\mu_2^{\text{c}_i} < \mu_1^{\text{c}_i}$ and smooth ramp $\epsilon=10$}
     \label{sim_image_mu1_less_mu2_with_ramp_eps_10}
\end{figure} 
%
Applying the Gambini algorithm from Fig.~\ref{sim_image_mu1_greater_mu2_with_ramp_eps_10_step_5_10} until Fig.~\ref{sim_image_mu1_greater_mu2_with_ramp_eps_10_step_5_10} with simulated image $\text{T}_5$ and $t{T}_6$ with $\mu_1^{\text{c}_i}> \mu_2^{\text{c}_i}$ with $\epsilon=10$, the edges have a little displacement to center the strip. 
 \begin{figure}[hbt]
	\centering
     \subfloat[T1 with $\mu_1^{\text{c}_1} > \mu_2^{\text{c}_1}$ and $\epsilon = 10$ and step from 5 to 10 \label{fig_c_1_mu1_less_mu2_eps_10_1}]{%
       \includegraphics[width=.5\linewidth]{fig_PI_c_1_smooth_ramp_mu1_greater_mu2_eps_10_radial_step_5_10}
     }
     \subfloat[T1 with $\mu_1^{\text{c}_1} > \mu_2^{\text{c}_1}$ and $\epsilon = 10$ and step from 5 to 10 \label{fig_c_1_mu1_less_mu2_eps_10_2}]{%
       \includegraphics[width=.5\linewidth]{fig_PI_c_1_smooth_ramp_mu1_greater_mu2_eps_10_negative_radial_step_5_10}
     }\\
     \subfloat[T1 with $\mu_1^{\text{c}_1} > \mu_2^{\text{c}_1}$ and $\epsilon = 10$ and step from 5 to 10\label{fig_c_2_mu1_less_mu2_eps_10}]{%
       \includegraphics[width=.5\linewidth]{fig_PI_c_2_smooth_ramp_mu1_greater_mu2_eps_10_radial_step_5_10}
     }
     \subfloat[T2 with $\mu_1^{\text{c}_1} > \mu_2^{\text{c}_1}$ and $\epsilon = 10$ and step from 5 to 10\label{fig_c_2_mu1_less_mu2_eps_10_negative}]{%
       \includegraphics[width=.5\linewidth]{fig_PI_c_2_smooth_ramp_mu1_greater_mu2_eps_10_negative_radial_step_5_10}
     }\\
     \subfloat[T1 with $\mu_1^{\text{c}_1} > \mu_2^{\text{c}_1}$ and $\epsilon = 10$ and step from 5 to 10\label{fig_c_2_mu1_less_mu2_eps_10}]{%
       \includegraphics[width=.5\linewidth]{fig_PI_c_3_smooth_ramp_mu1_greater_mu2_eps_10_radial_step_5_10}
     }
     \subfloat[T2 with $\mu_1^{\text{c}_1} > \mu_2^{\text{c}_1}$ and $\epsilon = 10$ and step from 5 to 10\label{fig_c_2_mu1_less_mu2_eps_10_negative}]{%
       \includegraphics[width=.5\linewidth]{fig_PI_c_3_smooth_ramp_mu1_greater_mu2_eps_10_negative_radial_step_5_10}
     }
     \caption{The intensities channels to simulated image with $\mu_1^{\text{c}_i} > \mu_2^{\text{c}_i}$ and smooth ramp $\epsilon=10$ and step 5 to 10}
     \label{sim_image_mu1_less_mu2_with_ramp_eps_10_step_5_10}
\end{figure}
%
Similar results happen from Fig.~\ref{sim_image_mu1_greater_mu2_with_ramp_eps_10_step_10_5} until Fig.~\ref{sim_image_mu1_greater_mu2_with_ramp_eps_10_step_5_10} with simulated image $\text{T}_7$ and $t{T}_8$ with $\mu_1^{\text{c}_i}> \mu_2^{\text{c}_i}$ with $\epsilon=10$, the edges have a little displacement to center the strip.
 \begin{figure}[hbt]
	\centering
     \subfloat[T1 with $\mu_2^{\text{c}_1} < \mu_1^{\text{c}_1}$ and $\epsilon = 10$ \label{fig_c_1_mu1_less_mu2_eps_10_1}]{%
       \includegraphics[width=.5\linewidth]{fig_PI_c_1_smooth_ramp_mu1_less_mu2_eps_10_radial_step_5_10}
     }
     \subfloat[T1 with $\mu_2^{\text{c}_1} < \mu_1^{\text{c}_1}$ and $\epsilon = 10$ \label{fig_c_1_mu1_less_mu2_eps_10_2}]{%
       \includegraphics[width=.5\linewidth]{fig_PI_c_1_smooth_ramp_mu1_less_mu2_eps_10_negative_radial_step_5_10}
     }\\
     \subfloat[T1 with $\mu_2^{\text{c}_1} < \mu_1^{\text{c}_1}$ and $\epsilon = 10$ \label{fig_c_2_mu1_less_mu2_eps_10}]{%
       \includegraphics[width=.5\linewidth]{fig_PI_c_2_smooth_ramp_mu1_less_mu2_eps_10_radial_step_5_10}
     }
     \subfloat[T2 with $\mu_2^{\text{c}_1} < \mu_1^{\text{c}_1}$ and $\epsilon = 10$\label{fig_c_2_mu1_less_mu2_eps_10_negative}]{%
       \includegraphics[width=.5\linewidth]{fig_PI_c_2_smooth_ramp_mu1_less_mu2_eps_10_negative_radial_step_5_10}
     }\\
     \subfloat[T1 with $\mu_2^{\text{c}_1} < \mu_1^{\text{c}_1}$ and $\epsilon = 10$ \label{fig_c_2_mu1_less_mu2_eps_10}]{%
       \includegraphics[width=.5\linewidth]{fig_PI_c_3_smooth_ramp_mu1_less_mu2_eps_10_radial_step_5_10}
     }
     \subfloat[T2 with $\mu_2^{\text{c}_1} < \mu_1^{\text{c}_1}$ and $\epsilon = 10$\label{fig_c_2_mu1_less_mu2_eps_10_negative}]{%
       \includegraphics[width=.5\linewidth]{fig_PI_c_3_smooth_ramp_mu1_less_mu2_eps_10_negative_radial_step_5_10}
     }
     \caption{The intensities channels to simulated negative image with $\mu_2^{\text{c}_i} < \mu_1^{\text{c}_i}$ and smooth ramp $\epsilon=10$ and step 10 to 5}
     \label{sim_image_mu1_less_mu2_with_ramp_eps_10_step_10_5}
\end{figure}
%
% Cfar results.
%
 \begin{figure}[hbt]
	\centering
     \subfloat[T1 with $\mu_1^{\text{c}_1} < \mu_2^{\text{c}_1}$ and $\epsilon = 0$ \label{fig_c_1_mu1_less_mu2_eps_10_1}]{%
       \includegraphics[width=.5\linewidth]{fig_PI_c_1_smooth_ramp_mu1_greater_mu2_eps_0_cfar_step_0_1}
     }
     \subfloat[T1 with $\mu_1^{\text{c}_1} < \mu_2^{\text{c}_1}$ and $\epsilon = 0$ \label{fig_c_1_mu1_less_mu2_eps_10_2}]{%
       \includegraphics[width=.5\linewidth]{fig_PI_c_1_smooth_ramp_mu1_greater_mu2_eps_0_negative_cfar_step_0_1}
     }\\
     \subfloat[T1 with $\mu_1^{\text{c}_2} < \mu_2^{\text{c}_2}$ and $\epsilon = 0$ \label{fig_c_2_mu1_less_mu2_eps_10}]{%
       \includegraphics[width=.5\linewidth]{fig_PI_c_2_smooth_ramp_mu1_greater_mu2_eps_0_cfar_step_0_1}
     }
     \subfloat[T2 with $\mu_1^{\text{c}_2} < \mu_2^{\text{c}_2}$ and $\epsilon = 0$\label{fig_c_2_mu1_less_mu2_eps_10_negative}]{%
       \includegraphics[width=.5\linewidth]{fig_PI_c_2_smooth_ramp_mu1_greater_mu2_eps_0_negative_cfar_step_0_1}
     }\\
     \subfloat[T1 with $\mu_1^{\text{c}_3} < \mu_2^{\text{c}_3}$ and $\epsilon = 0$ \label{fig_c_2_mu1_less_mu2_eps_10}]{%
       \includegraphics[width=.5\linewidth]{fig_PI_c_3_smooth_ramp_mu1_greater_mu2_eps_0_cfar_step_0_1}
     }
     \subfloat[T2 with $\mu_1^{\text{c}_3} < \mu_2^{\text{c}_3}$ and $\epsilon = 0$\label{fig_c_2_mu1_less_mu2_eps_10_negative}]{%
       \includegraphics[width=.5\linewidth]{fig_PI_c_3_smooth_ramp_mu1_greater_mu2_eps_0_negative_cfar_step_0_1}
     }
     \caption{CFAR threshold = 0.32. TThe intensities channels to simulated negative image with $\mu_2^{\text{c}_i} < \mu_1^{\text{c}_i}$ and smooth ramp $\epsilon=0$ and step 0 to 1}
     \label{sim_image_mu1_less_mu2_with_ramp_eps_10}
\end{figure} 
%
 \begin{figure}[hbt]
	\centering
     \subfloat[T1 with $\mu_1^{\text{c}_1} < \mu_2^{\text{c}_1}$ and $\epsilon = 10$ \label{fig_c_1_mu1_less_mu2_eps_10_1}]{%
       \includegraphics[width=.5\linewidth]{fig_PI_c_1_smooth_ramp_mu1_greater_mu2_eps_10_cfar_step_0_1}
     }
     \subfloat[T1 with $\mu_1^{\text{c}_1} < \mu_2^{\text{c}_1}$ and $\epsilon = 10$ \label{fig_c_1_mu1_less_mu2_eps_10_2}]{%
       \includegraphics[width=.5\linewidth]{fig_PI_c_1_smooth_ramp_mu1_greater_mu2_eps_10_negative_cfar_step_0_1}
     }\\
     \subfloat[T1 with $\mu_1^{\text{c}_2} < \mu_2^{\text{c}_2}$ and $\epsilon = 10$ \label{fig_c_2_mu1_less_mu2_eps_10}]{%
       \includegraphics[width=.5\linewidth]{fig_PI_c_2_smooth_ramp_mu1_greater_mu2_eps_10_cfar_step_0_1}
     }
     \subfloat[T2 with $\mu_1^{\text{c}_2} < \mu_2^{\text{c}_2}$ and $\epsilon = 10$\label{fig_c_2_mu1_less_mu2_eps_10_negative}]{%
       \includegraphics[width=.5\linewidth]{fig_PI_c_2_smooth_ramp_mu1_greater_mu2_eps_10_negative_cfar_step_0_1}
     }\\
     \subfloat[T1 with $\mu_1^{\text{c}_3} < \mu_2^{\text{c}_3}$ and $\epsilon = 10$ \label{fig_c_2_mu1_less_mu2_eps_10}]{%
       \includegraphics[width=.5\linewidth]{fig_PI_c_3_smooth_ramp_mu1_greater_mu2_eps_10_cfar_step_0_1}
     }
     \subfloat[T2 with $\mu_1^{\text{c}_3} < \mu_2^{\text{c}_3}$ and $\epsilon = 10$\label{fig_c_2_mu1_less_mu2_eps_10_negative}]{%
       \includegraphics[width=.5\linewidth]{fig_PI_c_3_smooth_ramp_mu1_greater_mu2_eps_10_negative_cfar_step_0_1}
     }
     \caption{CFAR threshold = 0.085.The intensities channels to simulated negative image with $\mu_2^{\text{c}_i} < \mu_1^{\text{c}_i}$ and smooth ramp $\epsilon=10$ }
     \label{sim_image_mu1_less_mu2_with_ramp_eps_10}
\end{figure}
%
 \begin{figure}[hbt]
	\centering
     \subfloat[T1 with $\mu_2^{\text{c}_1} < \mu_1^{\text{c}_1}$ and $\epsilon = 0$ \label{fig_c_1_mu1_less_mu2_eps_10_1}]{%
       \includegraphics[width=.5\linewidth]{fig_PI_c_1_smooth_ramp_mu1_less_mu2_eps_0_cfar_step_0_1}
     }
     \subfloat[T1 with $\mu_2^{\text{c}_1} < \mu_1^{\text{c}_1}$ and $\epsilon = 0$ \label{fig_c_1_mu1_less_mu2_eps_10_2}]{%
       \includegraphics[width=.5\linewidth]{fig_PI_c_1_smooth_ramp_mu1_less_mu2_eps_0_negative_cfar_step_0_1}
     }\\
     \subfloat[T1 with $\mu_2^{\text{c}_1} < \mu_1^{\text{c}_1}$ and $\epsilon = 0$ \label{fig_c_2_mu1_less_mu2_eps_10}]{%
       \includegraphics[width=.5\linewidth]{fig_PI_c_2_smooth_ramp_mu1_less_mu2_eps_0_cfar_step_0_1}
     }
     \subfloat[T2 with $\mu_2^{\text{c}_1} < \mu_1^{\text{c}_1}$ and $\epsilon = 0$\label{fig_c_2_mu1_less_mu2_eps_10_negative}]{%
       \includegraphics[width=.5\linewidth]{fig_PI_c_2_smooth_ramp_mu1_less_mu2_eps_0_negative_cfar_step_0_1}
     }\\
     \subfloat[T1 with $\mu_2^{\text{c}_1} < \mu_1^{\text{c}_1}$ and $\epsilon = 0$ \label{fig_c_2_mu1_less_mu2_eps_10}]{%
       \includegraphics[width=.5\linewidth]{fig_PI_c_3_smooth_ramp_mu1_less_mu2_eps_0_cfar_step_0_1}
     }
     \subfloat[T2 with $\mu_2^{\text{c}_1} < \mu_1^{\text{c}_1}$ and $\epsilon = 0$\label{fig_c_2_mu1_less_mu2_eps_10_negative}]{%
       \includegraphics[width=.5\linewidth]{fig_PI_c_3_smooth_ramp_mu1_less_mu2_eps_0_negative_cfar_step_0_1}
     }
     \caption{CFAR threshold = 0.32. TThe intensities channels to simulated negative image with $\mu_2^{\text{c}_i} < \mu_1^{\text{c}_i}$ and smooth ramp $\epsilon=0$ and step 0 to 1}
     \label{sim_image_mu1_less_mu2_with_ramp_eps_10}
\end{figure} 
%
 \begin{figure}[hbt]
	\centering
     \subfloat[T1 with $\mu_2^{\text{c}_1} < \mu_1^{\text{c}_1}$ and $\epsilon = 10$ \label{fig_c_1_mu1_less_mu2_eps_10_1}]{%
       \includegraphics[width=.5\linewidth]{fig_PI_c_1_smooth_ramp_mu1_less_mu2_eps_10_cfar_step_0_1}
     }
     \subfloat[T1 with $\mu_2^{\text{c}_1} < \mu_1^{\text{c}_1}$ and $\epsilon = 10$ \label{fig_c_1_mu1_less_mu2_eps_10_2}]{%
       \includegraphics[width=.5\linewidth]{fig_PI_c_1_smooth_ramp_mu1_less_mu2_eps_10_negative_cfar_step_0_1}
     }\\
     \subfloat[T1 with $\mu_2^{\text{c}_1} < \mu_1^{\text{c}_1}$ and $\epsilon = 10$ \label{fig_c_2_mu1_less_mu2_eps_10}]{%
       \includegraphics[width=.5\linewidth]{fig_PI_c_2_smooth_ramp_mu1_less_mu2_eps_10_cfar_step_0_1}
     }
     \subfloat[T2 with $\mu_2^{\text{c}_1} < \mu_1^{\text{c}_1}$ and $\epsilon = 10$\label{fig_c_2_mu1_less_mu2_eps_10_negative}]{%
       \includegraphics[width=.5\linewidth]{fig_PI_c_2_smooth_ramp_mu1_less_mu2_eps_10_negative_cfar_step_0_1}
     }\\
     \subfloat[T1 with $\mu_2^{\text{c}_1} < \mu_1^{\text{c}_1}$ and $\epsilon = 10$ \label{fig_c_2_mu1_less_mu2_eps_10}]{%
       \includegraphics[width=.5\linewidth]{fig_PI_c_3_smooth_ramp_mu1_less_mu2_eps_10_cfar_step_0_1}
     }
     \subfloat[T2 with $\mu_2^{\text{c}_1} < \mu_1^{\text{c}_1}$ and $\epsilon = 10$\label{fig_c_2_mu1_less_mu2_eps_10_negative}]{%
       \includegraphics[width=.5\linewidth]{fig_PI_c_3_smooth_ramp_mu1_less_mu2_eps_10_negative_cfar_step_0_1}
     }
     \caption{CFAR threshold = 0.082:w
      . TThe intensities channels to simulated negative image with $\mu_2^{\text{c}_i} < \mu_1^{\text{c}_i}$ and smooth ramp $\epsilon=10$ and step 0 to 1}
     \label{sim_image_mu1_less_mu2_with_ramp_eps_10}
\end{figure} 
%
Kernel = 3
 \begin{figure}[hbt]
	\centering
     \subfloat[T1 with $\mu_2^{\text{c}_1} < \mu_1^{\text{c}_1}$ and $\epsilon = 0$ \label{fig_c_1_mu1_less_mu2_eps_10_1}]{%
       \includegraphics[width=.5\linewidth]{fig_PI_c_1_smooth_ramp_mu1_greater_mu2_eps_0_cfar_coef_var_step_0_1}
     }
     \subfloat[T1 with $\mu_2^{\text{c}_1} < \mu_1^{\text{c}_1}$ and $\epsilon = 0$ \label{fig_c_1_mu1_less_mu2_eps_10_2}]{%
       \includegraphics[width=.5\linewidth]{fig_PI_c_1_smooth_ramp_mu1_greater_mu2_eps_0_negative_cfar_coef_var_step_0_1}
     }\\
     \subfloat[T1 with $\mu_2^{\text{c}_1} < \mu_1^{\text{c}_1}$ and $\epsilon = 0$ \label{fig_c_2_mu1_less_mu2_eps_10}]{%
       \includegraphics[width=.5\linewidth]{fig_PI_c_2_smooth_ramp_mu1_greater_mu2_eps_0_cfar_coef_var_step_0_1}
     }
     \subfloat[T2 with $\mu_2^{\text{c}_1} < \mu_1^{\text{c}_1}$ and $\epsilon = 0$\label{fig_c_2_mu1_less_mu2_eps_10_negative}]{%
       \includegraphics[width=.5\linewidth]{fig_PI_c_2_smooth_ramp_mu1_greater_mu2_eps_0_negative_cfar_coef_var_step_0_1}
     }\\
     \subfloat[T1 with $\mu_2^{\text{c}_1} < \mu_1^{\text{c}_1}$ and $\epsilon = 0$ \label{fig_c_2_mu1_less_mu2_eps_10}]{%
       \includegraphics[width=.5\linewidth]{fig_PI_c_3_smooth_ramp_mu1_greater_mu2_eps_0_cfar_coef_var_step_0_1}
     }
     \subfloat[T2 with $\mu_2^{\text{c}_1} < \mu_1^{\text{c}_1}$ and $\epsilon = 0$\label{fig_c_2_mu1_less_mu2_eps_10_negative}]{%
       \includegraphics[width=.5\linewidth]{fig_PI_c_3_smooth_ramp_mu1_greater_mu2_eps_0_negative_cfar_coef_var_step_0_1}
     }
     \caption{CFAR coef var - threshold = 0.5. TThe intensities channels to simulated negative image with $\mu_2^{\text{c}_i} < \mu_1^{\text{c}_i}$ and smooth ramp $\epsilon=0$ and step 0 to 1}
     \label{sim_image_mu1_less_mu2_with_ramp_eps_10}
\end{figure} 
%
Kernel = 5
 \begin{figure}[hbt]
	\centering
     \subfloat[T1 with $\mu_2^{\text{c}_1} < \mu_1^{\text{c}_1}$ and $\epsilon = 10$ \label{fig_c_1_mu1_less_mu2_eps_10_1}]{%
       \includegraphics[width=.5\linewidth]{fig_PI_c_1_smooth_ramp_mu1_greater_mu2_eps_10_cfar_coef_var_step_0_1}
     }
     \subfloat[T1 with $\mu_2^{\text{c}_1} < \mu_1^{\text{c}_1}$ and $\epsilon = 10$ \label{fig_c_1_mu1_less_mu2_eps_10_2}]{%
       \includegraphics[width=.5\linewidth]{fig_PI_c_1_smooth_ramp_mu1_greater_mu2_eps_10_negative_cfar_coef_var_step_0_1}
     }\\
     \subfloat[T1 with $\mu_2^{\text{c}_1} < \mu_1^{\text{c}_1}$ and $\epsilon = 10$ \label{fig_c_2_mu1_less_mu2_eps_10}]{%
       \includegraphics[width=.5\linewidth]{fig_PI_c_2_smooth_ramp_mu1_greater_mu2_eps_10_cfar_coef_var_step_0_1}
     }
     \subfloat[T2 with $\mu_2^{\text{c}_1} < \mu_1^{\text{c}_1}$ and $\epsilon = 10$\label{fig_c_2_mu1_less_mu2_eps_10_negative}]{%
       \includegraphics[width=.5\linewidth]{fig_PI_c_2_smooth_ramp_mu1_greater_mu2_eps_10_negative_cfar_coef_var_step_0_1}
     }\\
     \subfloat[T1 with $\mu_2^{\text{c}_1} < \mu_1^{\text{c}_1}$ and $\epsilon = 10$ \label{fig_c_2_mu1_less_mu2_eps_10}]{%
       \includegraphics[width=.5\linewidth]{fig_PI_c_3_smooth_ramp_mu1_greater_mu2_eps_10_cfar_coef_var_step_0_1}
     }
     \subfloat[T2 with $\mu_2^{\text{c}_1} < \mu_1^{\text{c}_1}$ and $\epsilon = 10$\label{fig_c_2_mu1_less_mu2_eps_10_negative}]{%
       \includegraphics[width=.5\linewidth]{fig_PI_c_3_smooth_ramp_mu1_greater_mu2_eps_10_negative_cfar_coef_var_step_0_1}
     }
     \caption{CFAR coef var - threshold =. The intensities channels to simulated negative image with $\mu_2^{\text{c}_i} < \mu_1^{\text{c}_i}$ and smooth ramp $\epsilon=10$ and step 0 to 1}
     \label{sim_image_mu1_less_mu2_with_ramp_eps_10}
\end{figure} 
\section{Conclusions and discussions}

\begin{enumerate}[label=(\roman*)]
	\item \label{item:conc_i} The visual inspection of the figures shows that edge detection is not depend on order relation between of the parameters $\mu_1^{\text{c}_i}$ and $\mu_2^{\text{c}_i}$. However, the edge detection. 
	\item \label{item:conc_ii} When the simulated image is based ad step 0 to 1 or step 1 to 0 the edge are detected near sample based in 0.
	\item \label{item:conc_iii} When the simulated image is based ad step 5 to 10 or step 10 to 5 the edge does repeat item \ref{item:conc_ii}. The edges position there are displacement to the strip center.
	\item  \label{item:conc_iv} Threshold gets the empiric way to CFARs methods.
	\item  \label{item:conc_v} The coefficient 
\end{enumerate}
 
\bibliographystyle{IEEEtran}
\bibliography{strings,refs}

\end{document}
