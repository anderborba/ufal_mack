\chapter{Métricas}\label{cap_metricas}

A matriz de confusão serve de origem para definirmos métricas como as seguintes:
\begin{equation}\label{cap_fusao_eq_01}
	tp_{rate}=\frac{TP}{P},
\end{equation}
\begin{equation}\label{cap_fusao_eq_02}
	fp_{rate}=\frac{FP}{N}
\end{equation}
\begin{equation}\label{cap_fusao_eq_02}
recall= tp_{rate}=\frac{TP}{P}=\frac{TP}{TP+FN}
\end{equation}
\begin{equation}\label{cap_fusao_eq_02}
precisão = \frac{TP}{Q}=\frac{TP}{TP+FP} 
\end{equation}
\begin{equation}\label{cap_fusao_eq_02}
acurácia = \frac{TP}{Q}=\frac{TP}{TP+FP} 
\end{equation}
\begin{equation}\nonumber
medida-F=\frac{2}{\frac{1}{recall}+\frac{1}{precisao}}
\end{equation}
\begin{equation}\nonumber
	medida-F=\frac{2}{\frac{Q+P}{TP}}
\end{equation}
\begin{equation}\nonumber
	medida-F=\frac{2TP}{Q+P}
\end{equation}
\begin{equation}\nonumber
	medida-F=\frac{2TP}{2TP+FP+FN}
\end{equation}

