\chapter{Aspectos gerais}\label{cap_asp_gerais}

Neste capítulo serão apresentadas algumas informações relevantes para situar e contextualizar a área de pesquisa desta tese. 
Serão abordados alguns assuntos introdutórios, traçando um breve panorama do desenvolvimento e aprimoramento dos radares RAR (\textit{Real Aperture Radar}),
SAR e PolSAR. 
Apresentamos também as características gerais e específicas dos radares, bem como informações de como são gerados e armazenados os dados provenientes da captação das imagens SAR e PolSAR. 

\section{Imagens SAR e PolSAR}

O SAR é o desenvolvimento tecnológico do RAR, o qual, de maneira geral, trabalha com sensores que transmitem sinais de micro-ondas e depois registram os sinais de retorno.

%%% ACF Divida este parágrafo em pelo menos duas frases bem formadas e coerentes
%%% AAB realizado
Os radares RAR para alcançar altas resoluções necessitam de grandes comprimentos de antenas o que inviabiliza o  seu uso em satélites e em aviões. Fato que configurou grande restrição para a área de sensoriamento remoto até os anos 50.
	
Na década de 50, pesquisadores fizeram avanços significativos para resolver as limi\-ta\-ções tecnológicas dos radares e desenvolveram uma técnica capaz de sintetizar o efeito de uma antena muito longa, em uma antena de tamanho viável, denominada SAR. 
Essa descoberta foi atribuída a Carl Wiley por volta de 1951 e, em 1954, foi registrada a patente do sistema SAR. 
Além disso, um sistema SAR foi operacionalizado em torno de 1958, impulsionando fortemente essa área de pesquisa. 

Os dados provenientes do SAR são gravados como uma faixa de posições para cada tempo avançado na direção da rota (azimute). 
No início do desenvolvimento dos sistemas SAR, o armazenamento de dados foi um grande problema, a solução foi colocar acoplado aos radares um sistema ótico para armazenar dados em filmes fotográficos. 
Atualmente, devido ao desenvolvimento da eletrônica, o problema de armazenamento de dados foi atenuado.

Nos sistemas SAR os dados das micro-ondas de retorno armazenados são representadas como imagens. 
Estas imagens são geradas de forma que em cada ponto da direção azimute, o radar envia um impulso em direção ao alvo, e depois recebe e armazena o sinal de retorno, que depende do espalhamento do sinal pelo alvo. 
Portanto, durante o tempo de aquisição das imagens são armazenadas as informações em uma linha de dados, gerando um mapeamento azimute \textit{versus} distância lateral das micro-ondas de retorno.

Essa estrutura de dados é típica de armazenamento em matrizes com as dimensões definidas pela resolução característica dos radares. 
As dimensões das imagens são, respectivamente, as resoluções na direção do azimute, dada pela rota do radar e pela direção perpendicular, definida como distância lateral. 
Desta forma, os radares geram imagens bidimensionais com resoluções distintas, dependendo do tipo de radar e da tecnologia empregada.

Neste trabalho usamos a generalização do sistema SAR conhecido como sistema polarimétrico SAR (PolSAR). 
O sistema PolSAR é  definido como a ciência de adquirir, processar e analisar o estado da polarização nas imagens de radar de abertura sintética. 

As imagens geradas por esses radares são formadas por micro-ondas de retorno, combinando as polarizações. 
O processo de polarização fornece mais informações sobre o alvo comparados com as imagens SAR. 
As imagens PolSAR têm o objetivo de melhorar o entendimento do efeito do espalhamento das micro-ondas pelos alvos, considerando as diferentes polarizações.

Os símbolos hh, hv, vh, e vv, representam as polarizações disponíveis nos radares, onde a primeira letra é a maneira como a onda é emitida e, a segunda letra é a maneira como a onda é recebida.

\section{Características gerais dos Radares SAR e PolSAR}

Os radares que têm a característica de emitir as micro-ondas são definidos como radares ativos e, se os radares transmitem e recebem os sinais com a mesma antena, são denominadas mono-estáticos. 
Importantes informações sobre radares podem ser encontradas na referência \citet{lp}.

O desenvolvimento dos radares pode ser resumidos pelos princípios a seguir:
\begin{itemize}
\item Possibilidade de uma antena transmitir micro-ondas em uma direção precisa; 
\item Possibilidade de detectar com grande precisão o sinal de retorno. Após a interação das micro-ondas com o alvo acontece o espalhamento das micro-ondas refletidas em todas as  direções. Assim o radar pode não captar alguns sinais de retorno. 
\item Capacidade de medir o tempo entre a transmissão  e a recepção das micro-ondas, consequentemente calcular a distância entre o alvo e o radar;
\item Habilidade de detectar vários alvos em grandes áreas percorridas.
\end{itemize}

As micro-ondas podem variar em comprimento e amplitude, dependendo da construção dos sensores para o radar em operação.
Uma característica importante dessas ondas é a capacidade de penetração no alvo analisado. 

Os radares SAR e PolSAR justificam seu uso por possuírem algumas características operacionais que podem ser resumidas nos seguintes itens:
\begin{itemize}
\item Podem estar em plataformas elevadas, aeronaves tripuladas ou não, satélites orbitando a terra ou outros planetas;
\item É uma técnica de produção de imagem viável e prática;
\item Podem gerar imagens de alta resolução espacial;
%\item Sintetiza longas aberturas de antenas; 
%%% ACF Me parece irrelevante, e confuso, dado o item anterior
%%% AAB Retirei concordo
\item Os radares produzem imagens dia e noite;
\item As condições atmosféricas interferem muito pouco na captação de imagens.
\end{itemize}

Na referência \cite{lkc} encontramos um amplo espectro de aplicações das imagens SAR e PolSAR na área de sensoriamento remoto.  Podemos listar algumas dessas aplicações: 
%%% ACF Seria interessante aqui citar um artigo tipo "survey" de aplicações de sensoriamento remoto com microondas
%%% AAB Encontro o livro que citei, achei interessante, mudei aslguns tópicos.
\begin{itemize}
\item Mapeamento do solo ou cobertura do solo;
\item Mapeamento Geológico;
\item Oceanografia;
\item Glaciologia;
\item Aplicações agrícola;
\item Acompanhamento de florestas;
\item Monitoramento ambiental;
\item Monitoramento de desastres naturais;
\item Controle de derramamento de petróleo;
\item Aplicações de recursos hídricos;
\item Aplicações da ecologia da vida selvagem;
\item Aplicações arqueológicas;  
\item E aplicações conjuntas com sistemas óticos.
\end{itemize}
\section{Características específicas dos radares SAR e PolSAR}
Os radares SAR e PolSAR trabalham no espectro das micro-ondas. 
A tabela~\ref{tab:micro_onda_espectro} definida no livro \cite{lp} mostra na primeira coluna a representação da banda, na segunda coluna das faixas de frequências que os radares são construídos. Na terceira coluna foi fixada uma frequência (f) para encontrar o comprimento de onda ($\lambda$) e mostrar na quarta coluna. Lembrando que o produto da frequência pelo comprimento de onda encontra a velocidade.   
\begin{table}[hbt] %%% ACF Leia a documentação do pacote siunitx e veja como digitar intervalos; você digitou diferenças
%%% AAB referenciei a fonte da tabela e modifiquei
	\centering
	\caption{Espectro eletromagnético para a faixa de micro-ondas.}\label{tab:micro_onda_espectro}
\begin{tabular}{@{}cccc@{}} \toprule
	banda & Frequência - f [\si{\giga\hertz}]     & f fixada& $\lambda$ [\si{\centi\meter}]. \\\midrule
	P&\numrange[range-phrase = --]{< 0.39}{ 0.39} &\num{0.3}& \num{100.0}  \\ 
	L&\numrange[range-phrase = --]{0.39}{1.55}    &\num{1.0}  & \num{30.0}\\ 
	S&\numrange[range-phrase = --]{1.55}{3.90}    &\num{3.0}  & \num{10.0}\\ 
	C&\numrange[range-phrase = --]{3.90}{5.75}    &\num{4.0}  & \num{7.5} \\ 
	X&\numrange[range-phrase = --]{5.75}{10.9}    &\num{10.0} & \num{3.0} \\ 
	K&\numrange[range-phrase = --]{10.9}{36.0}    &\num{30.0} & \num{1.0} \\ 
	Q&\numrange[range-phrase = --]{36.0}{46.0}    &\num{40.0} & \num{0.75} \\ 
	V&\numrange[range-phrase = --]{46.0}{56.0}    &\num{50.0} & \num{0.6} \\ 
	W&\numrange[range-phrase = --]{56.0}{> 56.0}  &\num{100.0}& \num{0.3} \\ \bottomrule 
\end{tabular}
\end{table}\index{Espectro Eletromagnético}

A descoberta da tecnologia SAR impulsionou um grande desenvolvimento da área, levando à construção do SEASAT, o primeiro SAR comercial projetado. 
Este projeto foi muito bem sucedido e estabeleceu definitivamente os sistemas SAR e PolSAR como área de pesquisa. 

%%% ACF Tabelas se referenciam com \ref; \eqref é apenas para equações
%%% AAB obrigado - ok
O lançamento do satélite foi em junho de 1978, a Tabela~\ref{tab:carac_seasat} mostra algumas características do SEASAT.
Para mais informações pode-se consultar o seguinte sítio eletrônico, \url{https://earth.esa.int/web/eoportal/satellite-missions/s/seasat}.  
\begin{table}[hbt]
	\centering
	\caption{Características do satélite SEASAT (SAR).}\label{tab:carac_seasat}
\begin{tabular}{@{}llr@{}} \toprule
	Características específicas& Valores operacionais  \\ \midrule
	Frequência           & \SI{1.275}{\GHz}  \\ 
	Altitude             & \SI{780}{\km}   \\
	Peso                 & \SI{2300}{\kilogram}   \\
	Ângulo de inclinação & \SI{\sim 23}{\degree}   \\
	Distância lateral    & \SI{100}{\km}  \\
	Largura de banda     & \SI{19}{\MHz}   \\
	Banda L              & \SI{23.5}{\cm} de comprimento de onda\\
	Polarização          & hh - Onda emitida e recebida na direção horizontal \\
	%%% ACF Por favor use os comandos nativos de siunitx; veja como escrever áreas
	%%% AAB - muito melhor
	Resolução            & \SI[product-units = brackets-power]{25 x 25}{\metre}  \\ \bottomrule
\end{tabular}
\end{table}

Outros projetos de sistema foram realizados e podem ser vistos na tabela \ref{tab:carac_radares}.% Neste trabalho serão usadas imagens capturadas pelo radar AIRSAR, cuja características estão na Tabela~\ref{tab:carac_radares}.

\begin{sidewaystable}
\footnotesize
	\centering
	\caption{Características operacionais dos satélites SAR ou PolSAR.}\label{tab:carac_radares}
\begin{tabular}{@{}llllllllr@{}} \toprule
Satélites      & 	SEASAT  &AIRSAR &SIR-C& Almaz&ERS-2& JERS-1& RADSAT-1&RADSAT-2 \\ \midrule
Nacionalidade       &EUA    &EUA&Alamanha-Itália&Rússia&Europa&Japão&Canadá&Canadá  \\ 
Lançamento          &\num{1978}   &\num{1988}   &\num{1990}  &\num{1992}  &\num{1995} &\num{1998}  &\num{1995}  & \num{2003}\\
Banda (\si{\cm})    &L      &P-L-C       &L-C-X &S     &C    &L     &C     & C \\
Polarização         &hh&hh-hv-vv&hh-hv-vv&hh&vv&hh&hh&hh-hv-vv\\
Ângulo de incidência&\num{23}&\numrange[range-phrase = --]{20}{60}&\numrange[range-phrase = --]{15}{55}&\numrange[range-phrase = --]{30}{60}&\num{23}&\num{35}&\numrange[range-phrase = --]{20}{59}&\numrange[range-phrase = --]{20}{60}\\
Distância (\si{\km})           &\num{100}&\numrange[range-phrase = --]{10}{17}&\numrange[range-phrase = --]{15}{90}&\num{350}&\num{100}&\num{75}&\numrange[range-phrase = --]{50}{500}&\numrange[range-phrase = --]{10}{500}\\
Resolução (\si{\m\squared})         &25&\numrange[range-phrase = --]{2}{8}&\numrange[range-phrase = --]{10}{60}&\numrange[range-phrase = --]{10}{30}&30&18&\numrange[range-phrase = --]{10}{100}&\numrange[range-phrase = --]{3}{100}\\ \bottomrule
\end{tabular}
\end{sidewaystable}
\section{Estrutura de dados para imagens PolSAR}
O sistema PolSAR armazena informações de retorno em uma matriz $\mathbf{S}_{2\times 2}$ para cada ponto de sua região de observação, onde as entradas da matriz são números complexos. Se o sistema é mono-estático, a matriz torna-se hermitiana (a matriz é igual a sua transposta conjugada), por esse motivo, podemos representar a matriz na forma de um vetor $\mathbf{s}$ de dimensão 3. O produto entre o vetor e seu hermitiano gera uma matriz hermitiana $3\times 3$ com suas entrada mostrada na Tabela \ref{tab:sistema_polsar}.
%%% ACF Precisa explicar o que é cada entrada da tabela, em palavras
%%% AAB Realizado
\begin{table}[hbt]
	\centering
	\caption{Informações do sistema PolSAR.}\label{tab:sistema_polsar}
\begin{tabular}{@{}lccc@{}} \toprule
	Polarização & hh  & hv & vv \\ \midrule
	hh & $\sigma_\text{hh}$ & $\sigma_\text{hhhv} + \hat{\sigma}_\text{hhhv}\hat{\jmath}$  & $\sigma_\text{hhvv} + \hat{\sigma}_\text{hhvv}\hat{\jmath}$\\ 
	hv &- &$\sigma_\text{hv}$ & $\sigma_\text{hvvv} + \hat{\sigma}_\text{hvvv}\hat{\jmath}$\\ 
	vv &- & -&$\sigma_\text{vv}$ \\ \bottomrule 
\end{tabular}
\end{table}

A estrutura de dados para receber as informações da tabela \ref{tab:sistema_polsar} pode ser uma matriz tridimensional, onde os dois primeiros índices da matriz localizam o pixel da imagem PolSAR correspondente ao azimute e a distância lateral. A resolução do sistema PolSAR define as dimensões na direção do azimute e da distância lateral. O terceiro índice da matriz tridimensional é definido como canal.

Para cada pixel fixo da imagem, os canais recebem os valores mostrados na Tabela~\ref{tab:canais}. O simbolo $C_k$ com $k=1,\dots,9$ define os canais da imagem.
%%% ACF \mathbb R e \mathbb I são conjuntos, não operadores. Você se refere a \Re e a \Im. Revise tudo e corrija onde precisar 
%%% AAB  realizado
\begin{table}[hbt]	
	\centering
	\caption{Ordem de armazenamento para os canais da imagem polSAR.}\label{tab:canais}
\begin{tabular}{@{}lcccccccc@{}} \toprule
	 $\text{C}_1$ &$\text{C}_2$&$\text{C}_3$ &$\text{C}_4$&$\text{C}_5$&$\text{C}_6$&$\text{C}_7$&$\text{C}_8$&$\text{C}_9$ \\ \midrule
	$\sigma_\text{hh}$&$\sigma_\text{hv}$&$\sigma_\text{vv}$&$\sigma_\text{hhhv}$&$\hat{\sigma}_\text{hhhv}$&$\sigma_\text{hhvv}$&$\hat{\sigma}_\text{hhvv}$&$\sigma_\text{hvvv}$&$\hat{\sigma}_\text{hvvv}$\\ \bottomrule
\end{tabular}
\end{table}

Neste trabalho serão usados os três primeiros canais da matriz que armazena os dados da imagem PolSAR, conhecidos como canais de intensidades. Os sensores AIRSAR, SIR-C, e RADSAT-2 fornecem informações deste tipo, mais características desses sensores podem ser encontrados na Tabela \label{tab:carac_radares}.
%%% ACF Aqui mencione quais sensores fornecem esse tipo de dados, ou dados em dois canais de intensidade
%%% AAB realiazado
%%% ACF Figuras são referenciadas com \ref, não com \eqref. Revise e corrija o texto todo
%%% AAB ok

Um exemplo de dados PolSAR é a imagem da baía de São Francisco (EUA), com suas respectivas polarizações em tons de cinza, mostradas na figura~\ref{fig:sf_hh_hv_vv}. A imagem foi captada pelo sensor aerotransportado AIRSAR polarimétrico na banda L. O AIRSAR apresenta resolução de \numrange[range-phrase = --]{2}{8} \si{\m\squared}.   
%%% ACF Precisa informar o sensor, banda, resolução espacial. Idealmente, acompanhe com uma imagem do Google Maps da mesma área.
%%% AAB Realizado
\begin{figure}[hbt]
\minipage{0.35\textwidth}
  \includegraphics[width=\linewidth]{sf_hh.pdf}
\endminipage
\minipage{0.35\textwidth}
	\includegraphics[width=\linewidth]{sf_vh.pdf}
\endminipage
\centering
\minipage{0.35\textwidth}
	\includegraphics[width=\linewidth]{sf_vv.pdf}
\endminipage
        \vspace{-2.0cm}
	\caption{Imagem PolSAR com polarizações hh, hv e vv.}\label{fig:sf_hh_hv_vv}
\end{figure}

A região de São Francisco capturada por um satélite ótico através do sítio eletrônico \url{https://www.google.com/maps} é mostrada na Figura \ref{fig:otica_google}.
\begin{figure}[hbt!]
	\centering
	\includegraphics[width=.5\linewidth]{san_francisco_google_maps_crop}%
	\caption{Imagem da região de São Francisco capturada por satélite}
\label{fig:otica_google}
\end{figure}

A visualização da região da baía de São Franscisco, usando a decomposição de Pauli (RBG), que será abordada no próximo capítulo,  pode ser vista na figura~\ref{cap_acf_sf_pauli}. Esta imagem também será usada para a realização do processamento de dados com o objetivo de testar os métodos propostos neste trabalho.

As Figuras~\ref{cap_acf_sf_hh_blue}, \ref{cap_acf_sf_hv_green} e~\ref{cap_acf_sf_vv_red} mostram, respectivamente, a decomposição RBG para cada canal, azul, verde e vermelho das imagens de PolSAR da baía de São Francisco. 

\begin{figure}[hbt]
\begin{minipage}[b]{0.450\linewidth}
\includegraphics[width=\linewidth]{polsar_teste.pdf}
\caption{Baía de São Francisco.}
\label{cap_acf_sf_pauli}
\end{minipage}\hfill
\begin{minipage}[b]{0.450\linewidth}
\includegraphics[width=\linewidth]{polsar_blue.pdf}
\caption{Polarização vv.}
\label{cap_acf_sf_hh_blue}
\end{minipage}
\end{figure}
%
\begin{figure}[hbt]
\begin{minipage}[b]{0.450\linewidth}
\includegraphics[width=\linewidth]{polsar_green.pdf}
\caption{Polarização hv.}
\label{cap_acf_sf_hv_green}
\end{minipage}\hfill
\begin{minipage}[b]{0.450\linewidth}
\includegraphics[width=\linewidth]{polsar_red.pdf}
\caption{Polarização hh.}
\label{cap_acf_sf_vv_red}
\end{minipage}
\end{figure}

Atualmente, os sistemas SAR e PolSAR são uma área de pesquisa muito ativa. 
Entre os interesses atuais, está o desenvolvimento de sistemas tipo SAR e PolSAR com alta resolução (VHR).
Um exemplo disso é o sistema Lynx, projetado pelo laboratório nacional Sandia, o qual alcança resoluções de $10$ a $30$ centímetros quadrados. O sitio eletrônico \url{https://www.sandia.gov/media/NewsRel/NR1999/Lynx.htm} fornece as informações da Tabela \ref{tab:carac_lynx} sobre o projeto lynx:
%%% ACF Fornecer referência ou link
\begin{table}[hbt]
	\centering
	\caption{Características do sistema  Lynx (SAR).}\label{tab:carac_lynx}
\begin{tabular}{@{}llr@{}} \toprule
	Características específicas& Valores operacionais  \\ \midrule
	Frequência           &  \numrange[range-phrase = --]{15.2}{18.2} \si{\GHz}\\ 
	Altitude             & \SI{7}{\km}   \\
	Peso                 & \SI{57}{\kilogram}   \\
	Ângulo de inclinação & \numrange[range-phrase = --]{45}{135}\si{\degree}   \\
	Distância lateral    & \numrange[range-phrase = --]{7}{30} \si{\km}  \\
	Banda                & $\text{K}_u$\\
	Polarização          & vv - Onda emitida e recebida na direção vertical \\
	Resolução inferior   & \SI[product-units = brackets-power]{10 x 10}{\cm}  \\
	Resolução superior   & \SI[product-units = brackets-power]{30 x 30}{\cm} \\ \bottomrule
\end{tabular}
\end{table}

O Lynx pode gerar imagem com alta resolução portanto com grandes dimensões, por exemplo, a imagem da Figura \ref{fig:sar_lynk} mostrando a região de Rio Grande Valley tem dimensão de $1068\times 2600$ pixeis e resolução de aproximadamente \SI[product-units = brackets-power]{30 x 30}{\cm}.
\begin{figure}[hbt!]
	\centering
	\includegraphics[width=.5\linewidth]{stripmap_lynx_sar_pdf}%
	\caption{Imagem da região do Rio Grande Valley}
\label{fig:sar_lynk}
\end{figure}

Outro exemplo, pode ser visto no artigo \cite{rijpbnm}, que mostra um sistema PolSAR que produzem imagens com resolução de \SI[product-units = brackets-power]{25 x 25}{\cm},
%%% ACF Use siunitx
%%% AAB realizado
conhecido como F-SAR, desenvolvido pelo centro aeroespacial da Alemanha (DLR). 
Esse sistema pode adquirir, simultaneamente, dados nas bandas X, C, S, L e P, gerando em torno de \SI{20}{\giga\byte\per\minute} %%% ACF Use siunitx %%%AAB Realizado 
de dados.    

%%% ACF Sempre detalhe o significado de uma sigla como SAOCOM
%%% AAB Realizado
Destacamos ainda o projeto \textit{SAtélite Argentino de Observación COn Microondas} (SAOCOM) pela atualidade e importância para o monitoramento da América Latina, informações podem ser obtidas no sítio web \url{https://directory.eoportal.org/web/eoportal/satellite-missions/s/saocom}. 
O SAOCOM-1B segundo satélite produzido pelo projeto foi colocado em órbita em 2020 conforme \url{https://www.space.com/spacex-saocom-1b-launch-rocket-landing-success.html}.

Como curiosidades, podemos citar: 
\begin{itemize}
\item O projeto de um mini-SAR para uso em veículos aéreos não tripulados que focam na evolução da micro-eletrônica para aumentar a eficiência e diminuir o peso, por exemplo, o projeto Lynx; %%% ACF Referência ou link %%% AAB realizado
\item As imagens SAR podem ser captadas por satélites, orbitando outros planetas, como o projeto Magellan SAR que orbita Vênus, o satélite foi lançado em 1989 com duração prevista de 5 anos. A figura \ref{fig:sar_magellan} mostra o planeta Vênus captura pelo satélite Magellan, mais informações sobre o projeto poder ser encontradas no Sítio eletrônico \url{https://www2.jpl.nasa.gov/magellan/};
  %%% ACF Referência ou link; orbita ainda?
  %%% AAB realizado
\item Outra tecnologia usada para a captação de imagens SAR denominada de interferométrica SAR (InSAR), usa dois ou mais radares de abertura sintética para obter imagens, podendo ser em posições diferentes ou tempos distintos. O sensor InSAR possibilita medir deslocamentos de objetos de interesse em uma região definida da imagem. Um tutorial sobre os sensores InSAR podem ser encontrado no Sítio eletrônico \url{https://www.esa.int/About_Us/ESA_Publications/InSAR_Principles_Guidelines_for_SAR_Interferometry_Processing_and_Interpretation_br_ESA_TM-19}.
 %%% ACF Referência ou link, e detalhe mais o que é o InSAR e para que serve
 %%% AAB Realizado
\end{itemize}

Os sistemas SAR e PolSAR apresentam algumas características inerentes à física do imageamento e à engenharia da sua implementação. 
E  podemos destacar como desvantagens:
\begin{itemize}
\item O sistema requer o conhecimento da rota do radar;
\item O sistema é sensível ao movimento do alvo;
\item O processamento para a geração de uma imagem é complexo.
\end{itemize}

\begin{figure}[hbt!]
	\centering
	\includegraphics[width=.5\linewidth]{venpole_small}%
	\caption{Imagem do planeta Vênus capturada pelo satélite Magellan}
\label{fig:sar_magellan}
\end{figure}

Entretanto, as desvantagens nos sistemas SAR e PolSAR não evitam os mesmos de serem largamente empregadas, o que impulsiona a produção de conhecimento nessa área, gerando um grande interesse, tanto em nível de aplicações, como em pesquisas científicas.