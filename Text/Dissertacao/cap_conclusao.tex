\chapter{Discussões, conclusões e contribuíções}\label{cap_conclusao}
Esta tese estudou e comparou seis métodos de fusão provenientes da detecção das evidências de bordas, nos canais de intensidade hh, hv e vv de imagens PolSAR múltiplas visadas. O método consiste em detectar evidências de bordas, isto é, pontos de transição em uma faixa de dados, a mais fina possível, idealmente com largura de um píxel, fornecendo dados que cobrem duas regiões usando o método MLE sob a distribuição de Wishart. Os métodos em questão são: média simples (MS), transformada \textit{wavelet} discreta multi-resolução (MR-DWT), análise de componente principal (PCA), estatísticas ROC, transformada \textit{wavelet} estacionária multi-resolução (MR-SWT), e um método de multi-resolução baseado na decomposição de valores singulares (MR-SVD). 

Uma etapa crucial para o desenvolvimento desta pesquisa foi a validação dos algoritmos usados em imagens simuladas, viabilizando assim, a realização dos testes em imagens adquiridas de sensores PolSAR, dando condições para que as etapas subsequentes pudessem ser realizadas. 

Na imagem simulada podemos notar que os métodos de fusão melhoram o desempenho em detectar bordas precisas com relação aos métodos de detectar as evidências de bordas para cada canal. Podemos confirmar assim o melhor desempenho dos métodos de fusão para imagens simuladas.

Prosseguindo a pesquisa, os dados fornecidos pelas linhas radiais das imagens foram selecionados e posteriormente foi aplicado o método MLE. Nesse processo ao ser usada a função de máxima verossimilhança total foram constatadas oscilações nas extremidades. Para contornar esse problema foram definidos, conforme a região de interesse, valores de folgas empiricamente. Essa alternativa mostrou-se suficiente para resolver o problema.      

Outro problema apresentado na aplicação da função de máxima verossimilhança total, se refere a sua característica de não ser diferenciável, fato que dificulta o uso de métodos de otimização que calculam a derivada da função. O problema foi resolvido usando o método GenSA.

Com os testes realizados, aplicando os métodos propostos nesta tese, foram encontradas bordas nas imagens de banda L, do sensor aerotransportado AIRSAR sobre Flevoland e São Francisco, comprovando a viabilidade dos métodos de fusão para a tarefa de detectar bordas.

Sobre os campos agrícolas de Flevoland, a melhor evidência de borda foi observada no canal hv. O canal hh forneceu as melhores estimativas de bordas entre as áreas urbanas e as áreas, de mar e vegetação, de São Francisco. Essa diversidade de conteúdo de informação justifica a necessidade de fusão das evidências de bordas.

Aplicamos métodos de fusão para reunir as evidências de bordas, obtidas nos três canais de intensidade. Ao comparar o desempenho dos métodos, os melhores resultados foram produzidos pelo método PCA e pelo método de múltipla resolução com decomposição em valores singulares, MR-SVD. Essa análise leva em conta o custo computacional em termos de tempo de processamento e a presença de \textit{outliers}. 

O método E-ROC mostrou desempenho inferior aos outros métodos, entretanto, seu desempenho pode melhorar a medida que outros canais são inseridos para a detecção de evidências de bordas.

Os resultados foram avaliados quantitativamente, verificando a proximidade dos pontos fundidos com a borda real e pela presença de \textit{outliers}. Embora a fusão por média e a fusão com PCA sejam semelhantes, com relação à probabilidade de detectar corretamente a borda, o última fornece uma ponderação mais eficaz das evidências de bordas. A partir desse fato, a fusão com PCA é capaz de descartar completamente as evidências enganosas, enquanto a fusão por média não.

É destacado as seguintes contribuição desta tese:
\begin{enumerate}[label=(\roman*)]
\item \label{item:contr_i} Propor a fusão de evidências de bordas, obtidas em cada canal, aprimorando o método de detecção de bordas;
\item \label{item:contr_ii} Aplicar e comparar os métodos de fusão: MS, MR--DWT, PCA, E--ROC, MR--SWT, e MR--SVD;
\item \label{item:contr_iii} Verificação da viabilidade do uso do método GenSA no método MLE com o algoritmo Gambini, aplicado aos canais de intensidades de uma imagem PolSAR, com o objetivo de detectar evidências de bordas, como proposto e analisado em \citet{gmbf, gmbf_sc, fbgm, nhfc};
\item \label{item:contr_iv} Realizar medidas de qualidade para a detecção em cada canal, e na fusão, usando uma métrica baseada na probabilidade de detecção de bordas, comparando com a imagem definida como referência (\textit{Ground Reference}).\index{Ground Truth}      
\item \label{item:contr_v} Identificação de duas características que dificultam a detecção de evidências de bordas: oscilação nas extremidades e não diferenciabilidade em vários pontos do domínio das funções de máxima verossimilhança total. 
Para contornar o problema propomos, respectivamente, as seguintes soluções: definir empiricamente constantes de folga nas extremidades, usar o método GenSA para encontrar o ponto de máximo da função.
\end{enumerate}

No decorrer da pesquisa alguns caminhos foram se apresentando como possíveis trabalhos futuros:
\begin{enumerate}[label=(\roman*)]
\item \label{item:pesq_fut_i} Aumentar o número de canais ou funções de distribuição de densidade para encontrar as evidências de bordas. Isso é possível, pois dados totalmente polarimétricos são mais ricos do que os canais de intensidades;
\item \label{item:pesq_fut_ii} Propor novas técnicas de fusão para evidências de bordas;
\item \label{item:pesq_fut_iii} Melhorar as medidas para aproveitamento ou descarte de canais no método de fusão; 
\item \label{item:pesq_fut_iv} Verificar os métodos de fusão inserindo textura nos modelos;
\item \label{item:pesq_fut_v} Classificar as regiões da imagem PolSAR, e usar as ideias propostas para refinar a detecção de bordas; 
\item \label{item:pesq_fut_vi} Pós-processamento, tanto para os métodos de detecção de evidências de bordas parciais, como também para os métodos de fusão das evidências de bordas.
\end{enumerate}