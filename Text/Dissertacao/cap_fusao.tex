\section{Métodos de fusão para as evidências de bordas}

As evidências de bordas estão armazenadas em  $n_c$ imagens binárias $\{\widehat{\bm\jmath}_c\}_{1\leq c\leq n_c}$, onde o pixel de valor 1 denota uma estimativa de borda e o pixel de valor 0 denota um elemento onde não foi detectada borda. 
As imagens (matrizes) têm tamanho  $m\times n$, e denotamos $\ell= mn$. 
Essas imagens foram usadas para a fusão resultando na imagem $I_\text{F}$ com as bordas detectadas.
%%% ACF Seria bom ilustrar aqui essas imagens binárias.

%%% ACF Você abusa da voz passiva, que torna o texto cansativo
Seis técnicas de fusão foram propostas:
\begin{itemize} 
\item Média simples; 
\item Transformada wavelet discreta multi-resolução (MR-DWT);
\item Transformada wavelet estacionária multi-resolução (MR-SWT); 
\item Análise de componentes principais (PCA);
\item Estatísticas ROC (E-ROC);
\item Decomposição em valores singulares com multi-resolução (MR-SVD).
\end{itemize}

As técnicas de fusão de evidência de bordas serão descritas nesta seção, com especial atenção para a estatística ROC. 

\subsection{Fusão por média simples}
A fusão por média simples realiza a média aritmética das evidências de bordas para cada canal,
$\bm I_\text{F}(x,y)=(n_c)^{-1}\sum_{c=1}^{n_c} \widehat{\bm\jmath}_c(x,y)$, onde $nc$ é o número de canais a serem utilizados na fusão. 

A figura~\ref{fig:cap_fusao_media_simples} apresenta o fluxograma para a fusão por média simples, para obter mais detalhes conferir \citet{mit}.

\pgfdeclarelayer{background}
\pgfdeclarelayer{foreground}
\pgfsetlayers{background,main,foreground}
%
\pgfdeclarelayer{background}
\pgfdeclarelayer{foreground}
\pgfsetlayers{background,main,foreground}
\tikzstyle{sensor}=[draw, fill=blue!20, text width=5em, 
    text centered, minimum height=2.5em,drop shadow]
\tikzstyle{ann} = [above, text width=5em, text centered]
\tikzstyle{wa} = [sensor, text width=15em, fill=red!20, 
    minimum height=6em, rounded corners, drop shadow]
\tikzstyle{sc} = [sensor, text width=13em, fill=red!20, 
    minimum height=10em, rounded corners, drop shadow]
\def\blockdist{2.3}
\def\edgedist{2.5}
	\begin{figure}[htb!]
\centering
\begin{tikzpicture}
	\node (wa) [wa]  {$\bm I_\text{F}(x,y)=(n_c)^{-1}\sum_{c=1}^{n_c} \widehat{\bm\jmath}_c(x,y)$};
	\path (wa.west)+(-3.2,1.5) node (e1) [sensor] {$\widehat{\bm\jmath}_1(x,y)$};
    \path (wa.west)+(-3.2,0.5) node (e2)[sensor] {$\widehat{\bm\jmath}_2(x,y)$};
    \path (wa.west)+(-3.2,-1.0) node (dots)[ann] {$\vdots$}; 
    \path (wa.west)+(-3.2,-2.0) node (e3)[sensor] {$\widehat{\bm\jmath}_3(x,y)$};    
%
    \path [draw, ->] (e1.east) -- node [above] {} 
        (wa.160) ;
    \path [draw, ->] (e2.east) -- node [above] {} 
        (wa.180);
    \path [draw, ->] (e3.east) -- node [above] {} 
        (wa.200);   
\end{tikzpicture}
	\caption{Fusão por média simples.}
\label{fig:cap_fusao_media_simples}
\end{figure}
%
\subsection{Fusão usando o método wavelet discreta multi-resolução -- MR-DWT} 

Este método de fusão é baseado no trabalho de~\cite{n_r}. 
Pada cada imagem binaria $\bm{\widehat\jmath}_c$ foram aplicados filtros DWT, sendo um filtro de passa baixa \textbf{L}  na direção vertical e um filtro de passa alta \textbf{H} na direção horizontal, ambos são \textit{down-sampled} para formar as matrizes de coeficientes $\bm{\widehat\jmath}_{c\text{L}}$ e $\bm{\widehat\jmath}_{c\text{H}}$.  
As operações são repetidas nas matrizes de coeficientes $\bm{\widehat\jmath}_{c\text{L}}$ e $\bm{\widehat\jmath}_{c\text{H}}$, resultando em $\bm{\widehat\jmath}_{c\text{LL}}$, $\bm{\widehat\jmath}_{c\text{LH}}$, $\bm{\widehat\jmath}_{c\text{HL}}$, e $\bm{\widehat\jmath}_{c\text{HH}}$.

A fusão de evidências de bordas tem os seguintes passos para cada nível de resolução:
\begin{enumerate}
\item Calcular a decomposiçao DWT $\bm{\widehat\jmath}_{c\text{LL}}$, $\bm{\widehat\jmath}_{c\text{LH}}$, $\bm{\widehat\jmath}_{c\text{HL}}$, and $\bm{\widehat\jmath}_{c\text{HH}}$, para cada canal.
\item Computar $\bm{\bar\jmath}_{c\text{HH}}$, a média pixel a pixel para todas as $\bm{\widehat\jmath}_{c\text{HH}}$ decomposições.
\item Encontrar o máximo pixel a pixel de $\bm{\widehat\jmath}_{c\text{LL}}$, $\bm{\widehat\jmath}_{c\text{LH}}$, $\bm{\widehat\jmath}_{c\text{HL}}$ resultando em $\bm{\bar\jmath}_{c\text{LL}}$, $\bm{\bar\jmath}_{c\text{LH}}$, e $\bm{\bar\jmath}_{c\text{HL}}$.
\item O resultado da fusão $I_\text{F}$ é a transformação inversa DWT para os coeficientes das matrizes $\bm{\bar\jmath}_{c\text{HH}}$, $\bm{\bar\jmath}_{c\text{LL}}$, $\bm{\bar\jmath}_{c\text{LH}}$, e $\bm{\bar\jmath}_{c\text{HL}}$.
\end{enumerate}

O fluxograma para o método de fusão MR-DWT pode ser visto na figura \ref{fig:cap_fusao_dwt}. Na figura abaixo W e $\text{W}^{-1}$ representam a transformação para obter a decomposição wavelet e a transformação inversa, respectivamente. 
\pgfdeclarelayer{background}
\pgfdeclarelayer{foreground}
\pgfsetlayers{background,main,foreground}
\tikzstyle{sensor}=[draw, fill=blue!20, text width=5.5em, 
    text centered, minimum height=2.5em,drop shadow]
\tikzstyle{ann} = [above, text width=5em, text centered]
\tikzstyle{wa} = [sensor, text width=5em, fill=red!20, 
    minimum height=3em, rounded corners, drop shadow]
\tikzstyle{sc} = [sensor, text width=10em, fill=red!20, 
    minimum height=7em, rounded corners, drop shadow]
\def\blockdist{2.3}
\def\edgedist{2.5}
	\begin{figure}[htb!]
\begin{tikzpicture}
	\path (wa.west)+(-3.0,1.5) node (swtnode1) [sensor] {$\text{Coef DWT}_1$};
	\path (wa.west)+(-3.0,0.5) node (swtnode2) [sensor] {$\text{Coef DWT}_2$};
	\path (wa.west)+(-3.0,-1.0) node (dots)[ann] {$\vdots$}; 
    \path (wa.west)+(-3.0,-2.0) node (swtnode3)[sensor] {$\text{Coef DWT}_{n_c}$};  
%
	\path (wa.west)+(-6.2,1.5) node (e1) [sensor] {$\widehat{\bm\jmath}_1(x,y)$};
    \path (wa.west)+(-6.2,0.5) node (e2)[sensor] {$\widehat{\bm\jmath}_2(x,y)$};
    \path (wa.west)+(-6.2,-1.0) node (dots)[ann] {$\vdots$}; 
    \path (wa.west)+(-6.2,-2.0) node (e3)[sensor] {$\widehat{\bm\jmath}_{n_c}(x,y)$};    
    \path (wa.west)+(1.5,1.0) node (swtnodefus) [wa] {Fusão dos coeficientes\\
                                                       wavelets};                                                      
    \path (wa.west)+(5.0,1.0) node (imagefus) [wa] {Imagem fusão};
    \path [draw, ->] (e1.east) -- node [above] {W} 
        (swtnode1.180) ;
    \path [draw, ->] (e2.east) -- node [above] {W} 
        (swtnode2.180);
    \path [draw, ->] (e3.east) -- node [above] {W} 
        (swtnode3.180);
%
    \path [draw, ->] (swtnode1.east) -- node [above] {} 
        (swtnodefus.160) ;
    \path [draw, ->] (swtnode2.east) -- node [above] {} 
        (swtnodefus.180);
    \path [draw, ->] (swtnode3.east) -- node [above] {} 
        (swtnodefus.200);      
    \path [draw, ->] (swtnodefus.east) -- node [above] {$W^{-1}$}      
        (imagefus.west);        
\end{tikzpicture}
	\caption{Fusão MR-DWT.}
\label{fig:cap_fusao_dwt}
\end{figure}

\subsection{Fusão usando o método wavelet discreta Multi-Resolução -- MR-SWT} 

Este método de fusão é  baseado nos trabalhos de \citet{n_r, jjly}. 
A diferença entre os métodos MR-DWT e MR-SWT é a substituição do operador transformada discreta wavelet (DWT) pelo operador transformada estacionária wavelet (SWT).

O fluxograma para o método de fusão MR-SWT pode ser visto na figura \ref{fig:cap_fusao_swt}. 
Na figura abaixo $\text{W}$ e $\text{W}^{-1}$ representam a transformação para obter a decomposição wavelet e a transformação inversa, respectivamente.

\pgfdeclarelayer{background}
\pgfdeclarelayer{foreground}
\pgfsetlayers{background,main,foreground}
\tikzstyle{sensor}=[draw, fill=blue!20, text width=5.5em, 
    text centered, minimum height=2.5em,drop shadow]
\tikzstyle{ann} = [above, text width=5em, text centered]
\tikzstyle{wa} = [sensor, text width=6em, fill=red!20, 
    minimum height=3em, rounded corners, drop shadow]
\tikzstyle{sc} = [sensor, text width=10em, fill=red!20, 
    minimum height=7em, rounded corners, drop shadow]
\def\blockdist{2.3}
\def\edgedist{2.5}
	\begin{figure}[htb!]
\begin{tikzpicture}
	\path (wa.west)+(-3.0,1.5) node (swtnode1) [sensor] {$\text{Coef SWT}_1$};
	\path (wa.west)+(-3.0,0.5) node (swtnode2) [sensor] {$\text{Coef SWT}_2$};
	\path (wa.west)+(-3.0,-1.0) node (dots)[ann] {$\vdots$}; 
    \path (wa.west)+(-3.0,-2.0) node (swtnode3)[sensor] {$\text{Coef SWT}_{n_c}$};  
	
	\path (wa.west)+(-6.2,1.5) node (e1) [sensor] {$\widehat{\bm\jmath}_1(x,y)$};
    \path (wa.west)+(-6.2,0.5) node (e2)[sensor] {$\widehat{\bm\jmath}_2(x,y)$};
    \path (wa.west)+(-6.2,-1.0) node (dots)[ann] {$\vdots$}; 
    \path (wa.west)+(-6.2,-2.0) node (e3)[sensor] {$\widehat{\bm\jmath}_{n_c}(x,y)$};    
    \path (wa.west)+(1.5,1.0) node (swtnodefus) [wa] {Fusão dos coeficientes\\
                                                       wavelets};                                                      
    \path (wa.west)+(5.0,1.0) node (imagefus) [wa] {Imagem fusão};
    \path [draw, ->] (e1.east) -- node [above] {W} 
        (swtnode1.180) ;
    \path [draw, ->] (e2.east) -- node [above] {W} 
        (swtnode2.180);
    \path [draw, ->] (e3.east) -- node [above] {W} 
        (swtnode3.180);
%
    \path [draw, ->] (swtnode1.east) -- node [above] {} 
        (swtnodefus.160) ;
    \path [draw, ->] (swtnode2.east) -- node [above] {} 
        (swtnodefus.180);
    \path [draw, ->] (swtnode3.east) -- node [above] {} 
        (swtnodefus.200);      
    \path [draw, ->] (swtnodefus.east) -- node [above] {$W^{-1}$}      
        (imagefus.west);        
\end{tikzpicture}
	\caption{Fusão MR-SWT.}
\label{fig:cap_fusao_swt}
\end{figure}

\subsection{Fusão usando o método análise das componentes principais -- PCA} 

O método PCA é baseada nos trabalhos de \citet{n_r,mit}, e pode ser resumido nos seguinte passos:
\begin{enumerate}
\item Armazenar as imagens binárias $\bm{\widehat\jmath}_c$ em colunas para obter a matriz $\bm X_{\ell\times n_c}$;
%%% ACF Cuidado com a notação para não confundir com a parte de polarimetria
\item Calcular a matriz de covariância $\bm C_{n_c\times n_c}$ de $\bm X_{\ell\times n_c}$;
\item Calcular as matrizes de autovalores ($\bm\Lambda$) e autovetores ($\bm V$) da matriz de covariância, ordene os autovalores e seus respectivos autovetores de maneira decrescente;
\item Encontrar o vetor $\bm P=(P(1),\dots,P(n_c))=(\sum_{c=1}^{n_c} V(c))^{-1}{\bm V}$, onde $\bm V$ é o autovetor associado com o maior autovalor de $\bm X$; note que $\sum_{c=1}^{n_c}\bm P_c=1$;
\item Realizar a fusão $\bm I_F(x,y)=\sum_{c=1}^{n_c} P(c)\bm{\widehat\jmath}_c(x,y)$.
\end{enumerate}

O fluxograma para a fusão usando o método PCA pode ser visto na figura \ref{fig:cap_fusao_pca}.
\pgfdeclarelayer{background}
\pgfdeclarelayer{foreground}
\pgfsetlayers{background,main,foreground}
\tikzstyle{sensor}=[draw, fill=blue!20, text width=5em, 
    text centered, minimum height=2.5em,drop shadow]
\tikzstyle{ann} = [above, text width=5em, text centered]
\tikzstyle{wa} = [sensor, text width=5em, fill=red!20, 
    minimum height=3em, rounded corners, drop shadow]
\tikzstyle{sc} = [sensor, text width=15em, fill=red!20, 
    minimum height=7em, rounded corners, drop shadow]
\def\blockdist{2.3}
\def\edgedist{2.5}
	\begin{figure}[htb!]
\begin{tikzpicture}
	\path (wa.west)+(-2.0,0.0) node (pcanode) [wa] {$\text{PCA}$};
	\path (wa.west)+(-6.2,1.5) node (e1) [sensor] {$\widehat{\bm\jmath}_1(x,y)$};
    \path (wa.west)+(-6.2,0.5) node (e2)[sensor] {$\widehat{\bm\jmath}_2(x,y)$};
    \path (wa.west)+(-6.2,-1.0) node (dots)[ann] {$\vdots$}; 
    \path (wa.west)+(-6.2,-2.0) node (e3)[sensor] {$\widehat{\bm\jmath}_{n_c}(x,y)$};    
    \path (wa.west)+(3.0,0.0) node (pcanodefus) [sc] {$\bm I_\text{F}(x,y)=\sum_{c=1}^{n_c} P(c)\bm{\widehat\jmath}_c(x,y)$};
    \path [draw, ->] (e1.east) -- node [above] {} 
        (pcanode.160) ;
    \path [draw, ->] (e2.east) -- node [above] {} 
        (pcanode.180);
    \path [draw, ->] (e3.east) -- node [above] {} 
        (pcanode.200);
        %
    \path [draw, ->] (pcanode.east) -- node [above] {} 
        (pcanodefus.180) ;  
\end{tikzpicture}
	\caption{Fusão PCA.}
\label{fig:cap_fusao_pca}
\end{figure}

\subsection{Fusão usando a estatística ROC}

O método baseado na estatística \textit{Receiver Operating Characteristics} (E-ROC) descrito nos trabalhos de \citet{gs,fawcett} segue os seguintes passos:
\begin{enumerate}
\item Adicionar as imagens binárias $\bm{\widehat\jmath}_c$ para produzir a matriz de frequência V; 
\item Aplicar os limiares $t=1,\dots,n_c$ em V para gerar as matrizes $\bm{\widehat\imath}_t$;
\item Comparar cada $\bm{\widehat\imath}_t$ com todos $\bm{\widehat\jmath}_c$, ache a matriz de confusão para gerar a curva ROC. O limiar ótimo corresponde ao ponto da curva ROC mais perto (no sentido da distância euclidiana) da linha de diagnóstico;
\item A fusão $\bm{I_\text{F}}$ é a matriz $\bm{\widehat\imath}_t$ que corresponde ao limiar ótimo.
\end{enumerate}

Nas próximas subseções serão abordados os aspectos teóricos do método E-ROC.
\subsubsection{A estatística ROC}
Conforme \citet{gs} e \citet{fawcett} as curvas ROC são técnicas para visualizar, organizar e selecionar classificadores aplicados em aprendizado de má\-qui\-na, visão computacional, inteligência artificial entre outras áreas similares, demostrando a capacidade do método para efetuar avaliações e comparações de algoritmos. Essa maleabilidade comprova que é possível aplicar o método para a fusão de evidências de bordas. 

A construção da curva ROC consiste em um problema de classificação com duas classes rotuladas como instâncias do conjunto $\{\mathbf{p},\mathbf{n}\}$, onde $\mathbf{p}$ representa a classe positiva e $\mathbf{n}$ representa a classe negativa. Estabelecendo as instâncias e o classificador, podemos definir a matriz de confusão $2\times 2$. 

A tabela \ref{tab:matrixz_conf} mostra a matriz de confusão.
\begin{table}[hbt]
	\centering
	\caption{Matriz de confusão.}\label{tab:matrixz_conf}
\begin{tabular}{@{}cll@{}} \toprule
	                        & \multicolumn{2}{c}{Classes definidas como verdadeiras}           \\ \midrule
	 Classes preditas       & $\mathbf{p}$                & $\mathbf{n}$                \\
                 $\mathbf{p}$& Positivos verdadeiros (TP) & Positivos falsos (FP)      \\ 
	             $\mathbf{n}$& Negativos falsos      (FN) & Negativos verdadeiros (TN) \\ \bottomrule 
\end{tabular}
\end{table}

Seguindo a regra de formação,
\begin{itemize}
    \item Se a instância é positiva e classificada como positiva, então é definida como positivo verdadeiro TP.
    \item Se a instância é positiva e classificada como negativa, então é definida como negativo falso FN.
	\item Se a instância é negativa e classificada como negativa, então é definida como negativo verdadeiro TN.
	\item Se a instância é negativa e classificada como positiva, então é definida como positivo falso FP.
\end{itemize}

Os valores da diagonal principal da matriz de confusão representam as classificações realizadas corretamente, enquanto os elementos da diagonal secundária, representam as classificações incorretas. 
%%% ACF Deixe claro que a matriz de confusão apresenta proporções, caso contrário a soma não é 1
A soma de todas as possibilidades dos resultados em uma classificação retorna o valor TP+FN+FP+TN=1.

Definimos a prevalência como a soma dos positivos verdadeiros com os negativos falsos, P=TP+FN, portanto, podemos afirmar que idealmente, a prevalência deveria aproximar-se de TP. Definindo-se o \text{Nível}-Q como a soma dos positivos verdadeiros com os positivos falsos, Q=TP+FP, o qual deveria aproximar-se de TP em situações ideais. 

De acordo com as definições acima, no detector de bordas otimizado, a prevalência e o \text{Nível-Q} deveriam ser iguais, isto é, P=\text{Nível}-Q. Além disso, se definir N=FP+TN, e recorrer ao fato TP+FN+FP+TN=1, garantimos que P+N=1.  

A matriz de confusão serve de base para a construção da curva ROC, com as seguintes métricas,
\begin{equation}\label{eq:tp_rate}
	tp_{\text{rate}}=\frac{\text{TP}}{\text{P}},
\end{equation}
\begin{equation}\label{eq:fp_rate}
	fp_{\text{rate}}=\frac{\text{FP}}{\text{N}}.
\end{equation}

 Na métrica conhecida por razão de positivos verdadeiros, taxa de acerto, \textit{recall} ou  sensibilidade,~$tp_{\text{rate}}=\frac{\text{TP}}{\text{P}}=\frac{\text{TP}}{\text{TP+FN}}$,~se o número de negativos falsos tenderem a zero, o valor da métrica aproxima-se do valor unitário. Assim como, na métrica conhecida como razão de positivos falsos,~$fp_{\text{rate}}=\frac{\text{FP}}{\text{N}}=\frac{\text{FP}}{\text{FP+TN}}$,~se o número de verdadeiros positivos tenderem a zero, o valor da métrica aproxima-se do valor unitário.
\subsubsection{Estatística ROC para a fusão das evidências de bordas}
O método de fusão de evidências baseado na estatística ROC, consiste em aplicar o método MLE em cada canal da imagem PolSAR, gerando imagens binárias com as evidências de bordas $\bm{\widehat\jmath}_c$, sendo $c=1,\dots,n_c$, onde $n_c$ é o número de canais. Após construir a matriz de frequência V de mesmo tamanho de $\bm{\widehat\jmath}_c$, tal que em cada pixel é armazenado um valor, correspondente a frequência de ocorrências de evidências de bordas em cada canal $\bm{\widehat\jmath}_c$, ou seja, a matriz V é a soma pixel a pixel de todas as imagens de  evidências de bordas $\bm{\widehat\jmath}_c$. Sendo assim, podemos afirmar que quanto maior o valor de um pixel, maior sua probabilidade de ser uma evidência de borda.

Na matriz de frequência V são aplicados limiares~$\text{CT}_t$, onde $t=1,\dots,n_c$, gerando matrizes chamadas mapas de evidências de bordas $\bm{\widehat\imath}_t$. O objetivo do método é estimar automaticamente o limiar correspondente CT ótimo, proveniente do conjunto de limiares parciais $\text{CT}_t$.

Resumindo, somando-se as matrizes $\bm{\widehat\jmath}_c$ temos a matriz de frequência $\text{V}=\sum_{c=1}^{n_c}\bm{\widehat\jmath}_c$. Na matriz V é aplicado os limiares $\text{CT}_t$ com $t=1,\dots,n_c$, gerando os mapas de evidências de bordas $\bm{\widehat\imath}_t$. Na figura \ref{fig:cap_fusao_roc1} pode-se visualizar o fluxograma correspondente a esse processo.
\pgfdeclarelayer{background}
\pgfdeclarelayer{foreground}
\pgfsetlayers{background,main,foreground}
\tikzstyle{sensor}=[draw, fill=blue!20, text width=5em, 
    text centered, minimum height=2.5em,drop shadow]
\tikzstyle{ann} = [above, text width=5em, text centered]
\tikzstyle{wa} = [sensor, text width=10em, fill=red!20, 
    minimum height=6em, rounded corners, drop shadow]
\tikzstyle{sc} = [sensor, text width=13em, fill=red!20, 
    minimum height=10em, rounded corners, drop shadow]
\def\blockdist{2.3}
\def\edgedist{2.5}

\begin{figure}[htb!]
\centering
\begin{tikzpicture}
	\node (wa) [wa]  {$V=\sum_{c=1}^{n_c}\bm{\widehat\jmath}_c$};
	\path (wa.west)+(-3.2,1.5) node (e1) [sensor] {$\bm{\widehat\jmath}_1$};
    \path (wa.west)+(-3.2,0.5) node (e2)[sensor] {$\bm{\widehat\jmath}_2$};
    \path (wa.west)+(-3.2,-1.0) node (dots)[ann] {$\vdots$}; 
    \path (wa.west)+(-3.2,-2.0) node (e3)[sensor] {$\bm{\widehat\jmath}_{n_c}$};    
%   
    \path (wa.east)+(3.2,1.5) node (m1) [sensor] {$\bm{\widehat\imath}_1$};
    \path (wa.east)+(3.2,0.5) node (m2) [sensor] {$\bm{\widehat\imath}_2$};
    \path (wa.east)+(3.2,-1.0) node (dots)[ann] {$\vdots$}; 
    \path (wa.east)+(3.2,-2.0) node (m3) [sensor] {$\bm{\widehat\imath}_{n_c}$};
%
    \path [draw, ->] (e1.east) -- node [above] {} 
        (wa.160) ;
    \path [draw, ->] (e2.east) -- node [above] {} 
        (wa.180);
    \path [draw, ->] (e3.east) -- node [above] {} 
        (wa.200);
	\path [draw, ->] (wa.east) -- node [above] {\tiny{$\text{CT}_1$}} 
        (m1.west);
	\path [draw, ->] (wa.east) -- node [above] {\tiny{$\text{CT}_2$}} 
        (m2.west);
	\path [draw, ->] (wa.east) -- node [right] {\tiny{$\text{CT}_{n_c}$}} 
        (m3.west);
%               
%    \path (wa.south) +(0,-\blockdist) node (asrs) {Estrutura geral da fusão de evidência proposta};
  
    \begin{pgfonlayer}{background}
        \path (e1.west |- e1.north)+(-0.5,0.3) node (a) {};
        \path (wa.south -| wa.east)+(+0.5,-0.3) node (b) {};
        \path (m3.east |- m3.east)+(+0.5,-0.75) node (c) {};
       %   
        \path[fill=yellow!20,rounded corners, draw=black!50, dashed]
            (a) rectangle (c);           
       %     
    \end{pgfonlayer}
   
\end{tikzpicture}
	\caption{Fluxograma para a aplicação dos limiares na fusão de evidência ROC.}
\label{fig:cap_fusao_roc1}
\end{figure}

As entradas para cada matriz de confusão são definidas como:  
\begin{equation}\label{eq:verdadeiro_positivo_j}
	\text{TP}^c_t= \frac{1}{m\cdot n}\sum_{k=1}^{m}\sum_{l=1}^{n} \bm{\widehat\imath}_{E_t}\cap \bm{\widehat\jmath}_{E_c}, \\
\end{equation}
\begin{equation}\label{eq:falso_positivo_j}
	\text{FP}^c_t= \frac{1}{m\cdot n}\sum_{k=1}^{m}\sum_{l=1}^{n} \bm{\widehat\imath}_{E_t}\cap \bm{\widehat\jmath}_{NE_c}, \\
\end{equation}
\begin{equation}\label{eq:verdadeiro_negativo_j}
	\text{TN}^c_t= \frac{1}{m\cdot n}\sum_{k=1}^{m}\sum_{l=1}^{n} \bm{\widehat\imath}_{NE_t}\cap \bm{\widehat\jmath}_{NE_c}, \\
\end{equation}
\begin{equation}\label{eq:falso_negativo_j}
	\text{FN}^c_t= \frac{1}{m\cdot n}\sum_{k=1}^{m}\sum_{l=1}^{n} \bm{\widehat\imath}_{NE_t}\cap \bm{\widehat\jmath}_{E_c},\\
\end{equation}
os índices E e NE, significam bordas detectadas e bordas não detectadas respectivamente.


Para cada matriz $\bm{\widehat\imath}_t$, com $t$ fixado arbitrariamente, foram comparadas a todas as matrizes $\bm{\widehat\jmath}_c$, e foi construído para cada comparação uma matriz de confusão, desta forma gerou-se $n_c$ matrizes de confusão, mostrado na tabela \ref{tab:matriz_conf_roc_j}.
\begin{table}[htb!]
	\centering
	\caption{Matriz de confusão para cada comparação}\label{tab:matriz_conf_roc_j}
\begin{tabular}{@{}lll@{}} \toprule
	& $\bm{\widehat\jmath}_{E_c}$  & $\bm{\widehat\jmath}_{NE_c}$  \\ \midrule
	$\bm{\widehat\imath}_{E_t}$    & $\text{TP}^c_t$ &  $\text{FP}^c_t$  \\ 
	$\bm{\widehat\imath}_{NE_t}$   & $\text{FN}^c_t$ &  $\text{TN}^c_t$\\ \bottomrule 
\end{tabular}
\end{table}

As médias com $c=1,\dots,n_c$ são:
\begin{equation}\label{eq:verdadeiro_positivo_media}
	\overline{\text{TP}}_t=\frac{1}{n_c}\sum_{c=1}^{n_c} \text{TP}^c_t;
\end{equation}
\begin{equation}\label{eq:falso_positivo_media}
	\overline{\text{FP}_t}=\frac{1}{n_c}\sum_{c=1}^{n_c} \text{TN}^c_t;
\end{equation}
\begin{equation}\label{eq:verdadeiro_negativo_media}
	\overline{\text{TN}_t}=\frac{1}{n_c}\sum_{c=1}^{n_c} \text{FP}^c_t;
\end{equation}
\begin{equation}\label{eq:falso_negativo_media}
	\overline{\text{FN}_t}=\frac{1}{n_c}\sum_{c=1}^{n_c} \text{FN}^c_t.
\end{equation}

A figura \ref{cap_fusao_fig02} mostra a comparação entre a imagem $\bm{\widehat\imath}_t$ fixada arbitrariamente com todos as imagens~$\bm{\widehat\jmath}_c$, gerando a métrica~$\text{TP}_t$. A notação $\overline{\cap \bm{\widehat\jmath}_c}$~significa que a comparação realizada é a intersecção pixel a pixel, entre $\bm{\widehat\imath}_t$ e $\bm{\widehat\jmath}_c$, fazendo a média de acordo com as equações  as~\eqref{eq:verdadeiro_positivo_media},~\eqref{eq:falso_positivo_media},~\eqref{eq:verdadeiro_negativo_media}~e~\eqref{eq:falso_negativo_media}. A notação $+$ refere-se à média de todos os $\text{TP}_t$, ou demais probabilidades $\text{FP}_t$, $\text{FN}_t$ e $\text{TN}_t$.
\tikzstyle{sensor_1}=[draw, fill=blue!20, text width=2.5em, 
    text centered, minimum height=2em,drop shadow]
\tikzstyle{ann_1} = [above, text width=5em, text centered]
\tikzstyle{wa_1} = [sensor, text width=2em, fill=red!20, 
    minimum height=2em, rounded corners, drop shadow]
\tikzstyle{wa1_1} = [sensor, text width=2em, fill=red!20, 
    minimum height=2em, rounded corners, drop shadow]
\begin{figure}[hbt!]
\centering
\begin{tikzpicture}
\node[wa_1] (wa_1) at (0.0,0.0) {$\bm{\widehat\imath}_t$};
\node[wa1_1] (wa1_1) at (4.0,0.0) {$\overline{\text{TP}}_t$};
%
    \path (wa_1.west)+(2.5,1.5) node (e1_1) [sensor_1] {$\text{TP}_1$};
    \path (wa_1.west)+(2.5,0.5) node (e2_1)[sensor_1] {$\text{TP}_2$};
    \path (wa_1.west)+(2.5,-1.0) node (dots)[ann_1] {$\vdots$}; 
    \path (wa_1.west)+(2.5,-2.0) node (e3_1)[sensor_1] {$\text{TP}_{n_c}$};    
%
	\path [draw, ->] (wa_1.east) -- node [left] {\tiny{$\overline{\cap \bm{\widehat\jmath}_1}$}} 
        (e1_1.180) ;
	\path [draw, ->] (wa_1.east) -- node [below] {\tiny{$\overline{\cap \bm{\widehat\jmath}_2}$}} 
        (e2_1.180);
	\path [draw, ->] (wa_1.east) -- node [right] {\tiny{$\overline{\cap \bm{\widehat\jmath}_{n_c}}$}} 
        (e3_1.180);
	\path [draw, ->] (e1_1.east) -- node [right] {\tiny{$+$}} 
        (wa1_1.160);
	\path [draw, ->] (e2_1.east) -- node [above] {\tiny{$+$}} 
        (wa1_1.180);
	\path [draw, ->] (e3_1.east) -- node [right] {\tiny{$+$}} 
        (wa1_1.200);
  
    \begin{pgfonlayer}{background}
        \path (wa_1.west |- wa_1.north)+(5.25,1.75) node (a) {};
        \path (e1_1.south -| e1_1.north)+(-2.75,-3.75) node (b) {};
        %\path (wa1.east |- wa1.east)+(+4.0,-0.5) node (c) {};
       %   
        \path[fill=yellow!20,rounded corners, draw=black!50, dashed]
            (a) rectangle (b);           
       %     
    \end{pgfonlayer}
    
\end{tikzpicture}
\caption{Estrutura para fusão de evidências com $j$ escolhido arbitrariamente.}
\label{cap_fusao_fig02}
\end{figure}

A curva ROC é encontrada calculando as razões de positivos verdadeiros, e a razão de positivos falsos para cada  mapa de bordas $\bm{\widehat\imath}_t$, definidas, respectivamente, como:  

\begin{equation}\label{eq:tp_rate_t}
	\text{TP}_{rate_{t}}=\frac{\overline{\text{TP}}_t}{\overline{\text{TP}}_t+\overline{\text{FN}}_t}, \\
\end{equation}
e
\begin{equation}\label{eq:tp_rate_t}
	\text{FP}_{rate_{t}}=\frac{\overline{\text{FP}}_t}{\overline{\text{FP}}_t+\overline{\text{TN}}_t}, \\
\end{equation}
onde a soma de $\overline{\text{TP}}_t+\overline{\text{FN}}_t$, representa o número médio de bordas verdadeiras para cada $\bm{\widehat\imath}_t$.

O gráfico para a curva ROC é bidimensional no qual os valores das razões de positivos falsos, $\text{FP}_{rate_t}$, são medidos no eixo horizontal, e as razões de positivos verdadeiros, $\text{TP}_{rate_t}$, são medidos no eixo vertical. Assim, cada mapa de borda $\bm{\widehat\imath}_t$ produz um ponto no gráfico $(\text{FP}_{rate_t}, \text{TP}_{rate_t})$ gerando a curva ROC. 

O trabalho de \cite{fawcett} mostra que o limiar ótimo ocorre na intersecção, ou perto da mesma, da curva ROC com a linha diagnóstico. Essa linha é formada conectando os pontos (P,P) e (0,1) representada pela linha vermelha na figura \ref{fig:curva_roc}. 

O limiar ótimo CT, correspondendo ao parâmetro $t$ que gera o ponto $(\text{FP}_{rate_t}, \text{TP}_{rate_t})$ mais próximo (métrica euclidiana) da reta diagnóstico determina a imagem $\bm{\widehat\imath}_t$ com maior acurácia de bordas detectadas. 

A figura \ref{fig:curva_roc} mostra a fusão dos canais hh, hv, e vv, para a imagem simulada, processo descrito no próximo capítulo. Para a imagem simulada o limiar ótimo encontrado $t=2$, resultou na imagem com melhor acurácia $\bm{\widehat\imath}_2$, por ser o ponto com menor distância euclidiana para a linha diagnóstico. 
%%% ACF seria interessante desenhar os segmentos representativos das distâncias dos pontos à linha diagnóstico

\begin{figure}[hbt]
\centering
\includegraphics[width=4.0in]{curva_roc_3_canais.pdf}
	\caption{Curva ROC para a imagem simulada de duas folhas.}
\label{fig:curva_roc}
\end{figure}
%
\subsection{Fusão usando o método decomposição em valores singulares multi-resolução -- MR-SVD}

O método fusão MR-SVD \citep{naidu} trabalha de maneira similar ao MR-DWT. 
A diferença consiste em mudar os filtros DWT por filtros SVD. 
O método pode ser resumido nos seguinte passos: 
\begin{enumerate}
\item Organizar as imagens binárias $\bm{\widehat\jmath}_c$, com a extração de blocos $2\times 2$ não sobrepostos, onde cada bloco é colocado como um vetor $4\times 1$, ordenado em colunas para formar a matriz de dados $\bm X_1$ com dimensão ${4\times{\ell}/{4}}$;   
\item Calcular a decomposição SVD de  $\bm X_1=\bm U_1 \bm S_1 \bm V_1^T$, onde $\bm U_1$ e $\bm V_1$ são unitárias e têm dimensões ${4\times 4}$ e ${\ell}/{4}\times{\ell}/{4}$, respectivamente. Os valores singulares são ordenados de maneira decrescente e são colocados na diagonal principal da matriz $\bm S_1$, as demais entradas da matriz são zeros;  
\item Transformar as linhas de $\widehat{\bm X}_1=\bm U_1^T\bm X_1=\bm S_1 \bm V_1^T$, sobre novas matrizes com dimensões ${m}/{2}\times{n}/{2}$: $\{\bm\Phi_1, \bm\Psi_{1\text{V}}, \bm\Psi_{1\text{H}}, \bm\Psi_{1\text{D}}\}$; 
\item Recomecar o procedimento em (1) para $\bm\Phi_r$ com $r=2$, até o menor nível de resolução $R$;
\item A MR-SVD decomposição em cada canal é  
\begin{equation}\nonumber
\widehat{\bm X}_c\rightarrow \left\{\bm \Phi_\text{R}^c,\{\bm\Psi_{r\text{V}}^c,\bm\Psi_{r\text{H}}^c,\bm\Psi_{r\text{D}}^c \}_{r=1}^\text{R},\{\bm U_r^c	\}_{r=1}^\text{R} \right\};
\end{equation}
\item Depois da decomposição ser aplicada em todos os canais, computar a média dos $\bm\Phi_R^c$ ($\bm\Phi_\text{R}^f$) no menor nível de resolução, e a média $\bm U_r^c$ ($\bm U_\text{r}^f$), para cada $r$, onde $f$ denota a fusão entre os canais;  
\item Achar o máximo pixel a pixel de $\bm\Psi_{r\text{V}}^c$, $\bm\Psi_{r\text{H}}^c$ e $\bm\Psi_{r\text{D}}^c$: $ \bm\Psi_{rV}^f$ $\bm\Psi_{r\text{V}}^f$, $\bm\Psi_{r\text{H}}^f$ e $\bm\Psi_{r\text{D}}^f$;
\item A fusão $\bm I_\text{F}$ é a transformação SVD para cada nível $r=\text{R},\dots,1$, 
\begin{equation}\nonumber
\bm I_\text{F}\leftarrow \left\{\bm \Phi_\text{R}^f,\{\bm\Psi_{r\text{V}}^f,\bm\Psi_{r\text{H}}^f,\bm\Psi_{r\text{D}}^f \}_{r=\text{R}}^1,\{\bm U_r^f\}_{r=\text{R}}^1 \right\}.
\end{equation}
\end{enumerate}