%/\chapter{Fusão de evidências na detecção de bordas em Imagens PolSAR}
\chapter{Introdução}
\label{cap_acf}
Os radares de abertura sintética (SAR -- \textit{Synthetic Aperture Radar}) e de abertura sintética polarimétricos (PolSAR) alcançaram uma posição essencial na área de sensoriamento remoto. 
Os dados que esses sensores fornecem requerem técnicas, especificamente adaptadas, para o seu processamento e análise. 
Entre essas técnicas, a detecção de bordas é uma das operações mais importantes para extrair informações. 
As bordas estão em um nível mais alto de abstração do que apenas os dados e, como tais, fornecem informações relevantes sobre a cena.

As imagens do tipo SAR e PolSAR possuem características inerentes ao seu processo de aquisição.
Como vantagens podemos destacar que a captação das imagens pode ser realizadas dia e noite (pois o sensor é ativo e carrega a própria fonte de iluminação), não dependem das condições do tempo (pois iluminam a cena com microondas, que são pouco afetadas por nuvens, chuva, bruma, fumaça, etc.) e possuem  excelente capacidade de penetração da cobertura vegetal. 
A desvantagem deste tipo de imagem está na presença do ruído \textit{speckle}, que gera um aspecto granular na imagem, dificultando a sua interpretação e análise. 

As técnicas de detecção de bordas disponíveis para imagens SAR e PolSAR estão sendo desenvolvidas, e são temas em recentes pesquisas. 
Entre as técnicas disponíveis para essas imagens, são destacadas: 
técnicas baseadas em redução do ruído \textit{speckle} \citet{sjx, lzly,wxbzw,cgaf}; 
campos aleatórios de Markov \citet{bf}; 
a  abordagem de \textit{deep learning} \citet{ztmxzxf} aplicada à segmentação e classificação; 
e técnicas estatísticas \citet{gmbf, fbgm, nhfc}.

Este trabalho de pesquisa segue a abordagem de modelagem estatística usando as técnicas descritas em \citet{gmbf, fbgm, nhfc} para encontrar evidências de bordas em imagens PolSAR. 
Destacamos que não usaremos técnicas de redução do \textit{speckle}, pois suas características serão usadas para potencializar a detecção de evidências de bordas. 
Além do que, em vez de lidar com dados totalmente polarimétricos, trataremos cada canal de intensidade da imagem separadamente, com o objetivo de obter evidências de bordas. 

Em seguida, será realizado a fusão das evidências de bordas obtidas, com o objetivo de produzir um único e mais acurado estimador da posição de bordas. 
Com isso, se pretende qualificar a contribuição fornecida por cada canal de intensidade da imagem PolSAR, na detecção de bordas. 
%Ver as referências \citet{mit, bmf_2020}. 
%%% ACF Aqui diga ao leitor o que ele deve "ver".
%%% AAB Realizado
A referência \citet{bmf_2020} fornece informações sobre os processos de fusão em imagens PolSAR, assim como o  desempenho de cada método para detectar bordas, enquanto a referência \citet{mit} mostra os fundamentos da fusão de imagens.


O Algoritmo Gambini, apresentado em \citet{gmbf_sc}, é uma excelente técnica para detecção de bordas. 
Com essa técnica é possível encontrar evidências de uma borda sobre uma faixa fina de dados.
Esta abordagem funciona com qualquer modelo, o que o torna adequado para dados SAR e PolSAR, e tem demostrando ter um desempenho melhor do que outras abordagens. 
Esse algoritmo consiste em traçar raios, para então, encontrar a evidência de borda maximizando uma função que, na proposta original, é a de máxima verossimilhança total.

Neste trabalho usaremos a verossimilhança total de duas amostras: uma para região interna e outra para a região externa da borda. 
Para não perder de generalidade, optou-se pela distribuição Wishart para as observações totalmente polarimétricas, derivando para as suas distribuições, do tipo Gamma, em cada canal de intensidade. 
A função de máxima verossimilhança total depende das estimativas que parametrizam as leis Gamma. 
As estimativas são realizadas pelo método de otimização BFGS
\citep[Broyden-Fletcher-Goldfarb-Shanno, ver][]{ht,nw}. 
Neste trabalho estimaremos os parâmetros que indexam as distribuições Gamma com o método de máxima verossimilhança (MLE -- \textit{Maximum Likelihood Estimator}).

A função de verossimilhança total não é diferenciável em vários pontos, e os métodos clássicos apresentam dificuldade em encontrar os seus máximos. 
Uma possível abordagem para resolver este problema é usar o método GenSA \citep[\textit{Generalized Simulated Annealing},][]{xgsh}.

Nesta pesquisa propomos, discutimos, e comparamos o uso de seis métodos de fusão: 
Média simples \citep[MS,][]{mit}; 
Transformada \textit{Wavelet} Discreta de Multi-resolução, \citep[MR--DWT,][]{n_r}; 
Análise de Componentes Principais, \citep[PCA,][]{n_r,mit}; 
Estatísticas baseadas na curva ROC \citep[\textit{Receiver Operating Characteristic},][]{gs}, o qual é denominado de E-ROC; 
Transformada \textit{Wavelet} Estacionária de Multi-resolução, \citep[MR--SWT,][]{n_r, jjly}; 
e Decomposição em valores singulares multi-resolução, \citep[MR--SVD,][]{naidu,demmel}.

\section{Objetivos}

O objetivo deste trabalho é estudar a viabilidade dos métodos para a fusão das evidências de bordas detectadas nos canais de intensidades hh, hv, e vv, de uma imagem PolSAR, como também usar os métodos de fusão para quantificar as contribuições de informações provenientes de cada canal.

Para alcançar tais objetivos deve-se aplicar nas imagem SAR e PolSAR os seguintes procedimentos:
\begin{enumerate}[label=(\roman*)]
	\item \label{item:obj_i}  Especificar manualmente ou automaticamente a região de interesse ROI;\index{ROI}
	\item \label{item:obj_ii} Em cada ROI calcular o centro de massa e, a partir desse ponto, traçar retas radiais cruzando duas amostras distintas. 
	Nas radiais serão extraídas informações para o passo \ref{item:obj_iii};
	\item \label{item:obj_iii} Nos dados extraídos em cada reta radial o método MLE é aplicado para estimar o ponto de transição entre duas amostras, chamado de evidência de borda;\index{Evidências de bordas}
	\item  \label{item:obj_iv} Métodos de fusão são aplicados nas evidências de bordas, em cada canal, com o objetivo de detectar as bordas na ROI escolhida.	  
\end{enumerate}

\section{Contribuições}
As contribuição desta tese são:
\begin{enumerate}[label=(\roman*)]
\item \label{item:contr_i} Propor a fusão de evidências de bordas, obtidas em cada canal, aprimorando o método de detecção de bordas;
\item \label{item:contr_ii} Aplicar e comparar os métodos de fusão: MS, MR--DWT, PCA, E--ROC, MR--SWT, e MR--SVD;
\item \label{item:contr_iii} Verificação da viabilidade do uso do método GenSA no método MLE com o algoritmo Gambini, aplicado aos canais de intensidades de uma imagem PolSAR, com o objetivo de detectar evidências de bordas, como proposto e analisado nas referências \citet{gmbf, fbgm, nhfc};
\item \label{item:contr_iv} Realizar medidas de qualidade para a detecção em cada canal, e na fusão, usando uma métrica baseada na probabilidade de detecção de bordas, comparando com a imagem definida como referência (\textit{Ground Truth}).\index{Ground Truth}      
\item \label{item:contr_v} Identificação de duas características que dificultam a detecção de evidências de bordas: oscilação nas extremidades e não diferenciabilidade em vários pontos do domínio das funções de máxima verossimilhança total. 
Para contornar o problema propomos, respectivamente, as seguintes soluções: definir empiricamente constantes de folga nas extremidades, usar o método GenSA para encontrar o ponto de máximo da função.
\end{enumerate}

Os resultados obtidos neste trabalho foram publicados na conferência \textit{IEEE Recent Advances in Geoscience and Remote Sensing : Technologies, Standards and Applications (TENGARSS)} com o seu identificador digital DOI (\textit{Digital Object Identifier}), e seu localizador URL (\textit{Uniform Resource Locator}), respectivamente, \url{https://doi.org/10.1109/TENGARSS48957.2019.8976040}, e  \url{https://ieeexplore.ieee.org/abstract/document/8976040.} E no periódico \textit{IEEE Geoscience and Remote Sensing Letters} com o seu identificador digital DOI, e seu localizador, respectivamente, \url{https://doi.org/10.1109/LGRS.2020.3022511}, e \url{https://ieeexplore.ieee.org/abstract/document/9203845}.
%%% ACF Citar por extenso, incluindo o DOI e a URL
%%% AAB Realizado
\section{Plataformas, validação do programas, recursos computacionais}

Neste trabalho foram usadas as seguintes plataformas:
\begin{itemize}
%%% ACF Atualize a referência do R
%%% AAB Realizado
\item R \citep{r_pack} é uma linguagem de programação com forte viés de mani\-pulação, análise e visualização de dados. 
Neste trabalho, a detecção de bordas nos canais de intensidades e a visualização dos dados foram programados usando esta linguagem.  
\item Matlab \citep{matlab} é uma linguagem de programação que tem seu ponto forte na manipulação de vetores e matrizes. 
Os programas de fusão para a detecção de bordas e as visualizações das imagens PolSAR, presentes neste trabalho, foram programados nesta linguagem.  
\end{itemize}

Os programas usados neste trabalho foram validados com auxílio de uma imagem de referência que consiste de duas classes, e para a qual é conhecido o posicionamento da borda.
Simulamos imagens observadas por um sensor PolSAR atribuindo a cada classe observações independes oriundas da distribuição Whisart. 
Os programas foram rodados para a imagem simulada e verificou-se que as bordas foram encontradas com precisão. 
Desta forma, os programas foram validados e aplicados em imagens PolSAR fornecidas por sensores operacionais.

O recurso computacional disponível consiste em computados Intel\copyright\ Core i7-9750HQ CPU \SI{2.6}{\giga\hertz} com \SI{16}{\giga\byte} de memória RAM.

\section{Reprodutibilidade e replicabilidade}
\index{Reprodutibilidade}\index{Replicabilidade}
A reprodutibilidade e replicabilidade são essenciais para o desenvolvimento da ciên\-cia, estes temas não são discutidos neste trabalho, porém realizamos as pesquisas de forma a respeitar esses conceitos.

Os trabalhos de \citet{br} e \citet{fgmed} mostram pesquisas sobre o tema. 
Na primeira referência encontramos as seguintes definições: 
\begin{itemize}
\item A reprodutibilidade é definida como a habilidade dos pesquisadores de obter resultados consistentes usando os mesmos dados de entrada, rotinas computacionais, métodos, programas e condições de análises.
\item A replicabilidade é definida como uma situação em que os pesquisadores obtêm resultados consistentes por meio de estudos que visam responder à mesma questão científica, cada um dos quais obteve seus próprios dados.
\end{itemize}
  
Os programas e dados usados neste trabalho estão disponíveis no sítio eletrônico \url{https://github.com/anderborba/Code_GRSL_2020_1}.
 
\section{Organização do texto} 

Os assuntos estudados para aplicação e análise dos métodos discutidos nesta pesquisa, assim como seus resultados foram distribuídos e organizados da seguinte forma:
\begin{enumerate}[label=(\roman*)]
%%% ACF use \label e \ref sempre
%%% AAB Realizado
\item  \label{item:org_text_i} Capítulo \ref{cap_asp_gerais}: Alguns conceitos introdutórios e informações relevantes sobre as imagens SAR e PolSAR e radares. 
\item \label{item:org_text_ii} Capítulo \ref{metodologia}: São apresentados os aspectos metodológicos, descrevendo o modelo, a detecção de bordas e as abordagens para fusão de evidências de bordas.
\item \label{item:org_text_iii} Capítulo \ref{cap_resultados}: Apresentação, descrição e análise dos resultados.
\item \label{item:org_text_iv} Capítulo \ref{cap_conclusao}: Discussões dos resultados obtidos e as direções para pesquisas futuras.        
\end{enumerate}
