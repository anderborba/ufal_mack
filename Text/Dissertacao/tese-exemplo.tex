% Arquivo LaTeX de exemplo de dissertação/tese a ser apresentados à CPG do IME-USP
% 
% Versão 5: Sex Mar  9 18:05:40 BRT 2012
%
% Criação: Jesús P. Mena-Chalco
% 
%  Revisão: Fabio Kon e Paulo Feofiloff
% Obs: Leia previamente o texto do arquivo README.txt
\documentclass[11pt,twoside,a4paper]{book}
% ---------------------------------------------------------------------------- %
% Pacotes 
%\usepackage[T1]{fontenc}
\usepackage[utf8]{inputenc}
\usepackage[brazil]{babel}
%%% ACF Esses dois comandos deveriam trocar "sin" por "sen" e "tan" por "tg", mas não estão funcionando. Verifique.
\renewcommand{\sin}{\hspace{2pt}\mathrm{sen}}
\renewcommand{\tan}{\hspace{2pt}\mathrm{tg}}

%\usepackage[latin1]{inputenc}
\usepackage[pdftex]{graphicx}           % usamos arquivos pdf/png como figuras
\usepackage{setspace}                   % espaçamento flexível
\usepackage{indentfirst}                % indentação do primeiro parágrafo
\usepackage{makeidx}                    % índice remissivo
\usepackage[nottoc]{tocbibind}          % acrescentamos a bibliografia/indice/conteudo no Table of Contents
\usepackage{merriweather}                    % usa o Adobe Courier no lugar de Computer Modern Typewriter
\usepackage{type1cm}                    % fontes realmente escaláveis
%\usepackage{listings}                   % para formatar código-fonte (ex. em Java)
\usepackage[font=small,format=plain,labelfont=bf,up,textfont=it,up]{caption}
\usepackage[usenames,svgnames,dvipsnames]{xcolor}
%\usepackage[a4paper,top=2.54cm,bottom=2.0cm,left=2.0cm,right=2.54cm]{geometry} % margens
\usepackage{geometry}
 \geometry{
 a4paper,
 right = 30.0mm,
 left=45.0mm,
 top=42.0mm,
 bottom=29mm,
 }
\usepackage{titletoc}
\usepackage{amsmath,amssymb}           % AAB inserido
\usepackage{booktabs}                   % AAB inserido
\usepackage{siunitx}                    % AAB inserido
\usepackage{rotating}                   % AAB inserido
\usepackage{tikz}                       % AAB inserido
\usetikzlibrary{shapes,arrows,shadows}  % AAB inserido
\usepackage{bm,bbm}                     % AAB inserido
\usepackage{float}                      % AAB inserido
\usepackage{enumitem}                   % AAB inserido
%\userpackage{pdfpages}                 % AAB inserido
\usepackage[final]{pdfpages}            % AAB inserido
%\usepackage[labelformat=empty]{caption}                  % AAB inserido
%\usepackage[caption=false,font=normalsize,labelfont=sf,textfont=sf]{subfig}
\usepackage{subfig}

%%%%
% Pacotes do artigo GRSL
\usepackage{times}
\sisetup{ detect-weight=true, binary-units=true }
%\usepackage{amsmath,amssymb,amsfonts}
%\usepackage[boxed]{algorithm2e}
%\usepackage{textcomp}
%\usepackage{mathtools}                       
%\usepackage[caption=false,font=footnotesize]{subfig}
%\usepackage[caption=false,font=footnotesize]{subfig}
%\usepackage{amsmath,bm,times}           % AAB inserido
%%%<
%\usepackage[alf,abnt-etal-cite=2]{abntcite} % AAB verificar as citacoes
%\usepackage[bf,small,compact]{titlesec} % cabeçalhos dos títulos: menores e compactos
%\usepackage[fixlanguage]{babelbib}
%\usepackage[pdftex,plainpages=false,pdfpagelabels,pagebackref,colorlinks=true,citecolor=black,linkcolor=black,urlcolor=black,filecolor=black,bookmarksopen=true]{hyperref} % links em preto
\usepackage[pdftex,plainpages=false,pdfpagelabels,pagebackref,colorlinks=true,citecolor=DarkGreen,linkcolor=NavyBlue,urlcolor=DarkRed,filecolor=green,bookmarksopen=true]{hyperref} % links coloridos
\usepackage[all]{hypcap}                    % soluciona o problema com o hyperref e capitulos
\usepackage[round,sort,nonamebreak]{natbib} % citação bibliográfica textual(plainnat-ime.bst)
\fontsize{60}{62}\usefont{OT1}{cmr}{m}{n}{\selectfont}
%
%\hyphenation{ma-ni-pu-la-cao}
% ---------------------------------------------------------------------------- %
% Cabeçalhos similares ao TAOCP de Donald E. Knuth
\usepackage{fancyhdr}
%\pagestyle{fancy}
%\fancyhf{}
%\fancyhf{}
%\fancyhead[LE,RO]{\thepage}
%\fancyhead[RE]{\textbf{ \nouppercase{Kapitel \thechapter: \leftmark}} }
%\fancyhead[LO]{\textbf{ \nouppercase{\rightmark}} }
%
%\fancypagestyle{plain}{ %
%  \fancyhf{} % remove everything
%  \renewcommand{\headrulewidth}{0pt} % remove lines as well
%  \renewcommand{\footrulewidth}{0pt}
%}
%
\renewcommand{\chaptermark}[1]{\markboth{\MakeUppercase{#1}}{}}
\renewcommand{\sectionmark}[1]{\markright{\MakeUppercase{#1}}{}}
\renewcommand{\headrulewidth}{0pt}
\newtheorem{definition}{Definição}[section]
\DeclareMathOperator{\traco}{tr}

% ---------------------------------------------------------------------------- %
\graphicspath{{./figuras/}}             % caminho das figuras (recomendável)
\frenchspacing                          % arruma o espaço: id est (i.e.) e exempli gratia (e.g.) 
\urlstyle{same}                         % URL com o mesmo estilo do texto e não mono-spaced
\makeindex                              % para o índice remissivo
\raggedbottom                           % para não permitir espaços extra no texto
\fontsize{60}{62}\usefont{OT1}{cmr}{m}{n}{\selectfont}
\cleardoublepage
\normalsize
% ---------------------------------------------------------------------------- %
% Opções de listing usados para o código fonte
% Ref: http://en.wikibooks.org/wiki/LaTeX/Packages/Listings
%\lstset{ %
%language=Java,                  % choose the language of the code
%basicstyle=\footnotesize,       % the size of the fonts that are used for the code
%numbers=left,                   % where to put the line-numbers
%numberstyle=\footnotesize,      % the size of the fonts that are used for the line-numbers
%stepnumber=1,                   % the step between two line-numbers. If it's 1 each line will be numbered
%numbersep=5pt,                  % how far the line-numbers are from the code
%showspaces=false,               % show spaces adding particular underscores
%showstringspaces=false,         % underline spaces within strings
%showtabs=false,                 % show tabs within strings adding particular underscores
%frame=single,	                % adds a frame around the code
%framerule=0.6pt,
%tabsize=2,	                    % sets default tabsize to 2 spaces
%captionpos=b,                   % sets the caption-position to bottom
%breaklines=true,                % sets automatic line breaking
%breakatwhitespace=false,        % sets if automatic breaks should only happen at whitespace
%escapeinside={\%*}{*)},         % if you want to add a comment within your code
%backgroundcolor=\color[rgb]{1.0,1.0,1.0}, % choose the background color.
%rulecolor=\color[rgb]{0.8,0.8,0.8},
%extendedchars=true,
%xleftmargin=10pt,
%xrightmargin=10pt,
%framexleftmargin=10pt,
%framexrightmargin=10pt
%}

% ---------------------------------------------------------------------------- %
% Corpo do texto
\begin{document}
\frontmatter 
% cabeçalho para as páginas das seções anteriores ao capítulo 1 (frontmatter)
%\fancyhead[RO]{{\footnotesize\rightmark}\hspace{2em}\thepage}
%\setcounter{tocdepth}{2}
%\fancyhead[LE]{\thepage\hspace{2em}\footnotesize{\leftmark}}
\fancyhead[RE,LO]{}
%\fancyhead[RO]{{\footnotesize\rightmark}\hspace{2em}\thepage}
\onehalfspacing  % espaçamento
\fancyhead[RE,LO]{\thesection}
\fancyhead[LO]{\thesection}
% ---------------------------------------------------------------------------- %
% CAPA
\thispagestyle{empty}
\begin{center}
    \vspace*{-3.7cm}
    \textbf{\Large{UNIVERSIDADE PRESBITERIANA MACKENZIE}}\\
    %\Large{Programa de Pós graduação em}\\
    %\Large{Engenharia Elétrica e Computação - PPGEEC}\\
    \vspace*{0.5cm}
    \large{ANDERSON ADAIME DE BORBA}\\
    \vspace*{5.0cm}
    \setlength{\baselineskip}{1.5\baselineskip}
\textbf{\large FUSÃO DE EVIDÊNCIAS DE BORDAS DOS CANAIS DE INTENSIDADES DE IMAGENS DE RADAR POLARIMÉTRICO DE ABERTURA SINTÉTICA}\\
    \vspace*{14.2cm}
    \large{São Paulo}\\
    %\vskip -0.4cm
    \large{2020}
\end{center}
\pagebreak
% CAPA
\thispagestyle{empty}
\vspace*{-3.7cm}
\begin{center}
\large  \textbf{UNIVERSIDADE PRESBITERIANA MACKENZIE}
\large  \textbf{PROGRAMA DE PÓS-GRADUAÇÃO EM}\\
\large  \textbf{ENGENHARIA ELÉTRICA E COMPUTAÇÃO}\\
\vskip 2.0cm
\text{\large ANDERSON ADAIME DE BORBA}\\
\vskip 4.0cm
\setlength{\baselineskip}{1.5\baselineskip}
\textbf{\large FUSÃO DE EVIDÊNCIAS DE BORDAS DOS CANAIS DE INTENSIDADES DE IMAGENS DE RADAR POLARIMÉTRICO DE ABERTURA SINTÉTICA}\\
\vskip 4.5cm
\end{center}
\hfill{\vbox{\hsize=8.5cm\noindent\strut
Tese apresentada ao Programa de Pós-Graduação\break
em Engenharia Elétrica e Computação da Universidade\break
 Presbiteriana Mackenzie, como requisito parcial à\break
obtenção do título de Doutor.}\\
\strut}
\vskip 3.0cm
\hspace{-0.7cm} \textbf{\normalsize Orientador: Prof. Dr. Maurício Marengoni}\\
\textbf{\normalsize  Coorientador: Prof. Dr. Alejandro César Frery Orgambide}
\begin{center}
São Paulo\\
2020\\
\end{center}
% ---------------------------------------------------------------------------- %
% Página de rosto (SÓ PARA A VERSÃO DEPOSITADA - ANTES DA DEFESA)
% Resolução CoPGr 5890 (20/12/2010)
%
% IMPORTANTE:
%   Coloque um '%' em todas as linhas
%   desta página antes de compilar a versão
%   final, corrigida, do trabalho
%
%
%\newpage
%\thispagestyle{empty}
%    \begin{center}
%        \vspace*{2.3 cm}
%        \textbf{\Large{Imagens PolSAR}}\\
%        \vspace*{2 cm}
%    \end{center}
%
%    \vskip 2cm
%
%    \begin{flushright}
%	Esta é a versão original da tese elaborada pelo\\
%	candidato Anderson Adaime de Borba, tal como \\
%	submetida à Comissão Julgadora.
%    \end{flushright}
%
%\pagebreak
% ---------------------------------------------------------------------------- %
% Página de rosto (SÓ PARA A VERSÃO CORRIGIDA - APÓS DEFESA)
% Resolução CoPGr 5890 (20/12/2010)
%
% Nota: O título para as dissertações/teses do IME-USP devem caber em um 
% orifício de 10,7cm de largura x 6,0cm de altura que há na capa fornecida pela SPG.
%
% IMPORTANTE:
%   Coloque um '%' em todas as linhas desta
%   página antes de compilar a versão do trabalho que será entregue
%   à Comissão Julgadora antes da defesa
%
%
%\begin{figure}
% \centering 
% \includepdf[pages=-]{file.pdf}
%\end{figure}
\pagebreak
\thispagestyle{empty}
%\begin{figure}
% \centering 
 \includepdf[pages=-]{ficha_Borba}
% \caption*{}
%\end{figure}
\pagebreak
\thispagestyle{empty}
%\begin{figure}
% \centering 
 \includepdf[pages=-]{folha_agencia_fomento}
% \caption*{}
%\end{figure}
\pagebreak
\thispagestyle{empty}
%\begin{figure}
% \centering 
 \includepdf[pages=-]{bancadoutoradoAnderson}
% \caption{}
%\end{figure}
\pagebreak
%\includepdf[page=-]{bancadoutoradoAnderson} 
%\pagebreak
%\thispagestyle{empty}
%\vspace*{-3.7cm}
%\begin{center}
%\text{\large ANDERSON ADAIME DE BORBA}\\
%\vskip 2.0cm
%\setlength{\baselineskip}{1.5\baselineskip}
%\textbf{\large FUSÃO DE EVIDÊNCIAS DE BORDAS DOS CANAIS DE INTENSIDADES DE IMAGENS DE RADAR POLARIMÉTRICO DE ABERTURA SINTÉTICA}\\
%\vskip 2.0cm
%\end{center}
%\hfill{\vbox{\hsize=8.5cm\noindent\strut
%Tese apresentada ao Programa de Pós-Graduação\break
%em Engenharia Elétrica e Computação da Universidade\break
% Presbiteriana Mackenzie, como requisito parcial à\break
%obtenção do título de Doutor.}\\
%\strut}
%    \vskip 1.5cm
%\text{\large Aprovada em 16 de Dezembro de 2020}\\
%    \begin{center}
%\text{\large BANCA EXAMINADORA}\\
%     \vskip 0.5cm
%     $\rule{10cm}{0.15mm}$ \\
%     Prof. Dr. Maurício Marengoni\\
%     Universidade Presbiteriana Mackenzie - UPM\\
%     \vskip 0.5cm
%     $\rule{10cm}{0.15mm}$\\
%     Prof$^{\underline{a}}$. Dr$^{\underline{a}}$. Ana Grasielle Dionísio Corrêa\\
%     Universidade Presbiteriana Mackenzie - UPM\\
%     \vskip 0.5cm
%     $\rule{10cm}{0.15mm}$\\
%     Prof. Dr. Carlos E. Thomaz\\
%     Centro Universitário FEI - FEI\\
%     \vskip 0.5cm
%     $\rule{10cm}{0.15mm}$\\
%     Prof$^{\underline{a}}$. Dr$^{\underline{a}}$. María Juliana Gambini\\
%     Instituto Tecnológico de Buenos Aires - ITBA\\
%     \vskip 0.5cm
%     $\rule{10cm}{0.15mm}$\\
%     Prof. Dr. Paulo Batista Lopes\\
%     Universidade Presbiteriana Mackenzie - UPM\\
%    \end{center}
%\pagebreak
%
%
\pagenumbering{roman}     % começamos a numerar 
% ---------------------------------------------------------------------------- %
% Dedicatoria:
% Se o candidato não quer fazer agradecimentos, deve simplesmente eliminar esta página 
\chapter*{}
\vskip 15.cm
\hfill{\vbox{\hsize=5.5cm\noindent\strut
Dedico esta tese à minha família, e\break
à minha companheira Michele.}\\
\strut}
% ---------------------------------------------------------------------------- %
%% Agradecimentos:
%% Se o candidato não quer fazer agradecimentos, deve simplesmente eliminar esta página 
\chapter*{Agradecimentos}
Agradeço à minha família pelo carinho e compreensão, à minha companheira por entender minha
ausência, sempre atenta e cuidadosa com convívio diário. Agradeço aos amigos que
participaram desse processo com conversas, cafés ou churrascos com uma boa salsa e, ao meu velho pai, para ao qual dedico a letra da música \textit{Naquela Mesa}, composta por Sérgio Bittencourt em homenagem a seu pai Jacob do Bandolim. \vspace{0.6cm}

\noindent Naquela mesa ele sentava sempre\\
E me dizia sempre, o que é viver melhor.\\
Naquela mesa ele contava histórias,\\
Que hoje na memória eu guardo e sei de cor.\\
\noindent Naquela mesa ele juntava gente\\
E contava contente o que fez de manhã.\\
E nos seus olhos era tanto brilho,\\
Que mais que seu filho, eu fiquei seu fã.\\
\noindent Eu não sabia que doía tanto\\
Uma mesa no canto, uma casa e um jardim.\\
Se eu soubesse o quanto doí a vida,\\
Essa dor tão doída não doía assim.\\
\noindent Agora resta uma mesa na sala\\
E hoje ninguém mais fala no seu bandolim.\\
Naquela mesa tá faltando ele\\
E a saudade dele tá doendo em mim.\\

E agradeço especialmente à Maurício Marengoni e à Alejandro Frery pela condução deste trabalho, sempre de forma muito generosa. Apreendi muito, como também, resgatei minha curiosidade e motivação com a pesquisa. Sinceramente, obrigado.
%%
%% Epígrafe:
%% Se o candidato não quer fazer agradecimentos, deve simplesmente eliminar esta página 
\chapter*{}
\vskip 15.cm
\hfill{\vbox{\hsize=5.5cm\noindent\strut
A chave de todas as ciências é inegavelmente o ponto de interrogação.\break
\textit{Honoré de Balzac}
}\\
\strut}
% ---------------------------------------------------------------------------- %
% Resumo
\chapter{Resumo}
%\thispagestyle{empty}
%
Os radares polarimétricos de abertura sintética (PolSAR) alcançaram uma posição essencial no sensoriamento remoto. As imagens que eles fornecem têm ruído \textit{speckle}, tornando as ações de processamento e de análise tarefas desafiadoras. Nesse contexto a investigação de métodos de fusão de evidências de bordas é importante para quantificar e qualificar as informações obtidas de cada canal da imagem. A obtenção desses dados possibilita a decisão de usar ou descartar as informações de um dado canal para melhorar o desempenho da detecção de bordas.  Neste trabalho foram estudados e comparados seis métodos de fusão de informações provenientes da detecção das evidências de bordas nos canais de intensidade hh, hv e vv de imagens PolSAR múltiplas visadas. O método para detectar evidências de bordas em cada canal consiste em detectar pontos de transição em uma faixa de dados, o mais fina possível, idealmente com largura de um píxel, fornecendo uma faixa de dados que cobre duas regiões usando o método estimativa de máxima verossimilhança sob a distribuição de Wishart. Os métodos de fusão das informações proveniente de cada canal usados são: média simples (MS), transformada \textit{wavelet} discreta multi-resolução (MR-DWT), análise de componente principal (PCA), estatísticas ROC, transformada \textit{wavelet} estacionária multi-resolução (MR-SWT), e um método de multi-resolução baseado na decomposição de valores singulares (MR-SVD). A comparação dos seis métodos de fusão foi realizada quantitativamente e qualitativamente, respectivamente, calculando-se a proximidade das bordas detectadas com as bordas definidas nas imagens \textit{Ground Reference} e pela presença de \textit{outliers}. Os resultados obtidos com as análises indicam que os métodos PCA e MR-SVD fornecem os melhores resultados, devido a precisão em detectar bordas e a baixa incidência de \textit{outliers}. 
%
\vskip 0.5cm
\noindent Palavras-chave: PolSAR image. Detecção de bordas. Estimativa de máxima verossimilhança. Métodos de fusão.
% ---------------------------------------------------------------------------- %
% Abstract
\chapter{Abstract}
%\thispagestyle{empty}
Polarimetric Synthetic Aperture Radar (PolSAR) sensors have reached an essential position in remote sensing. 
The images they provide have speckle noise, making their processing and analysis challenging tasks. In this context, the investigation of edge evidence fusion methods is essential to quantify and qualify the information obtained from each image channel. Getting this data enables the decision to use or discard the information from a given channel to improve the performance of edge detection. 
We discuss an edge detection method based on the fusion of evidences obtained in the intensity channels hh, hv, and vv of PolSAR multi-look images.
The method to detect evidence of edges in each channel consists of detecting transition points in the thinnest possible range of data that covers two regions using maximum likelihood under the Wishart distribution. 
The methods of fusion of the information coming from each channel used are:
simple average, 
multi-resolution discrete wavelet transform (MR-DWT),
principal component analysis (PCA), 
Receiver operating characteristic (ROC) statistics, 
multi-resolution stationary (MR-SWT) wavelet transform, 
and a multi-resolution method based on singular value decomposition (MR-SVD). 
The six fusion methods were compared quantitatively e qualitatively, rescpectively, calculating the distance of the detected edges to the edges defined in the Ground Reference images and by the presence of outliers.
The results obtained with the analyses suggest that the PCA and MR-SVD methods provide the best results due to the precision in detecting edges and the low incidence of outliers.
\vskip 0.5cm
\noindent Keywords: PolSAR. Edge detection. Maximum likelihood estimation, Fusion methods. 
% ---------------------------------------------------------------------------- %
% ---------------------------------------------------------------------------- %
%\thispagestyle{empty}
\chapter{Lista de Abreviaturas}
\begin{tabular}{ll}
RAR            & Radar de abertura real\\
SAR            & Radar de abertura sintética\\
PolSAR         & Radar de abertura sintética polarimétrica\\
PDF            & Função densidade de probabilidade\\
ROI            & Região de Interesse\\
LoG            & Detector de borda usando o laplaciano da gaussiana\\
MLE            & Método estimativa de máxima verossimilhança\\
MS             & Média simples\\
MR--DWT         & Transformada \textit{wavelet} discreta com múltiplas resoluções \\
PCA            & Analise das componentes principais\\
E--ROC          & Estatística ROC (\textit{Receiver operating characteristic}) \\
MR--SWT         & Transformada \textit{wavelet} estacionária com múltiplas resoluções \\
MR--SVD         & Decomposição em valores singulares com múltiplas resoluções\\
GenSA          & Método Simullated anneling\\
BFGS           & Método Broyden-Fletcher-Goldfarb-Shanno\\
VHR            & \textit{Very High Resolution}\\
DLR            & Centro espacial da Alemanha\\
BSA            & Sistema de coordenadas \textit{Back Scattering Alignment}\\
DOI            & \textit{Digital Object Identifier}\\
URL            & \textit{Uniform Resource Locator}\\
GR             & Ground Reference \\
FLEV--ROI-I    & Região de interesse I na imagem de Flevoland\\
FLEV--ROI-II   & Região de interesse II na imagem de Flevoland\\
SF--ROI        & Região de interesse na imagem de São Francisco\\
\end{tabular}
% ---------------------------------------------------------------------------- %
\chapter{Lista de Símbolos}
%\thispagestyle{empty}
\begin{tabular}{ll}
        $L$         & Número de visadas em uma imagem PolSAR\\
        $\Sigma$    & Mariz de covariância hermitiana e definida positiva \\
	$E[\cdot]$  & Valor esperado\\
	$\Gamma$    & Função Gamma \\
	$\Gamma_m$  & Função Gamma multivariada\\
	$W(\Sigma, L)$ & Distribuíção Wishart\\
	$\mathcal{L}$  & Função de máxima log-verossimilhança total\\
	$n_c$          & Número de canais utilizados na fusão de informações\\
	$\{\widehat{\bm\jmath}_c\}_{1\leq c\leq n_c}$ & Imagem binária para as evidências de bordas\\
	$\{\widehat{\bm\imath}_c\}_{1\leq c\leq n_c}$ & Mapas de evidências de bordas\\
	$I_\text{F}$   & Imagem com a fusão de informações \\
\end{tabular}
% ---------------------------------------------------------------------------- %
% Listas de figuras e tabelas criadas automaticamente
\listoffigures            
\listoftables            
% Sumário
\tableofcontents    % imprime o sumário
% ---------------------------------------------------------------------------- %
% Capítulos do trabalho
\mainmatter
\setcounter{secnumdepth}{4}
% cabeçalho para as páginas de todos os capítulos
%\fancyhead[RE,LO]{}
%\fancyhead[RO]{{\footnotesize\rightmark}\hspace{2em}\thepage}
%\onehalfspacing  % espaçamento
%\fancyhead[RE,LO]{\thesection}
\fancyhead[LO]{\thesection}
\fancyhead[RE,LO]{\thesection}
%\singlespacing              % espaçamento simples
\onehalfspacing            % espaçamento um e meio
%\pagenumbering{arabic}
%/\chapter{Fusão de evidências na detecção de bordas em Imagens PolSAR}
\chapter{Introdução}
\label{cap_acf}

\section{Aspectos gerais}

Neste trabalho estudaremos as imagens de radar de abertura sintética (\textit{Synthetic Aperture Radar} -- SAR) e as imagens de radar polarimétrico de abertura sintética (\textit{Polarimetric Synthetic Aperturadar} -- PolSAR).
Ambas requerem modelos e algoritmos adequados para o tratamento das suas características especiais.
Em particular, trabalharemos com técnicas de detecção de bordas.

No estudo de imagens PolSAR podemos citar diferentes técnicas, por exemplo, no trabalho de \citet{slf_2008} é usado modelagem eletromagnética para propor uma abordagem de detecção de bordas nas simulações de imagens PolSAR. 
%%% ACF Frase confusa:
Nos trabalhos de \citet{tlb, obw, flmc, fyf} podemos encontrar técnicas de detecção de bordas baseadas em métodos que estimam o gradiente, que funcionam usando uma janela deslizante, destacando através do cálculo do gradiente as bordas das imagens. 
Em \citet{obw} foi proposto o método de máxima verossimilhança para detectar a presença de bordas, em uma janela pré-definida de pixeis com a tarefa de encontrar bordas acuradas. 
O estudo de detecção de bordas, usando propriedades físicas das áreas estudadas, pode ser encontrado no trabalho de \citet{bf}, as quais utilizam cadeias de Markov. 
Em \citep{gfn} é descrita a comparação entre vários detectores de bordas que seguem a ideia deste trabalho. 
Técnicas baseadas nas modelagens estatísticas têm sido usadas na detecção de bordas em imagens SAR, podemos citar os trabalhos de \citet{gmbf, fbgm, horrit, gfn}. 
%%% ACF A revisão da literatura precisa ser mais informativa. São mencionadas abordagens completamente diferentes, sem que o leitor saiba as suas diferentes hipóteses de partida, nem os seus resultados. Essa parte é central para dar valor ao trabalho de doutorado.

Atualmente as pesquisas em \textit{Deep Learning} são muito importantes na área de sensoriamento remoto, aplicações são encontradas nas referências \citep{bac, ztmxzxf, tabmm, xstz}. 
As técnicas de \textit{Deep Learning} são usadas para segmentar ou classificar imagens, podendo auxiliar no processo de detecção de bordas. 

A área de fusão de imagens também é explorada neste trabalho. 
Um recente e interessante artigo, cujo autores são \citet{sglmla}, usa ideias do método \textit{random forest} aplicado em fusão de imagens PolSAR, adicionalmente, o artigo de \citet{sg} mostra outras técnicas de fusão de informação.  

O presente trabalho seguirá a abordagem de modelagem estatística, principalmente as técnicas descritas em \citep{fbgm, nhfc} usando a distribuição Wishart. Para realizar a fusão de informações temos como base as referências \citep{mit, sglmla, sg}. 

O objetivo deste trabalho é detectar bordas em cada polarização (canal) de uma imagem PolSAR e realizar a fusão das evidências de bordas, com a tarefa de melhorar a acurácia em relação a borda detectada, em cada polarização. A sequência de procedimentos pode ser descrita por:
\begin{itemize}%%% ACF Use o pacote "listing" para configurar e automatizar listas; com isso, poderá fazer referência a elas.
	\item[(i)] em cada polarização;
	\item[(ii)] especificar manualmente ou automaticamente a região de interesse (\textbf{ROI});
	\item[(iii)] em cada (\textbf{ROI}) calcular o centro de massa e a partir desse ponto traçar retas na direção radial de forma que garanta a existência de uma borda em cada reta;
	\item[(iv)] para todas as retas devemos analisar a variação das amostras, usando o método da máxima verossimilhança com o intuito de descobrir uma evidência de borda ou ponto de transição;
	\item[(v)] obtendo as evidências de bordas em todas as reta impostas, e repetindo o método em cada polarização, é realizado a média aritmética ou ponderada das evidências de bordas, para obter a fusão das mesmas;
	%%% ACF Aqui você se compromete com uma única técnica de fusão. É pouco para uma tese de doutorado.
	\item[(vi)] realizado a fusão das evidências de borda, teremos uma nuvem de pontos onde podemos aplicar métodos de regressão, usando para isso, o método dos quadrados mínimos para obter os resultados numéricos do presente trabalho.
\end{itemize}

Os itens (i) até (iv) tratam da detecção de borda que foram propostos nas seguintes referências bibliográficas \citep{gmbf, gmbf_sc, fbgm}, esses artigos usam a distribuição $G^{0}$ proposta por \citet{fmcs}. Neste trabalho usaremos a distribuição Wishart como na referência \citep{nhfc}. 

No item (v) a principal referência usada para fusão de evidências foi \citet{mit}, no item (vi) para o método dos quadrados mínimos foi usado a referência \citep{demmel}.

No presente trabalho o procedimento acima foi testado em imagens simuladas ou \textit{Phantons}. O resultado comparativo em relação as detecções de bordas nas polarizações nos mostrou uma estratégia de detecção de borda mais acurada. Os resultados alcançados devem ser testados em imagens PolSAR reais para verificar o desempenho e acurácia. 
\section{ Formação das imagens SAR}
O radar de abertura sintética (SAR) é o desenvolvimento tecnológico do radar de abertura real (RAR), o qual, de maneira geral trabalha com sensores ativos que transmitem micro-ondas e depois registram os ecos recebidos. Os radares usam plataformas tanto móveis como satélites, balões, veículos aéreos tripulados ou não quanto fixas, por exemplo, radares em aeroportos. Esses radares viajam em uma rota conhecida, transmitindo micro-ondas em direção ao alvo e recebendo micro-ondas ou ecos depois da interação com o alvo. 

Podemos afirmar que os radares têm fontes próprias de energia, ou seja, as micro-ondas são emitidas no próprio radar, por esse motivo são definidos como radares ativos, e adicionalmente, se possuirem uma antena para transmitir e receber os impulsos, são denominadas mono-estáticos. Importantes informações sobre radares podem ser encontradas na referência \citep{lp}, seus desenvolvimentos são norteados pelos princípios definidos a seguir:
\begin{itemize}
\item[(i)] possibilidade de uma antena transmitir um certo impulso eletromagnético (micro-ondas) em uma direção precisa; 
\item[(ii)] possibilidade de detectar com grande precisão o eco, depois da interação com o alvo, é atenuando com um processo que podemos chamar de espalhamento, onde a onda encontra o material do alvo induzindo uma corrente que gera uma energia eletromagnética irradiada em todas as direções, a maior parte dessa energia irradiada distancia-se do radar causando o espalhamento, não sendo possível detectar;
\item[(iii)] capacidade de medir o tempo entre a transmissão  e a recepção do impulso eletromagnético e, consequentemente, a distância entre o alvo e a antena;
\item[(iv)] habilidade de detectar vários alvos em grandes áreas a serem varridas.
\end{itemize}

A onda eletromagnética pode variar em comprimento e amplitude, dependendo da construção dos sensores para o radar em operação, uma característica importante dessas ondas é a capacidade de penetração no material analisado, dependendo do comprimento de onda, por exemplo, quanto maior o comprimento de onda, maior será a penetração no material analisado. 

Os radares podem produz imagens bidimensionais (2D) com resoluções distintas, dependendo do tipo de radar e da tecnologia empregada. As dimensões das imagens dependem respectivamente da resoluções nas direções de azimute e distância (\textit{range}). A direção de azimute é a mesma direção da rota do radar sendo perpendicular à direção da distância. A resolução na direção de azimute depende do comprimento $l$ da abertura antena de radar.  

Os radares que dependem diretamente dos comprimentos $l$ são os radares de abertura real (RAR), os quais têm uma grande restrição de uso em sensoriamento remoto, pois o comprimento da antena limita a resolução na direção de azimute, ou seja, para aumentar a resolução nesta direção, precisamos aumentar o comprimento da antena, o que pode ser inviável em satélites ou até mesmo em aviões, configurando-se em grande problema para essa área de pesquisa até os anos 50.  

Nos anos 50, pesquisadores fizeram avanços significativos para resolver a limitação tecnológica e desenvolveram uma técnica apta a sintetizar o efeito de uma antena muito longa, em uma antena de tamanho real. Para esse fim utilizaram o conhecimento na área de  processamentos de sinais em uma antena real para tornar possível a simulação de uma antena muito longa e, portanto, aumentar a resolução na direção do azimute. Essa descoberta é atribuída a Carl Wiley por volta de 1951.

O tipo de radar que utiliza a técnica, na qual uma antena real sintetiza uma antena longa, tornou-se conhecida como radar de abertura sintética (SAR), em 1954, Wiley registrou a patente do sistema SAR. O principal conceito físico para gerar a tecnologia dos sistema SAR, foi o efeito Doppler aplicados aos ecos dos radares. Devido ao avanço tecnológico no início da década de 50, um sistema SAR foi operacionalizado em torno de 1958. Consequentemente esse fato impulsionou fortemente a área de pesquisa relacionada com os sistemas SAR. 

Os dados provenientes do método de sintetização da antena real (SAR) são gravados de maneira única como uma faixa de posições para cada tempo avançado na direção da rota (azimute). No início do desenvolvimento dos sistemas SAR o armazenamento de dados foi um grande problema, para contorná-lo foi acoplado um sistema ótico nos radares para armazenar dados em filmes fotográficos. Atualmente devido ao desenvolvimento da eletrônica, o armazenamento de dados deixou de ser um problema nos sistemas SAR.

Nos sistemas SAR as ondas eletromagnéticas são representadas como imagens, que chamaremos de imagens SAR, geradas de forma que em cada ponto da direção azimute, o radar envia um impulso e recebe o sinal (eco), com o devido espalhamento. Esse sinal é armazenado ao longo da distância. Portanto, para cada tempo na direção do azimute geramos informações que serão armazenadas em uma linha de dados, tendo como resultado um mapeamento azimute \textit{versus} distância da energia recebida pelo radar, durante o tempo de aquisição de dados. Essas informações são típicas de armazenamentos em matrizes com as dimensões dependente da resolução inerente do radar.

Uma imagem SAR é visualizada em tons de cinza dependendo de como o alvo espalha a onda eletromagnética, sendo assim um alvo mais rugoso dá origem a pixeis mais claros, enquanto um alvo menos rugoso dá origem a pixeis mais escuros. 

Uma importante propriedade das imagens SAR é a transmissão e o recebimento de ondas eletromagnéticas em única direção, isto é, a onda pode ser transmitida e recebida tanto na direção horizontal como na vertical, configurando a polarização em uma das direções.

A generalização do processo SAR é conhecido como sistema polarimétrico SAR (PolSAR) definido como sendo a ciência de adquirir, processar e analisar o estado da polarização nas imagens de radar de abertura sintética, podemos dizer que é um sistema SAR capaz de medir mais de um estado de polarização. Em sistemas PolSAR as imagens de radares são formadas por ondas eletromagnéticas com várias combinações para as polarizações, tanto na transmissão como no recebimento das ondas, revelando uma melhor descrição do alvo em relação as imagens SAR. As imagens PolSAR têm o intuito de melhorar o entendimento do efeito do espalhamento de ondas pelos alvos levando, em conta as diferentes polarizações.

Os radares são usados de forma massiva desde os anos 40, principalmente para uso militar. Nos anos 50 houve a descoberta da tecnologia SAR, o que impulsionou um grande desenvolvimento na área levando a construção do primeiro radar de abertura sintética (SAR) comercial. O SEASAT foi o primeiro satélite orbital, operacional  e comercial projetado, seu lançamento foi em junho de 1978, a tabela (\ref{cap_acf_tab01}) mostra algumas características do SEASAT. 

\begin{table}[hbt]
	\centering
	\caption{Características do satélite SEASAT (SAR).}\label{cap_acf_tab01}
\begin{tabular}{@{}llr@{}} \toprule
	Características específicas& Valores operacionais  \\ \midrule
	Frequência           & \SI{1.275}{\GHz}  \\ 
	Altitude             & \SI{780}{\km}   \\
	Peso                 & \SI{2300}{\kilogram}   \\
	Ângulo de inclinação & \SI{\sim 23}{\degree}   \\
	Distância {\it range}& \SI{100}{\km}  \\
	Largura de banda     & \SI{19}{\MHz}   \\
	Banda - $L$          & \SI{23.5}{\cm} de comprimento de onda\\
	Polarização          & $HH$, onda emitida e recebida na direção horizontal \\
	Resolução            & $25 \times 25$  \\ \bottomrule

\end{tabular}
\end{table}
  
O projeto do SEASAT foi muito bem sucedido e estabeleceu definitivamente os sistemas SAR como área de pesquisa. Outros projetos de sistema SAR e PolSAR foram lançados e podem ser vistos na tabela (\ref{cap_acf_tab02}).

\begin{sidewaystable}
	\centering
	\caption{Características operacionais dos satélites SAR ou PolSAR.}\label{cap_acf_tab02}
\begin{tabular}{@{}llllllllr@{}} \toprule
Satélites      & 	SEASAT  &AIRSAR &SIR-C& Almaz&ERS-2& JERS-1& RADSAT-1&RADSAT-2 \\ \midrule
Nacionalidade       &EUA    &EUA&Alamanha-Itália&Rússia&Europa&Japão&Canadá&Canadá  \\ 
Lançamento          &1978   &1988        &1990  &1992  &1995 &1998  &1995  & 2003\\
C. de onda (\si{\cm}) (Banda) & 23.5 ($L$)&67 ($P$)/23.5 ($L$)/5.7 ($C$)&23.5 ($L$)/5.7 ($C$)/3.2 ($X$)&10 ($S$)&5.7 ($C$)&23.5 ($L$)&5.6 ($C$)&5.6 ($C$)\\
Polarização         &$HH$&$HH/HV/VV$&$HH/HV/VV$&$HH$&$VV$&$HH$&$HH$&$HH/HV/VV$\\
Ângulo de incidência&23&20-60&15-55&30-60&23&35&20-59&20-60\\
Distância (\si{\km})           &100&10-17&15-90&350&100&75&50-500&10-500\\
Resolução (\si{\m})         &25&2-8&10-60&10-30&30&18&10-100&3-100\\ \bottomrule
\end{tabular}
\end{sidewaystable}

Os símbolos $HH$, $HV$ e $VV$ representam as polarizações disponíveis nos radares, onde a primeira letra é a maneira como a onda é emitida e, a segunda letra é a maneira como a onda é recebida. Desta forma, quando aparece todas as combinações juntas temos um satélite com tecnologia PolSAR.

Os radares SAR e PolSAR possuem algumas características operacionais que podem ser resumidas nos seguintes itens:
\begin{itemize}
\item podem estar em plataformas elevadas, aeronaves tripuladas ou não, satélites orbitando a terra ou outros planetas;
\item é uma técnica de produção de imagem viável e prática;
\item possui alta resolução;
\item sintetiza longas aberturas de antenas;
\item os radares produzem imagens dia e noite;
\item o clima não interfere na captação de imagens;
\item os sistema de imagem SAR operam na região de micro-ondas do espectro eletromagnético, usualmente entre a banda $P-$ e a banda $K-$, a tabela (\ref{cap_acf_tab03}) mostra o espectro eletromagnético usado nas imagens SAR.
\end{itemize}
%%% ACF O texto está desordenado. As propriedades gerais do SAR/PolSAR deveriam estar antes das suas propriedades específicas
As aplicações das imagens SAR e PolSAR são intensas na área militar, porém existe um espectro de aplicações amplo, principalmente na iniciativa privada.  Devido a consolidação do satélite SEASAT podemos usar as imagens para o estudo de diversas áreas, como:
\begin{itemize}
\item sensoriamento remoto;
\item topografia;
\item oceanografia;
\item glaciologia;
\item agricultura;
\item geologia;
\item florestas;
\item alvos fixos ou em movimento;
\item monitoramento ambiental;
\item controle de derramamento de petróleo;
\item e no auxílio de sistemas óticos.
\end{itemize}
\begin{table}[hbt]
	\centering
	\caption{Espectro eletromagnético para a faixa de micro-ondas.}\label{cap_acf_tab03}
\begin{tabular}{@{}llc@{}} \toprule
	banda & Frequência $f$(\si{Ghz}) & Freq.$\times$C. de onda $\lambda(cm)$. \\\midrule
	$P$&$(<0.39, 0.39)$  & $0.3\times 100.0$  \\ 
	$L$&$(0.39-1.55)$  &  $1.0\times 30.0$\\ 
	$S$&$(1.55-3.90)$  &  $3.0\times 10.0$\\ 
	$C$&$(3.90-5.75)$  & $\sim(4.0\times 7.0)$ \\ 
	$X$&$(5.75-10.9)$  & $10.0\times 3.0$ \\ 
	$K$&$(10.9-36.0)$  & $30.0\times 1.0$ \\ 
	$Q$&$(36.0-46.0)$  & $\sim(40.0\times 0.8 )$ \\ 
	$V$&$(46.0-56.0)$  & $\sim(50.0\times 0.6)$ \\ 
	$W$&$(56.0- >56.0)$  & $100.0\times 0.3$ \\ \bottomrule 
\end{tabular}
\end{table}
%

Uma imagem PolSAR pode ser gerada considerando a existência de uma matriz de covariância $\Sigma_{3\times3}$ hermitiana, proveniente do processo de modelagem do sistema PolSAR. A matriz de covariância tem nas entradas da diagonal principal, valores reais adquiridos respectivamente nas polarizações $HH$, $HV$ e $VV$, as outras entradas são números complexos dispostos de maneira que respeite o fato da matriz ser hermitiana. A maneira como a matriz hermitiana é distribuída pode ser analisada na tabela (\ref{cap_acf_tab04}).
\begin{table}[hbt]
	\centering
	\caption{Parâmetros estatísticos da matriz de variância.}\label{cap_acf_tab04}
\begin{tabular}{@{}lccc@{}} \toprule
	Polarização & $HH$  & $HV$ & $VV$ \\ \midrule
	$HH$ & $\sigma_{hh}$ & $\sigma_{hhhv} + \bar{\sigma}_{hhhv}\vec{\jmath}$  & $\sigma_{hhvv} + \bar{\sigma}_{hhvv}\vec{\jmath}$\\ 
	$HV$ & &$\sigma_{HV}$ & $\sigma_{hvvv} + \bar{\sigma}_{hvvv}\vec{\jmath}$\\ 
	$VV$ & & &$\sigma_{VV}$ \\ \bottomrule 
\end{tabular}
\end{table}

Para gerar a imagem PolSAR é usado uma matriz tridimensional, onde as primeiras duas indexações da matriz armazenam os valores para o azimute e a distância, de acordo com a resolução do sistema PolSAR. Na terceira indexação, a qual podemos chamar de canal, é armazenado os valores aproximados da  matriz de variância, fixando arbitrariamente um pixel. Os canais estão dispostos conforma tabela (\ref{cap_acf_tab05}),
\begin{table}[hbt]
	\footnotesize
	\centering
	\caption{Ordem de armazenamento para os canais com um pixel fixo.}\label{cap_acf_tab05}
\begin{tabular}{@{}lcccccccc@{}} \toprule
	 $HH$ &$HV$&$VV$ &$HHHV(\mathbf{Re})$ &$HHHV(\mathbf{Im})$&$HHVV(\mathbf{Re})$&$HHVV(\mathbf{Im})$ &$HVVV(\mathbf{Re})$&  $HVVV(\mathbf{Im})$ \\ \midrule
	$\overline{\sigma_{hh}}$&$\overline{\sigma_{hv}}$&$\overline{\sigma_{vv}}$&$\overline{\sigma_{hhhv}}$&$\overline{\bar{\sigma}_{hhhv}}$&$\overline{\sigma_{hhvv}}$&$\overline{\bar{\sigma}_{hhvv}}$&$\overline{\sigma_{hvvv}}$&$\overline{\bar{\sigma}_{hvvv}}$\\ \bottomrule
\end{tabular}
\end{table}

Em imagens sintéticas podemos encontrar a aproximação $\overline{\sigma_{ij}}$, com $i$, $j$ $\in \{HH,HV,VV\}$ usando o método de monte carlo com o auxílio da matriz de covariância $\Sigma$. Em imagens reais teremos que analisar $\overline{\sigma_{ij}}$, dependendo da técnica utilizada para definição dos parâmetros estatísticos nas regiões de interesse, podemos citar os artigos \citep{gmbf} e \citep{nhfc} para analisar os diferentes métodos, reforçando que a descrição do método proposto no segundo artigo será aplicado ao longo deste trabalho.

Um exemplo clássico de imagem PolSAR é da baía de São Francisco (EUA), com suas respectivas polarizações em tons de cinza, mostradas na figura (\ref{cap_acf_sf_hh_hv_vv}), 
\begin{figure}[hbt]
\minipage{0.35\textwidth}
  \includegraphics[width=\linewidth]{sf_hh.pdf}
\endminipage
\minipage{0.35\textwidth}
	\includegraphics[width=\linewidth]{sf_vh.pdf}
\endminipage
\centering
\minipage{0.35\textwidth}
	\includegraphics[width=\linewidth]{sf_vv.pdf}
\endminipage
        \vspace{-2.0cm}
	\caption{Imagem PolSAR com polarizações $HH$, $HV$ e $VV$.}\label{cap_acf_sf_hh_hv_vv}
\end{figure}

A visualização usando a decomposição RBG é mostrado na figura (\ref{cap_acf_sf_pauli}), sendo a maneira clássica como é conhecida a imagem da baía de São Franscisco. As figuras (\ref{cap_acf_sf_hh_blue}), (\ref{cap_acf_sf_hv_green}) e (\ref{cap_acf_sf_vv_red}) são respectivamente a decomposição RBG para cada canal $HH$ (azul), $HV$ (verde) e $VV$ (vermelho) das imagens PolSAR. 

\begin{figure}[hbt]
\begin{minipage}[b]{0.450\linewidth}
\includegraphics[width=\linewidth]{polsar_teste.pdf}
\caption{Baía de São Francisco.}
\label{cap_acf_sf_pauli}
\end{minipage}\hfill
\begin{minipage}[b]{0.450\linewidth}
\includegraphics[width=\linewidth]{polsar_blue.pdf}
\caption{Polarização $HH$.}
\label{cap_acf_sf_hh_blue}
\end{minipage}
\end{figure}
%
\begin{figure}[hbt]
\begin{minipage}[b]{0.450\linewidth}
\includegraphics[width=\linewidth]{polsar_green.pdf}
\caption{Polarização $HV$.}
\label{cap_acf_sf_hv_green}
\end{minipage}\hfill
\begin{minipage}[b]{0.450\linewidth}
\includegraphics[width=\linewidth]{polsar_red.pdf}
\caption{Polarização $VV$.}
\label{cap_acf_sf_vv_red}
\end{minipage}
\end{figure}

Atualmente, podemos indicar tendências de aplicações de imagens SAR e PolSAR.  Existe interesse em desenvolver um sistema tipo SAR, com comprimentos de ondas óticos que podem ter uma resolução de $10\times 10$ centímetros quadrados, um exemplo é o  sistema Lynx projetado pelo laboratório nacional Sandia, o qual alcança resoluções de $10$ a $30$ centímetros quadrados. Existe ainda o mini-SAR para uso em veículos aéreos não tripulados que focam na evolução da micro-eletrônica para aumentar a eficiência e diminuir o peso. 

As imagens SAR podem ser captadas por satélites orbitando outros planetas como o projeto Magellan SAR orbitando Vênus.  Outra tecnologia usada para a captação  imagens SAR, chamada de interferométrica SAR (InSAR), que usa dois ou mais radares de abertura sintética para obter imagens.   
 
Os sistemas SAR e PolSAR apresentam algumas características inerentes do processo teórico e tecnológico que poderíamos destacar como desvantagens:
\begin{itemize}
\item requer o conhecimento da rota do radar;
\item o sistema SAR é sensível ao movimento do alvo;
\item o processamento para a geração de uma imagem é complexo.
\end{itemize}

Entretanto essas desvantagens nos sistemas SAR e PolSAR não evitam as mesmas de serem largamente empregadas, tornando a área do conhecimento muito ativa, gerando um grande interesse, tanto em nível de aplicação como em pesquisa científica.

               % AAB - Arquivo inserido
\chapter{Aspectos gerais}\label{cap_asp_gerais}

Neste capítulo serão apresentadas algumas informações relevantes para situar e contextualizar a área de pesquisa desta tese. 
Serão abordados alguns assuntos introdutórios, traçando um breve panorama do desenvolvimento e aprimoramento dos radares RAR (\textit{Real Aperture Radar}),
SAR e PolSAR. 
Apresentamos também as características gerais e específicas dos radares, bem como informações de como são gerados e armazenados os dados provenientes da captação das imagens SAR e PolSAR. 

\section{Imagens SAR e PolSAR}

O SAR é o desenvolvimento tecnológico do RAR, o qual, de maneira geral, trabalha com sensores que transmitem sinais de micro-ondas e depois registram os sinais de retorno.

%%% ACF Divida este parágrafo em pelo menos duas frases bem formadas e coerentes
%%% AAB realizado
Os radares RAR para alcançar altas resoluções necessitam de grandes comprimentos de antenas o que inviabiliza o  seu uso em satélites e em aviões. Fato que configurou grande restrição para a área de sensoriamento remoto até os anos 50.
	
Na década de 50, pesquisadores fizeram avanços significativos para resolver as limi\-ta\-ções tecnológicas dos radares e desenvolveram uma técnica capaz de sintetizar o efeito de uma antena muito longa, em uma antena de tamanho viável, denominada SAR. 
Essa descoberta foi atribuída a Carl Wiley por volta de 1951 e, em 1954, foi registrada a patente do sistema SAR. 
Além disso, um sistema SAR foi operacionalizado em torno de 1958, impulsionando fortemente essa área de pesquisa. 

Os dados provenientes do SAR são gravados como uma faixa de posições para cada tempo avançado na direção da rota (azimute). 
No início do desenvolvimento dos sistemas SAR, o armazenamento de dados foi um grande problema, a solução foi colocar acoplado aos radares um sistema ótico para armazenar dados em filmes fotográficos. 
Atualmente, devido ao desenvolvimento da eletrônica, o problema de armazenamento de dados foi atenuado.

Nos sistemas SAR os dados das micro-ondas de retorno armazenados são representadas como imagens. 
Estas imagens são geradas de forma que em cada ponto da direção azimute, o radar envia um impulso em direção ao alvo, e depois recebe e armazena o sinal de retorno, que depende do espalhamento do sinal pelo alvo. 
Portanto, durante o tempo de aquisição das imagens são armazenadas as informações em uma linha de dados, gerando um mapeamento azimute \textit{versus} distância lateral das micro-ondas de retorno.

Essa estrutura de dados é típica de armazenamento em matrizes com as dimensões definidas pela resolução característica dos radares. 
As dimensões das imagens são, respectivamente, as resoluções na direção do azimute, dada pela rota do radar e pela direção perpendicular, definida como distância lateral. 
Desta forma, os radares geram imagens bidimensionais com resoluções distintas, dependendo do tipo de radar e da tecnologia empregada.

Neste trabalho usamos a generalização do sistema SAR conhecido como sistema polarimétrico SAR (PolSAR). 
O sistema PolSAR é  definido como a ciência de adquirir, processar e analisar o estado da polarização nas imagens de radar de abertura sintética. 

As imagens geradas por esses radares são formadas por micro-ondas de retorno, combinando as polarizações. 
O processo de polarização fornece mais informações sobre o alvo comparados com as imagens SAR. 
As imagens PolSAR têm o objetivo de melhorar o entendimento do efeito do espalhamento das micro-ondas pelos alvos, considerando as diferentes polarizações.

Os símbolos hh, hv, vh, e vv, representam as polarizações disponíveis nos radares, onde a primeira letra é a maneira como a onda é emitida e, a segunda letra é a maneira como a onda é recebida.

\section{Características gerais dos Radares SAR e PolSAR}

Os radares que têm a característica de emitir as micro-ondas são definidos como radares ativos e, se os radares transmitem e recebem os sinais com a mesma antena, são denominadas mono-estáticos. 
Importantes informações sobre radares podem ser encontradas na referência \citet{lp}.

O desenvolvimento dos radares pode ser resumidos pelos princípios a seguir:
\begin{itemize}
\item Possibilidade de uma antena transmitir micro-ondas em uma direção precisa; 
\item Possibilidade de detectar com grande precisão o sinal de retorno. Após a interação das micro-ondas com o alvo acontece o espalhamento das micro-ondas refletidas em todas as  direções. Assim o radar pode não captar alguns sinais de retorno. 
\item Capacidade de medir o tempo entre a transmissão  e a recepção das micro-ondas, consequentemente calcular a distância entre o alvo e o radar;
\item Habilidade de detectar vários alvos em grandes áreas percorridas.
\end{itemize}

As micro-ondas podem variar em comprimento e amplitude, dependendo da construção dos sensores para o radar em operação.
Uma característica importante dessas ondas é a capacidade de penetração no alvo analisado. 

Os radares SAR e PolSAR justificam seu uso por possuírem algumas características operacionais que podem ser resumidas nos seguintes itens:
\begin{itemize}
\item Podem estar em plataformas elevadas, aeronaves tripuladas ou não, satélites orbitando a terra ou outros planetas;
\item É uma técnica de produção de imagem viável e prática;
\item Podem gerar imagens de alta resolução espacial;
%\item Sintetiza longas aberturas de antenas; 
%%% ACF Me parece irrelevante, e confuso, dado o item anterior
%%% AAB Retirei concordo
\item Os radares produzem imagens dia e noite;
\item As condições atmosféricas interferem muito pouco na captação de imagens.
\end{itemize}

Na referência \cite{lkc} encontramos um amplo espectro de aplicações das imagens SAR e PolSAR na área de sensoriamento remoto.  Podemos listar algumas dessas aplicações: 
%%% ACF Seria interessante aqui citar um artigo tipo "survey" de aplicações de sensoriamento remoto com microondas
%%% AAB Encontro o livro que citei, achei interessante, mudei aslguns tópicos.
\begin{itemize}
\item Mapeamento do solo ou cobertura do solo;
\item Mapeamento Geológico;
\item Oceanografia;
\item Glaciologia;
\item Aplicações agrícola;
\item Acompanhamento de florestas;
\item Monitoramento ambiental;
\item Monitoramento de desastres naturais;
\item Controle de derramamento de petróleo;
\item Aplicações de recursos hídricos;
\item Aplicações da ecologia da vida selvagem;
\item Aplicações arqueológicas;  
\item E aplicações conjuntas com sistemas óticos.
\end{itemize}
\section{Características específicas dos radares SAR e PolSAR}
Os radares SAR e PolSAR trabalham no espectro das micro-ondas. 
A tabela~\ref{tab:micro_onda_espectro} definida no livro \cite{lp} mostra na primeira coluna a representação da banda, na segunda coluna das faixas de frequências que os radares são construídos. Na terceira coluna foi fixada uma frequência (f) para encontrar o comprimento de onda ($\lambda$) e mostrar na quarta coluna. Lembrando que o produto da frequência pelo comprimento de onda encontra a velocidade.   
\begin{table}[hbt] %%% ACF Leia a documentação do pacote siunitx e veja como digitar intervalos; você digitou diferenças
%%% AAB referenciei a fonte da tabela e modifiquei
	\centering
	\caption{Espectro eletromagnético para a faixa de micro-ondas.}\label{tab:micro_onda_espectro}
\begin{tabular}{@{}cccc@{}} \toprule
	banda & Frequência - f [\si{\giga\hertz}]     & f fixada& $\lambda$ [\si{\centi\meter}]. \\\midrule
	P&\numrange[range-phrase = --]{< 0.39}{ 0.39} &\num{0.3}& \num{100.0}  \\ 
	L&\numrange[range-phrase = --]{0.39}{1.55}    &\num{1.0}  & \num{30.0}\\ 
	S&\numrange[range-phrase = --]{1.55}{3.90}    &\num{3.0}  & \num{10.0}\\ 
	C&\numrange[range-phrase = --]{3.90}{5.75}    &\num{4.0}  & \num{7.5} \\ 
	X&\numrange[range-phrase = --]{5.75}{10.9}    &\num{10.0} & \num{3.0} \\ 
	K&\numrange[range-phrase = --]{10.9}{36.0}    &\num{30.0} & \num{1.0} \\ 
	Q&\numrange[range-phrase = --]{36.0}{46.0}    &\num{40.0} & \num{0.75} \\ 
	V&\numrange[range-phrase = --]{46.0}{56.0}    &\num{50.0} & \num{0.6} \\ 
	W&\numrange[range-phrase = --]{56.0}{> 56.0}  &\num{100.0}& \num{0.3} \\ \bottomrule 
\end{tabular}
\end{table}\index{Espectro Eletromagnético}

A descoberta da tecnologia SAR impulsionou um grande desenvolvimento da área, levando à construção do SEASAT, o primeiro SAR comercial projetado. 
Este projeto foi muito bem sucedido e estabeleceu definitivamente os sistemas SAR e PolSAR como área de pesquisa. 

%%% ACF Tabelas se referenciam com \ref; \eqref é apenas para equações
%%% AAB obrigado - ok
O lançamento do satélite foi em junho de 1978, a Tabela~\ref{tab:carac_seasat} mostra algumas características do SEASAT.
Para mais informações pode-se consultar o seguinte sítio eletrônico, \url{https://earth.esa.int/web/eoportal/satellite-missions/s/seasat}.  
\begin{table}[hbt]
	\centering
	\caption{Características do satélite SEASAT (SAR).}\label{tab:carac_seasat}
\begin{tabular}{@{}llr@{}} \toprule
	Características específicas& Valores operacionais  \\ \midrule
	Frequência           & \SI{1.275}{\GHz}  \\ 
	Altitude             & \SI{780}{\km}   \\
	Peso                 & \SI{2300}{\kilogram}   \\
	Ângulo de inclinação & \SI{\sim 23}{\degree}   \\
	Distância lateral    & \SI{100}{\km}  \\
	Largura de banda     & \SI{19}{\MHz}   \\
	Banda L              & \SI{23.5}{\cm} de comprimento de onda\\
	Polarização          & hh - Onda emitida e recebida na direção horizontal \\
	%%% ACF Por favor use os comandos nativos de siunitx; veja como escrever áreas
	%%% AAB - muito melhor
	Resolução            & \SI[product-units = brackets-power]{25 x 25}{\metre}  \\ \bottomrule
\end{tabular}
\end{table}

Outros projetos de sistema foram realizados e podem ser vistos na tabela \ref{tab:carac_radares}.% Neste trabalho serão usadas imagens capturadas pelo radar AIRSAR, cuja características estão na Tabela~\ref{tab:carac_radares}.

\begin{sidewaystable}
\footnotesize
	\centering
	\caption{Características operacionais dos satélites SAR ou PolSAR.}\label{tab:carac_radares}
\begin{tabular}{@{}llllllllr@{}} \toprule
Satélites      & 	SEASAT  &AIRSAR &SIR-C& Almaz&ERS-2& JERS-1& RADSAT-1&RADSAT-2 \\ \midrule
Nacionalidade       &EUA    &EUA&Alamanha-Itália&Rússia&Europa&Japão&Canadá&Canadá  \\ 
Lançamento          &\num{1978}   &\num{1988}   &\num{1990}  &\num{1992}  &\num{1995} &\num{1998}  &\num{1995}  & \num{2003}\\
Banda (\si{\cm})    &L      &P-L-C       &L-C-X &S     &C    &L     &C     & C \\
Polarização         &hh&hh-hv-vv&hh-hv-vv&hh&vv&hh&hh&hh-hv-vv\\
Ângulo de incidência&\num{23}&\numrange[range-phrase = --]{20}{60}&\numrange[range-phrase = --]{15}{55}&\numrange[range-phrase = --]{30}{60}&\num{23}&\num{35}&\numrange[range-phrase = --]{20}{59}&\numrange[range-phrase = --]{20}{60}\\
Distância (\si{\km})           &\num{100}&\numrange[range-phrase = --]{10}{17}&\numrange[range-phrase = --]{15}{90}&\num{350}&\num{100}&\num{75}&\numrange[range-phrase = --]{50}{500}&\numrange[range-phrase = --]{10}{500}\\
Resolução (\si{\m\squared})         &25&\numrange[range-phrase = --]{2}{8}&\numrange[range-phrase = --]{10}{60}&\numrange[range-phrase = --]{10}{30}&30&18&\numrange[range-phrase = --]{10}{100}&\numrange[range-phrase = --]{3}{100}\\ \bottomrule
\end{tabular}
\end{sidewaystable}
\section{Estrutura de dados para imagens PolSAR}
O sistema PolSAR armazena informações de retorno em uma matriz $\mathbf{S}_{2\times 2}$ para cada ponto de sua região de observação, onde as entradas da matriz são números complexos. Se o sistema é mono-estático, a matriz torna-se hermitiana (a matriz é igual a sua transposta conjugada), por esse motivo, podemos representar a matriz na forma de um vetor $\mathbf{s}$ de dimensão 3. O produto entre o vetor e seu hermitiano gera uma matriz hermitiana $3\times 3$ com suas entrada mostrada na Tabela \ref{tab:sistema_polsar}.
%%% ACF Precisa explicar o que é cada entrada da tabela, em palavras
%%% AAB Realizado
\begin{table}[hbt]
	\centering
	\caption{Informações do sistema PolSAR.}\label{tab:sistema_polsar}
\begin{tabular}{@{}lccc@{}} \toprule
	Polarização & hh  & hv & vv \\ \midrule
	hh & $\sigma_\text{hh}$ & $\sigma_\text{hhhv} + \hat{\sigma}_\text{hhhv}\hat{\jmath}$  & $\sigma_\text{hhvv} + \hat{\sigma}_\text{hhvv}\hat{\jmath}$\\ 
	hv &- &$\sigma_\text{hv}$ & $\sigma_\text{hvvv} + \hat{\sigma}_\text{hvvv}\hat{\jmath}$\\ 
	vv &- & -&$\sigma_\text{vv}$ \\ \bottomrule 
\end{tabular}
\end{table}

A estrutura de dados para receber as informações da tabela \ref{tab:sistema_polsar} pode ser uma matriz tridimensional, onde os dois primeiros índices da matriz localizam o pixel da imagem PolSAR correspondente ao azimute e a distância lateral. A resolução do sistema PolSAR define as dimensões na direção do azimute e da distância lateral. O terceiro índice da matriz tridimensional é definido como canal.

Para cada pixel fixo da imagem, os canais recebem os valores mostrados na Tabela~\ref{tab:canais}. O simbolo $C_k$ com $k=1,\dots,9$ define os canais da imagem.
%%% ACF \mathbb R e \mathbb I são conjuntos, não operadores. Você se refere a \Re e a \Im. Revise tudo e corrija onde precisar 
%%% AAB  realizado
\begin{table}[hbt]	
	\centering
	\caption{Ordem de armazenamento para os canais da imagem polSAR.}\label{tab:canais}
\begin{tabular}{@{}lcccccccc@{}} \toprule
	 $\text{C}_1$ &$\text{C}_2$&$\text{C}_3$ &$\text{C}_4$&$\text{C}_5$&$\text{C}_6$&$\text{C}_7$&$\text{C}_8$&$\text{C}_9$ \\ \midrule
	$\sigma_\text{hh}$&$\sigma_\text{hv}$&$\sigma_\text{vv}$&$\sigma_\text{hhhv}$&$\hat{\sigma}_\text{hhhv}$&$\sigma_\text{hhvv}$&$\hat{\sigma}_\text{hhvv}$&$\sigma_\text{hvvv}$&$\hat{\sigma}_\text{hvvv}$\\ \bottomrule
\end{tabular}
\end{table}

Neste trabalho serão usados os três primeiros canais da matriz que armazena os dados da imagem PolSAR, conhecidos como canais de intensidades. Os sensores AIRSAR, SIR-C, e RADSAT-2 fornecem informações deste tipo, mais características desses sensores podem ser encontrados na Tabela \label{tab:carac_radares}.
%%% ACF Aqui mencione quais sensores fornecem esse tipo de dados, ou dados em dois canais de intensidade
%%% AAB realiazado
%%% ACF Figuras são referenciadas com \ref, não com \eqref. Revise e corrija o texto todo
%%% AAB ok

Um exemplo de dados PolSAR é a imagem da baía de São Francisco (EUA), com suas respectivas polarizações em tons de cinza, mostradas na figura~\ref{fig:sf_hh_hv_vv}. A imagem foi captada pelo sensor aerotransportado AIRSAR polarimétrico na banda L. O AIRSAR apresenta resolução de \numrange[range-phrase = --]{2}{8} \si{\m\squared}.   
%%% ACF Precisa informar o sensor, banda, resolução espacial. Idealmente, acompanhe com uma imagem do Google Maps da mesma área.
%%% AAB Realizado
\begin{figure}[hbt]
\minipage{0.35\textwidth}
  \includegraphics[width=\linewidth]{sf_hh.pdf}
\endminipage
\minipage{0.35\textwidth}
	\includegraphics[width=\linewidth]{sf_vh.pdf}
\endminipage
\centering
\minipage{0.35\textwidth}
	\includegraphics[width=\linewidth]{sf_vv.pdf}
\endminipage
        \vspace{-2.0cm}
	\caption{Imagem PolSAR com polarizações hh, hv e vv.}\label{fig:sf_hh_hv_vv}
\end{figure}

A região de São Francisco capturada por um satélite ótico através do sítio eletrônico \url{https://www.google.com/maps} é mostrada na Figura \ref{fig:otica_google}.
\begin{figure}[hbt!]
	\centering
	\includegraphics[width=.5\linewidth]{san_francisco_google_maps_crop}%
	\caption{Imagem da região de São Francisco capturada por satélite}
\label{fig:otica_google}
\end{figure}

A visualização da região da baía de São Franscisco, usando a decomposição de Pauli (RBG), que será abordada no próximo capítulo,  pode ser vista na figura~\ref{cap_acf_sf_pauli}. Esta imagem também será usada para a realização do processamento de dados com o objetivo de testar os métodos propostos neste trabalho.

As Figuras~\ref{cap_acf_sf_hh_blue}, \ref{cap_acf_sf_hv_green} e~\ref{cap_acf_sf_vv_red} mostram, respectivamente, a decomposição RBG para cada canal, azul, verde e vermelho das imagens de PolSAR da baía de São Francisco. 

\begin{figure}[hbt]
\begin{minipage}[b]{0.450\linewidth}
\includegraphics[width=\linewidth]{polsar_teste.pdf}
\caption{Baía de São Francisco.}
\label{cap_acf_sf_pauli}
\end{minipage}\hfill
\begin{minipage}[b]{0.450\linewidth}
\includegraphics[width=\linewidth]{polsar_blue.pdf}
\caption{Polarização vv.}
\label{cap_acf_sf_hh_blue}
\end{minipage}
\end{figure}
%
\begin{figure}[hbt]
\begin{minipage}[b]{0.450\linewidth}
\includegraphics[width=\linewidth]{polsar_green.pdf}
\caption{Polarização hv.}
\label{cap_acf_sf_hv_green}
\end{minipage}\hfill
\begin{minipage}[b]{0.450\linewidth}
\includegraphics[width=\linewidth]{polsar_red.pdf}
\caption{Polarização hh.}
\label{cap_acf_sf_vv_red}
\end{minipage}
\end{figure}

Atualmente, os sistemas SAR e PolSAR são uma área de pesquisa muito ativa. 
Entre os interesses atuais, está o desenvolvimento de sistemas tipo SAR e PolSAR com alta resolução (VHR).
Um exemplo disso é o sistema Lynx, projetado pelo laboratório nacional Sandia, o qual alcança resoluções de $10$ a $30$ centímetros quadrados. O sitio eletrônico \url{https://www.sandia.gov/media/NewsRel/NR1999/Lynx.htm} fornece as informações da Tabela \ref{tab:carac_lynx} sobre o projeto lynx:
%%% ACF Fornecer referência ou link
\begin{table}[hbt]
	\centering
	\caption{Características do sistema  Lynx (SAR).}\label{tab:carac_lynx}
\begin{tabular}{@{}llr@{}} \toprule
	Características específicas& Valores operacionais  \\ \midrule
	Frequência           &  \numrange[range-phrase = --]{15.2}{18.2} \si{\GHz}\\ 
	Altitude             & \SI{7}{\km}   \\
	Peso                 & \SI{57}{\kilogram}   \\
	Ângulo de inclinação & \numrange[range-phrase = --]{45}{135}\si{\degree}   \\
	Distância lateral    & \numrange[range-phrase = --]{7}{30} \si{\km}  \\
	Banda                & $\text{K}_u$\\
	Polarização          & vv - Onda emitida e recebida na direção vertical \\
	Resolução inferior   & \SI[product-units = brackets-power]{10 x 10}{\cm}  \\
	Resolução superior   & \SI[product-units = brackets-power]{30 x 30}{\cm} \\ \bottomrule
\end{tabular}
\end{table}

O Lynx pode gerar imagem com alta resolução portanto com grandes dimensões, por exemplo, a imagem da Figura \ref{fig:sar_lynk} mostrando a região de Rio Grande Valley tem dimensão de $1068\times 2600$ pixeis e resolução de aproximadamente \SI[product-units = brackets-power]{30 x 30}{\cm}.
\begin{figure}[hbt!]
	\centering
	\includegraphics[width=.5\linewidth]{stripmap_lynx_sar_pdf}%
	\caption{Imagem da região do Rio Grande Valley}
\label{fig:sar_lynk}
\end{figure}

Outro exemplo, pode ser visto no artigo \cite{rijpbnm}, que mostra um sistema PolSAR que produzem imagens com resolução de \SI[product-units = brackets-power]{25 x 25}{\cm},
%%% ACF Use siunitx
%%% AAB realizado
conhecido como F-SAR, desenvolvido pelo centro aeroespacial da Alemanha (DLR). 
Esse sistema pode adquirir, simultaneamente, dados nas bandas X, C, S, L e P, gerando em torno de \SI{20}{\giga\byte\per\minute} %%% ACF Use siunitx %%%AAB Realizado 
de dados.    

%%% ACF Sempre detalhe o significado de uma sigla como SAOCOM
%%% AAB Realizado
Destacamos ainda o projeto \textit{SAtélite Argentino de Observación COn Microondas} (SAOCOM) pela atualidade e importância para o monitoramento da América Latina, informações podem ser obtidas no sítio web \url{https://directory.eoportal.org/web/eoportal/satellite-missions/s/saocom}. 
O SAOCOM-1B segundo satélite produzido pelo projeto foi colocado em órbita em 2020 conforme \url{https://www.space.com/spacex-saocom-1b-launch-rocket-landing-success.html}.

Como curiosidades, podemos citar: 
\begin{itemize}
\item O projeto de um mini-SAR para uso em veículos aéreos não tripulados que focam na evolução da micro-eletrônica para aumentar a eficiência e diminuir o peso, por exemplo, o projeto Lynx; %%% ACF Referência ou link %%% AAB realizado
\item As imagens SAR podem ser captadas por satélites, orbitando outros planetas, como o projeto Magellan SAR que orbita Vênus, o satélite foi lançado em 1989 com duração prevista de 5 anos. A figura \ref{fig:sar_magellan} mostra o planeta Vênus captura pelo satélite Magellan, mais informações sobre o projeto poder ser encontradas no Sítio eletrônico \url{https://www2.jpl.nasa.gov/magellan/};
  %%% ACF Referência ou link; orbita ainda?
  %%% AAB realizado
\item Outra tecnologia usada para a captação de imagens SAR denominada de interferométrica SAR (InSAR), usa dois ou mais radares de abertura sintética para obter imagens, podendo ser em posições diferentes ou tempos distintos. O sensor InSAR possibilita medir deslocamentos de objetos de interesse em uma região definida da imagem. Um tutorial sobre os sensores InSAR podem ser encontrado no Sítio eletrônico \url{https://www.esa.int/About_Us/ESA_Publications/InSAR_Principles_Guidelines_for_SAR_Interferometry_Processing_and_Interpretation_br_ESA_TM-19}.
 %%% ACF Referência ou link, e detalhe mais o que é o InSAR e para que serve
 %%% AAB Realizado
\end{itemize}

Os sistemas SAR e PolSAR apresentam algumas características inerentes à física do imageamento e à engenharia da sua implementação. 
E  podemos destacar como desvantagens:
\begin{itemize}
\item O sistema requer o conhecimento da rota do radar;
\item O sistema é sensível ao movimento do alvo;
\item O processamento para a geração de uma imagem é complexo.
\end{itemize}

\begin{figure}[hbt!]
	\centering
	\includegraphics[width=.5\linewidth]{venpole_small}%
	\caption{Imagem do planeta Vênus capturada pelo satélite Magellan}
\label{fig:sar_magellan}
\end{figure}

Entretanto, as desvantagens nos sistemas SAR e PolSAR não evitam os mesmos de serem largamente empregadas, o que impulsiona a produção de conhecimento nessa área, gerando um grande interesse, tanto em nível de aplicações, como em pesquisas científicas.           % AAB - Arquivo inserido
\chapter{Metodologia}\label{metodologia}

\section{Modelagem estatística para dados PolSAR}\label{cap_acf_sec1}
Os sistemas SAR totalmente polarimétricos transmitem pulsos de micro-ondas polarizados ortogonalmente e medem componentes ortogonais do sinal recebido. Para cada pixel, a medida resulta em uma matriz de coeficientes de espalhamento. Esses coeficientes são números complexos que descrevem no sistema SAR a transformação do campo eletromagnético transmitido para o campo eletromagnético recebido.

A transformação pode ser representada como
\begin{equation*}
 \left[
\begin{array}{c}
	E_\text{h}^\text{r}   \\
	E_\text{v}^\text{r}    \\
\end{array}
\right]
 = \frac{e^{\hat{\imath} \text{kd}}}{\text{d}}\left[
\begin{array}{cc}
	S_{\text{hh}}   & S_{\text{hv}}   \\
	S_{\text{vh}}   & S_{\text{vv}}   \\
\end{array}
\right]
 \left[
\begin{array}{c}
	E_\text{h}^\text{t}   \\
	E_\text{v}^\text{t}   \\
\end{array}
\right],
\end{equation*}
onde k denota o número de onda, $\hat{\imath}$ é um número complexo e d é a distância entre o radar e o alvo. No campo eletromagnético com componentes $E_\text{i}^\text{j}$, o índice subscrito denota polarização horizontal (h) ou vertical (v), e o índice sobrescrito indica a onda recebida (r) ou transmitida (t). 

A matriz de espalhamento complexa $\mathbf{S}$ é definida por
\begin{equation}\label{matriz_de_espalhamento}
\mathbf{S} = \left[
\begin{array}{cc}
	S_{\text{hh}}   & S_{\text{hv}}   \\
	S_{\text{vh}}   & S_{\text{vv}}   \\
\end{array}
\right],
\end{equation}
onde as entradas da matriz $S_{\text{i,j}}$ são os coeficientes de espalhamento complexo, tal que os índices i e j são associados ao  recebimento e a transmissão das ondas, por exemplo, o coeficiente de espalhamento $S_{\text{hv}}$ está associado a onda transmitida na direção vertical (v) e recebida na direção horizontal (h).

Definindo a diagonal principal da matriz de espalhamento por co-polarização pois relaciona a polarização das ondas transmitidas e recebidas nas mesmas direções, e a os elementos da diagonal secundária da matriz de espalhamento por polarização cruzada relacionando os estados de polarizações ortogonais.
% ver~\citep{lp}.
 
A matriz $\mathbf{S}$ depende da definição do sistema de coordenadas, se a antena transmissora e receptora de sinal estão localizadas na mesma posição consideramos as medidas mono estáticas e consideramos o sistema de coordenada \textbf{BSA} - \textit{Back Scattering Alignment}, desta forma o sistema de coordenadas da transmissão e recepção de sinal são coincidentes.   
 
A potência total espalhada no caso de um sistema de radar polarimétrico é o chamado \textit{span}, sendo definido no caso mais geral como,
\begin{equation}\label{span_geral}
\mathbf{Span(S)} = \traco(SS^H)=|S_{hh}|^2+|S_{hv}|^2|+|S_{vh}|^2+|S_{vv}|^2,
\end{equation}
onde o operador $\traco(\cdot)$ é o traço de uma matriz.


\subsection{Matriz de coerência polarimétrica de Pauli ($T_4$) e matriz de covariância lexicográfica ($C_4$)}
A matriz de espalhamento $\mathbf{S}$ pode ser representada pela construção do vetor,
\begin{equation}\label{def_vet_espalhamento}
\mathbf{k}=\frac{1}{2}\left[\traco(S\Psi_1)\quad\traco(S\Psi_2)\quad \traco(S\Psi_3)\quad \traco(S\Psi_4)\right]^T,
\end{equation}
onde $\{\Psi_i\}_{i=1}^4$ é uma base para o espaço das matrizes hermitianas $2\times 2$.

Diferentes bases para o mesmo espaço matricial podem ser definidas, no presente trabalho serão consideradas duas bases chamadas respectivamente de  base de Pauli e base lexicográfica, a  base de Pauli é definida por,
\begin{equation}\label{base_de_pauli}
\{\Psi_P\} = \left\{
\sqrt{2}\left[\begin{array}{cc}
	1  & 0  \\
	0  & 1 \\
\end{array}\right],
\sqrt{2}\left[\begin{array}{cc}
	1  & 0  \\
	0  & -1  \\
\end{array}\right],
\sqrt{2}\left[\begin{array}{cc}
	0  & 1  \\
	1  & 0  \\
\end{array}\right],
\sqrt{2}\left[\begin{array}{cc}
	0       & -i  \\
	i  & 0  \\
\end{array}\right]
\right\},
\end{equation}
e a base lexicográfica é definida como,
\begin{equation}\label{base_de_lexicografica}
\{\Psi_L\} = \left\{
2\left[\begin{array}{cc}
	1  & 0  \\
	0  & 0 \\
\end{array}\right],
2\left[\begin{array}{cc}
	0  & 1  \\
	0  & 0  \\
\end{array}\right],
2\left[\begin{array}{cc}
	0  & 0  \\
	1  & 0 \\
\end{array}\right],
2\left[\begin{array}{cc}
	0  & 0  \\
	0  & 1  \\
\end{array}\right]
\right\}.
\end{equation}

Usando as bases (\ref{base_de_pauli}), (\ref{base_de_lexicografica}) e  a definição do vetor (\ref{def_vet_espalhamento}) representamos a matriz de espalhamento pelo vetor característico de Pauli $4$-D,
\begin{equation}\label{vetor_pauli_4d}
\mathbf{k}= \frac{1}{\sqrt{2}}\left[
	\begin{array}{cccc}
	S_{hh} + S_{vv}& S_{hh} - S_{vv}& S_{hv} + S_{vh} &i (S_{hv} - S_{vh})   \\
\end{array}\right]^T=\frac{1}{\sqrt{2}}[k_1\quad k_2\quad k_3\quad k_4]
\end{equation}
e pelo vetor característico lexicográfico $4$-D 
\begin{equation}\label{vetor_lexicografico_4d}
\mathbf{\Omega}= \left[
	\begin{array}{cccc}
	S_{hh}& S_{hv} &S_{hv}& S_{vv}   \\
\end{array}\right]^T=[\Omega_1\quad \Omega_2\quad \Omega_3\quad \Omega_4]
\end{equation}

A matriz de espalhamento pode ser relacionada com os vetores (\ref{vetor_pauli_4d}) e (\ref{vetor_lexicografico_4d}) da seguinte maneira,
\begin{equation}\label{mat_esp_rel_pauli_lex}
\mathbf{S} = \left[
\begin{array}{cc}
	S_{hh}   & S_{hv}   \\
	S_{vh}   & S_{vv}   \\
\end{array}
\right]=
\left[
\begin{array}{cc}
	\Omega_1   & \Omega_2   \\
	\Omega_3   & \Omega_4   \\
\end{array}
\right]=\frac{1}{\sqrt{2}}
\left[
\begin{array}{cc}
	 k_1+k_2  & k_3-ik_4   \\
	 k_3+ik_4 & k_1-k_2   \\
\end{array}
\right]
\end{equation}

As constantes $2$ e $\sqrt{2}$ nas equações (\ref{base_de_pauli}) e (\ref{base_de_lexicografica}) servem para manter a norma dos vetores de espalhamento iguais independente da escolha das bases. O produto interno escolhido é o padrão para o espaço vetorial dos vetores complexos de dimensão 4.

Podemos assim garantir que a invariância da potência total 
\begin{equation}\label{span_invariante}
\begin{array}{ccc}
\mathbf{Span(S)} &=& \traco(SS^H)\\
	   &=&  \traco(SS^H)=|S_{hh}|^2+|S_{hv}|^2|+|S_{vh}|^2+|S_{vv}|^2  \\
	   &=&  \mathbf{k}^H\mathbf{k}=|\mathbf{k}|^2\\
	   &=& \mathbf{\Omega}^H\mathbf{\Omega}=|\mathbf{\Omega}|^2.
\end{array}
\end{equation}

A transformação linear unitária $U_{4(L \rightarrow P)}$ é definida como uma transformação que aplica o vetor na base lexicográfica em um vetor na base de Pauli. Definimos a notação \textrm{SU(4)} para designar a transformação  unitária
\begin{equation}\label{trans_matriz_unit_su4}
\frac{1}{\sqrt{2}}\left[
\begin{array}{c}
	  S_{hh} +  S_{vv}\\  
	  S_{hh} -  S_{vv}\\
	  S_{hv} +  S_{vh} \\
        i(S_{hv} -  S_{vh}) \\
\end{array}
\right]=\frac{1}{\sqrt{2}}	
\left[
\begin{array}{rrrr}
	1   & 0 & 0 & 1  \\
	1   & 0 & 0 & -1  \\
	0   & 1 & 1 & 0  \\
	0   & i & -i &0   \\
\end{array}
\right]
\left[
\begin{array}{c}
	S_{hh} \\  
	S_{hv} \\
	S_{vh} \\
	S_{vv} \\
\end{array}
\right]
\end{equation}
desta maneira definimos a matriz unitária,
\begin{equation}\label{matriz_unit_su4}
U_{4(L \rightarrow P)}=	\frac{1}{\sqrt{2}}	
\left[
\begin{array}{rrrr}
	1   & 0 & 0 & 1  \\
	1   & 0 & 0 & -1  \\
	0   & 1 & 1 & 0  \\
	0   & i & -i &0   \\
\end{array}
\right]
\end{equation}


Para o caso biestático definimos a matriz de coerência polarimétrica de Pauli 
\begin{equation}\label{matriz_polar_pauli}
	\mathbf{T}_4=\mathbf{k}\mathbf{k}^H=	
\left[
\begin{array}{rrrr}
	|k_1|^2       & k_1\bar{k}_2  & k_1\bar{k}_3  & k_1\bar{k}_4  \\
	k_2\bar{k}_1  & |k_2|^2       & k_2\bar{k}_3  & k_2\bar{k}_4  \\
	k_3\bar{k}_1  & k_3\bar{k}_2  &    |k_3|^2    & k_3\bar{k}_4  \\
	k_4\bar{k}_1  & k_4\bar{k}_2  & k_4\bar{k}_3  & |k_4|^2   \\
\end{array}
\right]
\end{equation}
e a matriz de covariância lexicográfica
\begin{equation}\label{matriz_covar_lexic}
	\mathbf{C}_4=\mathbf{\Omega}\mathbf{\Omega}^H=	
\left[
\begin{array}{rrrr}
	|\Omega_1|^2       & \Omega_1\bar{\Omega}_2  & \Omega_1\bar{\Omega}_3  & \Omega_1\bar{\Omega}_4  \\
	\Omega_2\bar{\Omega}_1  & |\Omega_2|^2       & \Omega_2\bar{\Omega}_3  & \omega_2\bar{\Omega}_4  \\
	\Omega_3\bar{\Omega}_1  & \Omega_3\bar{\Omega}_2  &    |\Omega_3|^2    & \Omega_3\bar{\Omega}_4  \\
	\Omega_4\bar{\Omega}_1  & \Omega_4\bar{\Omega}_2  & \Omega_4\bar{\Omega}_3  & |\Omega_4|^2   \\
\end{array}
\right]
\end{equation}

Usando as definções e as propriedades acima teremos
\begin{equation}\nonumber
\begin{array}{rrrrr}
	\mathbf{T}_4&=&\mathbf{k}\mathbf{k}^H&=&\mathbf{U_4\Omega}(\mathbf{U_4\Omega})^H	\\
	   &=&\mathbf{U_4\Omega}\mathbf{\Omega}^H\mathbf{U_4}^H&&	\\
	   &=&\mathbf{U_4}\mathbf{C}_4\mathbf{U_4}^H&=&\mathbf{U_4}\mathbf{C}_4\mathbf{U_4}^{-1}\\
\end{array}
\end{equation}
\begin{equation}\label{matriz_simil_t4_c4}
\begin{array}{rrr}
	 \mathbf{T}_4  &=&\mathbf{U_4}\mathbf{C}_4\mathbf{U_4}^{-1}\\
\end{array}
\end{equation}
com isso, podemos concluir que
\begin{equation}\label{traco_t4_c4}
\begin{array}{rrrrr}
	\traco({\mathbf{T}_4})  &=&\traco({\mathbf{C}_4})&=&\mathbf{Span(S)}.\\
\end{array}
\end{equation}

\subsection{Matriz de coerência polarimétrica de Pauli ($T_3$) e matriz de covariância lexicográfica ($C_3$)}

Podemos entender as interações da ondas eletromagnéticas em alvos naturais sob a ótica do teorema da reciprocidade que considera o meio reciproco, de uma maneira geral as propriedades de transmissão e recebimento de uma antena são idênticos. Então podemos definir a igualdade dos termos complexos (polarização cruzada) $S_{hv}=S_{vh}$.  Veja \citet{lp}. 

O entendimento para extrair informação da matriz de espalhamento $\mathbf{S}$ pode ser alcançado coma construção de um sistema de vetores. A matriz de espalhamento $\mathbf{S}$ pode ser representada pela construção do vetor,
\begin{equation}\label{def_vet_espalhamento_3d}
\mathbf{k}=\frac{1}{2}\left[\traco(S\Psi_1)\quad\traco(S\Psi_2)\quad \traco(S\Psi_3)\right]^T,
\end{equation}
onde $\{\Psi_i\}_{i=1}^3$ é uma base para o espaço das matrizes hermitianas $2\times 2$.

Neste trabalho serão consideradas duas bases para os espaço das matrizes nomeadas como base de Pauli e base lexicográfica.	

A base de Pauli pode ser definida como,
\begin{equation}\label{base_de_pauli_3d}
\{\Psi_P\} = \left\{
\sqrt{2}\left[\begin{array}{cc}
	1  & 0  \\
	0  & 1 \\
\end{array}\right],
\sqrt{2}\left[\begin{array}{cc}
	1  & 0  \\
	0  & -1  \\
\end{array}\right],
\sqrt{2}\left[\begin{array}{cc}
	0  & 1  \\
	1  & 0  \\
\end{array}\right]
\right\}.
\end{equation}

A base lexicográfica pode ser definida como,
\begin{equation}\label{base_de_lexicografica_3d}
\{\Psi_L\} = \left\{
2\left[\begin{array}{cc}
	1  & 0  \\
	0  & 0 \\
\end{array}\right],
2\sqrt{2}\left[\begin{array}{cc}
	0  & 1  \\
	0  & 0  \\
\end{array}\right],
2\left[\begin{array}{cc}
	0  & 0  \\
	0  & 1  \\
\end{array}\right]
\right\}.
\end{equation}

Usando as bases (\ref{base_de_pauli_3d}), (\ref{base_de_lexicografica_3d}) e a equação (\ref{def_vet_espalhamento_3d}) geramos os seguintes vetores de espalhamento que pode ser representada pelo vetor característico de Pauli $3$-D,
\begin{equation}\label{vetor_pauli_3d}
\mathbf{k}= \frac{1}{\sqrt{2}}\left[
	\begin{array}{ccc}
	S_{hh} + S_{vv} & S_{hh} - S_{vv}& 2S_{hv}   \\
\end{array}\right]^T=\frac{1}{\sqrt{2}}[k_1\quad k_2\quad k_3],
\end{equation}
e pelo vetor característico lexicográfico $3$-D 
\begin{equation}\label{vetor_lexicografico_3d}
\mathbf{\Omega}= \frac{1}{\sqrt{2}}\left[
	\begin{array}{ccc}
	S_{hh}& S_{hv}& S_{vv}   \\
\end{array}\right]^T=[\Omega_1\quad \Omega_2\quad \Omega_3].
\end{equation}

As constantes $2$ e $\sqrt{2}$ nas equações (\ref{base_de_pauli_3d}) e (\ref{base_de_lexicografica_3d}) servem para manter a norma dos vetores de espalhamento iguais independente da escolha das bases. O produto interno escolhido é o padrão para o espaço vetorial dos vetores complexos de dimensão 3.

Podemos assim garantir que a invariância da potencia total,
\begin{equation}\label{span_invariante_3d}
\begin{array}{ccc}
\mathbf{Span(S)} &=& \traco(SS^H)\\
	   &=&  \traco(SS^H)=|S_{hh}|^2+2|S_{hv}|^2+|S_{vv}|^2  \\
	   &=&  \mathbf{k}^H\mathbf{k}=|\mathbf{k}|^2\\
	   &=& \mathbf{\Omega}^H\mathbf{\Omega}=|\mathbf{\Omega}|^2
\end{array}
\end{equation}

A transformação linear unitária $U_{3(L \rightarrow P)}$ é definida como uma transformação que aplica o vetor na base lexicográfica em um vetor na base de Pauli. Definimos a notação \textrm{SU(3)} para designar a transformação  unitária.

\begin{equation}\label{trans_matriz_unit_su3}
\frac{1}{\sqrt{2}}\left[
\begin{array}{c}
	  S_{hh} +  S_{vv}\\  
	  S_{hh} -  S_{vv}\\
	  2  S_{hv} \\
\end{array}
\right]=\frac{1}{\sqrt{2}}	
\left[
\begin{array}{rrr}
	1   & 0 & 1  \\
	1   & 0 & -1  \\
	0   & 2 &  0  \\
\end{array}
\right]
\left[
\begin{array}{c}
	S_{hh} \\  
	S_{hv} \\
	S_{vv}\\
\end{array}
\right]
\end{equation}
desta maneira definimos a matriz unitária,
\begin{equation}\label{matriz_unit_su3}
U_{3(L \rightarrow P)}=\frac{1}{\sqrt{2}}	
\left[
\begin{array}{rrr}
	1   & 0 & 1  \\
	1   & 0 & -1  \\
	0   & \sqrt{2} &  0  \\
\end{array}
\right].
\end{equation}

Para o caso mono estático definimos a matriz de coerência polarimétrica de Pauli 
\begin{equation}\label{matriz_polar_pauli_3}
	\mathbf{T}_3=\mathbf{k}\mathbf{k}^H=	
\left[
\begin{array}{rrr}
	|k_1|^2       & k_1\bar{k}_2  & k_1\bar{k}_3  \\
	k_2\bar{k}_1  & |k_2|^2       & k_2\bar{k}_3  \\
	k_3\bar{k}_1  & k_3\bar{k}_2  &    |k_3|^2    \\
\end{array}
\right],
\end{equation}
e a matriz de covariância lexicográfica
\begin{equation}\label{matriz_covar_lexic_3}
	\mathbf{C}_3=\mathbf{\Omega}\mathbf{\Omega}^H=	
\left[
\begin{array}{rrr}
	|\Omega_1|^2       & \Omega_1\bar{\Omega}_2  & \Omega_1\bar{\Omega}_3   \\
	\Omega_2\bar{\Omega}_1  & |\Omega_2|^2       & \Omega_2\bar{\Omega}_3  \\
	\Omega_3\bar{\Omega}_1  & \Omega_3\bar{\Omega}_2  &    |\Omega_3|^2      \\
\end{array}
\right].
\end{equation}

Usando as definções e as propriedades acima teremos
\begin{equation}\nonumber
\begin{array}{rrrrr}
	\mathbf{T}_3&=&\mathbf{k}\mathbf{k}^H&=&\mathbf{U_3\Omega}(\mathbf{U_3\Omega})^H	\\
	   &=&\mathbf{U_3\Omega}\mathbf{\Omega}^H\mathbf{U_3}^H&&	\\
	   &=&\mathbf{U_3}\mathbf{C}_3\mathbf{U_3}^H&=&\mathbf{U_3}\mathbf{C}_3\mathbf{U_3}^{-1}\\
\end{array},
\end{equation}
\begin{equation}\label{matriz_simil_t3_c3}
\begin{array}{rrr}
	 \mathbf{T}_3  &=&\mathbf{U_3}\mathbf{C}_3\mathbf{U_3}^{-1}\\
\end{array},
\end{equation}
com isso, podemos concluir que
\begin{equation}\label{traco_t3_c3}
\begin{array}{rrrrr}
	\traco({\mathbf{T}_3})  &=&\traco({\mathbf{C}_3})&=&\mathbf{Span(S)}.\\
\end{array}
\end{equation}

Podemos concluir, se
\begin{equation}\label{vetor_3d} 
\mathbf{s} = \left[
\begin{array}{c}
	S_{hh}      \\
        \sqrt{2}S_{vh}     \\
	S_{vv}      \\
\end{array}
\right],
\end{equation}
a potência total espalhada no caso de um sistema de radar polarimétrico em meio recíproco pode ser definido por
\begin{equation}\label{span_geral}
\mathbf{Span} = \traco(SS^H)=|S_{hh}|^2+2|S_{hv}|^2+|S_{vv}|^2.
\end{equation}

\section{Estatística do Ruido \textit{Speckle}}
O ruído \textit{speckle} causa uma variação de intensidade pixel a pixel imprimindo um aspecto granular nas imagens SAR / PolSAR.  

O \textit{speckle} dificulta a interpretação e analise das imagens reduzindo a efetividade da segmentação, classificação, ou detecção de mudanças de características  das imagens SAR / PolSAR. O entendimento do comportamento estatístico do \textit{speckle} é essencial para extrair boas informações das imagens e propor algoritmos efetivos para tratar o \textit{speckle}. Assim, podemos propor tarefas de criação de filtros para as imagens, estimativas de parâmetros geofísicos, classificação e segmentação de regiões, detectar bordas, entre outras.

\cite{lp} realizaram um estudo sistemático do speckle com o objetivo de entender os seus efeitos nas imagens SAR e PolSAR, o estudo usou os dados com simples visada ou com múltiplas visadas.

No presente trabalho usaremos as características do ruído \text{speckle} para auxiliar na detecção de borda, em oposição a trabalhos que tentam mitigar o efeito do \textit{speckle}.
 
\subsection{Formação do speckle}   
 A formação do speckle surge quando o radar ilumina uma superfície rugosa com escala do comprimento de onda do radar, o sinal de retorno consiste em ondas refletidas de muitos elementos de espalhamentos. 
 
 Os elementos de espalhamento têm geometrias complexas e distribuições aleatórias, tornando a modelagem estatística uma tarefa indispensável e desafiadora. Podemos considerar três tipos de processos de espalhamento da onda em alvos (elementos de espalhamento). A dispersão de superfície, a dispersão de volume e o espalhamento de volume-superfície. O primeiro é o espalhamento que acontece quando a onda eletromagnética atravessa uma mudança de meio de propagação. Segundo, consiste no espalhamento que acontece na profundidade de um meio, por exemplo, o espalhamento no interior de uma floresta. E por último, o espalhamento volume- superfície, que consiste em a onda atingir outra troca de meio de propagação, por exemplo o solo de uma floresta.
 
 As distancia entre os elementos de espalhamento e o recebimento no radar varia devido a natureza randômica da disposição desse elemento. A onda recebida de cada elemento espalhador embora coerente em frequência não são coerentes em fase. O sinal é forte se as ondas são construtivas, ou seja em fase, e fraco se a ondas não estão em fase.
 
 Podemos escrever um sinal complexo da seguinte forma.

\begin{equation}\label{eq:speckle_soma}
\sum_{i=1}^{M}(x_i+\dot{\jmath} y_i)=\sum_{i=1}^{M}x_i+\dot{\jmath}\sum_{i=1}^{M} y_i=x+\dot{\jmath} y=r\exp(\dot{\jmath}\theta),
\end{equation}
onde, $x_i+\dot{\jmath} y_i$ é o retorno do espalhamento para cada elemento $i$, $x+\dot{\jmath}y$ é o retorno dos $M$ espalhadores somados, e $r\exp(\dot{\jmath}\theta)$ é a decomposição de Euler para o número complexo $x+\dot{\jmath}y$.


\subsection{Modelo de Rayleigh para o \textit{speckle}}
Podemos determinar as seguintes condições para a modelagem,
\begin{itemize}
\item 1) um número grande de espalhadores na resolução de célula para um meio homogêneo,
\item 2) a distância de alcance é muito maior que o comprimento de onda do radar,
\item 3) A superfície tem rugosidade na escala do comprimento de onda de um radar.
\end{itemize}

O vetor soma $\eqref{eq:speckle_soma}$ de ondas refletidas de alvos podem ser definidas de forma que a sua fase seja distribuída uniformemente no intervalo $(-\pi,\pi)$. O \textit{speckle} possuindo esta propriedade são chamados de \textit{speckle} totalmente desenvolvido.  

O teorema do limite central para o \textit{speckle} completamente desenvolvido garante que as componentes $x$ e $y $ são independentemente e identicamente distribuídas gaussianas com média zero e variância $\frac{\sigma^2}{2}$. Podemos representar a sua probabilidade conjunta por,
\begin{equation}\label{eq:pdf_gaussiana_xy}
f_{XY}(x,y;\sigma^2)=f_X(x;\sigma^2)f_Y(y;\sigma^2)=\frac{1}{\sqrt{\pi}\sigma}\exp\left(-\frac{x^2}{\sigma^2}\right)\frac{1}{\sqrt{\pi}\sigma}\exp\left(-\frac{y^2}{\sigma^2}\right)=\frac{1}{\pi\sigma^2}\exp\left(-\frac{x^2+y^2}{\sigma^2}\right).
\end{equation}
sendo $x=A\cos(\theta)$ e $y=A\sin(\theta)$ teremos,
\begin{equation}\label{eq:pdf_gaussiana_Atheta}
f_{A}(z_A;\sigma^2)=\frac{z_A}{\pi\sigma^2}\exp\left(-\frac{z_A^2(\cos^2(\theta)+\sin^2(\theta)}{\sigma^2}\right)=\frac{z_A}{\pi\sigma^2}\exp\left(-\frac{z_A^2}{\sigma^2}\right),
\end{equation}

Integrando na variável $\theta$ no intervalo de $[-\pi,\pi]$ teremos a distribuição para a amplitude.
\begin{equation}\nonumber
f_A(z_A;\sigma^2)=\int_{-\pi}^{\pi}\frac{z_A}{\pi\sigma^2}\exp\left(-\frac{z_A^2}{\sigma^2}\right)d\theta=\frac{z_A}{\pi\sigma^2}\exp\left(-\frac{z_A^2}{\sigma^2}\right)\int_{-\pi}^{\pi}d\theta,
\end{equation}
definida como a distribuição Rayleigh com PDF
\begin{equation}\nonumber
f_A(z_A;\sigma^2)=\frac{2z_A}{\sigma^2}\exp\left(-\frac{z_A^2}{\sigma^2}\right).
\end{equation}

Podemos encontrar o valor esperado $$E[A]=\int_0^\infty z_Af(z_A)dA=\int_0^	\infty \frac{2z_A^2}{\sigma^2}\exp\left(-\frac{z_A^2}{\sigma^2}\right) dA=\frac{\sqrt{\pi}\sigma}{2},$$ e a variância $var= E[X^2]-E[X]^2$, sendo $$E[A^2]=\int_0^\infty z_A^2f(z_A)dA=\int_0^	\infty \frac{2z_A^3}{\sigma^2}\exp\left(-\frac{z_A^2}{\sigma^2}\right) dA=\sigma^2.$$
Então $$var=E[X^2]-E[x]^2=\sigma^2-\left(\frac{\sqrt{\pi}\sigma}{2}\right)^2=\sigma^2-\frac{\pi\sigma^2}{4}.$$

O coeficiente de variação $CV(Z_A) =\frac{\sqrt{var}}{E[A]}=\frac{\sigma^2-\frac{\pi\sigma^2}{4}}{\frac{\sqrt{\pi}\sigma}{2}}=\sqrt{\frac{\sigma^2-\frac{\pi\sigma^2}{4}}{\frac{\pi\sigma^2}{4}}}=\sqrt{\frac{4}{\pi}-1}=0,5227.$

Definindo $I=A^2$ a pdf para intensidade é 
\begin{equation}\nonumber
f_I(z_I;\sigma^2)=\frac{1}{\sigma^2}\exp\left(-\frac{z_I}{\sigma^2}\right),
\end{equation}
desta forma podemos calcular $E[I]=\sigma^2$, e $E[I^2]=2\sigma^2$ então $var=\sigma^4$, então o $CV(Z_I)=\frac{\sqrt{\sigma^2}}{\sigma^2}=1$. 

Comparando os valores $CV(Z_A)$ e $CV(Z_I)$ podemos afirmar que o valor do \textit{speckle} á mais pronunciado nas imagens de intensidade em relação com as imagens de amplitude.

\subsection{Estatística \textit{speckle} no processo de múltiplas visadas}

O processo de redução do ruído \textit{speckle} consiste em realizar a média aritmética de vários sinais de retorno e chamamos de múltiplas visadas. O método pode ser descrito como adquirir N imagens e realizar sua a média aritmética. As N imagens podem ser tomadas com a característica de serem estatisticamente independente. 

Definimos a função densidade de probabilidade para os canais de intensidade com múltiplas visadas,    
\begin{equation}\label{pdf_inten_multilook}
f_I(z_I;L,\sigma^2)=\frac{L^L z_I^{L-1}}{(L-1)!\sigma^{2L}}\exp\left(-L\frac{z_I}{\sigma^2}\right), z_I\geq 0.
\end{equation}

A média e a variância são $M_L(z_I)=\sigma^2$, $Var_L(z_I)=\frac{\sigma^4}{L}$ implicando que o desvio padrão será $SD_L(z_I)=\frac{\sigma^2}{\sqrt{L}}$. O coeficiente de variação é calcula como sendo a razão entre o desvio padrão e a média resultando em $CV_L(z_I)=\frac{1}{\sqrt{L}}$. Podemos observar que o desvio padrão é reduzido por $\sqrt{N}$ em relação ao processo de visada simples.

Definimos a função densidade de probabilidade para os canais de amplitude com múltiplas visadas,    
\begin{equation}\label{pdf_inten_multilook}
f_A(z_A;L,\sigma^2)=\frac{2L^L z_A^{2L-1}}{(L-1)!\sigma^{2L}}\exp\left(-L\frac{z_A^2}{\sigma^2}\right), z_A\geq 0.
\end{equation}

A média e a variância são $M_L(z_A)=\frac{\Gamma\left(L+\frac{1}{2}\right)}{\Gamma}\sqrt{\frac{\sigma^2}{L}}$ e $Var_L(z_A)=\left(L-\frac{\Gamma^2\left(L+\frac{1}{2}\right)}{\Gamma^2(L)}\right)\frac{\sigma^2}{L}$. O coeficiente de variação $CV_L(z_A)=\sqrt{\frac{L\Gamma^2(L)}{\Gamma^2\left(L+\frac{1}{2}\right)}-1}$, onde $\Gamma$ denota a função gamma

\begin{table}[hbt]
	\centering
	\caption{Coeficientes de variação.}\label{cv_multilook}
\begin{tabular}{@{}lccc@{}} \toprule
	Número de visadas & N-visadas(intensidade)   & N- visadas(amplitude) \\ \midrule
	1 & 1                  &   0.522723200877063\\ 
	2 &  0.707106781186547 &  0.362999289543428\\
	3 &  0.577350269189626 &  0.294104989486191\\
	4 & 0.500000000000000  & 0.253622399398351 \\
	5 & 0.447213595499958  & 0.226239950138330 \\
	6 &  0.408248290463863 &  0.206148101392413\\
	7 &  0.377964473009227 &  0.190600152599532\\ 
	8 & 0.353553390593274  &  0.178108152789829\\ \bottomrule
\end{tabular}
\end{table}

\subsection{Matriz de covariância segundo \citet{good}}

O vetor $\mathbf{S}$ pode ser rearranjado usando as partes reais e complexas de suas entradas em um vetor de dimensão $6$ onde cada entrada é representado por
\begin{equation}
\mathbf{S} = \left[
\begin{array}{c}
	R_{hh}     \\
    I_{hh}     \\
	R_{hv}     \\
	I_{hv}     \\
    R_{vv}     \\
	I_{vv}     \\
\end{array}
\right]
\end{equation}
e cujo produto, 
\begin{equation*}
\tiny
\mathbf{S} \mathbf{S}^H= \left[
\begin{array}{rrrrrr}
	R_{hh}R_{hh}  & R_{hh}I_{hh} &R_{hh}R_{hv} & R_{hh}I_{hv} &R_{hh}R_{vv} &R_{hh}I_{vv} \\
    I_{hh}R_{hh}  & I_{hh}I_{hh} &I_{hh}R_{hv} & I_{hh}I_{hv} &I_{hh}R_{vv} &I_{hh}I_{vv} \\
	R_{hv}R_{hh}  & R_{hv}I_{hh} &R_{hv}R_{hv} & R_{hv}I_{hv} &R_{hv}R_{vv} &R_{hv}I_{vv} \\
	I_{hv}R_{hh}  & I_{hv}I_{hh} &I_{hv}R_{hv} & I_{hv}I_{hv} &I_{hv}R_{vv} &I_{hh}I_{vv} \\
    R_{vv}R_{hh}  & R_{vv}I_{hh} &R_{vv}R_{hv} & R_{vv}I_{hv} &R_{vv}R_{vv} &R_{vv}I_{vv} \\
	I_{vv}R_{hh}  & I_{vv}I_{hh} &I_{vv}R_{hv} & I_{vv}I_{hv} &I_{vv}R_{vv} &I_{vv}I_{vv} \\
\end{array}
\right].
\end{equation*}

A distribuição gaussiana circular complexa multivariada com média zero pode ser definida de acordo com \citet{goodman}. Sendo $\mathbf{S}_= R_j+\hat{\imath}I_j$ definimos $R_j$ e $I_j$ com $j=1,2,3$ tenham distribuições conjuntas gaussianas e satisfaçam as seguintes condições

Na referência \cite{good} foi descrito a hipótese da distribuição gaussiana circular complexa multivariada com média zero. sendo $\mathbf{S}_{j}= R_j+\hat{\imath}I_j$ definimos $R_j$ e $I_j$ com $j=1,2,3$ tenham distribuições conjuntas gaussianas e satisfaçam as seguintes condições  
\begin{itemize}
	\item[I-] $E[R_{j}]=E[I_{j}]=0,$
	\item[II-] $E[R_j^2]=E[I_j^2],$ 
	\item[II-] $E[R_jI_j]=0,$  
	\item[IV-] $E[R_jR_i]=E[I_jI_i],$ e 
	\item[V-] $E[I_jR_i]=-E[R_jI_i].$
\end{itemize}
onde, $E[\cdot]$ denota o valor esperado. 

A hipótese da distribuição gaussiana circular complexa multivariada com média zero assumindo, resulta na matriz simétrica,  
\begin{equation*}
\tiny
\mathbf{S} \mathbf{S}^H= \left[
\begin{array}{rrrrrr}
	R_{hh}^2       & 0            & R_{hh}R_{hv}  &-R_{hh}I_{hv}  &R_{hh}R_{vv}  &-R_{hh}I_{vv}    \\
    0              & I_{hh}^2     & I_{hh}R_{hv}  & I_{hh}I_{hv}  &I_{hh}R_{vv}  & I_{hh}I_{vv}   \\
	R_{hv}R_{hh}   & R_{hv}I_{hh} & R_{hv}^2      & 0             &R_{hv}R_{vv}  &-R_{hv}I_{vv}     \\
   -I_{hv}R_{hh}   & I_{hv}I_{hh} & 0             & I_{hv}^2      &I_{hv}R_{vv}  & I_{hh}I_{vv} \\
    R_{vv}R_{hh}   & R_{vv}I_{hh} & R_{vv}R_{hv}  & R_{vv}I_{hv}  &R_{vv}^2      & 0            \\
   -I_{vv}R_{hh}   & I_{vv}I_{hh} &-I_{vv}R_{hv}  & I_{vv}I_{hv}  &0             & I_{vv}^2     \\
\end{array}
\right].
\end{equation*}

De acordo com \cite{good} a distribuição gaussiana complexa multivariada pode modelar adequadamente o comportamento estatístico de $\boldmath S$. Isto é chamado de {\it single-look complex PolSAR data representation} e podemos definir o vetor de espalhamento por $\mathbf{S}=[S_{hh},S_{hv},S_{vv}]^H$. 

Dados polarimétricos são usualmente submetidos a um processo de várias visadas com o intuito de melhorar a razão entre o sinal e o seu ruído. Para esse fim, matrizes positivas definidas hermitianas estimadas são obtidas computando a média de $L$ visadas independentes de uma mesma cena. Resultando na matriz de covariância amostral estimada \textbf{Z} conforme \cite{good, ade}
\begin{equation}
\begin{array}{ccc}
    \mathbf{Z}&=&\frac{1}{L}\displaystyle{\sum_{l=1}^{L} {\mathbf{s}_l}{\mathbf{s}_l}^H}, \\
\end{array}
\end{equation}
onde $\mathbf{s}_l$ com $l = 1, \dots, L$ são $L$ vetores complexos distribuídos como $\mathbf{S}$.

%Sendo $i=j$
%\begin{equation}
%\begin{array}{ccc}
%\mathbf{S}_{ii}\overline{\mathbf{S}}_{ii}&=& (R_{ii}+iI_{ii})\overline{(R_{ii}+iI_{jj})} \\
%\mathbf{S}_{ii}\overline{\mathbf{S}}_{ii}&=& (R_{ii}+iI_{ii})(R_{ii}-iI_{ii}) \\
%\mathbf{S}_{ii}\overline{\mathbf{S}}_{ii}&=& R_{ii}^2+I_{ii}^2 \\
%\end{array}
%\end{equation}
%e considerando $i \neq j$
%\begin{equation}
%\begin{array}{ccc}
%\mathbf{S}_{ii}\overline{\mathbf{S}}_{ij}&=& (R_{ii}+iI_{ii})\overline{(R_{ij}+iI_{ij})} \\
%\mathbf{S}_{ii}\overline{\mathbf{S}}_{ij}&=& (R_{ii}+iy_{ii})(I_{ij}-iI_{ij}) \\
%\mathbf{S}_{ii}\overline{\mathbf{S}}_{ij}&=& (R_{ii}R_{ij}+I_{ii}I_{ij})+i(R_{ij}I_{ii}-R_{ii}I_{ij}) \\
%\end{array}
%\end{equation}
% Definindo,
% \begin{equation}
%\begin{array}{ccc}
%	  RC_{ij}&=&  R_{ii}R_{ij}+I_{ii}I_{ij} 
%\end{array}
%\end{equation}
%e
%\begin{equation}
%\begin{array}{ccc}
%	  IC_{ij}&=& R_{ij}I_{ii}-R_{ij}I_{ii}
%\end{array}
%\end{equation}

%Sendo a variável randômica gaussiana complexa $\mathbf{C_{i,j}}=RC_{ij} + i IC_{ij}$, ou ainda, $\mathbf{C_{i,j}}=(R_{ii}R_{ij}+I_{ii}I_{ij}) + i(R_{ij}I_{ii}-R_{ij}I_{ii})$. Podemos escrever uma variável randômica gaussiana complexa $4-$variada $(R_{ii},R_{ij},I{ii},I_{ij})$.

De acordo com (\cite{good})
\begin{equation}
\mathbf{C} = \left[
\begin{array}{cc}
	E(X_iX_j)  & E(X_iY_j)  \\
	E(Y_iX_j)  & E(Y_iY_j)  \\
\end{array}
\right].
\end{equation}
Tal que
\begin{equation}
\mathbf{C} =
\left\{
\begin{array}{cc}
	\frac{1}{2}\left[
\begin{array}{cc}
	 1 & 0  \\
	 0 & 1  \\
\end{array}
	\right]\sigma^{2}_{j}  & \mbox{se}\quad i=j, \\
	& \\
	\frac{1}{2}\left[
\begin{array}{cc}
	\alpha_{ij} & -\beta_{ij}  \\
	 \beta_{ij} & \alpha_{ij}  \\
\end{array}
	\right]\sigma_j\sigma_k  & \mbox{se}\quad i\neq j.   \\
\end{array}
\right.
\end{equation}
pode ser escrito por
\begin{equation}
\mathbf{\sigma}_{ij} =
\left\{
\begin{array}{cc}
	\sigma^{2}_{j}  & \mbox{se}\quad i=j, \\
	(\alpha_{ij}+\hat{\imath}\beta_{ij})\sigma_i\sigma_j  & \mbox{se}\quad i\neq j.   \\
\end{array}
\right.
\end{equation}

{\bf Exemplo 1 -} Seja a distribuição gaussian complexa univariada $(p=1)$. Sendo $\xi^{T}=z_1=x_1+iy_1$. E a "matriz" de covariância $\Sigma_{\xi}=\sigma_{1}^{2}$ com determinante $|\Sigma_{\xi}|=\sigma_{1}^{2}$ e  "matriz inversa" $\Sigma_{\xi}^{-1}=\frac{1}{\sigma_{1}^{2}}$, Assim,

\begin{equation}\label{eqn70}
\begin{array}{ccc}
	\bar{\xi}^{T}\Sigma_{\xi}^{-1}\xi&=&(x_i-iy_1)\frac{1}{\sigma_1^2}(x_1+iy_1)  \\
	\bar{\xi}^{T}\Sigma_{\xi}^{-1}\xi&=&(x_i-iy_1)(x_1+iy_1)\frac{1}{\sigma_1^2}  \\
	\bar{\xi}^{T}\Sigma_{\xi}^{-1}\xi&=&\frac{x_1^2+y_1^2}{\sigma_1^2}  \\
\end{array}
\end{equation}


\begin{equation}\label{eqn71}
	p(\xi)=\frac{1}{\pi\Sigma_{\xi}^{2}}\exp\left(-\frac{x_1^2+y_1^2}{\sigma_1^2}\right)  \\
\end{equation}

{\bf Exemplo 2 -} Seja a distribuição gaussian complexa bivariada $(p=2)$. Sendo $\xi^{T}=(z_1, z_2)=(x_1 + iy_1, x_2 + iy_2)^{T}$. E a matriz de covariância 

$$
\Sigma_{\xi} = \left[
\begin{array}{cc}
	\sigma_1^2                                &  (\alpha_{12}+i\beta_{12})\sigma_1\sigma_2  \\
	(\alpha_{12}-i\beta_{12})\sigma_j\sigma_k & \sigma_2^2 \\
\end{array}
\right].
$$
com determinante $|\Sigma_{\xi}|=(1 - \sigma_{12}^{2}- \beta_{12}^2)\sigma_{1}^2\sigma_{2}^2$ e  matriz inversa 
$$
\Sigma_{\xi}^{-1} =\frac{1}{(1 - \sigma_{12}^{2}- \beta_{12}^2)\sigma_{1}^2\sigma_{2}^2} \left[
\begin{array}{cc}
	\sigma_2^2                                &  -(\alpha_{12}+i\beta_{12})\sigma_1\sigma_2  \\
	-(\alpha_{12}-i\beta_{12})\sigma_j\sigma_k & \sigma_1^2 \\
\end{array}
\right].
$$
\begin{equation}\label{eqn72}
\begin{array}{ccc}
	\bar{\xi}^{T}\Sigma_{\xi}^{-1}\xi&=&[z_1,z_2]^{H}\Sigma_{\xi}^{-1}
	\left[
\begin{array}{c}
	z_1  \\
	z_2 \\
\end{array}\right]\\
\end{array}
\end{equation}
\begin{equation}\label{eqn73}
\begin{array}{ccc}
	\bar{\xi}^{T}\Sigma_{\xi}^{-1}\xi&=&[z_1,z_2]^{H}\frac{1}{(1 - \sigma_{12}^{2}- \beta_{12}^2)\sigma_{1}^2\sigma_{2}^2} \left[
\begin{array}{cc}
	\sigma_2^2                                &  -(\alpha_{12}+i\beta_{12})\sigma_1\sigma_2  \\
	-(\alpha_{12}-i\beta_{12})\sigma_1\sigma_2 & \sigma_1^2 \\
\end{array}
\right]
	\left[
\begin{array}{c}
	z_1  \\
	z_2 \\
\end{array}\right]\\
\end{array}
\end{equation}

\begin{equation}\label{eqn74}
\begin{array}{ccc}
	\bar{\xi}^{T}\Sigma_{\xi}^{-1}\xi&=&\frac{1}{(1 - \sigma_{12}^{2}- \beta_{12}^2)\sigma_{1}^2\sigma_{2}^2} [z_1,z_2]^{H}\left[
\begin{array}{cc}
	\sigma_2^2                                &  -(\alpha_{12}+i\beta_{12})\sigma_1\sigma_2  \\
	-(\alpha_{12}-i\beta_{12})\sigma_1\sigma_2 & \sigma_1^2 \\
\end{array}
\right]
	\left[
\begin{array}{c}
	z_1  \\
	z_2 \\
\end{array}\right]\\
\end{array}
\end{equation}

\begin{equation}\label{eqn75}
\begin{array}{ccc}
	\bar{\xi}^{T}\Sigma_{\xi}^{-1}\xi&=&\frac{1}{(1 - \sigma_{12}^{2}- \beta_{12}^2)\sigma_{1}^2\sigma_{2}^2} [z_1,z_2]^{H}\left[
\begin{array}{cc}
	\sigma_2^2z_1-(\alpha_{12}+i\beta_{12})\sigma_1\sigma_2z_2  \\
	-(\alpha_{12}-i\beta_{12})\sigma_1\sigma_2z_1+\sigma_1^2z_2 \\
\end{array}
\right]
\end{array}
\end{equation}

\begin{equation}\label{eqn76}
	\bar{\xi}^{T}\Sigma_{\xi}^{-1}\xi=\frac{1}{(1 - \sigma_{12}^{2}- \beta_{12}^2)\sigma_{1}^2\sigma_{2}^2}\left(
	\sigma_2^2\bar{z_1}z_1-(\alpha_{12}+i\beta_{12})\sigma_1\sigma_2\bar{z_1}z_2 
	-(\alpha_{12}-i\beta_{12})\sigma_1\sigma_2\bar{z_2}z_1+\sigma_1^2\bar{z_2}z_2 \right)
\end{equation}

\begin{equation}\label{eqn77}
	\bar{\xi}^{T}\Sigma_{\xi}^{-1}\xi=\frac{1}{(1 - \sigma_{12}^{2}- \beta_{12}^2)\sigma_{1}^2\sigma_{2}^2}\left(
	\sigma_2^2|z_1|^2+\sigma_1^2|z_2|^2-2\alpha_{12}\sigma_1\sigma_2\bar{z_1}z_2 	\right)
\end{equation}

\begin{equation}\label{eqn78}
	\bar{\xi}^{T}\Sigma_{\xi}^{-1}\xi=\frac{\sigma_2^2|z_1|^2+\sigma_1^2|z_2|^2-2\alpha_{12}\sigma_1\sigma_2\bar{z_1}z_2}{(1 - \sigma_{12}^{2}- \beta_{12}^2)\sigma_{1}^2\sigma_{2}^2}
\end{equation}

Assim, a função densidade de probabilidade ({\bf pdf}) 

\begin{equation}\label{eqn79}
	p(\xi)=\frac{1}{\pi^2(1 - \sigma_{12}^{2}- \beta_{12}^2)\sigma_{1}^2\sigma_{2}^2}\exp\left(-\frac{\sigma_2^2|z_1|^2+\sigma_1^2|z_2|^2-2\alpha_{12}\sigma_1\sigma_2\bar{z_1}z_2}{(1 - \sigma_{12}^{2}- \beta_{12}^2)\sigma_{1}^2\sigma_{2}^2}
\right)  
\end{equation}

{\bf Distribuição complexa de Wishart}

A distribuição complexa de Wishart descrita no artigo \cite{good}, define agora uma amostra de  $n$ vetores com valores complexos $\xi_1,\xi_2,\dots,\xi_n$ então a matriz hermitiana de covariância é 

\begin{equation}\label{eqn80}
	\hat{\Sigma}_{\xi}=\frac{1}{n}\sum_{j=1}^{n}\xi_j\bar{\xi}_{j}^{T} . \\
\end{equation}

A matriz $\hat{\Sigma}_{\xi}$ é uma "maximum likelihood" para $\Sigma_{\xi}$ sendo uma estatística suficiente para a matriz hermitiana de covariância.

Considerando $A=||A_{jkR}+iA{jkI}||=n\hat{\Sigma}_{\xi}$ chamaremos a matriz $A$ de distribuição complexa de Wishart. A função densidade de probabilidade de $A$ é
\begin{equation}\label{eqn81}
	p_W(A)=\frac{|A|^{n-p}}{I(\Sigma_{\xi})} \exp(-tr(\Sigma_{\xi}^{-1}A)), 
\end{equation}
onde
\begin{equation}\label{eqn82}
	I(\Sigma_{\xi})=\pi^{\frac{1}{2}p(p-1)}\Gamma(n)\cdots\Gamma(n-p+1)|\Sigma_{\xi}|^n, 
\end{equation}
sendo $\Gamma(\cdot)$ a função Gamma.



\begin{sidewaystable}
	\centering
	\caption{Tabela}
\begin{tabular}{@{}lcccccc@{}} \toprule
	     &$R_{hh}$        & $I_{hh}$ & $R_{hv}$&$I_{hv}$                            &$R_{vv}$                           &$I_{vv}$ \\ \midrule
$R_{hh}$ &$\sigma_{hh}^2$ & 0                  &$\rho_{hh,hv}\sigma_{hh}\sigma_{hv}$ &$\eta_{hh,hv}\sigma_{hh}\sigma_{hv}$ & $\rho_{hh,vv}\sigma_{hh}\sigma_{vv}$&$\eta_{hh,vv}\sigma_{hh}\sigma_{vv}$  \\ 
	$I_{hh}$ & 0 & $\sigma_{hh}^2$ &$-\eta_{hh,hv}\sigma_{hh}\sigma_{hv}$ &$\rho_{hh,hv}\sigma_{hh}\sigma_{hv}$ &$-\eta_{hh,vv}\sigma_{hh}\sigma_{vv}$ &$\rho_{hh,vv}\sigma_{hh}\sigma_{vv}$  \\ 
	$R_{hv}$ &$\rho_{hh,hv}\sigma_{hh}\sigma_{hv}$   &$-\eta_{hh,hv}\sigma_{hh}\sigma_{hv}$  &$\sigma_{hv}^2$ &0 &$\rho_{hv,vv}\sigma_{hv}\sigma_{vv}$ &$\eta_{hv,vv}\sigma_{hv}\sigma_{vv}$  \\ 
	$I_{hv}$ &$\eta_{hh,hv}\sigma_{hh}\sigma_{hv}$  &$\rho_{hh,hv}\sigma_{hh}\sigma_{hv}$  &0 &$\sigma_{hv}^2$ &$-\eta_{hv,vv}\sigma_{hv}\sigma_{vv}$ &$\rho_{hv,vv}\sigma_{hv}\sigma_{vv}$ \\ 
	$R_{vv}$ &$\rho_{hh,vv}\sigma_{hh}\sigma_{vv}$  &$-\eta_{hh,vv}\sigma_{hh}\sigma_{vv}$  &$\rho_{hv,vv}\sigma_{hv}\sigma_{vv}$ &$-\eta_{hv,vv}\sigma_{hv}\sigma_{vv}$ & $\sigma_{vv}^2$ &0 \\ 
    $I_{vv}$ &$\eta_{hh,vv}\sigma_{hh}\sigma_{hv}$  &$\rho_{hh,vv}\sigma_{hh}\sigma_{vv}$  &$\eta_{hv,vv}\sigma_{hv}\sigma_{vv}$ &$\rho_{hv,vv}\sigma_{hv}\sigma_{vv}$ & 0 &$\sigma_{vv}^2$ \\ 	 \bottomrule
\end{tabular}
\end{sidewaystable}
\section{Funções de densidade}
\subsection{Função de densidade Wishart para os canais de intensidade}
Para os canais $(hh)$, $(hv)$ e $(vv)$ vamos usar a distribuição Wishart (PDF) descrita por

\begin{equation}
    f_{\mathbf{Z}}(\mathbf{Z};\mathbf{\Sigma_{s}},L)=\frac{L^{mL}|\mathbf{Z}|^{L-m}}{|\mathbf{\Sigma_{s}}|^{L}\Gamma_m(L)} \exp(-L\traco(\mathbf{\Sigma_{s}}^{-1}\mathbf{Z})), \\
\end{equation} 

onde, $\traco(\cdot)$ é o operador traço de uma matriz, $\Gamma_m(L)$ é uma função Gamma multivariada definida por
\begin{equation}\label{cap_acf_10}
	\Gamma_m(L)=\pi^{\frac{1}{2}m(m-1)} \prod_{i=0}^{m-1}\Gamma(L-i) \\
\end{equation}
e $\Gamma(\cdot)$ é a função Gamma. Podemos afirmar que $\mathbf{Z}$ é distribuído como uma distribuição Wishart denotando por $\mathbf{Z}\sim W(\mathbf{\Sigma_{s}}, L)$ e satisfazendo $E[\mathbf{Z}]=\mathbf{\Sigma_{s}}$. Sem perda de generalidade para o texto vamos usar o simbolo $\mathbf{\Sigma}$ em detrimento a $\mathbf{\Sigma_{s}}$ para representar a matriz de covariância associada a $\mathbf{S}$.

\begin{figure}[hbt]
\centering
\includegraphics[width=4.0in]{dist_intensidade_multi_visadas.pdf}
	\caption{Distribuição intensidade multiplas visadas com $\sigma=0.01$.}
\label{fig2}
\end{figure}



\subsection{Função de densidade para cada canal complexo}

Sendo $(R_{ii}, R_{ij})\sim N2(0, C_{ij})$ podemos observar na tabela anterior que 
\begin{equation}
C_{ij}=\left[
\begin{array}{cc}
	\sigma_{ii}^2   &  \rho_{ii,ij}\sigma_{ii}\sigma_{ij}  \\
	\rho_{ii,ij}\sigma_{ii}\sigma_{ij} & \sigma_{ij}^2   \\
\end{array}
\right],
\end{equation}

A pdf para esta distribuição normal é:

\begin{equation}
\begin{array}{ccc}
	f_{Z_{R_{ii}R_{ij}}}(z)&=&\frac{1}{\pi\sigma_{ii}\sigma_{ij}\sqrt{1-\rho_{ii,ij}^2}}\exp\left(\frac{\rho_{ii,ij}z}{\sigma_{ii}\sigma_{ij}(1-\rho_{ii,ij})^2}\right)\\
	&&K_0\left(\frac{|z|}{\sigma_{ii}\sigma_{ij}(1-\rho_{ii,ij})^2}\right).
\end{array}
\end{equation}


Sendo $(I_{ii}, I_{ij})\sim N2(0, C_{ij})$ podemos observar na tabela anterior que 
\begin{equation}
C_{ij}=\left[
\begin{array}{cc}
	\sigma_{ii}^2   &  \rho_{ii,ij}\sigma_{ii}\sigma_{ij}  \\
	\rho_{ii,ij}\sigma_{ii}\sigma_{ij} & \sigma_{ij}^2   \\
\end{array}
\right],
\end{equation}
\begin{equation}
\begin{array}{ccc}
	f_{Z_{R_{ii}R_{ij}}}(z)&=&\frac{1}{\pi\sigma_{ii}\sigma_{ij}\sqrt{1-\rho_{ii,ij}^2}}\exp\left(\frac{\rho_{ii,ij}z}{\sigma_{ii}\sigma_{ij}(1-\rho_{ii,ij})^2}\right)\\
	&&K_0\left(\frac{|z|}{\sigma_{ii}\sigma_{ij}(1-\rho_{ii,ij})^2}\right).
\end{array}
\end{equation}

Definindo o funcional $\Theta(z;\sigma_p,\sigma_q,\gamma)$
\begin{equation}
\begin{array}{ccc}
	\Theta(z;\sigma_p,\sigma_q,\gamma)&=&\frac{1}{\pi\sigma_p\sigma_q\sqrt{1-\gamma^2}}\exp\left(\frac{\gamma z}{\sigma_p\sigma_q(1-\gamma)^2}\right)\\
	&&K_0\left(\frac{|z|}{\sigma_p\sigma_q(1-\gamma)^2}\right).
\end{array}
\end{equation}
onde,  $\sigma_p,\sigma_q,\gamma$ são  parâmetros da função.
\subsection{distribuição conjunta para  $(R_{ii}, R_{ij})\sim N2(0, C_{ij})$}
Sendo $(R_{ii}, R_{ij},I_{ii}, I_{ij})\sim N4(0, C_{ii,ij})$ podemos observar na tabela anterior que 
\begin{equation}
C_{ii,ij}=\left[
\begin{array}{cccc}
	\sigma_{ii}^2   &  \rho_{ii,ij}\sigma_{ii}\sigma_{ij} & 0&\eta_{ii,ij}\sigma_{ii}\sigma_{ij}\\
	\rho_{ii,ij}\sigma_{ii}\sigma_{ij} & \sigma_{ij}^2  & -\eta_{ii,ij}\sigma_{ii}\sigma_{ij}&0 \\
	0&-\eta_{ii,ij}\sigma_{ii}\sigma_{ij}&\sigma_{ii}^2&\rho_{ii,ij}\sigma_{ii}\sigma_{ij}\\
	\eta_{ii,ij}\sigma_{ii}\sigma_{ij}&0&\rho_{ii,ij}\sigma_{ii}\sigma_{ij}&\sigma_{ij}^2\\
\end{array}
\right],
\end{equation}
Realizar a transformação 
\begin{equation}
\left[
\begin{array}{ccc}
	 Z = R_{ii}R_{ij}+I_{ii}I_{ij} \\
	 U_1 = R_{ii}\\
	 U_2 = R_{ij}\\
	 U_3 = I_{ii}\\
\end{array}
\right],
\end{equation}

%%%%%%%%%%%%%%%%%%%%%%%%%%%%%%%%%%%%%%%%%%%%%%%%%%%%%%%%%%%%%%
De acordo com \citet{good} e \citet{lee} esta distribuição pode modelar adequadamente o comportamento estatístico de $\mathbf{s}$. A hipotêse de ser gaussiana e circular foi comprovada para dados SAR polarimétricos no artigo \citet{sarabendi}.   

A função densidade de probabilidade ({\bf pdf}) da distribuição gaussiana complexa $m$-variada é dada por
\begin{equation}\label{cap_acf_4}
    p({\bf s})=\frac{1}{\pi^m|\bf{\Sigma}_{{\bf s}}|}\exp(-{\bf s}^{H}\bf{\Sigma}_{{\bf s}}^{-1}{\bf s}), 
\end{equation}
sendo $|\cdot|$ o determinante de uma matriz ou o valor absoluto de um escalar, e $\mathbf{\Sigma}_{\bf{s}}$ é a matriz de covariância associada a $\mathbf{s}$ definida por
\begin{equation}\label{cap_acf_5}
	{\bf \Sigma_{{\bf s}}} = E[{\mathbf s}{\mathbf s}^H] = \left[
\begin{array}{cccc}
	E[S_1\overline{S_1}]  & E[{S_1}\overline{S_2}] &\hdots & E[S_1\overline{S_m}] \\
	E[S_2\overline{S_1}]  & E[{S_2}\overline{S_2}] &\hdots &E[S_2\overline{S_m}]\\
        \vdots&\vdots &\ddots &\vdots\\
	E[S_m\overline{S_1}]  & E[S_m\overline{S_2}] &\hdots &E[S_m\overline{S_m}]\\
\end{array}
\right]
\end{equation}
talque, $E[\cdot]$ denota o valor esperado e $\overline{\cdot}$ denota o conjugado complexo. A matriz de covariância é hermitiana positiva definida e contém todas as informações necessárias para caracterizar o retroespalhamento, podemos consultar mais informações em \citep{mfp}. 

Nas imagens PolSAR serão consideradas três componentes para o vetor $\mathbf{s}=[S_{hh},S_{vh},S_{vv}]^T$ e a multiplicação de $\mathbf{s}=[S_{hh},S_{vh},S_{vv}]$ pelo seu conjugado transposto $\mathbf{s}=[S_{hh},S_{vh},S_{vv}]^H$, isto é, a hermitiana do vetor, 

\begin{equation}\label{cap_acf_6}
\mathbf{s}\mathbf{s}^H = \left[
\begin{array}{c}
	S_{hh}      \\
        S_{vh}     \\
	S_{vv}      \\
\end{array}
\right]
\left[
\begin{array}{ccc}
	S_{hh}  & S_{vh}  & S_{vv}      \\
\end{array}
\right]^H = 
\left[
\begin{array}{ccc}
	S_{hh}\overline{S_{hh}} & S_{hh} \overline{S_{vh}} & S_{hh}  \overline{S_{vv}}     \\
	S_{vh} \overline{S_{hh}}  & S_{vh} \overline{S_{vh}}  & S_{vh} \overline{S_{vv}}      \\
	S_{vv} \overline{S_{hh}}  & S_{vv} \overline{S_{vh}}  & S_{vv}  \overline{S_{vv}}     \\
\end{array}
\right].
\end{equation}

A  matriz $\mathbf \Sigma_{{\mathbf s}}$ tem dimensão $3\times 3$, e pode ser definida como sendo a matriz de covariância associada a $\mathbf{s}$.
\begin{equation}\label{cap_acf_7}
\mathbf{\Sigma_{{\mathbf s}}} = E[\mathbf{s}\mathbf{s}^H] =
\left[
\begin{array}{ccc}
	E[S_{hh}\overline{S_{hh}}] & E[S_{hh} \overline{S_{vh}}] & E[S_{hh}  \overline{S_{vv}}]     \\
	E[S_{vh} \overline{S_{hh}}]  & E[S_{vh} \overline{S_{vh}}]  & E[S_{vh} \overline{S_{vv}}]      \\
	E[S_{vv} \overline{S_{hh}}]  & E[S_{vv} \overline{S_{vh}}]  & E[S_{vv}  \overline{S_{vv}}]     \\
\end{array}
\right].  
\end{equation}

Dados polarimétricos são usualmente sujeitados a um processo de várias visadas com o intuito de melhorar a razão entre o sinal e o seu ruído. Para esse fim, matrizes positivas definidas hermitianas estimadas são obtidas computando a média de $L$ visadas independentes de uma mesma cena. Resultando na matriz de covariância amostral estimada {\bf Z} conforme \citep{good, ade}
\begin{equation}\label{cap_acf_8}
\begin{array}{ccc}
    \mathbf{Z}&=&\frac{1}{L}\displaystyle{\sum_{i=1}^{L} {\mathbf{s}_i}{\mathbf{s}_i}^H}, \\
\end{array}
\end{equation}
onde $\mathbf{s}_i$ com $i = 1, \dots, L$ é uma amostra de $\mathit{L}$ vetores complexos distribuídos como $\mathbf{s}$, assim a matriz de covariância amostral associada a $\mathbf{s}_i$, com $i=1,\dots,L$ denotam o espalhamento para cada visada $L$ seguindo uma distribuição complexas de Wishart. 

Sendo agora $\mathbf{\Sigma_{s}}$ e $L$ parâmetros conhecidos a função densidade de probabilidade da distribuição Wishart por  
%
\begin{equation}\label{cap_acf_9}
    f_{\mathbf{Z}}(\mathbf{Z};\mathbf{\Sigma_{s}},L)=\frac{L^{mL}|\mathbf{Z}|^{L-m}}{|\mathbf{\Sigma_{s}}|^{L}\Gamma_m(L)} \exp(-L\traco(\mathbf{\Sigma_{s}}^{-1}\mathbf{Z})), \\
\end{equation}
onde, $\traco(\cdot)$ é o operador traço de uma matriz, $\Gamma_m(L)$ é uma função Gamma multivariada definida por
\begin{equation}\label{cap_acf_10}
	\Gamma_m(L)=\pi^{\frac{1}{2}m(m-1)} \prod_{i=0}^{m-1}\Gamma(L-i) \\
\end{equation}
e $\Gamma(\cdot)$ é a função Gamma. Podemos afirmar que $\mathbf{Z}$ é distribuído como uma distribuição Wishart denotando por $\mathbf{Z}\sim W(\mathbf{\Sigma_{s}}, L)$ e satisfazendo $E[\mathbf{Z}]=\mathbf{\Sigma_{s}}$. Sem perda de generalidade para o texto vamos usar o simbolo $\mathbf{\Sigma}$ em detrimento a $\mathbf{\Sigma_{s}}$ para representar a matriz de covariância associada a $\mathbf{s}$.

Seja a função densidade de probabilidade da distribuição complexa Wishart (\ref{cap_acf_9}) na qual vamos aplicar o logaritmo natural e suas propriedades com o intuito de reescrever a função na forma adequada para aplicar o método de estimativa de máxima verossimilhança. Assim,
\begin{equation}\label{cap_acf_11}
\begin{array}{ccc}
    \ln{f_{\mathbf{Z}}(\mathbf{Z};\mathbf{\Sigma},L)}&=&\ln{\left(\frac{L^{mL}|\mathbf{Z}|^{L-m}}{|\mathbf{\Sigma}|^{L}\Gamma_m(L)} \exp(-L\traco(\mathbf{\Sigma}^{-1}\mathbf{Z}))\right)}, \\
        \ln{\left(f_{\mathbf{Z}}(\mathbf{Z};\mathbf{\Sigma},L)\right)}&=&\ln{\left(\frac{L^{mL}|\mathbf{Z}|^{L-m}}{|\mathbf{\Sigma}|^{L}\Gamma_m(L)}\right)}+\ln{\left( \exp(-L\traco(\mathbf{\Sigma}^{-1}\mathbf{Z}))\right)}, \\
        \ln{\left(f_{\mathbf{Z}}(\mathbf{Z};\mathbf{\Sigma},L)\right)}&=&\ln{\left(L^{mL}|\mathbf{Z}|^{L-m}\right)} - \ln{\left(|\mathbf{\Sigma}|^{L}\Gamma_m(L)\right)}-L\traco(\mathbf{\Sigma}^{-1}\mathbf{Z}), \\
        \ln{\left(f_{\mathbf{Z}}(\mathbf{Z};\mathbf{\Sigma},L)\right)}&=&mL\ln{L}+(L-m)\ln{\left(|\mathbf{Z}|\right)} - \ln{\left(|\mathbf{\Sigma}|^{L}\right)}-\ln{\left(\Gamma_m(L)\right)}-L\traco(\mathbf{\Sigma}^{-1}\mathbf {Z}), \\
	\ln{\left(f_{\mathbf{Z}}(\mathbf{Z};\mathbf{\Sigma},L)\right)}&=&mL\ln{L}+L\ln{\left(|\mathbf{Z}|\right)}-m\ln{\left(|\mathbf{Z}|\right)} - L\ln{\left(|\mathbf{\Sigma}|\right)}-\ln{\left(\Gamma_m(L)\right)}-L\traco(\mathbf{\Sigma}^{-1}\mathbf{Z}), \\
\end{array}
\end{equation}
lembrando que a função Gamma multivariada é definida na equação (\ref{cap_acf_10}) então, podemos rescrever a equação da seguinte forma
\begin{equation}\label{cap_acf_12}
\begin{array}{cll}
	\ln{\left(f_{\mathbf{Z}}(\mathbf{Z};\mathbf{\Sigma},L)\right)}&=&mL\ln{L}+L\ln{\left(|\mathbf{Z}|\right)}-m\ln{\left(|\mathbf{Z}|\right)} - L\ln{\left(|\mathbf{\Sigma}|\right)}-\ln{\left(\Gamma_m(L)\right)}-L\traco(\mathbf{\Sigma}^{-1}\mathbf{Z}), \\
	\ln{\left(f_{\mathbf{Z}}(\mathbf{Z};\mathbf{\Sigma},L)\right)}&=&mL\ln{L}+L\ln{\left(|\mathbf{Z}|\right)}-m\ln{\left(|\mathbf{Z}|\right)} - L\ln{\left(|\mathbf{\Sigma}|\right)}\\
	&-&\ln{\left(\pi^{\frac{1}{2}m(m-1)} \prod_{i=0}^{m-1}\Gamma(L-i)\right)}-L\traco(\mathbf{\Sigma}^{-1}\mathbf{Z}),\\
	\ln{\left(f_{\mathbf{Z}}(\mathbf{Z};\mathbf{\Sigma},L)\right)}&=&mL\ln{L}+L\ln{\left(|\mathbf{Z}|\right)}-m\ln{\left(|\mathbf{Z}|\right)} - L\ln{\left(|\mathbf{\Sigma}|\right)}\\
        &-&\ln{\left(\pi^{\frac{1}{2}m(m-1)}\right)}-\ln{\left( \prod_{i=0}^{m-1}\Gamma(L-i)\right)}-L\traco(\mathbf{\Sigma}^{-1}\mathbf{Z}), \\
	\ln{\left(f_{\mathbf{Z}}(\mathbf{Z};\mathbf{\Sigma},L)\right)}&=&mL\ln{L}+L\ln{\left(|\mathbf{Z}|\right)}-m\ln{\left(|\mathbf{Z}|\right)} - L\ln{\left(|\mathbf{\Sigma}|\right)}\\
        &-&\frac{1}{2}m(m-1)\ln{\left(\pi\right)}-\sum_{i=0}^{m-1}\ln{\left(\Gamma(L-i)\right)}-L\traco(\mathbf{\Sigma}^{-1}\mathbf{Z}),\\
\end{array}
\end{equation}
equação equivalente pode ser encontrada em \citep{fnc2011}.


\section{Modelos para dados dados}

A matriz de espalhamento complexa {\boldmath S} é definida por
$$
\mathbf{ s} = \left[
\begin{array}{cc}
	S_{hh}   & S_{hv}   \\
	S_{vh}   & S_{vv}   \\
\end{array}
\right].
$$
% % % ACF Dizer o que significam "h" e "v"

Usaremos o caso do meio de propagação ser recíproco, isto é, $S_{hv}=S_{vh}$ tornando a matriz de espalhamento simétrica. Podemos facilitar a notação representando a matriz de espalhamento por um vetor da seguinte forma
$$
\mathbf{s} = \left[
\begin{array}{c}
	S_{vv}      \\
	S_{vh}     \\
	S_{hh}      \\
\end{array}
\right].
$$
% % % ACF Não é sempre um fato. Ocorre na maioria das vezes, e em português é "meio recíproco".

De acordo com \cite{good} a distribuição gaussiana complexa multivariada pode modelar adequadamente o comportamento estatístico de $\boldmath S$. Isto é chamado de {\it single-look complex PolSAR data representation} e podemos definir o vetor de espalhamento por $\mathbf{s}=[S_1,S_2,\dots,S_p]^T$. 
% % % ACF Acrescentei "complex"

A função densidade de probabilidade ({\boldmath pdf}) da distribuição gaussiana complexa $p-$variada é dada por
\begin{equation}\label{eqn1}
	p(\mathbf{s})=\frac{1}{\pi^p|\Sigma_{\mathbf{s}}|}\exp(-\bar{\mathbf{s}}^{T}\Sigma_{\mathbf{s}}^{-1}\mathbf{s}).
\end{equation}
O parâmetro que indexa a distribuição é a matriz de covariância, que é definida por:
\begin{equation}\label{eqn2}
	\mathbf{ \Sigma_{s}} = E[\mathbf{ss}^H] = \left[
\begin{array}{cccc}
	E(\mathbf{s_1s_1}^H)  & E(\mathbf{s_1s_2}^H) &\hdots & E({\mathbf s_1s_p}^H) \\
	E(\mathbf{ s_2s_1}^H)  & E(\mathbf {s_2 s_2}^H) &\hdots &E(\mathbf {s_2 s_p}^H)\\
        \vdots&\vdots &\ddots &\vdots\\
	E(\mathbf{ s_ps_1}^H)  & E(\mathbf {s_ps_2}^H) &\hdots &E(\mathbf {s_ps_p}^H)\\
\end{array}
\right].
\end{equation}
onde $E(\cdot)$ e $(\cdot)^H$ denotam o valor esperado e o conjugado transposto.

A matriz {\boldmath$\Sigma_{\mathbf{s}}$} é hermitiana pois se $\mathbf {S_j}= x_j+iy_j $

\begin{equation}\label{eqn3}
\begin{array}{ccc}
\mathbf{S}_j\overline{\mathbf{S}}_j&=& (x_j+iy_j)\overline{(x_j+iy_j)} \\
\mathbf{S}_j\overline{\mathbf{S}}_j&=& (x_j+iy_j)(x_j-iy_j) \\
\mathbf{S}_j\overline{\mathbf{S}}_j&=& x_j^2+y_j^2 \\
\end{array}
\end{equation}
considerando $j \neq k$
\begin{equation}\label{eqn4}
\begin{array}{ccc}
\mathbf{S}_j\overline{\mathbf{S}}_k&=& (x_j+iy_j)\overline{(x_k+iy_k)} \\
\mathbf{S}_j\overline{\mathbf{S}}_k&=& (x_j+iy_j)(x_k-iy_k) \\
\mathbf{S}_j\overline{\mathbf{S}}_k&=& (x_jx_k+y_jy_k)+i(x_ky_j-x_jy_k) \\
\end{array}
\end{equation}
ainda,
\begin{equation}\label{eqn5}
\begin{array}{ccc}
	\overline{\mathbf{S}_k\overline{\mathbf{S}}}_j&=&\overline{ (x_k+iy_k)\overline{(x_j+iy_j)} }\\
	\overline{\mathbf{S}_k\overline{\mathbf{S}}}_j&=&\overline{ (x_k+iy_k)(x_j-iy_j)} \\
	\overline{\mathbf{S}_k\overline{\mathbf{S}}}_j&=&\overline{ (x_kx_j+y_ky_j)+i(x_jy_k-x_ky_j) }\\
	\overline{\mathbf{S}_k\overline{\mathbf{S}}}_j&=&(x_kx_j+y_ky_j)-i(x_jy_k-x_ky_j) \\
	\overline{\mathbf{S}_k\overline{\mathbf{S}}}_j&=&(x_kx_j+y_ky_j)+i(x_ky_j-x_jy_k) \\
\end{array}
\end{equation}
Portanto1
\begin{equation}\label{eqn6}
\begin{array}{ccc}
	\mathbf{S}_j\overline{\mathbf{S}}_j&=&\overline{\mathbf{S}_j\overline{\mathbf {S}}}_j \\
	\mathbf{S}_j\overline{\mathbf{S}}_k&=&\overline{\mathbf{S}_k\overline{\mathbf {S}}}_j \\
\end{array}
\end{equation}
Assim com $j$ e $k$ varrendo toda a matriz podemos afirmar que $\mathbf{\Sigma_{\mathbf{s}}}=\mathbf{\Sigma_{ s}}^H$ portanto hermitiana.


Dados polarimétricos são usualmente sujeitados a um processo {\it multilook} com o intuito de melhorar a razão sinal-ruído.
% % % ACF "relação senhal-ruído"
Para esse fim, matrizes positivas definidas hermitianas são obtidas computando a médias de $L$ visadas 
% % % ACF "visadas", mas ninguém usa
independentes de uma mesma cena. Isto resulta na matriz de covariância {\boldmath{Z}} dada por:
\begin{equation}\label{eqn7}
	\mathbf{Z}=\frac{1}{L}\sum_{i=1}^{L} \mathbf{s_is_i}^H .
\end{equation}
% % % ACF Código LaTeX simplificado

\subsubsection{Coeficiente de correlação {\it Multilook}}

O coeficiente de correlação complexo é um importante parâmetro para descrever a função de densidade de probabilidade. 
Podemos defini-lo como
\begin{equation}\label{eqn8}
	\rho_c=\frac{E[\mathbf{s_is_j}^H]}{\sqrt{E[|\mathbf{s_i}|^2]E[|\mathbf{s_j}|^2]}} =|\rho_c|e^{i\theta}.
\end{equation}
% % % ACF Não complique LaTeX
em que {\boldmath $s_i$} e {\boldmath $s_j$} 
% % % ACF Repare que o negrito matemático se faz de outra maneira
são duas componentes da matriz de espalhamento ou dois retorno do radar polarimétrico ou interferométrico SAR. 
Para dados de radar polarimétricos representado pela matriz de Mueller, $\rho_c$ pode ser calculado encontrando a média da vizinhança de um pixel de uma matriz Mueller. A magnitude de $\rho_c$ pode também ser estimada usando duas intensidade  {\it multilook} $Z_{ii}$ e $Z_{jj}$. O coeficiente de correlação de dados $L$ looks intensidade é definida como   
% % % ACF Eram L looks, agora são n. Unificar a notação
\begin{equation}\label{eqn9}
	\rho_I^{(n)}=\frac{E[(Z_{ii}-\overline{Z_{ii}})(Z_{jj}-\overline{Z_{jj}})]}{\sqrt{E[(Z_{ii}-\overline{Z_{ii}})^2][(Z_{jj}-\overline{Z_{jj}})^2]}}. \\
\end{equation}

No apêndice do artigo \cite{lee} foi mostrado que 

\begin{equation}\label{eqn10}
	\rho_I^{(n)}= |\rho_c|^2\\
\end{equation}

Sendo 

\begin{equation}\label{eqn11}
\begin{array}{ccc}
	\mathbf{S_i}&=&a_{R}+ia_{I} \\
        \mathbf{S_j}&=&b_{R}+ib_{I} \\
\end{array}
\end{equation}

Assim a equação (\ref{eqn8}) pode ser reescrita

\begin{equation}\label{eqn12}
\begin{array}{ccc}
	\rho_c&=&\frac{E[(a_{R}+ia_{I})\overline{(b_{R}+ib_{I})}]}{\sqrt{E[a_{R}^2+a_{I}^2]E[b_{R}^2+b_{I}^2]}}. \\
	\rho_c&=&\frac{E[(a_{R}+ia_{I})(b_{R}-ib_{I})]}{\sqrt{E[a_{R}^2+a_{I}^2]E[b_{R}^2+b_{I}^2]}}. \\
	\rho_c&=&\frac{E[a_{R}b_{R}+ia_{I}b_{R}-ia_{R}b_{I}+a_{I}b_{I}]}{\sqrt{E[a_{R}^2+a_{I}^2]}\sqrt{E[b_{R}^2+b_{I}^2]}}. \\
	\rho_c&=&\frac{E[a_{R}b_{R}+ia_{I}b_{R}-ia_{R}b_{I}+a_{I}b_{I}]}{\sqrt{E[a_{R}^2+a_{I}^2]}\sqrt{E[b_{R}^2+b_{I}^2]}}. \\
\end{array}
\end{equation}
Definindo os desvios padrões,
\begin{equation}\label{eqn13}
\begin{array}{ccc}
	\sigma_{a}	&=&\sqrt{E[a_{R}^2+a_{I}^2]} \\
	\sigma_{b}      &=&\sqrt{E[b_{R}^2+b_{I}^2]} \\
\end{array}
\end{equation}

\begin{equation}\label{eqn14}
\begin{array}{ccc}
	\rho_c&=&\frac{E[a_{R}b_{R}+ia_{I}b_{R}-ia_{R}b_{I}+a_{I}b_{I}]}{\sigma_a\sigma_b}. \\
	\rho_c&=&\frac{E[a_{R}b_{R}+a_{I}b_{I}+i(a_{I}b_{R}-a_{R}b_{I})]}{\sigma_a\sigma_b}. \\
	\rho_c&=&\frac{E[a_{R}b_{R}]+E[a_{I}b_{I}]+i(E[a_{I}b_{R}]-E[a_{R}b_{I})]}{\sigma_a\sigma_b}. \\
	\rho_c&=&\frac{E[a_{R}b_{R}]}{\sigma_a\sigma_b}+\frac{E[a_{I}b_{I}]}{\sigma_a\sigma_b}+i\left(\frac{E[a_{I}b_{R}]}{\sigma_a\sigma_b}-\frac{E[a_{R}b_{I}]}{\sigma_a\sigma_b}\right). \\
\end{array}
\end{equation}
Definindo
\begin{equation}\label{eqn15}
\begin{array}{ccccccccc}
	\rho_{RR}=\frac{E[a_{R}b_{R}]}{\sigma_a\sigma_b},&&\rho_{II}=\frac{E[a_{I}b_{I}]}{\sigma_a\sigma_b},&&\rho_{IR}=\frac{E[a_{I}b_{R}]}{\sigma_a\sigma_b},&&\rho_{RI}=\frac{E[a_{R}b_{I}]}{\sigma_a\sigma_b}. \\
\end{array}
\end{equation}

Portanto, 
\begin{equation}\label{eqn16}
	\rho_c=\frac{(\rho_{RR}+\rho_{II})+i(\rho_{IR}-\rho_{RI})}{2}. \\
\end{equation}

\textcolor{red}{obs:Explicar melhor o fator 2}

Devido a condição de ser gaussiana circular

\begin{equation}\label{eqn17}
	\rho_{RR}=\rho_{II},\quad \rho_{IR}=-\rho_{RI}. \\
\end{equation}
podemos escrever $\rho_c$
\begin{equation}\label{eqn18}
	\rho_c=\rho_{RR}+i\rho_{IR}. \\
\end{equation}

Portanto

\begin{equation}\label{eqn19}
	|\rho_c|^2=\rho_{RR}^2+\rho_{IR}^2. \\
\end{equation}

O processo de {\bf Multilook} produz
\begin{equation}\label{eqn20}
\begin{array}{ccc}
	A_n&=&\frac{1}{n}\sum_{k=1}^{n} [a_{R}^2(k)+a_{I}^2(k)]. \\
	B_n&=&\frac{1}{n}\sum_{k=1}^{n} [b_{R}^2(k)+b_{I}^2(k)]. \\
\end{array}
\end{equation}

Assumindo a independência estatística entre amostras, a média e o desvio padrão podem ser definidos por
\begin{equation}\label{eqn21}
\begin{array}{cccccccccccc}
	\overline{A_n}&=&E[A_n]&=&2E[a_{R}^2(k)]&=&2\sigma_a^2,&SD[A_n]&=&\frac{2\sigma_a^2}{\sqrt{n}}.\\
	\overline{B_n}&=&E[B_n]&=&2E[b_{R}^2(k)]&=&2\sigma_b^2,&SD[B_n]&=&\frac{2\sigma_b^2}{\sqrt{n}}.\\
\end{array}
\end{equation}

O coeficiente de correlação {\it Multilook} para intensidade (equação (\ref{eqn9})) pode ser escrito por:

\begin{equation}\label{eqn22}
	\rho_I^{(n)}=\frac{E[(A_n-\overline{A_n})(B_n-\overline{B_n})]}{SD[A_n]SD[B_n]}. \\
	%\rho_I^{(n)}&=&\frac{E[(A_nB_n-A_n\overline{B_n}-\overline{A_n}B_n+\overline{A_n}\overline{B_n}]}{SD[A_n]SD[B_n]}. \\
	%\rho_I^{(n)}&=&\frac{E[(A_nB_n]-E[A_n\overline{B_n}]-E[\overline{A_n}B_n]+E[\overline{A_n}\overline{B_n}]}{SD[A_n]SD[B_n]}. \\
	%\rho_I^{(n)}&=&\frac{E[(A_nB_n]-E[A_n\overline{B_n}]-E[\overline{A_n}B_n]+E[\overline{A_n}\overline{B_n}]}{SD[A_n]SD[B_n]}. \\
\end{equation}

Assumindo a independência entre as amostras e depois de algumas manipulações algébricas para o numerador da equação (\ref{eqn22}). 
\begin{equation}\label{eqn23}
	E[(A_n-\overline{A_n})(B_n-\overline{B_n})]=\frac{1}{n^2}\sum_{k=1}^{n}[E[(a_{R}^2(k)+a_{I}^2(k))(b_{R}^2(k)+b_{I}^2(k))]-4\sigma_a^2\sigma_b^2] \\
\end{equation}
\textcolor{red}{OBS: Entender melhor a equação (\ref{eqn23}) e (\ref{eqn24})}.
\begin{equation}\label{eqn24}
	E[(A_n-\overline{A_n})(B_n-\overline{B_n})]=\frac{4}{n}\sigma_a^2\sigma_b^2|\rho_c|^2\\
\end{equation}

Agora substituindo em (\ref{eqn22})

\begin{equation}\label{eqn25}
\begin{array}{ccc}
	\rho_I^{(n)}&=&\frac{\frac{4}{n}\sigma_a^2\sigma_b^2|\rho_c|^2}{SD[A_n]SD[B_n]}. \\
	\rho_I^{(n)}&=&\frac{\frac{4}{n}\sigma_a^2\sigma_b^2|\rho_c|^2}{\frac{2\sigma_a^2}{\sqrt{n}}\frac{2\sigma_b^2}{\sqrt{n}}}. \\
\end{array}
\end{equation}

completando as simplificaçãoes

\begin{equation}\label{eqn26}
	\rho_I^{(n)}=|\rho_c|^2. \\
\end{equation}

\textcolor{blue}{OBS: Esta relação mostra que o coeficiente de correlação da intensidade não depende dos {\it nlooks}.}



\subsubsection{Distribuição conjunta do {\it Multilook} $|S_i|^2$ e $|S_j|^2$ }

O $PDF$ conjunto retorna de dois canais correlacionados dos radares polarimétricos e interferométricos são importantes. As $PDF's$ conjuntas conduzem a derivação da intensidade e amplitude razão $PDF's$. Da equação (\ref{eqn42}) temos que as intensidades {\it multilook} sejam 

\begin{equation}\label{eqn59}
\begin{array}{ccccc}
	R_1&=&\frac{1}{n}\sum_{k=1}^{n}|S_i(k)|^2&=&\frac{B_1C_{11}}{n}\\
	R_2&=&\frac{1}{n}\sum_{k=1}^{n}|S_j(k)|^2&=&\frac{B_2C_{22}}{n}\\
\end{array}
\end{equation}

Integrando a equação (\ref{eqn52}) em relação a $\eta$ e $\psi$. A $PDF$ é

\begin{equation}\label{eqn60}
	p(B_1,B_2)=\frac{\left(B_1B_2\right)^{\frac{n-1}{2}}\exp\left(-\frac{B_1+B_2}{1-|\rho_c|^2}\right)}{\Gamma(n)(1-|\rho_c|^2)|\rho_c|^{n-1}}I_{n-1}\left(2\sqrt{B_1B_2}\frac{|\rho_c|}{1-|\rho_c|^2}\right)
\end{equation}

Sendo
\begin{equation}\label{eqn61}
	I_{\mu}(Z)=\frac{(\frac{z}{2})^{\mu}}{\Gamma(\mu+1)} F_{1}^{0}[-;\mu+1;\frac{z^2}{4}]
\end{equation}

\textcolor{red}{OBS: As integrações na equação (\ref{eqn52}) não foram realizadas neste estudo.}
\begin{equation}\label{eqn62}
	p(B_1,B_2)=\frac{n^{n+1}\left(R_1R_2\right)^{\frac{n-1}{2}}\exp\left(-\frac{n(\frac{R_1}{C_{11}}+\frac{R_2}{C_{22}})}{1-|\rho_c|^2}\right)}{(C_{11}C_{22})^{\frac{n+1}{2}}\Gamma(n)(1-|\rho_c|^2)|\rho_c|^{n-1}}I_{n-1}\left(2n\sqrt{\frac{R_1R_2}{C_{11}C_{22}}}\frac{|\rho_c|}{1-|\rho_c|^2}\right)
\end{equation}

\textcolor{red}{OBS: Verificar o surgimento de um fator $\frac{n^2}{C_{11}C_{22}}$ na equação  (\ref{eqn62}) - Mudança de variável!!!!!.}

\subsubsection{Distribuição razão intensidade e amplitude para {\it multilook}}

A razão de intensidade e amplitude entre $S_{hh}$ e $S_{vv}$ são importantes no estudo de radares polarimétricos. A $PDF's$ razão de intensidade e amplitude normalizada será mostrada agora

\begin{equation}\label{eqn63}
\begin{array}{ccccccc}
	\mu&=&\frac{B_1}{B_2}&=&\frac{\sum_{k=1}^{n}\frac{|S_i(k)|^2}{C_{11}}}{\sum_{k=1}^{n}\frac{|S_j(k)|^2}{C_{22}}}&=&\frac{\sum_{k=1}^{n}|S_i(k)|^2}{\tau\sum_{k=1}^{n}|S_j(k)|^2}\\
\end{array}
\end{equation}

Onde $\tau=\frac{C_{11}}{C_{22}}$.

A $PDF$ razão intensidade {\it multlook} normalizada é mostrada no apêndice $(C)$ do artigo \cite{lee}  


\begin{equation}\label{eqn64}
	p^{(n)}(\mu)=\frac{\Gamma(2n)(1-|\rho_c|^2)^{n}(1+\mu)\mu^{n-1}}{\Gamma(n)\Gamma(n)\left[(1+\mu)^2-4|\rho_c|^2\mu \right]^{\frac{2n+1}{2}}}\\
\end{equation}

\textcolor{red}{OBS: Não realizei as contas do apêndice $(C)$.}

Realizando a troca de variável $\nu=\sqrt{\mu}$ a equação (\ref{eqn64}) pode ser rescrita por
\begin{equation}\label{eqn65}
	p^{(n)}(\nu)=\frac{2\Gamma(2n)(1-|\rho_c|^2)^{n}(1+\nu^2)\nu^{2n-1}}{\Gamma(n)\Gamma(n)\left[(1+\nu^2)^2-4|\rho_c|^2\nu^2 \right]^{\frac{2n+1}{2}}}\\
\end{equation}
As $PDF's$ razão de intensidade e amplitude entre os {\it multilook} $\mathbf{S_1}$ e $\mathbf{S_2}$ podem ser facilmente deduzidas das seguintes definições e posterior aplicação nas equações (\ref{eqn64}) e (\ref{eqn65}). definindo 
\begin{equation}\label{eqn66}
\begin{array}{ccccc}
	w&=&\frac{\sum_{k=1}^{n}|S_i(k)|^2}{\sum_{k=1}^{n}|S_i(k)|^2}&=&\tau\mu\\
	z&=&\sqrt{w}&=&\sqrt{\tau}\nu
\end{array}
\end{equation}
Portanto a distribuição da razão $w$ de intensidade {\it multilook} é
\begin{equation}\label{eqn67}
	p^{(n)}(w)=\frac{\tau^{n}\Gamma(2n)(1-|\rho_c|^2)^{n}(\tau+w)w^{n-1}}{\Gamma(n)\Gamma(n)\left[(\tau+w)^2-4\tau|\rho_c|^2w \right]^{\frac{2n+1}{2}}}.
\end{equation}
Portanto a distribuição da razão $z$ de amplitude {\it multilook} é
\begin{equation}\label{eqn68}
	p^{(n)}(z)=\frac{\tau^{n}\Gamma(2n)(1-|\rho_c|^2)^{n}(\tau+z^2)z^{2n-1}}{\Gamma(n)\Gamma(n)\left[(\tau+z^2)^2-4\tau|\rho_c|^2z^2 \right]^{\frac{2n+1}{2}}}.
\end{equation}

A discusão será limitada para estatística da razão $\nu$ amplitude normalizada. A figura (\ref{fig2}) mostra a distribuição razão amplitude apresentada na equação  (\ref{eqn65}). Notadamente a medida que $n$ aumenta tendemos a ter uma aproximação da "função" delta de Dirac e uma concentração em torno da abscissa $\nu=1$.

\textcolor{blue}{OBS: Processos de {\it multilook} reduzem a variância.}


\section{Metodologia}

Os modelos marginais são derivados dos modelos mariados Gamma como a distribuíção complexa Wishart.
 Nesta seção serão apresentados os modelos marginais usados no trabalho.

\section{Número Equivalente de Visadas}
O coeficiente de variação definido como a razão do desvio padrãoem uma imagem com a média para as intenidades na imagem PolSAR $\text{CV}=\frac{\sqrt{\text{Var(I)}}}{\text{E(I)}}$ é um bom indicador para no nível de ruído \textit{Speckle}. O coeficiente de variação (CV) será usado para definir o número equivalente de visadas para a intensidade (ENL(I)), 
\begin{equation}\label{eq:ENL}
	\text{ENL(I)} = \frac{1}{\text{CV}^2}, \\
\end{equation}
para mais detalhes codemos consultar \citet{ljdwo}.
\section{Modelos de Densidades de Probabilidades Marginais}
\subsection{Função Densidade de Probabilidade Univariada Gamma}
Considerando a função distribuição de densidade univariada gaussiana 
\begin{equation}\label{pdf_gauss_univ}
	f_{Z}(z;\mu,L)=\frac{L^{L}}{\Gamma(L)\mu^{L}} z^{L-1}\exp\left\{-L\frac{z}{\mu}\right\}, \\
\end{equation}
onde, $\mu>0$ e $L>0$. Aplicando o logaritmo natural na equação~\eqref{pdf_gauss_univ}  e realizando algumas manipulações algébricas teremos:
\begin{equation}\nonumber
\begin{array}{ccl}
	\ln f_{Z}(z;\mu,L)&=&\ln \left(\frac{L^{L}}{\Gamma(L)\mu^{L}} z^{L-1}\exp\left\{-L\frac{z}{\mu}\right\}\right), \\
	                                         &=&\ln\left(\frac{L}{\mu}\right)^L-\ln\Gamma(L)+ \ln z^{L-1} + \ln \exp\left\{-L\frac{z}{\mu}\right\}, \\
%	                                         &=&L\ln\frac{L}{\mu}-\ln\Gamma(L)+(L-1)\ln z - \frac{L}{\mu} z,\\	                                         
\end{array}
\end{equation}
resultanto na função,
\begin{equation}\label{func_log_univ_gaussiana}
	\ln f_{Z}(z;\mu,L)=L\ln\frac{L}{\mu}-\ln\Gamma(L)+(L-1)\ln z - \frac{L}{\mu} z.
\end{equation}

A função log-verossimilhança pode ser deduzida da seguinte maneira, dado a amostra $\bm z = (z_1,\dots,z_n)$, 
\begin{equation}\nonumber
\begin{split}
  \ell(\bm z;\mu, L)=\ln\prod_{k=1}^{n}f_Z(z_k;\mu,L)\\
  \ell(\bm z;\mu, L)=\sum_{k=1}^{n}\ln f_Z(z_k;\mu,L),
 \end{split}
 \end{equation}
usando a função~\eqref{func_log_univ_gaussiana} teremos,
\begin{equation}\nonumber
\begin{split}
    \ell(\bm z;\mu, L)&=\sum_{k=1}^{n}\ln f_Z(z_k;\mu,L)\\
                      &=\sum_{k=1}^{n}\left[L\ln\frac{L}{\mu}-\ln\Gamma(L)+(L-1)\ln z_k - \frac{L}{\mu} z_k\right]\\
                      &=\sum_{k=1}^{n}L\ln\frac{L}{\mu}-\sum_{k=1}^{n}\ln\Gamma(L)+(L-1)\sum_{k=1}^{n}\ln z_k - \frac{L}{\mu}\sum_{k=1}^{n} z_k\\
                      &=L\ln\frac{L}{\mu}\sum_{k=1}^{n}1-\ln\Gamma(L)\sum_{k=1}^{n}1+(L-1)\sum_{k=1}^{n}\ln z_k - \frac{L}{\mu}\sum_{k=1}^{n} z_k\\
                      &=L\ln\frac{L}{\mu}n-\ln\Gamma(L)n+(L-1)\sum_{k=1}^{n}\ln z_k - \frac{L}{\mu}\sum_{k=1}^{n} z_k.\\                
 \end{split}
 \end{equation}

Definimos a equação log-verossimilhança para a PDF univariada~(\ref{func_log_univ_gaussiana}).
\begin{equation}\nonumber
    \ell(\bm z;\mu, L)=n\left[L\ln\frac{L}{\mu}-\ln\Gamma(L)\right]+(L-1)\sum_{k=1}^{n}\ln z_k - \frac{L}{\mu}\sum_{k=1}^{n} z_k,\\                
 \end{equation}
e a forma reduzida, 
\begin{equation}
\ell(\bm z;\mu, L) = 
n \left[L\ln \frac{L}{\mu} - \ln \Gamma(L)\right]
+L \sum_{k=1}^{n}\ln z_k -\frac{L}{\mu}\sum_{k=1}^{n} z_k.
\label{eq:LogLikelihoodGamma_red}
\end{equation}


Vamos obter $(\widehat L, \widehat \mu)$, o estimador de máxima verossimilhança (MLE) de $(L, \mu)$ baseado $\bm z$, por maximizar~\eqref{eq:LogLikelihoodGamma_red} com o método BFGS implementado no pacote \texttt{maxLik}~\citep{ht}. Vamos preferir otimização resolvendo $\nabla\ell=\bm 0$ com intuito de melhorar a estabilidade numérica.

O função é a log-verossimilhança reduzida para as amostras internas e externas da faixa de dados denotadas respectivamento como $\bm z_\text{I}$ e $\bm z_\text{E}$. Cada faixa de dados $\bm z = (z_1,z_2,\dots,z_n)$ é particionada em duas amostras disjuntas na posição $j$:  
$$
\bm z = (\underbrace{z_1,z_2,\dots,z_j}_{\bm z_\text{I}}, 
\underbrace{z_{j+1}, z_{j+2},\dots,z_n}_{\bm z_\text{E}}).
$$
Vamos assumir dois diferentes modelos para cada partição:
$\bm Z_\text{I} \sim \Gamma(\mu_\text{I},L_\text{I})$, e
$\bm Z_\text{E} \sim \Gamma(\mu_\text{E},L_\text{E})$.
Vamos estimar $(\mu_\text{I},L_\text{I})$ e $(\mu_\text{E},L_\text{E})$ com $\bm z_\text{I}$ e $\bm z_\text{E}$, respectivamente, maximizando~\eqref{eq:LogLikelihoodGamma_red}, e obtendo $(\widehat{\mu}_\text{I}, \widehat{L}_\text{I})$ e $(\widehat{\mu}_\text{E}, \widehat{L}_\text{E})$.


A log-verossimilhança no ponto $j$ é, então
\begin{equation}\label{eq:TotalLogLikelihood}
\begin{split}
%\ell(j&;\bm z_\text{I},\bm z_\text{E}) = \\
\ell(j&;\widehat{\mu}_I, \widehat{L}_I,\widehat{\mu}_E, \widehat{L}_E)=\\
&j \big[\widehat{L}_\text{I}\ln (\widehat{L}_\text{I} / \widehat{\mu}_\text{I}) - \ln \Gamma(\widehat{L}_\text{I})\big]
+\widehat{L}_\text{I} \sum_{k=1}^{j}\ln z_k -\frac{\widehat{L}_\text{I}}{\widehat{\mu}_\text{I}}\sum_{k=1}^{j} z_k +\\
&(n-j) \big[\widehat{L}_\text{E}\ln (\widehat{L}_\text{E} / \widehat{\mu}_\text{E}) - \ln \Gamma(\widehat{L}_\text{E})\big]
+\widehat{L}_\text{E} \sum_{k=j+1}^{n}\ln z_k - \frac{\widehat{L}_\text{E}}{\widehat{\mu}_\text{E}}\sum_{k=j+1}^{n} z_k.
\raisetag{2.2em}
\end{split}
\end{equation}
Vamos aplicar o método GenSA para encontrar
$$
\widehat{\jmath}= \arg\max\limits_{j\in [\min_s,N-\min_s]}\ell(j;\widehat{\mu}_I, \widehat{L}_I,\widehat{\mu}_E, \widehat{L}_E),
$$ 
onde $\min_s$ é o tamanho mínimo da amostra definido por $14$.

Desta maneira, vamos obter uma estimativa para a borda em cada canal de intensidade.
Note que esse método pode ser estendido e/ou modificado para lidar com qualquer tipo de dados.

%\subsubsection*{Aplicação em imagem simulada}
%
%O método da máxima verossimilhança \eqref{eq:TotalLogLikelihood} foi aplicado na imagem simulada com duas amostras, e as evidências de bordas estão mostradas na figura. \textcolor{red}{Base de dados gamf}
% \begin{figure*}[hbt]
%	\centering
%     \subfloat[Evidências no canal $\text{hh}$ \label{evidencias_hh_hv_vv_gamf:a}]{%
%       \includegraphics[width=0.5\linewidth]{im_sim_gamf_hh_evid_param_L_mu_14_pixel}
%     }
%     \subfloat[Evidências no canal $\text{hv}$ \label{evidencias_hh_hv_vv_gamf:b}]{%
%       \includegraphics[width=0.5\linewidth]{im_sim_gamf_hv_evid_param_L_mu_14_pixel}
%     }      
%   %  \subfloat[Evidências no canal $\text{vv}$ \label{evidencias_hh_hv_vv_gamf:c}]{%
%    %   \includegraphics[width=0.5\linewidth]{im_sim_gamf_hh_evid_param_L_mu_14_pixel}
%    % }
%    \caption{Evidências de bordas para os três canais de intensidade}
%     \label{evidencias_hh_hv_vv_gamf} 
%   \end{figure*}
%   
%   \begin{figure*}[hbt]
%	\centering
%     \subfloat[Evidências no canal $\text{vv}$ \label{evidencias_hh_hv_vv_gamf:c}]{%
%       \includegraphics[width=0.5\linewidth]{im_sim_gamf_vv_evid_param_L_mu_14_pixel}
%     }
%    \caption{Evidências de bordas para os três canais de intensidade}
%     \label{evidencias_hh_hv_vv_gamf} 
%   \end{figure*}
\subsection{Função Densidade de Probabilidade Univariada produto de magnitudes das intensidades}
%A magnitude do produto $\mathbf{S}_i$ e $\mathbf{S}_j$ é uma importante medida para as imagem SAR polarimétrica, definimos a magnitude normalizada por 
%\begin{equation}
%	\bm\xi = \frac{\left|\frac{1}{L} \sum_{k=1}^L\mathbf{S}_i(k)\mathbf{S}_j^H(k) \right|}{\sqrt{E[|\mathbf{S}_i|^2]E[|\mathbf{S}_i|^2]}}=\frac{g}{h}.
%\end{equation}
%onde é definido por $g=|\mathbf{S}_i\mathbf{S}_j^H|$ e $h=\sqrt{E[|\mathbf{S}_i|^2]E[|\mathbf{S}_i|^2]}$.

A  função densidade de probabilidade univariada produto de magnitudes das intensidades $z$ é definida por
\begin{equation}\label{eq:pdf_mag_prod}
\begin{array}{lcl}
	f(z;\rho, L)&=&\frac{4L^{L+1}z^L}{\Gamma(L)(1-|\rho|^2)}I_0\left(\frac{2|\rho|Lz}{1-|\rho|^2}\right)K_{L-1}\left(\frac{2Lz}{1-|\rho|^2}\right),
		\end{array}
\end{equation}
onde $I_0$ e $K_{L-1}$ são funções de Bessel modificadas,e onde, $\rho>0$ e $L>0$.


%A figura \eqref{fig:pdf_mag_prod} mostra a função densidade magnitude do produto com a variação das visadas $L=2,3,4$, e $8$, 
%\begin{figure}[hbt]
%\centering
%\includegraphics[width=4.0in]{dist_interferograma_multi_visadas.pdf}
%	\caption{Distribuição magnitude do produto múltiplas visadas.}
%\label{fig:pdf_mag_prod}
%\end{figure}

Aolicando  o logaritmo natural na equação~\eqref{eq:pdf_mag_prod} com o objetivo de construir o método de máxima verossimilhança, teremos
\begin{equation}\nonumber
\begin{split}
	\ln f(z;\rho,L)&=\ln\left(\frac{4L^{L+1}z^L}{\Gamma(L)(1-|\rho|^2)}I_0\left(\frac{2|\rho|Lz}{1-|\rho|^2}\right)K_{L-1}\left(\frac{2Lz}{1-|\rho|^2}\right)\right),\\
	&=\ln\left(\frac{4L^{L+1}z^L}{\Gamma(L)(1-|\rho|^2)}\right)+\ln I_0\left(\frac{2|\rho|Lz}{1-|\rho|^2}\right)+ \ln K_{L-1}\left(\frac{2Lz}{1-|\rho|^2}\right),\\
	&=\ln (4L^{L+1}z^L)-\ln(\Gamma(L)(1-|\rho|^2))+\ln I_0\left(\frac{2|\rho|Lz}{1-|\rho|^2}\right)+ \ln K_{L-1}\left(\frac{2Lz}{1-|\rho|^2}\right),\\
     &=\ln (4)+\ln L^{L+1}+\ln z^L-\ln\Gamma(L)-\ln(1-|\rho|^2)\\
     &+\ln I_0\left(\frac{2|\rho|Lz}{1-|\rho|^2}\right)+ \ln K_{L-1}\left(\frac{2Lz}{1-|\rho|^2}\right),\\
	&=\ln (4)+(L+1)\ln L+L\ln z-\ln\Gamma(L)-\ln(1-|\rho|^2)\\
	&+\ln I_0\left(\frac{2|\rho|Lz}{1-|\rho|^2}\right)+ \ln K_{L-1}\left(\frac{2Lz}{1-|\rho|^2}\right).
		\end{split}
\end{equation}

Definimos a função logarítmica para a densidade de probabilidade univariada produto de magnitudes das intensidades
\begin{equation}\label{eq:log_vero_mag_produto}
\begin{split}
	\ln f(\bm\xi; \rho,\L)&=\ln (4)+(L+1)\ln L+L\ln \bm\xi-\ln\Gamma(L)-\ln(1-|\rho|^2)\\
	                      &+\ln I_0\left(\frac{2|\rho|L\bm\xi}{1-|\rho|^2}\right)+ \ln K_{L-1}\left(\frac{2L\bm\xi}{1-|\rho|^2}\right).
	\end{split}
\end{equation}

A função log-verossimilhança pode ser deduzida da seguinte maneira, dado a amostra $\bm z = (z_1,\dots,z_n)$, 
\begin{equation}\nonumber
\begin{split}
  \ell(\bm z;\rho, L)=\ln\prod_{k=1}^{n}f(z_k;\rho,L)\\
  \ell(\bm z;\rho, L)=\sum_{k=1}^{n}\ln f(z_k;\rho,L),
 \end{split}
 \end{equation}
usando a função~\eqref{eq:log_vero_mag_produto} teremos,
\begin{equation}\nonumber
\begin{split}
    \ell(\bm z;\rho, L)&=\sum_{k=1}^{n}\ln f(z_k;\rho,L)\\
                         &=\sum_{k=1}^{n}\left[\ln (4)+(L+1)\ln L+L\ln z_k-\ln\Gamma(L)-\ln(1-|\rho|^2)\right.\\
                         &\left.+\ln I_0\left(\frac{2|\rho|Lz_k}{1-|\rho|^2}\right)+ \ln K_{L-1}\left(\frac{2Lz_k}{1-|\rho|^2}\right)\right],\\
	 \end{split}
 \end{equation}
 
 \begin{equation}\nonumber
\begin{split}
    \ell(\bm \xi;\rho, L)&=\ln (4)\sum_{k=1}^{n}1+(L+1)\ln L\sum_{k=1}^{n}1+L\sum_{k=1}^{n}\ln \xi_k-\ln\Gamma(L)\sum_{k=1}^{n}1-\ln(1-|\rho|^2)\sum_{k=1}^{n}1\\
                         &+\sum_{k=1}^{n}\ln I_0\left(\frac{2|\rho|L\xi_k}{1-|\rho|^2}\right)+ \sum_{k=1}^{n}\ln K_{L-1}\left(\frac{2L\xi_k}{1-|\rho|^2}\right)\\
                         &=n\ln (4)+n(L+1)\ln L+L\sum_{k=1}^{n} \ln\xi_k-n\ln\Gamma(L)-n\ln(1-|\rho|^2)\\
                         &+\sum_{k=1}^{n}\ln I_0\left(\frac{2|\rho|L\xi_k}{1-|\rho|^2}\right)+ \sum_{k=1}^{n}\ln K_{L-1}\left(\frac{2L\xi_k}{1-|\rho|^2}\right)\\
                         &=n\left[\ln (4)+(L+1)\ln L-\ln\Gamma(L)-\ln(1-|\rho|^2)\right]+L\sum_{k=1}^{n} \ln\xi_k\\
                         &+\sum_{k=1}^{n}\ln I_0\left(\frac{2|\rho|L\xi_k}{1-|\rho|^2}\right)+ \sum_{k=1}^{n}\ln K_{L-1}\left(\frac{2L\xi_k}{1-|\rho|^2}\right).\\
\end{split}
 \end{equation}
 
Definimos a equação log-verossimilhança para a PDF univariada~(\ref{eq:pdf_mag_prod}).
\begin{equation}\nonumber
\begin{split}
    \ell(\bm z;\rho, L)&=n\left[\ln (4)+(L+1)\ln L-\ln\Gamma(L)-\ln(1-|\rho|^2)\right]+L\sum_{k=1}^{n} \ln z_k\\
                         &+\sum_{k=1}^{n}\ln I_0\left(\frac{2|\rho|Lz_k}{1-|\rho|^2}\right)+ \sum_{k=1}^{n}\ln K_{L-1}\left(\frac{2Lz_k}{1-|\rho|^2}\right),\\
\end{split}
 \end{equation}
e a forma reduzida,
\begin{equation}\label{eq:eq_log_vero_mag_prod_red}
\begin{split}
    \ell(\bm z;\rho, L)&=n\left[(L+1)\ln L-\ln\Gamma(L)-\ln(1-|\rho|^2)\right]+L\sum_{k=1}^{n} \ln z_k\\
                         &+\sum_{k=1}^{n}\ln I_0\left(\frac{2|\rho|L z_k}{1-|\rho|^2}\right)+ \sum_{k=1}^{n}\ln K_{L-1}\left(\frac{2Lz_k}{1-|\rho|^2}\right).\\
\end{split}
 \end{equation} 


%
%As figuras \eqref{fig:prod_mag_l_50_r_35} até \eqref{fig:prod_mag_l_350_r_35} mostram a função de log-verossimilhança da pdf magnitude de produtos \eqref{eq:eq_log_vero_mag_prod_red} aplicada para a amostra de duas folhas simulada. No processo fixamos arbitrariamente a linha 35 da amostra e variamos o número de pixel entre 50, 150, 250 e 250. Desta maneira construímos duas funções $\ell{j}$, uma para cada lado da amostra.
% \begin{figure*}[hbt]
%	\centering
%     \subfloat[Pixel variando de 1 até 50 na linha 35.  \label{fig:prod_mag_l_rho_1_50}]{%
%       \includegraphics[width=0.50\linewidth]{fig_pdf_mag_prod_r_35_1_to_50}
%     }
%     \subfloat[Pixel variando de 51 até 400 na linha 35. \label{fig:prod_mag_l_rho_51_400}]{%
%       \includegraphics[width=0.50\linewidth]{fig_pdf_mag_prod_r_35_51_to_400}
%     }
%     \caption{Funções de máxima verossimilhança produto de magnitude com pixel fixo 50.}
%     \label{fig:prod_mag_l_50_r_35} 
%   \end{figure*}
%   
%   \begin{figure*}[hbt]
%	\centering
%     \subfloat[Pixel variando de 1 até 150 na linha 35.  \label{fig:prod_mag_l_rho_1_150}]{%
%       \includegraphics[width=0.50\linewidth]{fig_pdf_mag_prod_r_35_1_to_150}
%     }
%     \subfloat[Pixel variando de 151 até 400 na linha 35. \label{fig:prod_mag_l_rho_151_400}]{%
%       \includegraphics[width=0.50\linewidth]{fig_pdf_mag_prod_r_35_151_to_400}
%     }
%     \caption{Funções de máxima verossimilhança produto de magnitude com pixel fixo 150.}
%     \label{fig:prod_mag_l_150_r_35} 
%   \end{figure*}
%
%\begin{figure*}[hbt]
%	\centering
%     \subfloat[Pixel variando de 1 até 250 na linha 35.  \label{fig:prod_mag_l_rho_1_250}]{%
%       \includegraphics[width=0.50\linewidth]{fig_pdf_mag_prod_r_35_1_to_250}
%     }
%     \subfloat[Pixel variando de 251 até 400 na linha 35. \label{fig:prod_mag_l_rho_251_400}]{%
%       \includegraphics[width=0.50\linewidth]{fig_pdf_mag_prod_r_35_251_to_400}
%     }
%     \caption{Funções de máxima verossimilhança produto de magnitude com pixel fixo 250.}
%     \label{fig:prod_mag_l_250_r_35} 
%   \end{figure*}
%   \begin{figure*}[hbt]
%	\centering
%     \subfloat[Pixel variando de 1 até 350 na linha 35.  \label{fig:prod_mag_l_rho_1_350}]{%
%       \includegraphics[width=0.50\linewidth]{fig_pdf_mag_prod_r_35_1_to_350}
%     }
%     \subfloat[Pixel variando de 351 até 400 na linha 35. \label{fig:prod_mag_l_rho_351_400}]{%
%       \includegraphics[width=0.50\linewidth]{fig_pdf_mag_prod_r_35_351_to_400}
%     }
%     \caption{Funções de máxima verossimilhança produto de magnitude com pixel fixo 350.}
%     \label{fig:prod_mag_l_350_r_35} 
%   \end{figure*}

%As figuras \eqref{fig:prod_mag_l_25_r_50_hh}, \eqref{fig:prod_mag_l_25_r_50_hv} e \eqref{fig:prod_mag_l_25_r_50_vv} mostram a função de log-verossimilhança da pdf magnitude de produtos \eqref{eq:eq_log_vero_mag_prod_red} aplicada na ROI da imagem de flevoland. No processo fixamos arbitrariamente a radial 50 da amostra e fixamos o número de pixel 50. Desta maneira construímos duas funções $\ell{j}$ em cada lado da radial. Os gráficos das funções foram gerados para os canais de intensidades.

%Na figura \eqref{fig:prod_mag_l_rho_1_25_hh} podemos identificar o problema da função ser plana dificultando muito o processo de encontrar o ponto de máximo gerando oscilação na função $\ell(j)$. Na \eqref{fig:prod_mag_l_rho_26_120_hh} ocorre o problema das funções de bessel serem infinitas quando seu argumento assume valores grande, também dificultando o cálculo do valor máximo.  

%\begin{figure*}[hbt]
%	\centering
%     \subfloat[Pixel variando de 1 até 25 na radial 50 no canal (hh).  \label{fig:prod_mag_l_rho_1_25_hh}]{%
%       \includegraphics[width=0.50\linewidth]{fig_pdf_mag_prod_r_50_1_to_25_flev}
%     }
%     \subfloat[Pixel variando de 26 até 120 na radial 50 no canal (hh). \label{fig:prod_mag_l_rho_26_120_hh}]{%
%       \includegraphics[width=0.50\linewidth]{fig_pdf_mag_prod_r_50_26_to_120_flev}
%     }
%     \caption{Funções de máxima verossimilhança produto de magnitude com pixel fixo 25 no canal (hh).}
%     \label{fig:prod_mag_l_25_r_50_hh} 
%   \end{figure*}
   
%Nas figuras de \eqref{fig:prod_mag_l_25_r_50_hv} destacamos o problema da função ser plana. Assim como em no gráfico da função \eqref{fig:prod_mag_l_rho_1_25_hh} 
%\begin{figure*}[hbt]
%	\centering
%     \subfloat[Pixel variando de 1 até 25 na radial 50 no canal (hv).  \label{fig:prod_mag_l_rho_1_25_hv}]{%
%       \includegraphics[width=0.50\linewidth]{fig_pdf_mag_prod_r_50_1_to_25_flev_hv}
%     }
%     \subfloat[Pixel variando de 26 até 120 na radial 50 no canal (hv). \label{fig:prod_mag_l_rho_26_120_hv}]{%
%       \includegraphics[width=0.50\linewidth]{fig_pdf_mag_prod_r_50_26_to_120_flev_hv}
%     }
%     \caption{Funções de máxima verossimilhança produto de magnitude com pixel fixo 25 no canal (hv).}
%     \label{fig:prod_mag_l_25_r_50_hv} 
%   \end{figure*}
%
%Nas figuras de \eqref{fig:prod_mag_l_25_r_50_vv} destacamos o problema da função ser plana. Assim como em no gráfico da função \eqref{fig:prod_mag_l_rho_1_25_hh} 
%
%\begin{figure*}[hbt]
%	\centering
%     \subfloat[Pixel variando de 1 até 25 na radial 50 no canal (vv).  \label{fig:prod_mag_l_rho_1_25_vv}]{%
%       \includegraphics[width=0.50\linewidth]{fig_pdf_mag_prod_r_50_1_to_25_flev_vv}
%     }
%     \subfloat[Pixel variando de 26 até 120 na radial 50 no canal (vv). \label{fig:prod_mag_l_rho_26_120_vv}]{%
%       \includegraphics[width=0.50\linewidth]{fig_pdf_mag_prod_r_50_26_to_120_flev_vv}
%     }
%     \caption{Funções de máxima verossimilhança produto de magnitude com pixel fixo 25 no canal (vv).}
%     \label{fig:prod_mag_l_25_r_50_vv} 
%   \end{figure*}


Vamos obter $(\widehat \rho, \widehat L)$, o estimador de máxima verossimilhança (MLE) de $(\rho, L)$ baseado $\bm \xi$, por maximizar~\eqref{eq:eq_log_vero_mag_prod_red} com o método BFGS implementado no pacote \texttt{maxLik}~\citep{ht}. Preferimos otimizar resolvendo $\nabla\ell=\bm 0$ com objetivo de melhorar a estabilidade numérica do método.

A função otimizada é a log-verossimilhança reduzida para as amostras internas e externas da faixa de dados denotadas respectivamente como $\bm z_\text{I}$ e $\bm z_\text{E}$. Cada faixa de dados $\bm z = (z_1,z_2,\dots,z_n)$ é particionada em duas amostras disjuntas na posição $j$,  
$$
\bm z = (\underbrace{z_1,z_2,\dots,z_j}_{\bm z_\text{I}}, 
\underbrace{z_{j+1}, z_{j+2},\dots,z_n}_{\bm z_\text{E}}).
$$

Vamos estimar $(\rho_\text{I},L_\text{I})$ e $(\rho_\text{E},L_\text{E})$ com $\bm z_\text{I}$ e $\bm z_\text{E}$, maximizando~\eqref{eq:eq_log_vero_mag_prod_red}, e obtendo $(\widehat{\rho}_\text{I}, \widehat{L}_\text{I})$ e $(\widehat{\rho}_\text{E}, \widehat{L}_\text{E})$.

A log-verossimilhança no ponto $j$ é
\begin{equation}\label{eq:TotalLogLikelihood_prod_mag}
\begin{split}
\ell(j;\widehat{\rho}_\text{I}, \widehat{L}_\text{I}, \widehat{\rho}_\text{E}, \widehat{L}_\text{E})&
=j\left[(\widehat{L}_\text{I}+1)\ln \widehat{L}_\text{I}-\ln\Gamma(\widehat{L}_\text{I})-\ln(1-|\widehat{\rho}_\text{I}|^2)\right]
+\widehat{L}_\text{I}\sum_{k=1}^{j} \ln z_k\\
&+\sum_{k=1}^{j}\ln I_0\left(\frac{2|\widehat{\rho}_\text{I}|\widehat{L}_\text{I}z_k}{1-|\widehat{\rho}_\text{I}|^2}\right)
+ \sum_{k=1}^{j}\ln K_{\widehat{L}_\text{I}-1}\left(\frac{2\widehat{L}_\text{I}z_k}{1-|\widehat{\rho}_\text{I}|^2}\right)\\
&+(n-j)\left[(\widehat{L}_\text{E}+1)\ln \widehat{L}_\text{E}-\ln\Gamma(\widehat{L}_\text{E})-\ln(1-|\widehat{\rho}_\text{E}|^2)\right]
+\widehat{L}_\text{E}\sum_{k=j+1}^{n} \ln z_k\\
&+\sum_{k=j+1}^{n}\ln I_0\left(\frac{2|\widehat{\rho}_\text{E}|\widehat{L}_\text{E}z_k}{1-|\widehat{\rho}_\text{E}|^2}\right)
+ \sum_{k=j+1}^{n}\ln K_{\widehat{L}_\text{E}-1}\left(\frac{2\widehat{L}_\text{E}z_k}{1-|\widehat{\rho}_\text{E}|^2}\right).\\
\end{split}
\end{equation}

Vamos aplicar o método GenSA para encontrar
$$
\widehat{\jmath}= \arg\max\limits_{j\in [\min_s,N-\min_s]}\ell(j;\widehat{\rho}_I, \widehat{L}_I,\widehat{\rho}_E, \widehat{L}_E),
$$ 
onde $\min_s$ é o tamanho mínimo da amostra definido por $14$.

Podemos concluir que o uso da função densidade magnitude do produtos não é adequado para gerar a função log-verossimilhança. Essas funções resultantes tem características de serem planas dificultando o cálculo do valor máximo. O cálculo inadequado dos coeficientes $L$  e $\rho$ geram uma função \eqref{eq:TotalLogLikelihood_prod_mag} com muita oscilação degenerando o ponto onde o argumento é máximo.


\subsection{Método da verossimilhança aplicado na PDF univariada razão de intensidades múltiplas visadas}
A razão de intensidades ou amplitudes entre $\mathbf{S}_i$ e $\mathbf{S}_j$ são importantes no estudo de radares polarimétricos. Seja a razão de intensidade normalizada,
\begin{equation}\label{eq:razao_intensidades}
 \mu=\frac{\sum_{k=1}^{n}\frac{|\mathbf{S}_i(k)|^2}{\Sigma_{11}}}{\sum_{k=1}^{n}\frac{|\mathbf{S}_j(k)|^2}{\Sigma_{22}}}=\frac{\sum_{k=1}^{n}|\mathbf{S}_i(k)|^2}{\tau\sum_{k=1}^{n}|\mathbf{S}_j(k)|^2},\\
\end{equation}
onde $\tau=\frac{\Sigma_{11}}{\Sigma_{22}}$.
  
Considerando a função densidade de probabilidade univariada razão de intensidades múltiplas visadas,
\begin{equation}\label{eq:pdf_razao_intensidades}
	f(\mu;\rho,L)=\frac{\Gamma(2L)(1-|\rho|^2)^{L}(1+\mu)\mu^{L-1}}{\Gamma(L)\Gamma(L)\left[(1+\mu)^2-4|\rho|^2\mu \right]^{\frac{2L+1}{2}}}\\
\end{equation}
onde, $\rho>0$ e $L>0$. 
Podemos definir 
\begin{equation}\label{eq:razao_intensidades_w}
 w=\frac{\sum_{k=1}^{n}\frac{|\mathbf{S}_i(k)|^2}{\Sigma_{11}}}{\sum_{k=1}^{n}\frac{|\mathbf{S}_j(k)|^2}{\Sigma_{22}}}=\frac{\sum_{k=1}^{n}|\mathbf{S}_i(k)|^2}{\tau\sum_{k=1}^{n}|\mathbf{S}_j(k)|^2}=\tau \mu,\\ 
\end{equation}
realizando a mudança de variável na PDF, \eqref{eq:pdf_razao_intensidades} teremos, 
\begin{equation}\label{eq:pdf_razao_intensidades_tau_w}
	f(w;\rho,L,\tau)=\frac{\tau^L\Gamma(2L)(1-|\rho|^2)^{L}(\tau+w)w^{L-1}}{\Gamma(L)\Gamma(L)\left[(\tau+w)^2-4\tau|\rho|^2w \right]^{\frac{2L+1}{2}}}\\
\end{equation}

Aplicando o logaritmo natural na equação~\eqref{eq:pdf_razao_intensidades_tau_w} e realizando algumas manipulações algébricas teremos:

\begin{equation}\nonumber
\begin{split}
	\ln f(w;\rho,L,\tau)&=\ln\left(\frac{\tau^L\Gamma(2L)(1-|\rho|^2)^{L}(\tau+w)w^{L-1}}{\Gamma(L)\Gamma(L)\left[(\tau+w)^2-4\tau|\rho|^2w \right]^{\frac{2L+1}{2}}}\right),\\
	                &=\ln\left(\tau^L\Gamma(2L)(1-|\rho|^2)^{L}(\tau+w)w^{L-1}\right)\\
	                &-\ln\left(\Gamma(L)\Gamma(L)\left[(\tau+w)^2-4\tau|\rho|^2w \right]^{\frac{2L+1}{2}}\right),\\
	                &=\ln\tau^L + \ln\Gamma(2L) +\ln(1-|\rho|^2)^{L}+\ln(\tau+w)+\ln w^{L-1}\\
	                &-\left(\ln\Gamma(L)+\ln\Gamma(L)+\ln\left[(\tau+w)^2-4\tau|\rho|^2w \right]^{\frac{2L+1}{2}}\right),\\
	                &=L\ln\tau + \ln\Gamma(2L) +L\ln(1-|\rho|^2)+\ln(\tau+w)+(L-1)\ln w\\
	                &-2\ln\Gamma(L)-\frac{2L+1}{2}\ln\left[(\tau+w)^2-4\tau|\rho|^2w \right].\\
\end{split}
\end{equation}

Definimos a equação log-verossimilhança para a PDF univariada razão de intensidades múltiplas visadas,
\begin{equation}\label{eq_log_vero_razao_intensidade_tau_w}
\begin{split}	
	\ln f(w;\rho,L,\tau)&=L\ln\tau + \ln\Gamma(2L) +L\ln(1-|\rho|^2)+\ln(\tau+w)+(L-1)\ln w\\
	                      &-2\ln\Gamma(L)-\frac{2L+1}{2}\ln\left[(\tau+w)^2-4\tau|\rho|^2w \right].\\
\end{split}
\end{equation}

A função log-verossimilhança pode ser deduzida da seguinte maneira, dado a amostra $\bm w = (w_1,\dots,w_n)$, 
\begin{equation}\nonumber
\begin{split}
  \ell(\bm w;\rho, L)=\ln\prod_{k=1}^{n}f(w_k;\rho,L)\\
  \ell(\bm w;\rho, L)=\sum_{k=1}^{n}\ln f(w_k;\rho,L),
 \end{split}
 \end{equation}
usando a função~\eqref{eq_log_vero_razao_intensidade_tau_w} teremos,
\begin{equation}\nonumber
\begin{split}
    \ell(\bm w;\rho, L, \tau)&=\sum_{k=1}^{n}\ln f(\mu_k;\rho, L)\\
                         &=\sum_{k=1}^{n}\left[L\ln\tau + \ln\Gamma(2L) +L\ln(1-|\rho|^2)+\ln(\tau+w)+(L-1)\ln w\right.\\
	                     &-\left.2\ln\Gamma(L)-\frac{2L+1}{2}\ln\left[(\tau+w)^2-4\tau|\rho|^2w \right]\right].\\
 \end{split}
 \end{equation}
 
 \begin{equation}\nonumber
\begin{split} 
    \ell(\bm w;\rho, L, \tau)&=L\ln\tau\sum_{k=1}^{n}1+\ln\Gamma(2L)\sum_{k=1}^{n} 1+L\ln(1-|\rho|^2)\sum_{k=1}^{n} 1+\sum_{k=1}^{n}\ln(\tau+w_k)+(L-1)\sum_{k=1}^{n}\ln w_k\\
                         &-2\ln\Gamma(L)\sum_{k=1}^{n} 1-\frac{2L+1}{2}\sum_{k=1}^{n}\ln\left[(\tau+w_k)^2-4\tau|\rho|^2w_k\right]\\
                         &=n\left(L\ln\tau+\ln\Gamma(2L)+L\ln(1-|\rho|^2)-2\ln\Gamma(L)\right)+\sum_{k=1}^{n}\ln(\tau+w_k)\\
                         &+L\sum_{k=1}^{n}\ln w_k-\sum_{k=1}^{n}\ln w_k-\frac{2L+1}{2}\sum_{k=1}^{n} \ln\left[(\tau+ w_k)^2-4\tau|\rho|^2w_k\right].\\
\end{split}
\end{equation}
 
Definimos a equação log-verossimilhança para a PDF univariada~(\ref{eq_log_vero_razao_intensidade_tau_w}).
\begin{equation}\nonumber
\begin{split}
    \ell(\bm w;\rho, L, \tau)&=n\left(L\ln\tau + \ln\Gamma(2L)+L\ln(1-|\rho|^2)-2\ln\Gamma(L)\right)+\sum_{k=1}^{n}\ln(\tau+w_k)\\
                         &+L\sum_{k=1}^{n}\ln w_k-\sum_{k=1}^{n}\ln w_k-\frac{2L+1}{2}\sum_{k=1}^{n} \ln\left[(\tau+w_k)^2-4\tau|\rho|^2w_k\right]\\
\end{split}
 \end{equation}
e a forma reduzida,
\begin{equation}\label{eq_log_vero_razao_intensidade_red_tau_w}
\begin{split}
    \ell(\bm w;\rho, L, \tau)&=n\left(L\ln\tau +\ln\Gamma(2L)+L\ln(1-|\rho|^2)-2\ln\Gamma(L)\right)\\
                         &+\sum_{k=1}^{n}\ln(\tau+w_k)+L\sum_{k=1}^{n}\ln w_k-\frac{2L+1}{2}\sum_{k=1}^{n} \ln\left[(\tau+w_k)^2-4\tau|\rho|^2w_k\right]\\
\end{split}
 \end{equation} 

Vamos obter $(\widehat \rho, \widehat L, \widehat \tau)$, o estimador de máxima verossimilhança (MLE) de $(\rho, L, \tau)$ baseado em $\bm w$ maximizando~\eqref{eq_log_vero_razao_intensidade_red_tau_w} com o método BFGS implementado no pacote \texttt{maxLik}~\citep{ht}. Vamos preferir otimização resolvendo $\nabla\ell=\bm 0$ com intuito de melhorar a estabilidade numérica.

A função é a log-verossimilhança reduzida para as amostras internas e externas da faixa de dados denotadas respectivamente como $\bm w_\text{I}$ e $\bm w_\text{E}$. Cada faixa de dados $\bm w = (w_1,w_2,\dots,w_n)$ é particionada em duas amostras disjuntas na posição $j$:  
$$
\bm w = (\underbrace{w_1,w_2,\dots,w_j}_{\bm w_\text{I}}, 
\underbrace{w_{j+1}, w_{j+2},\dots,w_n}_{\bm w_\text{E}}).
$$
%Vamos assumir dois diferentes modelos para cada partição:
%$\bm Z_\text{I} \sim \Gamma(\mu_\text{I},L_\text{I})$, e
%$\bm Z_\text{E} \sim \Gamma(\mu_\text{E},L_\text{E})$.
Vamos estimar $(\rho_\text{I},L_\text{I}, \tau_\text{I})$ e $(\rho_\text{E},L_\text{E}, \tau_\text{E})$ com $\bm w_\text{I}$ e $\bm w_\text{E}$, respectivamente, maximizando~\eqref{eq:eq_log_vero_mag_prod_red}, e obtendo $(\widehat{\rho}_\text{I}, \widehat{L}_\text{I}, \widehat{\tau}_\text{I})$ e $(\widehat{\rho}_\text{E}, \widehat{L}_\text{E}, \widehat{\tau}_\text{E})$.

A log-verossimilhança no ponto $j$ é, então
\begin{equation}\label{eq:TotalLogLikelihood}
\begin{split}
\ell(j;\widehat{\rho}_\text{I}, \widehat{L}_\text{I}, \widehat{\tau}_\text{I}, \widehat{\rho}_\text{E}, \widehat{L}_\text{E},\widehat{\tau}_\text{E})&=n\left(\widehat{L}_\text{I}\ln\widehat{\tau}_\text{I} +\ln\Gamma(2\widehat{L}_\text{I})+\widehat{L}_\text{I}\ln(1-|\widehat{\rho}_\text{I}|^2)-2\ln\Gamma(\widehat{L}_\text{I})\right)\\
                         &+\sum_{k=1}^{n}\ln(\widehat{\tau}_\text{I}+w_k)+\widehat{L}_\text{I}\sum_{k=1}^{n}\ln w_k-\frac{2\widehat{L}_\text{I}+1}{2}\sum_{k=1}^{n} \ln\left[(\widehat{\tau}_\text{I}+w_k)^2-4\widehat{\tau}_\text{I}|\widehat{\rho}_{I}|^2w_k\right]\\
                         &+n\left(\widehat{L}_\text{E}\ln\widehat{\tau}_\text{E}+\ln\Gamma(2\widehat{L}_\text{E})+\widehat{L}_\text{E}\ln(1-|\widehat{\rho}_\text{E}|^2)-2\ln\Gamma(\widehat{L}_\text{E})\right)\\
                         &+\sum_{k=1}^{n}\ln(\widehat{\tau}_\text{E}+w_k)+\widehat{L}_\text{E}\sum_{k=1}^{n}\ln w_k-\frac{2\widehat{L}_\text{E}+1}{2}\sum_{k=1}^{n} \ln\left[(\widehat{\tau}_\text{E}+w_k)^2-4\widehat{\tau}_\text{E}|\widehat{\rho}_\text{E}|^2w_k\right]
%\raisetag{2.2em}
\end{split}
\end{equation}

Vamos aplicar o método GenSA para encontrar
$$
\widehat{\jmath}= \arg\max\limits_{j\in [\min_s,N-\min_s]}\ell(j;\widehat{\rho}_I, \widehat{L}_I,\widehat{\tau}_I,\widehat{\rho}_E, \widehat{L}_E, \widehat{\tau}_E),
$$ 
onde $\min_s$ é o tamanho mínimo da amostra definido por $14$.

Desta maneira, vamos obter uma estimativa para a borda em cada canal de intensidade.
Note que esse método pode ser estendido e/ou modificado para lidar com qualquer tipo de dados.

\subsubsection{Aplicação na imagem simulada com duas amostras}


As figuras \eqref{fig:razao_tau_rho_1_50} até \eqref{fig:razao_tau_rho_51_400} mostram a função de log-verossimilhança da pdf razão de intensidades \eqref{eq:pdf_razao_intensidades_tau_w} aplicada na imagem de duas amostras simulada. No processo fixamos arbitrariamente a linha 80 da amostra e variamos o número de pixel entre 50, 150, 250 e 250. Desta maneira construímos duas funções $\ell{j}$, uma para cada lado da amostra. Nessas figuras fixamos o L = 4 e os canais (hh) e (vv).
 \begin{figure*}[hbt]
	\centering
     \subfloat[Pixel variando de 1 até 50 na linha 80.  \label{fig:razao_tau_rho_1_50}]{%
       \includegraphics[width=0.50\linewidth]{fig_pdf_razao_r_80_1_to_50}
     }
     \subfloat[Pixel variando de 51 até 400 na linha 80. \label{fig:razao_tau_rho_51_400}]{%
       \includegraphics[width=0.50\linewidth]{fig_pdf_razao_r_80_50_to_400}
     }
     \caption{Funções de máxima verossimilhança razão de intensidades com pixel fixo 50.}
     \label{fig:razao_l_50_r_80} 
   \end{figure*}
\begin{figure*}[hbt]
	\centering
     \subfloat[Pixel variando de 1 até 150 na linha 80.  \label{fig:razao_tau_rho_1_150}]{%
       \includegraphics[width=0.50\linewidth]{fig_pdf_razao_r_80_1_to_150}
     }
     \subfloat[Pixel variando de 151 até 400 na linha 80. \label{fig:razao_tau_rho_151_400}]{%
       \includegraphics[width=0.50\linewidth]{fig_pdf_razao_r_80_150_to_400}
     }
     \caption{Funções de máxima verossimilhança razão de intensidades com pixel fixo 150.}
     \label{fig:razao_l_150_r_80} 
   \end{figure*}   
   
   \begin{figure*}[hbt]
	\centering
     \subfloat[Pixel variando de 1 até 250 na linha 80.  \label{fig:razao_tau_rho_1_250}]{%
       \includegraphics[width=0.50\linewidth]{fig_pdf_razao_r_80_1_to_250}
     }
     \subfloat[Pixel variando de 251 até 400 na linha 80. \label{fig:razao_tau_rho_251_400}]{%
       \includegraphics[width=0.50\linewidth]{fig_pdf_razao_r_80_250_to_400}
     }
     \caption{Funções de máxima verossimilhança razão de intensidades com pixel fixo 250.}
     \label{fig:razao_l_250_r_80} 
   \end{figure*}   
   
   \begin{figure*}[hbt]
	\centering
     \subfloat[Pixel variando de 1 até 350 na linha 80.  \label{fig:razao_tau_rho_1_350}]{%
       \includegraphics[width=0.50\linewidth]{fig_pdf_razao_r_80_1_to_350}
     }
     \subfloat[Pixel variando de 351 até 400 na linha 80. \label{fig:razao_tau_rho_351_400}]{%
       \includegraphics[width=0.50\linewidth]{fig_pdf_razao_r_80_350_to_400}
     }
     \caption{Funções de máxima verossimilhança razão de intensidades com pixel fixo 350.}
     \label{fig:razao_l_250_r_80} 
   \end{figure*}   
   

O método da máxima verossimilhança \eqref{eq:TotalLogLikelihood} foi aplicado na imagem simulada com duas amostras, e as evidências de bordas estão mostradas na figura. \textcolor{red}{Base de dados gamf}
 \begin{figure*}[hbt]
	\centering
     \subfloat[Evidências no canal $\text{hh}$ \label{evidencias_hh_hv_vv_gamf:a}]{%
       \includegraphics[width=0.5\linewidth]{im_sim_gamf_hh_hv_param_tau_rho_14_pixel}
     }
     \subfloat[xxxxxxxxxxxxxxxxxx $\text{hv}$ \label{evidencias_hh_hv_vv_gamf:b}]{%
       \includegraphics[width=0.5\linewidth]{im_sim_gamf_hh_vv_param_tau_rho_14_pixel}
     }      
   %  \subfloat[Evidências no canal $\text{vv}$ \label{evidencias_hh_hv_vv_gamf:c}]{%
    %   \includegraphics[width=0.5\linewidth]{im_sim_gamf_hh_evid_param_L_mu_14_pixel}
    % }
    \caption{Evidências de bordas para os três canais de intensidade}
     \label{evidencias_hh_hv_vv_gamf} 
   \end{figure*}
   
   \begin{figure*}[hbt]
	\centering
     \subfloat[Evidências no canal $\text{vv}$ \label{evidencias_hh_hv_vv_gamf:c}]{%
       \includegraphics[width=0.5\linewidth]{im_sim_gamf_hv_vv_param_tau_rho_14_pixel}
     }
    \caption{Evidências de bordas para os três canais de intensidade}
     \label{evidencias_hh_hv_vv_gamf} 
   \end{figure*}   
   
   
\subsection{Distribuição bivariada produto de intensidades - Lee } 

Seja a função distribuição de probabilidade 
\begin{equation}\label{func_biv_produto_inten_b1_b2}
	f(B_1,B_2;\rho, L)=\frac{\left(B_1B_2\right)^{\frac{L-1}{2}}\exp\left(-\frac{B_1+B_2}{1-|\rho|^2}\right)}{\Gamma(L)(1-|\rho|^2)|\rho|^{L-1}}I_{L-1}\left(2\sqrt{B_1B_2}\frac{|\rho|}{1-|\rho|^2}\right)
\end{equation}
Considerando as seguintes relações 
\begin{equation}\label{eqn59}
\begin{array}{ccccc}
	R_1&=&\frac{1}{L}\sum_{k=1}^{L}|S_i(k)|^2&=&\frac{B_1\Sigma_{11}}{L}\\
	R_2&=&\frac{1}{L}\sum_{k=1}^{L}|S_j(k)|^2&=&\frac{B_2\Sigma_{22}}{L}\\
\end{array}
\end{equation}
\begin{equation}\label{fun_pdf_biv_inten}
	f(R_1,R_2;\rho,L, \Sigma_{11}, \Sigma_{22})=\frac{L^{L+1}\left(R_1R_2\right)^{\frac{L-1}{2}}\exp\left(-\frac{L\left(\frac{R_1}{\Sigma_{11}}+\frac{R_2}{\Sigma_{22}}\right)}{1-|\rho|^2}\right)}{(\Sigma_{11}\Sigma_{22})^{\frac{L+1}{2}}\Gamma(L)(1-|\rho|^2)|\rho|^{L-1}}I_{L-1}\left(2L\sqrt{\frac{R_1R_2}{\Sigma_{11}\Sigma_{22}}}\frac{|\rho|}{1-|\rho|^2}\right)
\end{equation}

Aplicando o logaritmo natural na equação em ambos os lados da  (\ref{fun_pdf_biv_inten})
\begin{equation}\nonumber
\begin{split}
	\ln f(R_1,R_2;\rho, L, \Sigma_{11}, \Sigma_{22})&=\ln\left(\frac{L^{L+1}\left(R_1R_2\right)^{\frac{L-1}{2}}\exp\left(-\frac{L\left(\frac{R_1}{\Sigma_{11}}+\frac{R_2}{\Sigma_{22}}\right)}{1-|\rho|^2}\right)}{(\Sigma_{11}\Sigma_{22})^{\frac{L+1}{2}}\Gamma(L)(1-|\rho|^2)|\rho_c|^{L-1}}I_{L-1}\left(2L\sqrt{\frac{R_1R_2}{\Sigma_{11}\Sigma_{22}}}\frac{|\rho|}{1-|\rho|^2}\right)\right)\\
	\end{split}
\end{equation}
\begin{equation}\nonumber
\begin{split}
	\ln f(R_1,R_2;\rho,L, \Sigma_{11}, \Sigma_{22})&=\ln\left(\frac{L^{L+1}\left(R_1R_2\right)^{\frac{L-1}{2}}\exp\left(-\frac{L\left(\frac{R_1}{\Sigma_{11}}+\frac{R_2}{\Sigma_{22}}\right)}{1-|\rho|^2}\right)}{(\Sigma_{11}\Sigma_{22})^{\frac{L+1}{2}}\Gamma(L)(1-|\rho|^2)|\rho|^{L-1}}\right)
	 +\ln I_{L-1}\left(2L\sqrt{\frac{R_1R_2}{\Sigma_{11}\Sigma_{22}}}\frac{|\rho|}{1-|\rho|^2}\right)\\
	             &=\ln\left(L^{L+1}\left(R_1R_2\right)^{\frac{L-1}{2}}\exp\left(-\frac{L\left(\frac{R_1}{\Sigma_{11}}+\frac{R_2}{\Sigma_{22}}\right)}{1-|\rho|^2}\right)\right)-\ln\left((\Sigma_{11}\Sigma_{22})^{\frac{L+1}{2}}\Gamma(L)(1-|\rho|^2)|\rho|^{L-1}\right) \\
	&+\ln I_{L-1}\left(2L\sqrt{\frac{R_1R_2}{\Sigma_{11}\Sigma_{22}}}\frac{|\rho|}{1-|\rho|^2}\right)
	\end{split}
\end{equation}
\begin{equation}\nonumber
\begin{split}
	\ln f(R_1,R_2;\rho,L, \Sigma_{11}, \Sigma_{22})&=\ln\left(L^{L+1}\left(R_1R_2\right)^{\frac{L-1}{2}}\right) + \ln \exp\left(-\frac{L\left(\frac{R_1}{\Sigma_{11}}+\frac{R_2}{\Sigma_{22}}\right)}{1-|\rho|^2}\right)-\ln\left((\Sigma_{11}\Sigma_{22})^{\frac{L+1}{2}}\Gamma(L)(1-|\rho|^2)|\rho|^{L-1}\right) \\
	&+\ln I_{L-1}\left(2L\sqrt{\frac{R_1R_2}{\Sigma_{11}\Sigma_{22}}}\frac{|\rho|}{1-|\rho|^2}\right)\\
	&=\ln L^{L+1} + \ln (R_1R_2)^{\frac{L-1}{2}} -\frac{L\left(\frac{R_1}{\Sigma_{11}}+\frac{R_2}{\Sigma_{22}}\right)}{1-|\rho|^2}-\ln(\Sigma_{11}\Sigma_{22})^{\frac{L+1}{2}} - \ln\Gamma(L)- \ln(1-|\rho|^2)-\ln|\rho|^{L-1} \\
	&+\ln I_{L-1}\left(2L\sqrt{\frac{R_1R_2}{\Sigma_{11}\Sigma_{22}}}\frac{|\rho|}{1-|\rho|^2}\right)\\
	&=(L+1)\ln L +\frac{L-1}{2} \ln (R_1R_2) -\frac{LR_1}{\Sigma_{11}(1-|\rho|^2)}-\frac{LR_2}{\Sigma_{22}(1-|\rho|^2)}\\
	&-\frac{L+1}{2}\ln(\Sigma_{11}\Sigma_{22}) - \ln\Gamma(L)- \ln(1-|\rho|^2)-(L-1)\ln|\rho|\\
	&+\ln I_{L-1}\left(2L\sqrt{\frac{R_1R_2}{\Sigma_{11}\Sigma_{22}}}\frac{|\rho|}{1-|\rho|^2}\right)
\end{split}
\end{equation}
	 	
\begin{equation}\label{fun_log_biv_inten}
\begin{split}
	\ln f(R_1,R_2;\rho,L, \Sigma_{11}, \Sigma_{22})&=(L+1)\ln L +\frac{L-1}{2} \ln R_1 +\frac{L-1}{2} \ln R_2 -\frac{LR_1}{\Sigma_{11}(1-|\rho|^2)}-\frac{LR_2}{\Sigma_{22}(1-|\rho|^2)}\\
	&-\frac{L+1}{2}\ln\Sigma_{11}-\frac{L+1}{2}\ln\Sigma_{22} - \ln\Gamma(L)- \ln(1-|\rho|^2)-(L-1)\ln|\rho|\\
	&+\ln I_{L-1}\left(2L\sqrt{\frac{R_1R_2}{\Sigma_{11}\Sigma_{22}}}\frac{|\rho|}{1-|\rho|^2}\right)
\end{split}
\end{equation}


A função log-verossimilhança pode ser deduzida da seguinte maneira, dado a amostra $\bm\mu = (\mu_1,\dots,\mu_n)$, 
\begin{equation}\nonumber
\begin{split}
  \ell(R_1, R_2;\rho, L, \Sigma_{11}, \Sigma_{22})=\ln\prod_{k=1}^{n}f(R_1, R_2;\rho,L, \Sigma_{11}, \Sigma_{22})\\
  \ell(R_1, R_2;\rho, L, \Sigma_{11}, \Sigma_{22})=\sum_{k=1}^{n}\ln f(R_1, R_2;\rho,L, \Sigma_{11}, \Sigma_{22}),
 \end{split}
 \end{equation}
usando a função~\eqref{fun_log_biv_inten} teremos,
\begin{equation}\nonumber
\begin{split}
    \ell(R_1, R_2;\rho, L, \Sigma_{11}, \Sigma_{22})&=\sum_{k=1}^{n}\ln f(R_1, R_2;\rho, L, \Sigma_{11}, \Sigma_{22})\\
                         &=\sum_{k=1}^{n}\left[(L+1)\ln L +\frac{L-1}{2} \ln R_1 +\frac{L-1}{2} \ln R_2 -\frac{LR_1}{\Sigma_{11}(1-|\rho|^2)}-\frac{LR_2}{\Sigma_{22}(1-|\rho|^2)}\right.\\
	&-\frac{L+1}{2}\ln\Sigma_{11}-\frac{L+1}{2}\ln\Sigma_{22} - \ln\Gamma(L)- \ln(1-|\rho|^2)-(L-1)\ln|\rho|\\
	&\left.+\ln I_{L-1}\left(2L\sqrt{\frac{R_1R_2}{\Sigma_{11}\Sigma_{22}}}\frac{|\rho|}{1-|\rho|^2}\right)\right]
 \end{split}
 \end{equation}
 
 \begin{equation}\nonumber
\begin{split} 
  \ell(R_1, R_2;\rho, L, \Sigma_{11}, \Sigma_{22})&=(L+1)\ln L\sum_{k=1}^{n}1 +\frac{L-1}{2}\sum_{k=1}^{n} \ln R_1 +\frac{L-1}{2} \sum_{k=1}^{n}\ln R_2\\
                        &-L\sum_{k=1}^{n}\frac{R_1}{\Sigma_{11}(1-|\rho|^2)}-L\sum_{k=1}^{n}\frac{R_2}{\Sigma_{22}(1-|\rho|^2)}\\
	&-\frac{L+1}{2}\sum_{k=1}^{n}\ln\Sigma_{11}-\frac{L+1}{2}\sum_{k=1}^{n}\ln\Sigma_{22} \\
	&- \ln\Gamma(L)\sum_{k=1}^{n}1- \ln(1-|\rho|^2)\sum_{k=1}^{n}1-(L-1)\ln|\rho|\sum_{k=1}^{n}1\\
	&+\sum_{k=1}^{n}\ln I_{L-1}\left(2L\sqrt{\frac{R_1R_2}{\Sigma_{11}\Sigma_{22}}}\frac{|\rho|}{1-|\rho|^2}\right)
\end{split}
\end{equation}

Definimos a equação log-verossimilhança para a PDF univariada~(\ref{fun_log_biv_inten}).
\begin{equation}\nonumber
\begin{split} 
  \ell(R_1, R_2;\rho, L, \Sigma_{11}, \Sigma_{22})&=n\left[(L+1)\ln L - \ln\Gamma(L)- \ln(1-|\rho|^2)-(L-1)\ln|\rho|\right] \\
                        &+\frac{L-1}{2}\sum_{k=1}^{n} \ln R_1 +\frac{L-1}{2} \sum_{k=1}^{n}\ln R_2\\
                        &-L\sum_{k=1}^{n}\frac{R_1}{\Sigma_{11}(1-|\rho|^2)}-L\sum_{k=1}^{n}\frac{R_2}{\Sigma_{22}(1-|\rho|^2)}\\
	&-\frac{L+1}{2}\sum_{k=1}^{n}\ln\Sigma_{11}-\frac{L+1}{2}\sum_{k=1}^{n}\ln\Sigma_{22} \\
	&+\sum_{k=1}^{n}\ln I_{L-1}\left(2L\sqrt{\frac{R_1R_2}{\Sigma_{11}\Sigma_{22}}}\frac{|\rho|}{1-|\rho|^2}\right)
\end{split}
\end{equation} 
e a forma reduzida,
\begin{equation}\label{eq_log_vero_biv_prod_red}
\begin{split}
\ell(R_1, R_2;\rho, L, \Sigma_{11}, \Sigma_{22})&=n\left[(L+1)\ln L - \ln\Gamma(L)- \ln(1-|\rho|^2)-(L-1)\ln|\rho|\right. \\
	                    &-\left.\frac{L}{2}\ln\Sigma_{11}-\frac{L}{2}\ln\Sigma_{22}\right] \\
                        &+\frac{L}{2}\sum_{k=1}^{n} \ln R_1 +\frac{L}{2} \sum_{k=1}^{n}\ln R_2\\
                        &-\frac{L}{\Sigma_{11}(1-|\rho|^2)}\sum_{k=1}^{n}R_1-\frac{L}{\Sigma_{22}(1-|\rho|^2)}\sum_{k=1}^{n}R_2\\
	&+\sum_{k=1}^{n}\ln I_{L-1}\left(2L\sqrt{\frac{R_1R_2}{\Sigma_{11}\Sigma_{22}}}\frac{|\rho|}{1-|\rho|^2}\right)
\end{split}
 \end{equation} 

Vamos obter $(\widehat \rho, \widehat L)$, o estimador de máxima verossimilhança (MLE) de $(\rho, L)$ baseado $\bm \mu$, por maximizar~\eqref{eq_log_vero_razao_intensidade_red} com o método BFGS implementado no pacote \texttt{maxLik}~\citep{ht}. Vamos preferir otimização resolvendo $\nabla\ell=\bm 0$ com intuito de melhorar a estabilidade numérica.

O função é a log-verossimilhança reduzida para as amostras internas e externas da faixa de dados denotadas respectivamento como $\bm \mu_\text{I}$ e $\bm \mu_\text{E}$. Cada faixa de dados $\bm \mu = (\mu_1,\mu_2,\dots,\mu_n)$ é particionada em duas amostras disjuntas na posição $j$:  
$$
\bm \mu = (\underbrace{\mu_1,\mu_2,\dots,\mu_j}_{\bm \mu_\text{I}}, 
\underbrace{\mu_{j+1}, \mu_{j+2},\dots,\mu_n}_{\bm \mu_\text{E}}).
$$
%Vamos assumir dois diferentes modelos para cada partição:
%$\bm Z_\text{I} \sim \Gamma(\mu_\text{I},L_\text{I})$, e
%$\bm Z_\text{E} \sim \Gamma(\mu_\text{E},L_\text{E})$.
Vamos estimar $(\rho_\text{I},L_\text{I})$ e $(\rho_\text{E},L_\text{E})$ com $\bm \mu_\text{I}$ e $\bm \mu_\text{E}$, respectivamente, maximizando~\eqref{eq:eq_log_vero_mag_prod_red}, e obtendo $(\widehat{\rho}_\text{I}, \widehat{L}_\text{I})$ e $(\widehat{\rho}_\text{E}, \widehat{•}t{L}_\text{E})$.

A log-verossimilhança no ponto $j$ é, então
\begin{equation}\label{eq:TotalLogLikelihood}
\begin{split}
\ell(j;\widehat{\rho}_\text{I}, \widehat{L}_\text{I}, \widehat{\rho}_\text{E}, \widehat{L}_\text{E})&=n\left[(\widehat{L}_\text{I}+1)\ln \widehat{L}_\text{I} - \ln\Gamma(\widehat{L}_\text{I})- \ln(1-|\widehat{\rho}_\text{I}|^2)-(\widehat{L}_\text{I}-1)\ln|\widehat{\rho}_\text{I}|\right] \\
                        &+\frac{\widehat{L}_\text{I}}{2}\sum_{k=1}^{n} \ln R_1 +\frac{\widehat{L}_\text{I}}{2} \sum_{k=1}^{n}\ln R_2\\
                        &-\frac{\widehat{L}_\text{I}}{1-|\widehat{\rho}_\text{I}|^2}\sum_{k=1}^{n}\frac{R_1}{\Sigma_{11}}-\frac{\widehat{L}_\text{I}}{1-|\widehat{\rho}_\text{I}|^2}\sum_{k=1}^{n}\frac{R_2}{\Sigma_{22}}\\
	&-\frac{\widehat{L}_\text{I}}{2}\sum_{k=1}^{n}\ln\Sigma_{11}-\frac{\widehat{L}_\text{I}}{2}\sum_{k=1}^{n}\ln\Sigma_{22} \\
	&+\sum_{k=1}^{n}\ln I_{\widehat{L}_\text{I}-1}\left(2\widehat{L}_\text{I}\sqrt{\frac{R_1R_2}{\Sigma_{11}\Sigma_{22}}}\frac{|\widehat{\rho}_\text{I}|}{1-|\widehat{\rho}_\text{I}|^2}\right)\\
	&+n\left[(\widehat{L}_\text{E}+1)\ln \widehat{L}_\text{E} - \ln\Gamma(\widehat{L}_\text{E})- \ln(1-|\widehat{\rho}_\text{E}|^2)-(\widehat{L}_\text{E}-1)\ln|\widehat{\rho}_\text{E}|\right] \\
                        &+\frac{\widehat{L}_\text{E}}{2}\sum_{k=1}^{n} \ln R_1 +\frac{\widehat{L}_\text{E}}{2} \sum_{k=1}^{n}\ln R_2\\
                        &-\frac{\widehat{L}_\text{E}}{1-|\widehat{\rho}_\text{E}|^2}\sum_{k=1}^{n}\frac{R_1}{\Sigma_{11}}-\frac{\widehat{L}_\text{E}}{1-|\widehat{\rho}_\text{E}|^2}\sum_{k=1}^{n}\frac{R_2}{\Sigma_{22}}\\
	&-\frac{\widehat{L}_\text{E}}{2}\sum_{k=1}^{n}\ln\Sigma_{11}-\frac{\widehat{L}_\text{E}}{2}\sum_{k=1}^{n}\ln\Sigma_{22} \\
	&+\sum_{k=1}^{n}\ln I_{\widehat{L}_\text{E}-1}\left(2\widehat{L}_\text{E}\sqrt{\frac{R_1R_2}{\Sigma_{11}\Sigma_{22}}}\frac{|\widehat{\rho}_\text{E}|}{1-|\widehat{\rho}_\text{E}|^2}\right)
\end{split}
\end{equation}

Vamos aplicar o método GenSA para encontrar
$$
\widehat{\jmath}= \arg\max\limits_{j\in [\min_s,N-\min_s]}\ell(j;\widehat{\rho}_I, \widehat{L}_I,\widehat{\rho}_E, \widehat{L}_E),
$$ 
onde $\min_s$ é o tamanho mínimo da amostra definido por $14$.

Desta maneira, vamos obter uma estimativa para a borda em cada canal de intensidade.
Note que esse método pode ser estendido e/ou modificado para lidar com qualquer tipo de dados.

\section{Resultados numéricos para o método MLE aplicado em cada distribuíção}

Usamos uma imagem AIRSAR PolSAR de Flevoland, banda L, de $750\times 1024$ pixels para os testes.  Fig.~\ref{flevoland_radial_4look} mostra o ROI, com as linhas radiais onde as bordas são detectadas. Fig.~\ref{flevoland_flevoland} mostra a referência do solo em vermelho.  
  \begin{figure}[hbt]
   \centering
     \subfloat[Imagem, Região de Interesse (ROI), and radiais. \label{flevoland_radial_4look}]{%
%       \includegraphics[viewport= 0 50 500 550, clip=true, width=0.23\textwidth]{flevoland_radial_4_look_black_crop}}      
       \includegraphics[width=0.53\textwidth]{flevoland_radial_4_look_black_crop}}
     \subfloat[Ground reference\label{gt_flevoland}]{%
       \includegraphics[width=0.5\textwidth]{gt_flevoland_crop}
     }
    \caption{Decomposição de Pauli para a imagem de Flevoland, região de interesse, e referência \textit{ground}}
    \label{roi_gt}
\end{figure}

\subsection{Método da verossimilhança aplicado na pdf univariada $\Gamma$.}
Resolvendo o problema,
$$
\widehat{\jmath}= \arg\max\limits_{j\in [\min_s,N-\min_s]}\ell(j;\widehat{\rho}_I, \widehat{L}_I,\widehat{\rho}_E, \widehat{L}_E),
$$

Figs.~\ref{evidencias_hh_hv_vv}\subref{evidencias_hh_hv_vv:a},~\ref{evidencias_hh_hv_vv}\subref{evidencias_hh_hv_vv:b}, e~\ref{evidencias_hh_hv_vv:a}\subref{evidencias_hh_hv_vv:c}, mostram, respectivamente, as evidências de borda nos canais $\text{hh}$, $\text{hv}$ e $\text{vv}$ como obtidos pela MLE.

Vale notar que a GenSA identificou com precisão o valor máximo de $\eqref{eq:TotalLogLikelihood}$, mesmo na presença de múltiplos máximos locais. 
Uma avaliação visual leva à conclusão de que os melhores resultados são fornecidos por $\text{hv}$, embora com alguns pontos longe da borda real.

 \begin{figure*}[hbt]
	\centering
     \subfloat[Evidências no canal $\text{hh}$ \label{evidencias_hh_hv_vv:a}]{%
       \includegraphics[width=0.32\linewidth]{flevoland_hh_evid_param_L_mu_14_pixel_crop}
     }
     \subfloat[Evidências no canal $\text{hv}$ \label{evidencias_hh_hv_vv:b}]{%
       \includegraphics[width=0.32\linewidth]{flevoland_hv_evid_param_L_mu_14_pixel_crop}
     }
     \subfloat[Evidências no canal $\text{vv}$ \label{evidencias_hh_hv_vv:c}]{%
       \includegraphics[width=0.32\linewidth]{flevoland_vv_evid_param_L_mu_14_pixel_crop}
     }
     \caption{Evidências de bordas para os três canais de intensidade}
     \label{evidencias_hh_hv_vv} 
   \end{figure*}

\subsection{Método da verossimilhança aplicado na pdf magnitude do produto.}


 \begin{figure*}[hbt]
	\centering
     \subfloat[Evidências no canal $\text{hh}$ \label{evidencias_hh_hv_vv:a}]{%
       \includegraphics[width=0.32\linewidth]{}
     }
     \subfloat[Evidências no canal $\text{hv}$ \label{evidencias_hh_hv_vv:b}]{%
       \includegraphics[width=0.32\linewidth]{flevoland_hv_evid_param_L_mu_14_pixel_crop}
     }
     \subfloat[Evidências no canal $\text{vv}$ \label{evidencias_hh_hv_vv:c}]{%
       \includegraphics[width=0.32\linewidth]{flevoland_vv_evid_param_L_mu_14_pixel_crop}
     }
     \caption{Evidências de bordas para os três canais de intensidade}
     \label{evidencias_hh_hv_vv} 
   \end{figure*}









%\begin{figure}[hbt]
%\minipage{0.5\textwidth}
%  \includegraphics[width=\linewidth]{funv_max_ver_j_10_flev_razao.pdf}
%  	\caption{$\sigma= 12.3426$.}\label{cap_acf_fig04}
%\endminipage\hfill
%\minipage{0.5\textwidth}
%  \includegraphics[width=\linewidth]{funv_max_ver_j_20_flev_razao.pdf}
%		\caption{$\sigma=2.1029 $.}\label{cap_acf_fig05}
%\endminipage\hfill
%\centering
%\minipage{0.5\textwidth}
%  \includegraphics[width=\linewidth]{funv_max_ver_j_30_flev_razao.pdf}
%  	\caption{$\sigma=1.4999 $.}\label{cap_acf_fig04}
%\endminipage\hfill
%\minipage{0.5\textwidth}
%  \includegraphics[width=\linewidth]{funv_max_ver_j_40_flev_razao.pdf}
%		\caption{$\sigma=12.6414 $.}\label{cap_acf_fig05}
%\endminipage\hfill
%\end{figure}
%\begin{figure}[hbt]
%\minipage{0.5\textwidth}
%  \includegraphics[width=\linewidth]{funv_max_ver_j_50_flev_razao.pdf}
%  	\caption{$\sigma=10.4523$.}\label{cap_acf_fig04}
%\endminipage\hfill
%\minipage{0.5\textwidth}
%  \includegraphics[width=\linewidth]{funv_max_ver_j_60_flev_razao.pdf}
%		\caption{$\sigma= 14.2156$.}\label{cap_acf_fig05}
%\endminipage\hfill
%\centering
%\minipage{0.5\textwidth}
%  \includegraphics[width=\linewidth]{funv_max_ver_j_70_flev_razao.pdf}
%  	\caption{$\sigma=9.8405 $.}\label{cap_acf_fig04}
%\endminipage\hfill
%\minipage{0.5\textwidth}
%  \includegraphics[width=\linewidth]{funv_max_ver_j_80_flev_razao.pdf}
%		\caption{$\sigma=13.1298 $.}\label{cap_acf_fig05}
%\endminipage\hfill
%\end{figure}
%\subsection{Distribuição univariada da magnitude do produto}
%A magnitude do produto $\mathbf{S}_i$ e $\mathbf{S}_j$ é uma importante medida para as imagem SAR polarimétrica. Definimos a magnitude normalizada por 
%
%\begin{equation}
%	\xi = \frac{\left|\frac{1}{L} \sum_{k=1}^L\mathbf{S}_i(k)\mathbf{S}_j^H(k) \right|}{\sqrt{E[|\mathbf{S}_i|^2]E[|\mathbf{S}_i|^2]}}=\frac{g}{h}.
%\end{equation}
%onde é definido por $g=|\mathbf{S}_i\mathbf{S}_j^H|$ e $h=\sqrt{E[|\mathbf{S}_i|^2]E[|\mathbf{S}_i|^2]}$.
%\begin{equation}
%\begin{array}{ccc}
%	f(\xi)&=&\frac{4L^{L+1}\xi^L}{\Gamma(L)(1-|\rho|^2)}I_0\left(\frac{2|\rho|L\xi}{1-|\rho|^2}\right)K_{L-1}\left(\frac{2L\xi}{1-|\rho|^2}\right).
%		\end{array}
%\end{equation}
%\begin{equation}
%\begin{array}{ccc}
%	\ln f(\xi)&=&\ln\left(\frac{4L^{L+1}\xi^L}{\Gamma(L)(1-|\rho|^2)}I_0\left(\frac{2|\rho|L\xi}{1-|\rho|^2}\right)K_{L-1}\left(\frac{2L\xi}{1-|\rho|^2}\right)\right).
%		\end{array}
%\end{equation}
%\begin{equation}
%\begin{array}{ccc}
%	\ln f(\xi)&=&\ln\left(\frac{4L^{L+1}\xi^L}{\Gamma(L)(1-|\rho|^2)}\right)+\ln I_0\left(\frac{2|\rho|L\xi}{1-|\rho|^2}\right)+ \ln K_{L-1}\left(\frac{2L\xi}{1-|\rho|^2}\right).
%		\end{array}
%\end{equation}
%
%\begin{equation}
%\begin{array}{ccc}
%	\ln f(\xi)&=&\ln (4L^{L+1}\xi^L)-\ln(\Gamma(L)(1-|\rho|^2))+\ln I_0\left(\frac{2|\rho|L\xi}{1-|\rho|^2}\right)+ \ln K_{L-1}\left(\frac{2L\xi}{1-|\rho|^2}\right).
%		\end{array}
%\end{equation}
%
%\begin{equation}
%\begin{array}{ccc}
%	\ln f(\xi)&=&\ln (4)+\ln L^{L+1}+\ln \xi^L-\ln\Gamma(L)-\ln(1-|\rho|^2)+\ln I_0\left(\frac{2|\rho|L\xi}{1-|\rho|^2}\right)+ \ln K_{L-1}\left(\frac{2L\xi}{1-|\rho|^2}\right).
%		\end{array}
%\end{equation}
%
%\begin{equation}
%\begin{array}{ccc}
%	\ln f(\xi)&=&\ln (4)+(L+1)\ln L+L\ln \xi-\ln\Gamma(L)-\ln(1-|\rho|^2)+\ln I_0\left(\frac{2|\rho|L\xi}{1-|\rho|^2}\right)+ \ln K_{L-1}\left(\frac{2L\xi}{1-|\rho|^2}\right).
%		\end{array}
%\end{equation}
%
%\begin{figure}[hbt]
%\minipage{0.5\textwidth}
%  \includegraphics[width=\linewidth]{funv_max_ver_j_10_flev_produto.pdf}
%  	\caption{$\sigma= 0.0001241$.}\label{cap_acf_fig04}
%\endminipage\hfill
%\minipage{0.5\textwidth}
%  \includegraphics[width=\linewidth]{funv_max_ver_j_20_flev_produto.pdf}
%		\caption{$\sigma= 0.0021969$.}\label{cap_acf_fig05}
%\endminipage\hfill
%\centering
%\minipage{0.5\textwidth}
%  \includegraphics[width=\linewidth]{funv_max_ver_j_30_flev_produto.pdf}
%  	\caption{$\sigma=0.0047520 $.}\label{cap_acf_fig04}
%\endminipage\hfill
%\minipage{0.5\textwidth}
%  \includegraphics[width=\linewidth]{funv_max_ver_j_40_flev_produto.pdf}
%		\caption{$\sigma= 0.0123943$.}\label{cap_acf_fig05}
%\endminipage\hfill
%\end{figure}
%\begin{figure}[hbt]
%\minipage{0.5\textwidth}
%  \includegraphics[width=\linewidth]{funv_max_ver_j_50_flev_produto.pdf}
%  	\caption{$\sigma= 0.0002715$.}\label{cap_acf_fig04}
%\endminipage\hfill
%\minipage{0.5\textwidth}
%  \includegraphics[width=\linewidth]{funv_max_ver_j_60_flev_produto.pdf}
%		\caption{$\sigma= 0.0001922$.}\label{cap_acf_fig05}
%\endminipage\hfill
%\centering
%\minipage{0.5\textwidth}
%  \includegraphics[width=\linewidth]{funv_max_ver_j_70_flev_produto.pdf}
%  	\caption{$\sigma= 0.0004329$.}\label{cap_acf_fig04}
%\endminipage\hfill
%\minipage{0.5\textwidth}
%  \includegraphics[width=\linewidth]{funv_max_ver_j_80_flev_produto.pdf}
%		\caption{$\sigma= 0.0002790$.}\label{cap_acf_fig05}
%\endminipage\hfill
%\end{figure}


%\subsection{Distribuição bivariada produto de intensidades - Lee } 
%
%O $PDF$ conjunto retorna de dois canais correlacionados dos radares polarimétricos e interferométricos são importantes. As $PDF's$ conjuntas conduzem a derivação da intensidade e amplitude razão $PDF's$. Da equação (\ref{eqn42}) temos que as intensidades {\it multilook} sejam 
%
%\begin{equation}\label{eqn59}
%\begin{array}{ccccc}
%	R_1&=&\frac{1}{n}\sum_{k=1}^{n}|S_i(k)|^2&=&\frac{B_1C_{11}}{n}\\
%	R_2&=&\frac{1}{n}\sum_{k=1}^{n}|S_j(k)|^2&=&\frac{B_2C_{22}}{n}\\
%\end{array}
%\end{equation}
%
%Integrando a equação (\ref{eqn52}) em relação a $\eta$ e $\psi$. A $PDF$ é
%
%\begin{equation}\label{eqn60}
%	p(B_1,B_2)=\frac{\left(B_1B_2\right)^{\frac{n-1}{2}}\exp\left(-\frac{B_1+B_2}{1-|\rho_c|^2}\right)}{\Gamma(n)(1-|\rho_c|^2)|\rho_c|^{n-1}}I_{n-1}\left(2\sqrt{B_1B_2}\frac{|\rho_c|}{1-|\rho_c|^2}\right)
%\end{equation}

%Sendo
%\begin{equation}\label{eqn61}
%	I_{\mu}(Z)=\frac{(\frac{z}{2})^{\mu}}{\Gamma(\mu+1)} F_{1}^{0}[-;\mu+1;\frac{z^2}{4}]
%\end{equation}
%
%\begin{equation}\label{eqn62}
%	p(B_1,B_2)=\frac{n^{n+1}\left(R_1R_2\right)^{\frac{n-1}{2}}\exp\left(-\frac{n(\frac{R_1}{C_{11}}+\frac{R_2}{C_{22}})}{1-|\rho_c|^2}\right)}{(C_{11}C_{22})^{\frac{n+1}{2}}\Gamma(n)(1-|\rho_c|^2)|\rho_c|^{n-1}}I_{n-1}\left(2n\sqrt{\frac{R_1R_2}{C_{11}C_{22}}}\frac{|\rho_c|}{1-|\rho_c|^2}\right)
%\end{equation}
%\subsection{Distribuição $\Gamma$ trivariada - Hagedorn }
%\begin{equation}\label{eqn62}
%\begin{array}{ccc}
%	p(I_1,I_2,I_3)&=& \frac{\exp(-\frac{1}{2}(a_1I_1+b_1I_2+c_1I_3))}{8(n-1)|C|^{\frac{n}{2}}(d_1d_2d_3)^{n-1}}\sum_{k=n-1}^{\infty}k(-1)^{k-n+1}C_{k-n+1}^{n-1}(cos(\gamma))\\
%	&&I_k(d_1\sqrt{I_1I_2})I_k(d_2\sqrt{I_2I_3})I_k(d_3\sqrt{I_1I_3})
%\end{array}
%\end{equation}
%
%
%
%Para cada $i$:
%  
%Estimar $(\mu_i, L_i )$ por $(\hat{\mu}_i, \hat{L}_i)(Z_I)$ em uma primeira metade da faixa de dados.
%
%Estimar $(\mu_i, L_i )$ por $(\hat{\mu}_i, \hat{L}_i)(Z_E)$ em uma segunda metade da faixa de dados.
%
%Usando o estimador de máxima verossimilhança, 
%\begin{equation}\label{cap_acf_16}
%    (\hat{\mu}_i, \hat{L}_i)(Z_{\bigodot})= \text{arg}\,\max\limits_{(\mu, L)\in \mathbb{R}^{+}\times \mathbb{R}^{+}}%\ell(\mu,L;Z_i).\\
%\end{equation}
%Assim
%\begin{equation}\label{cap_acf_16}
%\begin{array}{ccc}
% \ell(\mu, L)&=&\ln\left(\prod_{k=1}^{n}f_{Z_{i}}(Z_{k};\mu,L)\right)\\
%  \ell(\mu, L)&=&\sum_{k=1}^{n}\ln\left(f_{Z_{i}}(Z_{k};\mu,L)\right)
% \end{array}
% \end{equation}
%Temos duas amostras
%\begin{equation}\label{cap_acf_16}
% \begin{array}{lll}
%\ell(\mu_I, L_I,\mu_E, L_E, j)&=&\sum_{k=1}^{j}     \left[   L_I\ln L_I +(L_I   - 1) \ln Z_{i}-L_I \ln \mu_I-\ln \Gamma(L_i) -%\frac{L_I}{\mu_I} Z_i \right]\\
%                                               &+&\sum_{k=j+1}^{N}\left[   L_E\ln L_E +(L_E - 1) \ln Z_{i}-L_E \ln \mu_E-\ln \Gamma(L_E) -\frac{L_E}{\mu_E} Z_i \right]\\
%\ell(\mu_I, L_I,\mu_E, L_E, j)&=&  L_I\ln L_I \sum_{k=1}^{j} 1 +(L_I   - 1) \sum_{k=1}^{j}  \ln Z_{i}-L_I \ln \mu_I\sum_{k=1}^{j} 1-\ln \Gamma(L_i) \sum_{k=1}^{j} 1  -\frac{L_I}{\mu_I} \sum_{k=1}^{j}   Z_i \\
%                                               &+&  L_E\ln L_E \sum_{k=j+1}^{N}1+(L_E - 1) \sum_{k=j+1}^{N}\ln Z_{i}- \ln %\mu_E\sum_{k=j+1}^{N}1-\ln \Gamma(L_E)\sum_{k=j+1}^{N} 1-\frac{L_E}{\mu_E} \sum_{k=j+1}^{N}Z_i \\
%\ell(\mu_I, L_I,\mu_E, L_E, j)&=&  L_I\ln L_I j-L_I \ln \mu_I j-\ln \Gamma(L_i) j \\
%&+& (L_I  - 1) \sum_{k=1}^{j}  \ln Z_{i}  -\frac{L_I}{\mu_I} \sum_{k=1}^{j}   Z_i \\
%                                               &+&  L_E\ln L_E (N-j)-L_E \ln \mu_E(N-j)-\ln \Gamma(L_E)(N-j)- \\
%                                               &+& (L_E - 1) \sum_{k=j+1}^{N}\ln Z_{i}-\frac{L_E}{\mu_E} \sum_{k=j+1}^{N}Z_i \\
%                                                \end{array}
% \end{equation}
%
%
%\section{Imagens PolSAR reais}
%\begin{figure}[hbt]
%\centering
%\includegraphics[width=4.0in]{grafico_pdf_lee_1994_razao_amplitude.pdf}
%	\caption{Distribuição razão de amplitudes {\it L- visadas}.}
%\label{fig2}
%\end{figure}
%\begin{figure}[hbt]
%\centering
%	\includegraphics[width=4.0in]{sf_amostras_b_r_y.pdf}
%	\vspace{-2.5cm}
%	\caption{Regiões de interesses (ROIs).}
%\label{fig2}
%\end{figure}
% A tabela mostra os coeficientes de correlação para as regiões destacadas na figura. Os coeficientes correlacionam os canais $(hh-hv)$, $(hh-vv)$ e $(vv-hv)$ respectivamente para o mar ( ROI azul), floresta (ROI vermelho) e zona urbana (ROI amarelo).
%\begin{table}[hbt]
%	\centering
%	\caption{Coeficientes de correlação.}\label{cap_acf_tab04}
%\begin{tabular}{@{}lccc@{}} \toprule
%	Coeficiente de correlação & Mar  & Floresta & Zona urbana \\ \midrule
%	$(hh-hv)$ & 0.5548 & 0.7024 &  0.7177 \\ 
%	$(hh-vv)$ & 0.8743 & 0.6633 &  0.6483\\ 
%	$(vv-hv)$ & 0.5128 & 0.6065 &  0.6175\\ \bottomrule
%\end{tabular}
%\end{table}
%
%A figura acima mostra a baía de San Francisco ($450 \times 600$) com três regiões de interesses destacadas oceâno, floresta e zona urbana respectivamente nas cores azul, vermelho e amarelo. As ROI's têm dimensão ($50 \times 50$) adquirindo dados nos três canais de intensidade. O histograma e as pdf's teóricas são mostradas na figura abaixo. No cálculo das pdf's foi usado a equação 
%\begin{equation}\label{cap_acf_23}
%	f_{Z_{i}}(Z_{i};\frac{L}{\sigma_{i}^2},L)=\frac{L^{L}Z_{i}^{L-1}}{\sigma_{i}^{2L}\Gamma(L)} \exp(-L\frac{Z_{i}}{\sigma_{i}^2}), \\
%\end{equation}
%sendo $L=\{2,3,4\}$ e $\sigma_{i}^2$ a média de todos as entradas das respectivas regiões de interesses conforme \cite{nhfc}.
%
%\begin{figure}[hbt]
%\centering
%	\includegraphics[width=4.0in]{graf_pdf_roi_mar_hh.pdf}
%	\caption{Regiões de interesses (ROIs).}
%\label{fig2}
%\end{figure}
%\textcolor{red}{OBS: Tem algo errado no gráfico, provavelmente a estimativa de $\sigma_{i}$.}
% ****************************************************************

%\chapter{Fusão de evidências} \label{cap_fusao}
\section{Métodos de fusão para as evidências de bordas}
As evidências de bordas estão armazenadas em  $n_c$ imagens binárias $\{\widehat{\bm\jmath}_c\}_{1\leq c\leq n_c}$ em que o pixel de valor 1 denota uma estimativa de borda e o pixel de valor 0 denota um elemento onde não foi detectado borda. As imagens (matrizes) têm tamanho  $m\times n$; onde podemos definir $\ell= mn$. Estas imagens serão usadas para a fusão resultando na imagem binária $I_\text{F}$.

Comparamos os resultados de seis técnicas de fusão:
\begin{itemize} 
\item média simples, 
\item transformada wavelet discreta multi-resolução (MR-DWT),
\item transformada wavelet estacionária multi-resolução (MR-SWT), 
\item análise de componentes principais (PCA),
\item estatísticas ROC, e
\item decomposição em valores singulares com multi-resolução (MR-SVD).
\end{itemize}
\subsection{Fusão por média simples}
A fusão por média simples realiza a média aritmética das evidências de bordas para cada cada canal,
$\bm I_\text{F}(x,y)=(n_c)^{-1}\sum_{c=1}^{n_c} \widehat{\bm\jmath}_c(x,y)$, onde $nc$ é o número de canais a serem utilizados na fusão. A figura~\eqref{fig:cap_fusao_media_simples} apresenta o fluxograma para a fusão por média simples, podemos obter mais detalhes na referência \cite{mit}.

\pgfdeclarelayer{background}
\pgfdeclarelayer{foreground}
\pgfsetlayers{background,main,foreground}
%
\pgfdeclarelayer{background}
\pgfdeclarelayer{foreground}
\pgfsetlayers{background,main,foreground}
\tikzstyle{sensor}=[draw, fill=blue!20, text width=5em, 
    text centered, minimum height=2.5em,drop shadow]
\tikzstyle{ann} = [above, text width=5em, text centered]
\tikzstyle{wa} = [sensor, text width=15em, fill=red!20, 
    minimum height=6em, rounded corners, drop shadow]
\tikzstyle{sc} = [sensor, text width=13em, fill=red!20, 
    minimum height=10em, rounded corners, drop shadow]
\def\blockdist{2.3}
\def\edgedist{2.5}
	\begin{figure}[htb!]
\centering
\begin{tikzpicture}
	\node (wa) [wa]  {$\bm I_\text{F}(x,y)=(n_c)^{-1}\sum_{c=1}^{n_c} \widehat{\bm\jmath}_c(x,y)$};
	\path (wa.west)+(-3.2,1.5) node (e1) [sensor] {$\widehat{\bm\jmath}_1(x,y)$};
    \path (wa.west)+(-3.2,0.5) node (e2)[sensor] {$\widehat{\bm\jmath}_2(x,y)$};
    \path (wa.west)+(-3.2,-1.0) node (dots)[ann] {$\vdots$}; 
    \path (wa.west)+(-3.2,-2.0) node (e3)[sensor] {$\widehat{\bm\jmath}_3(x,y)$};    
%
    \path [draw, ->] (e1.east) -- node [above] {} 
        (wa.160) ;
    \path [draw, ->] (e2.east) -- node [above] {} 
        (wa.180);
    \path [draw, ->] (e3.east) -- node [above] {} 
        (wa.200);   
\end{tikzpicture}
	\caption{Fusão por média simples.}
\label{fig:cap_fusao_media_simples}
\end{figure}
%
\subsection{Fusão usando o método wavelet discreta Multi-Resolução -- MR-DWT} 
Esta seção é baseada em~\cite{n_r}. Vamos aplicar os filtros DWT em cada imagem binaria $\bm{\widehat\jmath}_c$: um filtro de passa baixa \textbf{L}  na direção vertical, e um filtro de passa alta \textbf{H} na direção horizontal, então ambos são \textit{down-sampled} para formar as matrizes de coeficientes~$\bm{\widehat\jmath}_{c\text{L}}$~e~$\bm{\widehat\jmath}_{c\text{H}}$.  

As operações são repetidas nas matrizes de coeficientes~$\bm{\widehat\jmath}_{c\text{L}}$~e~$\bm{\widehat\jmath}_{c\text{H}}$ resultando em $\bm{\widehat\jmath}_{c\text{LL}}$, $\bm{\widehat\jmath}_{c\text{LH}}$, $\bm{\widehat\jmath}_{c\text{HL}}$,~e~$\bm{\widehat\jmath}_{c\text{HH}}$.

A fusão de evidências de bordas tem os seguintes passos para cada nível de resolução:
\begin{enumerate}
\item Calculate a decomposiçao DWT $\bm{\widehat\jmath}_{c\text{LL}}$, $\bm{\widehat\jmath}_{c\text{LH}}$, $\bm{\widehat\jmath}_{c\text{HL}}$, and $\bm{\widehat\jmath}_{c\text{HH}}$, para cada canal.
\item Compute $\bm{\bar\jmath}_{c\text{HH}}$, a média pixel a pixel para todas as $\bm{\widehat\jmath}_{c\text{HH}}$ decomposições.
\item Encontrar o máximo pixel a pixel de $\bm{\widehat\jmath}_{c\text{LL}}$, $\bm{\widehat\jmath}_{c\text{LH}}$, $\bm{\widehat\jmath}_{c\text{HL}}$ resultando em $\bm{\bar\jmath}_{c\text{LL}}$, $\bm{\bar\jmath}_{c\text{LH}}$, e $\bm{\bar\jmath}_{c\text{HL}}$.
\item O resultado da fusão $I_\text{F}$ é a transformação inversa DWT para os coeficientes das matrizes $\bm{\bar\jmath}_{c\text{HH}}$, $\bm{\bar\jmath}_{c\text{LL}}$, $\bm{\bar\jmath}_{c\text{LH}}$, e $\bm{\bar\jmath}_{c\text{HL}}$.
\end{enumerate}

O fluxograma para o método de fusão MR-DWT pode ser visto na figura \eqref{fig:cap_fusao_dwt}. Na figura W e $\text{W}^{-1}$ representam a transformação para obter a decomposição wavelet e a transformação inversa respectivamente, 
\pgfdeclarelayer{background}
\pgfdeclarelayer{foreground}
\pgfsetlayers{background,main,foreground}
\tikzstyle{sensor}=[draw, fill=blue!20, text width=5.5em, 
    text centered, minimum height=2.5em,drop shadow]
\tikzstyle{ann} = [above, text width=5em, text centered]
\tikzstyle{wa} = [sensor, text width=7em, fill=red!20, 
    minimum height=3em, rounded corners, drop shadow]
\tikzstyle{sc} = [sensor, text width=10em, fill=red!20, 
    minimum height=7em, rounded corners, drop shadow]
\def\blockdist{2.3}
\def\edgedist{2.5}
	\begin{figure}[htb!]
\begin{tikzpicture}
	\path (wa.west)+(-3.0,1.5) node (swtnode1) [sensor] {$\text{Coef DWT}_1$};
	\path (wa.west)+(-3.0,0.5) node (swtnode2) [sensor] {$\text{Coef DWT}_2$};
	\path (wa.west)+(-3.0,-1.0) node (dots)[ann] {$\vdots$}; 
    \path (wa.west)+(-3.0,-2.0) node (swtnode3)[sensor] {$\text{Coef DWT}_N$};  
%
	\path (wa.west)+(-6.2,1.5) node (e1) [sensor] {$\widehat{\bm\jmath}_1(x,y)$};
    \path (wa.west)+(-6.2,0.5) node (e2)[sensor] {$\widehat{\bm\jmath}_2(x,y)$};
    \path (wa.west)+(-6.2,-1.0) node (dots)[ann] {$\vdots$}; 
    \path (wa.west)+(-6.2,-2.0) node (e3)[sensor] {$\widehat{\bm\jmath}_N(x,y)$};    
    \path (wa.west)+(1.5,1.0) node (swtnodefus) [wa] {Fusão dos coeficientes\\
                                                       wavelets};                                                      
    \path (wa.west)+(6.0,1.0) node (imagefus) [wa] {Imagem fusão};
    \path [draw, ->] (e1.east) -- node [above] {W} 
        (swtnode1.180) ;
    \path [draw, ->] (e2.east) -- node [above] {W} 
        (swtnode2.180);
    \path [draw, ->] (e3.east) -- node [above] {W} 
        (swtnode3.180);
%
    \path [draw, ->] (swtnode1.east) -- node [above] {} 
        (swtnodefus.160) ;
    \path [draw, ->] (swtnode2.east) -- node [above] {} 
        (swtnodefus.180);
    \path [draw, ->] (swtnode3.east) -- node [above] {} 
        (swtnodefus.200);      
    \path [draw, ->] (swtnodefus.east) -- node [above] {$W^{-1}$}      
        (imagefus.west);        
\end{tikzpicture}
	\caption{Fusão MR-DWT.}
\label{fig:cap_fusao_dwt}
\end{figure}
%
\subsection{Fusão usando o método wavelet discreta Multi-Resolução -- MR-SWT} 
Esta sectão é baseada em~\cite{n_r, jjly}. A diferenção entre os métodos MR-DWT e MR-SWT é a substituição do operador transformada discrete wavelet (DWT) por operador transformada estacionária wavelet (SWT).

O fluxograma para o método de fusão MR-SWT pode ser visto na figura \eqref{fig:cap_fusao_swt}. Na figura W e $\text{W}^{-1}$ representam a transformação para obter a decomposição wavelet e a transformação inversa respectivamente
\pgfdeclarelayer{background}
\pgfdeclarelayer{foreground}
\pgfsetlayers{background,main,foreground}
\tikzstyle{sensor}=[draw, fill=blue!20, text width=5.5em, 
    text centered, minimum height=2.5em,drop shadow]
\tikzstyle{ann} = [above, text width=5em, text centered]
\tikzstyle{wa} = [sensor, text width=7em, fill=red!20, 
    minimum height=3em, rounded corners, drop shadow]
\tikzstyle{sc} = [sensor, text width=10em, fill=red!20, 
    minimum height=7em, rounded corners, drop shadow]
\def\blockdist{2.3}
\def\edgedist{2.5}
	\begin{figure}[htb!]
\begin{tikzpicture}
	\path (wa.west)+(-3.0,1.5) node (swtnode1) [sensor] {$\text{Coef SWT}_1$};
	\path (wa.west)+(-3.0,0.5) node (swtnode2) [sensor] {$\text{Coef SWT}_2$};
	\path (wa.west)+(-3.0,-1.0) node (dots)[ann] {$\vdots$}; 
    \path (wa.west)+(-3.0,-2.0) node (swtnode3)[sensor] {$\text{Coef SWT}_N$};  
	
	\path (wa.west)+(-6.2,1.5) node (e1) [sensor] {$\widehat{\bm\jmath}_1(x,y)$};
    \path (wa.west)+(-6.2,0.5) node (e2)[sensor] {$\widehat{\bm\jmath}_2(x,y)$};
    \path (wa.west)+(-6.2,-1.0) node (dots)[ann] {$\vdots$}; 
    \path (wa.west)+(-6.2,-2.0) node (e3)[sensor] {$\widehat{\bm\jmath}_N(x,y)$};    
    \path (wa.west)+(1.5,1.0) node (swtnodefus) [wa] {Fusão dos coeficientes\\
                                                       wavelets};                                                      
    \path (wa.west)+(6.0,1.0) node (imagefus) [wa] {Imagem fusão};
    \path [draw, ->] (e1.east) -- node [above] {W} 
        (swtnode1.180) ;
    \path [draw, ->] (e2.east) -- node [above] {W} 
        (swtnode2.180);
    \path [draw, ->] (e3.east) -- node [above] {W} 
        (swtnode3.180);
%
    \path [draw, ->] (swtnode1.east) -- node [above] {} 
        (swtnodefus.160) ;
    \path [draw, ->] (swtnode2.east) -- node [above] {} 
        (swtnodefus.180);
    \path [draw, ->] (swtnode3.east) -- node [above] {} 
        (swtnodefus.200);      
    \path [draw, ->] (swtnodefus.east) -- node [above] {$W^{-1}$}      
        (imagefus.west);        
\end{tikzpicture}
	\caption{Fusão MR-SWT.}
\label{fig:cap_fusao_swt}
\end{figure}
%
\subsection{Fusão usando o método análise das componentes principais -- PCA} 
Esta seção é baseada em~\cite{n_r,mit}. O método pode ser resumido nos seguinte passos:
\begin{enumerate}
\item Armazene as imagens binárias $\bm{\widehat\jmath}_c$ em colunas para obter am matriz $\bm X_{\ell\times n_c}$.
\item Calcule a matriz de covariância $\bm C_{n_c\times n_c}$ de $\bm X_{\ell\times n_c}$.
\item Calcule as matrizes de autovalores ($\bm\Lambda$) e autovetores ($\bm V$) da matriz de covariância, ordene os autovalores e seus respectivos autovalores de maneira decrescente.
\item Encontre o vetor $\bm P=(P(1),\dots,P(n_c))=(\sum_{c=1}^{n_c} V(c))^{-1}{\bm V}$, onde $\bm V_c$ é o autovetor associado com o maior autovalor de $\bm X$; note que $\sum_{c=1}^{n_c}\bm P_c=1$.
\item Realize a fusão $\bm I_F(x,y)=\sum_{c=1}^{n_c} P(c)\bm{\widehat\jmath}_c(x,y)$.
\end{enumerate}

O fluxograma para a fusão usando o método PCA pode ser visto na figura \eqref{fig:cap_fusao_pca}.
\pgfdeclarelayer{background}
\pgfdeclarelayer{foreground}
\pgfsetlayers{background,main,foreground}
\tikzstyle{sensor}=[draw, fill=blue!20, text width=5em, 
    text centered, minimum height=2.5em,drop shadow]
\tikzstyle{ann} = [above, text width=5em, text centered]
\tikzstyle{wa} = [sensor, text width=7em, fill=red!20, 
    minimum height=3em, rounded corners, drop shadow]
\tikzstyle{sc} = [sensor, text width=15em, fill=red!20, 
    minimum height=7em, rounded corners, drop shadow]
\def\blockdist{2.3}
\def\edgedist{2.5}
	\begin{figure}[htb!]
\begin{tikzpicture}
	\path (wa.west)+(-2.0,0.0) node (pcanode) [wa] {$\text{PCA}$};
	\path (wa.west)+(-6.2,1.5) node (e1) [sensor] {$\widehat{\bm\jmath}_1(x,y)$};
    \path (wa.west)+(-6.2,0.5) node (e2)[sensor] {$\widehat{\bm\jmath}_2(x,y)$};
    \path (wa.west)+(-6.2,-1.0) node (dots)[ann] {$\vdots$}; 
    \path (wa.west)+(-6.2,-2.0) node (e3)[sensor] {$\widehat{\bm\jmath}_N(x,y)$};    
    \path (wa.west)+(4.0,0.0) node (pcanodefus) [sc] {$\bm I_\text{F}(x,y)=\sum_{c=1}^{n_c} P(c)\bm{\widehat\jmath}_c(x,y)$};
    \path [draw, ->] (e1.east) -- node [above] {} 
        (pcanode.160) ;
    \path [draw, ->] (e2.east) -- node [above] {} 
        (pcanode.180);
    \path [draw, ->] (e3.east) -- node [above] {} 
        (pcanode.200);
        %
    \path [draw, ->] (pcanode.east) -- node [above] {} 
        (pcanodefus.180) ;  
\end{tikzpicture}
	\caption{Fusão PCA.}
\label{fig:cap_fusao_pca}
\end{figure}

\subsection{Fusão usando a estatística ROC}
O método ROC foi descrito em~\cite{gs,fawcett}, de maneira geral podemos afirmar que o método baseado em estatística ROC funciona da seguinte maneira:
\begin{enumerate}
\item  Adicione as imagens binárias $\bm{\widehat\jmath}_c$ para produzir a matriz de frequência ($\bm V$). 
\item Aplique os limiares $t=1,\dots,n_c$ em $\bm V$ para gerar as matrizes $\bm{\text{M}_t}$.
\item Compare cada $\bm{\widehat\imath}_t$ com todos $\bm{\widehat\jmath}_c$, ache a matriz de confusão para gerar a curva ROC. O limiar ótimo corresponde ao ponto da curva ROC mais perto (no sentido da distancia euclidiana) da linha de diagnóstico.
\item A fusão $\bm{I_\text{F}}$ é a matriz $\bm{\widehat\imath}_t$ que corresponde ao limiar ótimo.
\end{enumerate}

Com o objetivo de descrever o método desenvolvemos os aspectos teóricos da estatísticas ROC.
\subsubsection{A estatística ROC}
As teorias dos artigos \citet{gs} e \citet{fawcett} serão descritas baseadas nas propriedades estatísticas da curva \textit{Receiver Operating Characteristics} (ROC) aplicadas em imagens PolSAR.

As curvas ROC são técnicas para visualizar, organizar e selecionar classificadores aplicados em aprendizado de máquina, visão computacional, inteligência artificial entre outras áreas similares demostrando a capacidade do método para efetuar avaliações e comparações de algoritmos. 

A construção da curva ROC consiste em um problema de classificação com duas classe rotuladas como instâncias do conjunto $\{\mathbf{p},\mathbf{n}\}$, onde $\mathbf{p}$ representa a classe positiva e $\mathbf{n}$ representa a classe negativa. Estabelecendo as instâncias e o classificador podemos definir a matriz de confusão $2\times 2$. 

A tabela (\ref{tab:matrixz_conf}) mostra a matriz de confusão, onde
\begin{enumerate}
    \item Se a instância é positiva e classificada como positiva então é definida como positivo verdadeiro TP.
    \item Se a instância é positiva e classificada como negativa então é definida como negativo falso FN.
	\item Se a instância é negativa e classificada como negativa então é definida como negativo verdadeiro TN
	\item Se a instância é negativa e classificada como positiva então é definida como positivo falso FP
\end{enumerate}

\begin{table}[hbt]
	\centering
	\caption{Matriz de confusão.}\label{tab:matrixz_conf}
\begin{tabular}{@{}cll@{}} \toprule
	                        & \multicolumn{2}{c}{Classes definidas como verdadeiras}           \\ \midrule
	 Classes preditas       & $\mathbf{p}$                & $\mathbf{n}$                \\
                 $\mathbf{p}$& Positivos verdadeiros (TP) & Positivos falsos (FP)      \\ 
	             $\mathbf{n}$& Negativos falsos      (FN) & Negativos verdadeiros (TN) \\ \bottomrule 
\end{tabular}
\end{table}

Os valores da diagonal principal matriz de confusão representam as classificações realizadas corretamente, enquanto os elementos da diagonal secundária representam as classificações incorretas. A soma de todas as possibilidades de resultados em uma classificação retorna o valor TP+FN+FP+TN=1.

Definimos a prevalência como a soma dos positivos verdadeiros com os negativos falsos, P=TP+FN, portanto podemos afirmar que idealmente a prevalência deveria aproximar-se de TP. E definimos \text{Nível}-Q como a soma dos positivos verdadeiros com os positivos falsos, Q=TP+FP, o qual deveria aproximar-se de TP em situações ideais. De acordo com as definições acima, no detector de bordas otimizado a prevalência e o \text{Nível-Q} deverian ser iguais, isto é, P=\text{Nível}-Q. Além disso, se definir N=FP+TN, e recorrer ao fato TP+FN+FP+TN=1, garantimos que P+N=1.  

A matriz de confusão serve de base para a construção da curva ROC, definindo as seguintes métricas,
\begin{equation}\label{eq:tp_rate}
	tp_{\text{rate}}=\frac{\text{TP}}{\text{P}},
\end{equation}
\begin{equation}\label{eq:fp_rate}
	fp_{\text{rate}}=\frac{\text{FP}}{\text{N}}.
\end{equation}

 Na métrica conhecida por razão de positivos verdadeiros, taxa de acerto, \textit{recall} ou  sensibilidade,~$tp_{\text{rate}}=\frac{\text{TP}}{\text{P}}=\frac{\text{TP}}{\text{TP+FN}}$,~se o número de negativos falsos tendem a zero o valor da métrica aproxima-se do valor unitário. Assim como, na métrica conhecida como razão de positivos falsos,~$fp_{\text{rate}}=\frac{\text{FP}}{\text{N}}=\frac{\text{FP}}{\text{FP+TN}}$,~se o número de verdadeiros positivos tendem a zero o valor da métrica aproxima-se do valor unitário.
\subsubsection{Fusão das evidências de bordas usando a estatística ROC}
O método de fusão de evidências baseado na estatística ROC consiste em aplicar a estimativa de máxima verossimilhança em cada canal da imagem PolSAR, gerando imagens binárias com as evidências de bordas $\bm{\widehat\jmath}_c$, sendo $c=1,\dots,n_c$, onde $n_c$ é o número de canais. Após, construímos a matriz de frequência V de mesmo tamanho de $\bm{\widehat\jmath}_c$, tal que em cada pixel é armazenado um valor ao qual corresponde a frequência de ocorrências de evidências de bordas em cada canal $\bm{\widehat\jmath}_c$. Ou seja, a matriz V é a soma pixel a pixel de todas as imagens de  evidências de bordas $\bm{\widehat\jmath}_c$. Podemos afirmar que quanto maior o valor de um pixel, maior sua probabilidade de ser uma borda.

Na matriz de frequência V são aplicados limiares~$\text{CT}_t$, onde $t=1,\dots,n_c$ gerando matrizes chamadas mapas de evidências de bordas $\bm{\widehat\imath}_t$. O objetivo do método é estimar automaticamente o limiar correspondente CT ótimo, proveniente do conjunto de limiares parciais $\text{CT}_t$.

O detector de bordas baseado nas estatística ROC é proposto como um método de fusão de evidências de bordas em diferentes canais da imagem PolSAR, e usado como ferramenta para detectar bordas.

Definições são propostas para desenvolver o algoritmo de detecção de bordas usando a estatística ROC. Então, de maneira similar a tabela \eqref{tab:matrixz_conf} definimos $\{e, ne\}$ respectivamente como bordas e não bordas reais na imagem, e $\{E, NE\}$ respectivamente como bordas preditas e não bordas preditas pelo classificador. A tabela~\ref{tab:matriz_conf_roc} redefini a matriz de confusão usando o método da máxima verossimilhança como classificador.
\begin{table}[hbt]
	\centering
	\caption{Matriz de confusão para as evidências de bordas.}\label{tab:matriz_conf_roc}
\begin{tabular}{@{}lll@{}} \toprule
	     & $e$  & $ne$  \\ \midrule
	$E$  & Positivos verdadeiros (TP) & Positivos falsos     (FP)  \\ 
	$NE$ & Negativos falsos       (FN)& Negativos verdadeiros (TN)\\ \bottomrule  
\end{tabular}
\end{table}

Para construir a curva ROC aplicamos o método da estimativa de máxima verossimilhança em cada canal da imagem PolSAR, ou componentes derivados desse canais, gerando assim imagens binárias com evidências de bordas $\bm{\widehat\jmath}_c$. Como descrito acima, realizando a soma de todas a matrizes $\bm{\widehat\jmath}_c$ resultando na matriz de frequência $\text{V}=\sum_{c=1}^{n_c}\bm{\widehat\jmath}_c$. Na matriz V é aplicado os limiares $\text{CT}_t$ com $t=1,\dots,n_c$ gerando os mapas de evidências de bordas $\bm{\widehat\imath}_t$. O fluxograma do  processo é mostrado na figura \eqref{fig:cap_fusao_roc1}.

\pgfdeclarelayer{background}
\pgfdeclarelayer{foreground}
\pgfsetlayers{background,main,foreground}
\tikzstyle{sensor}=[draw, fill=blue!20, text width=5em, 
    text centered, minimum height=2.5em,drop shadow]
\tikzstyle{ann} = [above, text width=5em, text centered]
\tikzstyle{wa} = [sensor, text width=10em, fill=red!20, 
    minimum height=6em, rounded corners, drop shadow]
\tikzstyle{sc} = [sensor, text width=13em, fill=red!20, 
    minimum height=10em, rounded corners, drop shadow]
\def\blockdist{2.3}
\def\edgedist{2.5}

\begin{figure}[htb!]
\centering
\begin{tikzpicture}
	\node (wa) [wa]  {$V=\sum_{c=1}^{N}\bm{\widehat\jmath}_c$};
	\path (wa.west)+(-3.2,1.5) node (e1) [sensor] {$\bm{\widehat\jmath}_1$};
    \path (wa.west)+(-3.2,0.5) node (e2)[sensor] {$\bm{\widehat\jmath}_2$};
    \path (wa.west)+(-3.2,-1.0) node (dots)[ann] {$\vdots$}; 
    \path (wa.west)+(-3.2,-2.0) node (e3)[sensor] {$\bm{\widehat\jmath}_N$};    
%   
    \path (wa.east)+(3.2,1.5) node (m1) [sensor] {$\bm{\widehat\imath}_1$};
    \path (wa.east)+(3.2,0.5) node (m2) [sensor] {$\bm{\widehat\imath}_2$};
    \path (wa.east)+(3.2,-1.0) node (dots)[ann] {$\vdots$}; 
    \path (wa.east)+(3.2,-2.0) node (m3) [sensor] {$\bm{\widehat\imath}_N$};
%
    \path [draw, ->] (e1.east) -- node [above] {} 
        (wa.160) ;
    \path [draw, ->] (e2.east) -- node [above] {} 
        (wa.180);
    \path [draw, ->] (e3.east) -- node [above] {} 
        (wa.200);
	\path [draw, ->] (wa.east) -- node [above] {\tiny{$\text{CT}_1$}} 
        (m1.west);
	\path [draw, ->] (wa.east) -- node [above] {\tiny{$\text{CT}_2$}} 
        (m2.west);
	\path [draw, ->] (wa.east) -- node [right] {\tiny{$\text{CT}_N$}} 
        (m3.west);
%               
%    \path (wa.south) +(0,-\blockdist) node (asrs) {Estrutura geral da fusão de evidência proposta};
  
    \begin{pgfonlayer}{background}
        \path (e1.west |- e1.north)+(-0.5,0.3) node (a) {};
        \path (wa.south -| wa.east)+(+0.5,-0.3) node (b) {};
        \path (m3.east |- m3.east)+(+0.5,-0.75) node (c) {};
       %   
        \path[fill=yellow!20,rounded corners, draw=black!50, dashed]
            (a) rectangle (c);           
       %     
    \end{pgfonlayer}
   
\end{tikzpicture}
	\caption{Fluxograma para a aplicação dos limiares na fusão de evidência ROC.}
\label{fig:cap_fusao_roc1}
\end{figure}

Definimos as entradas para cada matriz de confusão $t$,  
\begin{equation}\label{eq:verdadeiro_positivo_j}
	\text{TP}^c_t= \frac{1}{m\cdot n}\sum_{k=1}^{m}\sum_{l=1}^{n} \bm{\widehat\imath}_{E_t}\cap \bm{\widehat\jmath}_{E_c}, \\
\end{equation}
\begin{equation}\label{eq:falso_positivo_j}
	\text{FP}^c_t= \frac{1}{m\cdot n}\sum_{k=1}^{m}\sum_{l=1}^{n} \bm{\widehat\imath}_{E_t}\cap \bm{\widehat\jmath}_{NE_c}, \\
\end{equation}
\begin{equation}\label{eq:verdadeiro_negativo_j}
	\text{TN}^c_t= \frac{1}{m\cdot n}\sum_{k=1}^{m}\sum_{l=1}^{n} \bm{\widehat\imath}_{NE_t}\cap \bm{\widehat\jmath}_{NE_c}, \\
\end{equation}
\begin{equation}\label{eq:falso_negativo_j}
	\text{FN}^c_t= \frac{1}{m\cdot n}\sum_{k=1}^{m}\sum_{l=1}^{n} \bm{\widehat\imath}_{NE_t}\cap \bm{\widehat\jmath}_{E_c}, \\
\end{equation}
e construímos a matriz de confusão para cada $j$ fixado arbitrariamente, desta forma, teremos $n_c$ matrizes de confusão como a mostrada na tabela \eqref{tab:matriz_conf_roc_j},
\begin{table}[htb!]
	\centering
	\caption{Matriz de confusão para cada $\bm{\widehat\imath}_t$}\label{tab:matriz_conf_roc_j}
\begin{tabular}{@{}lll@{}} \toprule
	& $\bm{\widehat\jmath}_{e_c}$  & $\bm{\widehat\jmath}_{ne_c}$  \\ \midrule
	$\bm{\widehat\imath}_{E_t}$    & $TP^c_t$ &  $FP^c_t$  \\ 
	$\bm{\widehat\imath}_{NE_t}$   & $FN^c_t$ &  $TN^c_t$\\ \bottomrule 
\end{tabular}
\end{table}
definimos as médias para $c=1,\dots,n_c$,
\begin{equation}\label{eq:verdadeiro_positivo_media}
	\overline{\text{TP}}_t=\frac{1}{n_c}\sum_{c=1}^{n_c} \text{TP}^c_t
\end{equation}
\begin{equation}\label{eq:falso_positivo_media}
	\overline{\text{FP}_t}=\frac{1}{n_c}\sum_{c=1}^{n_c} \text{TN}^c_t
\end{equation}
\begin{equation}\label{eq:verdadeiro_negativo_media}
	\overline{\text{TN}_t}=\frac{1}{n_c}\sum_{c=1}^{n_c} \text{FP}^c_t
\end{equation}
\begin{equation}\label{eq:falso_negativo_media}
	\overline{\text{FN}_t}=\frac{1}{n_c}\sum_{c=1}^{n_c} \text{FN}^c_t
\end{equation}

A figura \eqref{cap_fusao_fig02} mostra a comparação entre a imagem $\bm{\widehat\imath}_t$ fixada arbitrariamente com todos as imagens~$\bm{\widehat\jmath}_c$ gerando a métrica~$\text{TP}_t$. A notação $\overline{\cap \bm{\widehat\jmath}_c}$~significa que a comparação realizada é a intersecção pixel a pixel entre $\bm{\widehat\imath}_t$ e $\bm{\widehat\jmath}_c$, fazendo a média de acordo com as equações  as~\eqref{eq:verdadeiro_positivo_media},~\eqref{eq:falso_positivo_media}~\eqref{eq:verdadeiro_negativo_media}~e~\eqref{eq:falso_negativo_media}. A notação $+$ refere-se a média de todos os $\text{TP}_t$, ou demais probabilidades $\text{FP}_t$, $\text{FN}_t$ e $\text{TN}_t$.
\tikzstyle{sensor_1}=[draw, fill=blue!20, text width=2.5em, 
    text centered, minimum height=2em,drop shadow]
\tikzstyle{ann_1} = [above, text width=5em, text centered]
\tikzstyle{wa_1} = [sensor, text width=2em, fill=red!20, 
    minimum height=2em, rounded corners, drop shadow]
\tikzstyle{wa1_1} = [sensor, text width=2em, fill=red!20, 
    minimum height=2em, rounded corners, drop shadow]
\begin{figure}[hbt!]
\centering
\begin{tikzpicture}
\node[wa_1] (wa_1) at (0.0,0.0) {$\bm{\widehat\imath}_t$};
\node[wa1_1] (wa1_1) at (4.0,0.0) {$\overline{\text{TP}}_t$};
%
    \path (wa_1.west)+(2.5,1.5) node (e1_1) [sensor_1] {$\text{TP}_1$};
    \path (wa_1.west)+(2.5,0.5) node (e2_1)[sensor_1] {$\text{TP}_2$};
    \path (wa_1.west)+(2.5,-1.0) node (dots)[ann_1] {$\vdots$}; 
    \path (wa_1.west)+(2.5,-2.0) node (e3_1)[sensor_1] {$\text{TP}_N$};    
%
	\path [draw, ->] (wa_1.east) -- node [left] {\tiny{$\overline{\cap \bm{\widehat\jmath}_1}$}} 
        (e1_1.180) ;
	\path [draw, ->] (wa_1.east) -- node [below] {\tiny{$\overline{\cap \bm{\widehat\jmath}_2}$}} 
        (e2_1.180);
	\path [draw, ->] (wa_1.east) -- node [right] {\tiny{$\overline{\cap \bm{\widehat\jmath}_3}$}} 
        (e3_1.180);
	\path [draw, ->] (e1_1.east) -- node [right] {\tiny{$+$}} 
        (wa1_1.160);
	\path [draw, ->] (e2_1.east) -- node [above] {\tiny{$+$}} 
        (wa1_1.180);
	\path [draw, ->] (e3_1.east) -- node [right] {\tiny{$+$}} 
        (wa1_1.200);
  
    \begin{pgfonlayer}{background}
        \path (wa_1.west |- wa_1.north)+(5.25,1.75) node (a) {};
        \path (e1_1.south -| e1_1.north)+(-2.75,-3.75) node (b) {};
        %\path (wa1.east |- wa1.east)+(+4.0,-0.5) node (c) {};
       %   
        \path[fill=yellow!20,rounded corners, draw=black!50, dashed]
            (a) rectangle (b);           
       %     
    \end{pgfonlayer}
    
\end{tikzpicture}
\caption{Estrutura para fusão de evidências com $j$ escolhido arbitrariamente.}
\label{cap_fusao_fig02}
\end{figure}

A curva ROC é encontrada calculando as razões de positivos verdadeiros, e a razão de positivos falsos para cada  mapas de bordas $\bm{\widehat\imath}_t$, definidas respectivamente como:  

\begin{equation}\label{eq:tp_rate_t}
	\text{TP}_{rate_{t}}=\frac{\overline{TP}_t}{\overline{TP}_t+\overline{FN}_t}. \\
\end{equation}
e
\begin{equation}\label{eq:tp_rate_t}
	\text{FP}_{rate_{t}}=\frac{\overline{FP}_t}{\overline{FP}_t+\overline{TN}_t}. \\
\end{equation}
onde $\overline{TP}_t+\overline{FN}_t$ representa o número médio de bordas verdadeiras para cada $\bm{\widehat\imath}_t$. Este número é sempre o mesmo independentemente de $t$ e denotado por P.

O gráfico para acurva ROC é bidimensional no qual os valores das razões de positivos falsos $\text{FP}_{rate_t}$ são medidos no eixo horizontal, e as razões de positivos verdadeiros $\text{TP}_{rate_t}$ são medidos no eixo vertical. Assim, cada mapa de borda $\bm{\widehat\imath}_t$ produz um ponto no gráfico $(\text{FP}_{rate_t}, \text{TP}_{rate_t})$ no plano ROC (eixos cartesianos) gerando a curva ROC. A referência~\cite{fawcett} mostra que o limiar ótimo ocorre na intersecção da curva ROC, ou perto da mesma, com a linha diagnóstico, linha vermelha na figura \eqref{fig:curva_roc}, formada conectando os pontos $(P,P)$ e $(0,1)$ no plano ROC. Escolhemos o limiar ótimo $CT$ correspondendo ao parâmetro $t$ que gera o ponto $(\text{FP}_{rate_t}, \text{TP}_{rate_t})$ mais próximo (métrica euclidiana) da reta diagnóstico determinando a imagem $\bm{\widehat\imath}_t$ com maior acurácia de bordas detectadas. 

A figura \eqref{fig:curva_roc} mostra a fusão baseada na estatística ROC dos canais simulados hh, hv, e vv descritos no capítulo (3). O limiar ótimo aconteceu quando  $t=2$ por ser o ponto com menor distância euclidiana para a linha diagnóstico (linha vermelha) resultando na imagem com melhor acurácia $\bm{\widehat\imath}_2$. 
\begin{figure}[hbt]
\centering
\includegraphics[width=4.0in]{curva_roc_3_canais.pdf}
	\caption{Curva ROC para a imagem simulada de duas folhas.}
\label{fig:curva_roc}
\end{figure}
%
\subsection{Fusão usando o método decomposição em valores singulares multi-resolução -- MR-SVD}
O método fusão MR-SVD, veja~\cite{naidu}, trabalha de maneira similar ao MR-DWT. A diferença consiste em mudar os filtros DWT por filtros SVD. O método pode ser resumido nos seguinte passos: 
\begin{enumerate}
\item Organize as imagens binárias $\bm{\widehat\jmath}_c$ armazenando como blocos $2\times 2$ não sobrepostos, onde cada bloco é colocado como um vetor $4\times 1$ ordenando em colunas para formar a matriz de dados $\bm X_1$ com dimensão ${4\times{\ell}/{4}}$.   
\item Ache a decomposição SVD de  $\bm X_1=\bm U_1 \bm S_1 \bm V_1^T$, onde $\bm U_1$ e $\bm V_1$ são unitárias e têm dimensões ${4\times 4}$ e ${\ell}/{4}\times{\ell}/{4}$ respectivamente. Os valores singulares são ordenados de maneira decrescente, e são colocados na diagonal principal da matriz $\bm S_1$, as demais entradas da matriz são zeros.  
\item Transforme a linhas de $\widehat{\bm X}_1=\bm U_1^T\bm X_1=\bm S_1 \bm V_1^T$ sobre novas matrizes com dimensões ${m}/{2}\times{n}/{2}$: $\{\bm\Phi_1, \bm\Psi_{1\text{V}}, \bm\Psi_{1\text{H}}, \bm\Psi_{1\text{D}}\}$. 
\item Recomece o procedimento em (1) para $\bm\Phi_r$ com $r=2$ até o menor nível de resolução $R$.
\item A MR-SVD decomposição em cada canal é  
\begin{equation}\nonumber
\widehat{\bm X}_c\rightarrow \left\{\bm \Phi_\text{R}^c,\{\bm\Psi_{r\text{V}}^c,\bm\Psi_{r\text{H}}^c,\bm\Psi_{r\text{D}}^c \}_{r=1}^\text{R},\{\bm U_r^c	\}_{r=1}^\text{R} \right\}.
\end{equation}
\item Depois da decomposição ser aplicadas em todos os canais, compute a média dos $\bm\Phi_R^c$ ($\bm\Phi_\text{R}^f$) no menor nível de resolução, e a média $\bm U_r^c$ ($\bm U_\text{r}^f$), para cada $r$, onde $f$ denota a fusão entre os canais.  
\item Ache o máximo pixel a pixel de $\bm\Psi_{r\text{V}}^c$, $\bm\Psi_{r\text{H}}^c$ e $\bm\Psi_{r\text{D}}^c$: $ \bm\Psi_{rV}^f$ $\bm\Psi_{r\text{V}}^f$, $\bm\Psi_{r\text{H}}^f$ e $\bm\Psi_{r\text{D}}^f$.
\item A fusão $\bm I_\text{F}$ é a transformação SVD para cada nível $r=\text{R},\dots,1$, 
\begin{equation}\nonumber
\bm I_\text{F}\leftarrow \left\{\bm \Phi_\text{R}^f,\{\bm\Psi_{r\text{V}}^f,\bm\Psi_{r\text{H}}^f,\bm\Psi_{r\text{D}}^f \}_{r=\text{R}}^1,\{\bm U_r^f\}_{r=\text{R}}^1 \right\}.
\end{equation}
\end{enumerate}               % AAB - Arquivo inserido 
\chapter{Resultados Numéricos}\label{resul_numericos}

\section{Aplicação em imagem simulada}
A metodologia (MLE) para a detecção será aplicada para uma imagem simulada baseada em~\cite{nhfc,gamf}. A imagem tem $400\times400$ pixels e foi gerada por duas amostras obedecendo a distribuíção Wishart. Para cada par de matrizes de covariância $\Sigma_{k_1}$, $\Sigma_{k_2}$ a imagem $I_{k_1,k_2}$ é simulada de acordo com, amostras de $W_G(\Sigma_{k_1}, 4)$ para a metade esquerda da imagem, e  amostras $W_G(\Sigma_{k_2}, 4)$ para a metade direita da imagem.


A imagem $400 \times 400$ pixels foi gerada
\begin{figure}[hbt]
	\centering
	\includegraphics[width=.5\linewidth]{phanton_gamf_dec_pauli}%
	\vspace{-2.5cm}
	\caption{Decomposição de Pauli aplicada a imagem simulada}
\label{fig:simulada_gamf_dec_pauli}
\end{figure}

A decomposição de Pauli é baseada na combinação linear dos canais de intensidades, 
$$(\mathbf{I_\text{hh}+I_{\text{vv}}}, \mathbf{I_\text{hh}-I_{\text{vv}}}, \mathbf{I_\text{hv}}).$$ 
Esta decomposição mostra a evidência de bordas em uma linha média da imagem, como apresentado na figura~\ref{fig:simulada_gamf_dec_pauli}. 

A imagem simulada  usa as intensidades da matriz de covariancia $\Sigma_{k_1}$ e $\Sigma_{k_2}$ definidas por
\begin{equation}\label{matriz_sigma_gamf_1}
	\hspace{2.75cm} \Sigma_{u}= \left[
\begin{array}{lll}
0.042811            & 0.000072-0.003180i & 0.010435+0.005022i\\
0.000072+0.003180i  & 0.035977           & 0.000784+0.004886i\\
0.010435-0.005022i  & 0.000784-0.004886i & 0.066498
\end{array}
\right],
\end{equation}
\begin{equation}\label{matriz_sigma_gamf_2}
 \Sigma_{f}= \left[
\begin{array}{lll}
0.014380            & 0.001333-0.000076i & -0.000755+0.001570i\\
0.001333+0.000076i  & 0.002789           & -0.001044+0.001101i\\
-0.000755-0.001570i &-0.001044-0.001101i & 0.015387
\end{array}
\right],
\end{equation}
as intensidades são as entradas da diagonal principal.  


\subsection{Função Densidade de Probabilidade Univariada Gamma}\label{pdf_uni_gamma_apl_simulada}

\subsubsection{Estimativa para os parâmetros ($\mu$, L)}\label{pdf_uni_gamma_apl_sim_param_mu_L}
As figuras~\eqref{fig:l_uni_gamma_param_mu_l}\subref{fig:l_uni_gamma_param_mu_l:1a},~\eqref{fig:l_uni_gamma_param_mu_l}\subref{fig:l_uni_gamma_param_mu_l:1b},~\eqref{fig:l_uni_gamma_param_mu_l}\subref{fig:l_uni_gamma_param_mu_l:1c} mostram os gráficos das funções \eqref{eq:TotalLogLikelihood} indicando as evidências de bordas nos canais considerados, os quais foram gerados com a estimativa dos parâmetros $(\mu,L)$, fixando arbitrariamente a linha 150 da imagem simulada, e definindo o coeficiente de folga para as extremidades igual a 14 pixeis. A característica comum dessas funções é não ser diferenciáveis, dificultando o uso de métodos de otimização que calculam derivada. O problema foi resolvido usando o método \textit{Simulated Annealing} generalizado (GenSA)~\cite{xgsh}, adequado para funções não diferenciáveis.

\begin{figure}[hbt]
	\centering
     \subfloat[Canal $\text{hh}$]{%
       \includegraphics[width=0.32\linewidth]{grafico_l_gamf_2017_sigmahh_param_mu_L_linha_150}\label{fig:l_uni_gamma_param_mu_l:1a}}
     \subfloat[Canal $\text{hv}$]{%
       \includegraphics[width=0.32\linewidth]{grafico_l_gamf_2017_sigmahv_param_mu_L_linha_150}\label{fig:l_uni_gamma_param_mu_l:1b}}
     \subfloat[Canal $\text{vv}$]{%
       \includegraphics[width=0.32\linewidth]{grafico_l_gamf_2017_sigmavv_param_mu_L_linha_150}\label{fig:l_uni_gamma_param_mu_l:1c}}
     \caption{Funções log-verossimilhanças com a estimativa de parâmetros $(\mu, L)$}
     \label{fig:l_uni_gamma_param_mu_l}
   \end{figure}	
 
O método estimativa de máxima verossimilhança (MLE) foi aplicado para cada linha da imagem simulada com duas amostras, e as evidências de bordas estão mostradas na figura \eqref{evidencias_hh_hv_vv_uni_gamma_gamf_mu_L}.
 \begin{figure}[H]
     \centering
     \subfloat[Evidências no canal $\text{hh}$]{
       \includegraphics[width=0.35\linewidth]{im_sim_gamf_hh_evid_param_mu_L_14_pixel} \label{evidencias_hh_hv_vv_uni_gamma_gamf_mu_L:a}
     }
     \subfloat[Evidências no canal $\text{hv}$]{
       \includegraphics[width=0.35\linewidth]{im_sim_gamf_hv_evid_param_mu_L_14_pixel} \label{evidencias_hh_hv_vv_uni_gamma_gamf_mu_L:b}
     }
     \subfloat[Evidências no canal $\text{vv}$]{
       \includegraphics[width=0.35\linewidth]{im_sim_gamf_vv_evid_param_mu_L_14_pixel} \label{evidencias_hh_hv_vv_uni_gamma_gamf_mu_L:c}
     }
    \caption{Evidências de bordas para os três canais de intensidade com ($\mu, L$) estimados.}
     \label{evidencias_hh_hv_vv_uni_gamma_gamf_mu_L} 
   \end{figure}
\subsubsection{Estimativa para o parâmetro $\mu$ com L fixo} \label{pdf_uni_gamma_apl_sim_param_mu}
As figuras~\eqref{fig:l_uni_gamma_param_mu}\subref{fig:l_uni_gamma_param_mu:1a},~\eqref{fig:l_uni_gamma_param_mu}\subref{fig:l_uni_gamma_param_mu:1b},~\eqref{fig:l_uni_gamma_param_mu}\subref{fig:l_uni_gamma_param_mu:1c} mostram os gráficos das funções~\eqref{eq:TotalLogLikelihood_L_fixo} indicando as evidências de bordas a serem encontradas nos canais considerados, as quais foram geradas com a estimativa do parâmetro $(\mu)$, com a definição $L=4$, fixando arbitrariamente a linha 150 da imagem simulada, e definindo o coeficiente de folga paras as extremidades igual a 14 pixeis. A característica comum dessas funções é não ser diferenciáveis, dificultando o uso de métodos de otimização que calculam a derivada da função. O problema foi resolvido usando o método \textit{Simulated Annealing} generalizado (GenSA)~\cite{xgsh}, adequado para funções não diferenciáveis.
\begin{figure}[hbt]
	\centering
     \subfloat[Canal $\text{hh}$ \label{fig:l_uni_gamma_param_mu:1a}]{%
       \includegraphics[width=0.32\linewidth]{grafico_l_gamf_2017_sigmahh_param_mu}}
     \subfloat[Canal $\text{hv}$ \label{fig:l_uni_gamma_param_mu:1b}]{%
       \includegraphics[width=0.32\linewidth]{grafico_l_gamf_2017_sigmahv_param_mu}}
     \subfloat[Canal $\text{vv}$ \label{fig:l_uni_gamma_param_mu:1c}]{%
       \includegraphics[width=0.32\linewidth]{grafico_l_gamf_2017_sigmavv_param_mu}}
     \caption{Funções log-verossimilhanças com a estimativa do parâmetro $(\mu)$ e L fixo}
     \label{fig:l_uni_gamma_param_mu}
   \end{figure}	

O método da estimativa de máxima verossimilhança (MLE) foi aplicado nas linhas da imagem simulada com duas amostras, e as evidências de bordas estão mostradas na figura \eqref{evidencias_hh_hv_vv_gamf_mu_estimado}.
 \begin{figure*}[hbt]
	\centering
     \subfloat[Evidências no canal $\text{hh}$  \label{evidencias_hh_hv_vv_gamf_mu_estimado:a}]{%
       \includegraphics[width=0.35\linewidth]{im_sim_gamf_hh_evid_param_mu_14_pixel}
     }
     \subfloat[Evidências no canal $\text{hv}$ \label{evidencias_hh_hv_vv_gamf_mu_estimado:b}]{%
       \includegraphics[width=0.35\linewidth]{im_sim_gamf_hv_evid_param_mu_14_pixel}
     }      
     \subfloat[Evidências no canal $\text{vv}$ \label{evidencias_hh_hv_vv_gamf_mu_estimado:c}]{%
       \includegraphics[width=0.35\linewidth]{im_sim_gamf_vv_evid_param_mu_14_pixel}
     }
    \caption{Evidências de bordas para os três canais de intensidade com $\mu$ estimado e L fixo.}
     \label{evidencias_hh_hv_vv_gamf_mu_estimado} 
   \end{figure*}
%   
\begin{figure*}[hbt]
	\centering
     \subfloat[Evidências no canal $\text{hh}$  \label{evidencias_hh_hv_vv_gamf_mu_estimado:a}]{%
       \includegraphics[width=0.35\linewidth]{im_sim_gamf_hh_evid_media_mu_14_pixel}
     }
     \subfloat[Evidências no canal $\text{hv}$ \label{evidencias_hh_hv_vv_gamf_mu_estimado:b}]{%
       \includegraphics[width=0.35\linewidth]{im_sim_gamf_hv_evid_media_mu_14_pixel}
     }      
     \subfloat[Evidências no canal $\text{vv}$ \label{evidencias_hh_hv_vv_gamf_mu_estimado:c}]{%
       \includegraphics[width=0.35\linewidth]{im_sim_gamf_vv_evid_media_mu_14_pixel}
     }
    \caption{Evidências de bordas para os três canais de intensidade com $\mu$ estimado usando a média.}
     \label{evidencias_hh_hv_vv_gamf} 
   \end{figure*}
%
\clearpage
\subsection{Distribuição univariada produto de intensidades}\label{sec:pdf_uni_prod_int_apl_sim_param}
Observamos que o uso da função densidade magnitude do produtos não é adequado para gerar a função log-verossimilhança, as mesmas tem características de serem planas dificultando o cálculo do valor máximo. O cálculo inadequado dos parâmetros geram uma função \eqref{eq:TotalLogLikelihood_prod_mag} com oscilação degenerando o ponto ótimo no processo de otimização.
\subsubsection{Estimativas para os parâmetros ($\rho$, L)}\label{sec:pdf_uni_prod_int_apl_sim_param_rho_L}
O método estimativa de máxima verossimilhança(MLE) usa a pdf \eqref{eq:pdf_mag_prod} para encotrar a função log-verossimilhança \eqref{eq:eq_log_vero_mag_prod_red}, para qual realizamos as estimativas dos parâmetros ($\rho$, L) usando o método de otimização BFGS.

Notamos que a forma das funções \eqref{eq:eq_log_vero_mag_prod_red} dificultam encontrar o ponto de máximo, devido a suas características de serem planas. Para gerar a figura~\ref{fig:loglikelihood_im_sim_gamma} escolhemos a linha 35 da imagem simulada e o pixel $j=150$, foram realizados testes com outras linhas da imagem simulada, assim como outros pixeis $j$ e os gráficos apresentavam o mesmo comportamento. 
   \begin{figure}[hbt]
   \centering
     \subfloat[Amostra a esquerda do pixel 150\label{fig:loglikelihood_im_sim_gamma_l_1_150}]{%
       \includegraphics[width=0.50\textwidth]{fig_pdf_mag_prod_r_35_1_to_150}}
       \subfloat[Amostra a direita do pixel 150\label{fig:loglikelihood_im_sim_gamma_l_100_400}]{%
       \includegraphics[width=0.50\textwidth]{fig_pdf_mag_prod_r_35_151_to_400}
     }
    \caption{Função log-verossimilhança.} \label{fig:loglikelihood_im_sim_gamma}   
\end{figure}
\subsubsection{Estimativa para o parâmetros ($\rho$)}\label{sec:pdf_uni_prod_int_apl_sim_param_rho}
O mesmo comportamento é observado quando fixamos o número de visadas e estimamos o parâmetro $\rho$, isto é, o uso da função densidade magnitude do produtos com o número de visadas fixo não é adequado para gerar a função log-verossimilhança, as mesmas tem características de serem planas dificultando o cálculo do valor máximo. O cálculo inadequado do coeficiente $\rho$ gera uma função \eqref{eq:TotalLogLikelihood_prod_mag_L_fixo} com muita oscilação, degenerando o ponto ótimo no processo de otimização.

O método estimativa de máxima verossimilhança(MLE) usa a pdf \eqref{eq:pdf_mag_prod} com visadas fixas, para encotrar a função log-verossimilhança \eqref{eq:eq_log_vero_mag_prod_red_L_fixo}, para qual realizamos a estimativa do parâmetro $\rho$ usando o método de otimização BFGS.

A figura \eqref{fig:pdf_prod_mag_flat_rho} mostra a função log-verossimilhança para distribuição univariada produto de magnitudes com radial(linha) 35 fixa, na radial escolhemos arbitrariamente o pixel igual a 150, e definimos o número de visadas(L) em 1,2, 4, e 8 para construir as funções. Notamos nas figuras que a função é plana evidenciando a dificuldade para encontrar o $\rho$ máximo. Esse fato tem consequência na construção da função de log-verossimilhança total \eqref{eq:TotalLogLikelihood_prod_mag_L_fixo}  apresentando forte oscilação e dificultando encontrar evidências de bordas para a distribuição produto de intensidades. 
\begin{figure}[hbt]
	\centering
     \subfloat[Função $\ell$ para a amostra dos pixeis 1 até 150 \label{fig:pdf_prod_mag_flat_rho:1a}]{%
       \includegraphics[width=0.50\linewidth]{log_like_prod_mag_L_fixo_por_1_to_150_l_35}}
     \subfloat[Função $\ell$ para a amostra dos pixeis 150 até 400 \label{fig:pdf_prod_mag_flat_rho:1b}]{%
       \includegraphics[width=0.50\linewidth]{log_like_prod_mag_L_fixo_por_150_to_400_l_35}}
     \caption{Função log-verossimilhança para distribuição produto de intensidades com diferentes L fixos}
     \label{fig:pdf_prod_mag_flat_rho}
   \end{figure}	

Podemos concluir que o uso da função densidade magnitude do produtos tanto com estimativas de dois parâmetros como estimativa de um parâmetro não é adequado para gerar a função log-verossimilhança. Nesse ponto sugerimos uma melhor investigação deste tipo de funções para contornar esse problema, a referência~\cite{fcs} relata problemas semelhantes.  
\subsection{Distribuição univariada razão de intensidades com L fixo}
As figuras~\eqref{fig:l_uni_razao_rho_tau}\subref{fig:l_uni_razao_rho_tau:1a},~\eqref{fig:l_uni_razao_rho_tau}\subref{fig:l_uni_razao_rho_tau:1b},~\eqref{fig:l_uni_razao_rho_tau}\subref{fig:l_uni_razao_rho_tau:1c} mostram os gráficos das funções \eqref{eq:TotalLogLikelihood_razao_L_fixo} indicando as evidências de bordas nos canais considerados, os quais foram gerados com a estimativa dos parâmetros $(\rho,\tau)$, fixando arbitrariamente a linha 150 da imagem simulada, e definindo o coeficiente de folga para as extremidades igual a 14 pixeis. A característica comum dessas funções é não ser diferenciáveis, dificultando o uso de métodos de otimização que calculam derivada. O problema foi resolvido usando o método \textit{Simulated Annealing} generalizado (GenSA)~\cite{xgsh}, adequado para funções não diferenciáveis.
\begin{figure}[hbt]
	\centering
     \subfloat[Canal hh/hv \label{fig:l_uni_razao_rho_tau:1a}]{%
       \includegraphics[width=0.32\linewidth]{grafico_l_gamf_razao_hh_hv_param_rho_tau}}
     \subfloat[Canal hh/vv \label{fig:l_uni_razao_rho_tau:1b}]{%
       \includegraphics[width=0.32\linewidth]{grafico_l_gamf_razao_hh_vv_param_rho_tau}}
     \subfloat[Canal hv/vv \label{fig:l_uni_razao_rho_tau:1c}]{%
       \includegraphics[width=0.32\linewidth]{grafico_l_gamf_razao_hv_vv_param_rho_tau}}
     \caption{Funções log-verossimilhanças para PDF razão de intensidades}
     \label{fig:l_uni_razao_rho_tau}
   \end{figure}	

O método da estimativa de máxima verossimilhança (MLE) aplicado nas linhas da imagem simulada com duas amostras, e as evidências de bordas estão mostradas na figura \eqref{fig:evidencias_razao_int_rho_tau}.
\begin{figure}[hbt]
	\centering
     \subfloat[Evidências no canal hh/hv  \label{fig:evidencias_razao_int_rho_tau:a}]{%
       \includegraphics[width=0.35\linewidth]{im_sim_gamf_hh_hv_param_tau_rho_14_pixel}
     }%
     \subfloat[Evidências no canal hh/vv \label{fig:evidencias_razao_int_rho_tau:b}]{%
       \includegraphics[width=0.35\linewidth]{im_sim_gamf_hh_vv_param_tau_rho_14_pixel}
     }%      
     \subfloat[Evidências no canal hv/vv \label{fig:evidencias_razao_int_rho_tau:c}]{%
       \includegraphics[width=0.35\linewidth]{im_sim_gamf_hv_vv_param_tau_rho_14_pixel}
     }%
    \caption{Evidências de bordas para os três canais de intensidade com $\mu$ estimado.}
     \label{fig:evidencias_razao_int_rho_tau} 
   \end{figure}

As figuras~\eqref{fig:l_uni_razao_param_rho_tau_1_to_150}\subref{fig:l_uni_razao_param_rho_tau_1_to_150:1a},~\eqref{fig:l_uni_razao_param_rho_tau_1_to_150}\subref{fig:l_uni_razao_param_rho_tau_1_to_150:1b},~\eqref{fig:l_uni_razao_param_rho_tau_1_to_150}\subref{fig:l_uni_razao_param_rho_tau_1_to_150:1c} mostram os gráficos para a razão  das funções \eqref{eq:log_razao_int_w_L_fixo}
com 
\begin{figure}[hbt]
	\centering
     \subfloat[Razão hh/hv]{%
       \includegraphics[width=0.32\linewidth]{grafico_l_gamf_razao_hh_hv_param_rho_tau_1_to_150}\label{fig:l_uni_razao_param_rho_tau_1_to_150:1a}}
     \subfloat[Razão hh/vv]{%
       \includegraphics[width=0.32\linewidth]{grafico_l_gamf_razao_hh_vv_param_rho_tau_1_to_150}\label{fig:l_uni_razao_param_rho_tau_1_to_150:1b}}
     \subfloat[Razão hv/vv ]{%
       \includegraphics[width=0.32\linewidth]{grafico_l_gamf_razao_hv_vv_param_rho_tau_1_to_150}\label{fig:l_uni_razao_param_rho_tau_1_to_150:1c}}
     \caption{Funções log-verossimilhanças com a estimativa de parâmetros $(\rho, \tau)$ e amostra a esquerda do pixel fixado 150}
     \label{fig:l_uni_razao_param_rho_tau_1_to_150}
\end{figure}	
%
\begin{figure}[hbt]
	\centering
     \subfloat[Razão hh/hv]{%
       \includegraphics[width=0.32\linewidth]{grafico_l_gamf_razao_hh_hv_param_rho_tau_150_to_400}\label{fig:l_uni_razao_param_rho_tau_1_to_150:1a}}
     \subfloat[Razão hh/vv]{%
       \includegraphics[width=0.32\linewidth]{grafico_l_gamf_razao_hh_vv_param_rho_tau_150_to_400}\label{fig:l_uni_razao_param_rho_tau_1_to_150:1b}}
     \subfloat[Razão hv/vv ]{%
       \includegraphics[width=0.32\linewidth]{grafico_l_gamf_razao_hv_vv_param_rho_tau_150_to_400}\label{fig:l_uni_razao_param_rho_tau_1_to_150:1c}}
     \caption{Funções log-verossimilhanças com a estimativa de parâmetros $(\rho, \tau)$ e amostra a direita do pixel fixado 150}
     \label{fig:l_uni_razao_param_rho_tau_1_to_150}
   \end{figure}	
\clearpage
\subsection{Distribuição bivariada produto de intensidades com L fixo}
Observamos que o uso da função densidade de probabilidade bivariada para intensidades mostrou que não é adequada para gerar a função log-verossimilhança, as mesmas tem características de serem planas dificultando o cálculo do valor máximo. O cálculo inadequado dos parâmetros geram uma função \eqref{eq:TotalLogLikelihood_biv_prod} com oscilação degenerando o ponto ótimo no processo de otimização. Nesse processo usamos $\text{L}=4$ fixado a priori.
\subsubsection{Estimativas para os parâmetros ($\Sigma_{11}$,$\Sigma_{22}$)}\label{sec:pdf_biv_prod_int_apl_sim_param_s1_s2}
O método estimativa de máxima verossimilhança(MLE) usa a pdf \eqref{func_biv_produto_inten_r1_r2} para encotrar a função log-verossimilhança \eqref{eq:log_vero_biv_prod_red_L_fixo}, para qual realizamos as estimativas dos parâmetros ($\rho$, $\Sigma_{11}$, $\Sigma_{22}$) usando o método de otimização BFGS.

Notamos que a forma das funções \eqref{eq:TotalLogLikelihood_biv_prod} dificultam encontrar o ponto de máximo, devido a suas características de serem planas. Para gerar a figura~\ref{fig:loglikelihood_im_sim_biv_prod_esq} escolhemos a linha 35 da imagem simulada e o pixel $j=150$, foram realizados testes com outras linhas da imagem simulada, assim como outros pixeis $j$ e os gráficos apresentavam o mesmo comportamento. 
\begin{figure}[hbt]
   \centering
     \subfloat[Amostra a esquerda do do produto bivariada 150 para $\rho=0.1$\label{fig:loglikelihood_im_sim_biv_prod_esq_rho_0p1}]{%
       \includegraphics[width=0.50\textwidth]{fig_pdf_biv_prod_l_4_fixo_r_35_j_150_rho_0p1}}
       \subfloat[Amostra a esquerda do pixel 150 para $\rho=0.5$\label{fig:loglikelihood_im_sim_biv_prod_esq_rho_0p5}]{%
       \includegraphics[width=0.50\textwidth]{fig_pdf_biv_prod_l_4_fixo_r_35_j_150_rho_0p5}}\\
         \subfloat[Amostra a esquerda do pixel 150 para $\rho=0.9$\label{fig:loglikelihood_im_sim_biv_prod_esq_rho_0p9}]{%
       \includegraphics[width=0.50\textwidth]{fig_pdf_biv_prod_l_4_fixo_r_35_j_150_rho_0p9}}
    \caption{Função log-verossimilhança para a pdf a bivariada produto de intensidades .} \label{fig:loglikelihood_im_sim_biv_prod_esq}   
\end{figure}
\begin{figure}[hbt]
   \centering
     \subfloat[Amostra a direita do produto bivariada 150 para $\rho=0.1$\label{fig:loglikelihood_im_sim_biv_prod_dir_rho_0p1}]{%
       \includegraphics[width=0.50\textwidth]{fig_pdf_biv_prod_l_4_fixo_r_35_j_150_to_400_rho_0p1}}
       \subfloat[Amostra a direita do pixel 150 para $\rho=0.5$\label{fig:loglikelihood_im_sim_biv_prod_dir_rho_0p5}]{%
       \includegraphics[width=0.50\textwidth]{fig_pdf_biv_prod_l_4_fixo_r_35_j_150_to_400_rho_0p5}}\\
         \subfloat[Amostra a direita do pixel 150 para $\rho=0.9$\label{fig:loglikelihood_im_sim_biv_prod_dir_rho_0p9}]{%
       \includegraphics[width=0.50\textwidth]{fig_pdf_biv_prod_l_4_fixo_r_35_j_150_to_400_rho_0p9}}
    \caption{Função log-verossimilhança para a pdf a bivariada produto de intensidades .} \label{fig:loglikelihood_im_sim_biv_prod_dir}   
\end{figure}
%
\subsection{Distribuição univariada span com L fixo}        

As figuras~\eqref{fig:l_uni_gamma_param_mu_l}\subref{fig:l_uni_gamma_param_mu_l:1a},~\eqref{fig:l_uni_gamma_param_mu_l}\subref{fig:l_uni_gamma_param_mu_l:1b}, mostram os gráficos das funções \eqref{eq:TotalLogLikelihood_L_fixo_span} indicando as evidências de bordas. Para gerar o primeiro gráfico usamos a média para da amostra para estimar os parâmetros $\mu$ utilizados, para gerar o segundo gráfico usamos o programa BFGS para estimar o parâmetro $\mu$. Fixamos arbitrariamente a linha 150 da imagem simulada, e definindo o coeficiente de folga para as extremidades igual a 14 pixeis. A característica comum dessas funções é não ser diferenciáveis, dificultando o uso de métodos de otimização que calculam derivada. O problema foi resolvido usando o método \textit{Simulated Annealing} generalizado (GenSA)~\cite{xgsh}, adequado para funções não diferenciáveis.
\begin{figure}[hbt]
	\centering
     \subfloat[Canal $\text{hh}$]{%
       \includegraphics[width=0.32\linewidth]{grafico_l_span_gamf_2017_sigma_soma_mu_L_fixo_linha_150}\label{fig:l_span_uni_gamma_param_mu_l:1a}}
     \subfloat[Canal $\text{hv}$]{%
       \includegraphics[width=0.32\linewidth]{grafico_l_span_gamf_2017_sigma_param_mu_L_fixo_linha_150}\label{fig:l_span_uni_gamma_param_mu_l:1b}}
     \caption{Funções log-verossimilhanças com soma e estimativa de parâmetros}
     \label{fig:l_span_uni_gamma_param_mu_l}
   \end{figure}	
 
O método estimativa de máxima verossimilhança (MLE) foi aplicado para cada linha da imagem simulada com duas amostras, e as evidências de bordas para as duas técnicas de estimativa de parâmetros estão mostradas na figura \eqref{fig:evidencias_span_gamf}.  
\begin{figure}[hbt]
	\centering
     \subfloat[Estimativa de $\mu$ usando média]{%
       \includegraphics[width=0.5\linewidth]{im_sim_gamf_span_media_mu_14_pixel}\label{fig:evidencias_span_gamf:a}}
     \subfloat[Estimativa de $\mu$ usando o BFGS]{%
       \includegraphics[width=0.5\linewidth]{im_sim_gamf_span_param_mu_14_pixel}\label{fig:evidencias_span_gamf:b}}
     \caption{Funções log-verossimilhanças com soma e estimativa de parâmetros}
     \label{fig:evidencias_span_gamf}
   \end{figure}	
%
\section{Medida dos erros para a detecção de evidências de bordas nas imagem simuladas}

O erro é medido calculando a menor distância euclidiana do pixel de referência em cada radial (linha) na imagem Ground Truth (GT) para todos os pixeis detectado na mesma radial. Armazenamos as menores distâncias em um vetor de frequência e encontramos o vetor $H(k)$ denotando o número de replicações para qual o vetor de frequência é menor que um número de pixel fixado $k_s$. A estimativa de probabilidade é encontrada por $f(k)={H(k)}/{n_r}$, onde  $n_r$ é o número de radiais, e o índice $k$ varia entre $1$ e $k_s$ definido. O algoritmo é descrito em \citet{fbgm} e definimos $k_s=10$ para encontrar todas as probabilidades para o erro.

O gráfico do erro é mostrado na figura (\ref{metrica_sim_port}),
\begin{figure}[hbt]
	\centering
	\includegraphics[width=.60\linewidth]{metricas_simuladas_port}%
	\caption{Métricas para a detecção em imagem simulada}
\label{metrica_sim_port}
\end{figure}

O gráfico mostra acurado desempenho do algoritmo no canal (hv), assim como a detecção deficitária para o canal formado pela distribuição razão entre as intensidades (hh) e (vv), fato que já poderíamos inferir pela inspeção visual. 

\pagebreak  
\section{Aplicação em imagem de adquiridas por sensores PolSAR}
A imagem da região de Flevoland de dimensão $750\times 1024$ pixeis é uma imagem PolSAR capturada pelo sensor AIRSAR com banda-L. Usamos a imagem capturando várias sub-regiões para realizar teste numéricos. A imagem capturada no sensor é mostrada na figura~\ref{flevoland_4_look} 
\begin{figure}[hbt]
	\centering
	\includegraphics[width=.40\linewidth]{flevoland_4_look}%
	\caption{Imagem da região de Flevoland}
\label{flevoland_4_look}
\end{figure}

O primeiro teste é realizado na região destacada na figura ~\ref{flevoland_4_look_radial} onde é mostrado as radiais usada para extrair as informações. 
\begin{figure}[hbt]
	\centering
	\includegraphics[width=.40\linewidth]{flevoland_radial_4_look}%
	\caption{Imagem da região de Flevoland com radiais}
\label{flevoland_4_look_radial}
\end{figure}
\clearpage
Para facilitar a visualização é capturado uma região de interesse na imagem onde destacamos as radiais que podem ser vista na figura~\ref{roi_gt}\subref{flevoland_radial_4_look_crop}. A figura~\ref{roi_gt}\subref{gt_flevoland_crop} mostra em pixeis vermelhos as bordas que usaremos de referência, a qual é conhecida por \textit{Ground Truth} (GT). Chamaremos de região de interesse I (ROI-I)  
\begin{figure}[hbt]
   \centering
     \subfloat[Imagem, ROI, e radiais. \label{flevoland_radial_4_look_crop}]{%  
       \includegraphics[width=0.229\textwidth]{flevoland_radial_4_look_black_crop}}
     \subfloat[Ground truth\label{gt_flevoland_crop}]{%
       \includegraphics[width=0.216\textwidth]{gt_flevoland_crop}
     }
    \caption{Decomposição de Pauli para a imagem, e \textit{ground truth} de referência para a (ROI-I)}
    \label{roi_gt}
\end{figure}

As Figuras.~\ref{evidencias_hh_hv_vv}\subref{evidencias_hh_hv_vv:a}, \ref{evidencias_hh_hv_vv}\subref{evidencias_hh_hv_vv:b}, e~\ref{evidencias_hh_hv_vv}\subref{evidencias_hh_hv_vv:c} mostram, respectivamente, as evidências de bordas no canais $\text{hh}$, $\text{hv}$ e $\text{vv}$ obtidos pelo método estimativa de máxima verossimilhança MLE. Para a ROI-I estabelecemos 100 radiais com comprimento de 120 pixeis. Utilizamos em cada radial uma folga nas extremidades de  14 pixeis, a folga é usada pois constatamos uma forte oscilação da função de máxima verossimilhança nos pixeis das extremidades das radiais. Escolhemos esse valor empiricamente e pode variar de acordo com a região selecionada da imagem, canal, sensor e imagem. Na ROI-I os 14 pixeis escolhidos em cada extremidade foram suficientes para contornar o problema da oscilação.

É digno de nota, o esquema proposto no qual a estimativa de parâmetros é realizada pelo programa MaxLik aplicado nas equações reduzidas, e posterior uso das estimativas nas funções de máxima verossimilhança total onde encontramos seu máximo usando o programa GenSA obteve as evidências de bordas acuradamente.   
\begin{figure}[hbt]
	\centering
    \subfloat[Canal $\text{hh}$ \label{evidencias_hh_hv_vv:a}]{%
    	\includegraphics[width=0.32\linewidth]{flevoland_hh_evid_param_L_mu_14_pixel_crop}
     	}
    \subfloat[Canal $\text{hv}$ \label{evidencias_hh_hv_vv:b}]{%
       	\includegraphics[width=0.32\linewidth]{flevoland_hv_evid_param_L_mu_14_pixel_crop}
     	}
    \subfloat[Canal $\text{vv}$ \label{evidencias_hh_hv_vv:c}]{%
       	\includegraphics[width=0.32\linewidth]{flevoland_vv_evid_param_L_mu_14_pixel_crop}
     	}
    \caption{Evidências de bordas para os três canais de intensidades na imagem de Flevoland}
    \label{evidencias_hh_hv_vv} 
\end{figure}


\clearpage
Figuras~\ref{fusion_met}\subref{fusion_met:a}, \subref{fusion_met:b}, \subref{fusion_met:c}, \subref{fusion_met:d}, \subref{fusion_met:e}, and~\subref{fusion_met:f} mostram os resultados numéricos para os métodos de fusão para evidência de bordas propostos. 
\begin{figure}[H]
	\centering
     \subfloat[Fusão por média\label{fusion_met:a}]{%
       \includegraphics[width=0.3\linewidth]{flevoland_fus_media_param_L_mu_14_pixel_crop}
     }
     \subfloat[Fusão DWT\label{fusion_met:b}]{%
       \includegraphics[width=0.3\linewidth]{flevoland_fus_dwt_param_L_mu_14_pixel_crop}
     }\\
     \subfloat[Fusão PCA \label{fusion_met:c}]{%
       %\includegraphics[width=0.2\textwidth]{example-image-a}
       \includegraphics[width=0.3\linewidth]{flevoland_fus_pca_param_L_mu_14_pixel_crop}       
     }
     \subfloat[Fusão ROC\label{fusion_met:d}]{%
       \includegraphics[width=0.3\linewidth]{flevoland_fus_roc_param_L_mu_14_pixel_crop}
     }\\
     \subfloat[Fusão MR-SWT \label{fusion_met:e}]{%
       \includegraphics[width=0.3\linewidth]{flevoland_fus_swt_param_L_mu_14_pixel_crop}
     }
     \subfloat[Fusão MR-SVD\label{fusion_met:f}]{%
       \includegraphics[width=0.3\linewidth]{flevoland_fus_svd_param_L_mu_14_pixel_crop}
     }
     \caption{Resultados da apliação dos seis métodos de fusão}
     \label{fusion_met}
\end{figure}

Os métodos de fusão por média e fusão PCA produzem resultados similares. MR-SVD produz uma considerável vantagem em descartar os  \text{outliers}, porém o tempo de processamento é maior em comparação com os demais métodos. O método usando a estatística ROC produz bordas acuradas, com poucos \text{outliers}, porém de forma esparsa. O autor acredita que é um método com potencial quando consideramos mais canais, ou outras funções de densidades de probabilidades para obter evidências de bordas. Ambos métodos baseados em \textit{wavelets} produzem densas bordas e muitos \text{outliers}, destacamos que podemos melhorar a detecção com o uso de pós-processamento, ver na referência~\cite{fbgm}. Observamos que o pós-processamento pode ser usado em todos em métodos. 
%\clearpage      

%%% Imagem de são francisco
A imagem da baía de São Francisco de dimensão $450\times 600$ pixeis é uma imagem PolSAR capturada pelo sensor AIRSAR com banda-L. Usamos a imagem para realizar teste numéricos. A imagem capturada pelo sensor é mostrada na figura~\ref{san_francisco}
\begin{figure}[hbt]
	\centering
	\vspace{-2.5cm}
	\includegraphics[width=.5\linewidth]{san_francisco_2020}%
    \vspace{-3.2cm}
	\caption{Imagem da baía de São Francisco}
\label{san_francisco}
\end{figure}
%% Teste com a região I San Francisco
%%% radial lenght 120 - com folga 25

Na figura~\ref{roi_san_fran_gt}\subref{san_francisco_radial_25} é destacado a região de interesse (ROI) escolhida com 25 radiais. Em cada radial vamos extrair os dados para obter as informações sobre as localizações das evidências de bordas. A figura~\ref{roi_san_fran_gt}\subref{gt_san_fran_r1_crop} mostra a imagem \textit{Ground Truth} (GT) que foi gerada para realizar os resultados numéricos.  
\begin{figure}[hbt]
   \centering
     \subfloat[Imagem, ROI e radiais. \label{san_francisco_radial_25}]{%  
       \includegraphics[width=0.216\textwidth]{san_francisco_radial_25_crop}}
     \subfloat[Ground Truth\label{gt_san_fran_r1_crop}]{%
       \includegraphics[width=0.216\textwidth]{gt_san_fran_r1_crop}
     }%
     \caption{Decomposição de Pauli para imagem de São Francisco, e a ground truth}
    \label{roi_san_fran_gt}
\end{figure}

%%% AAB inserido 
% Figura da região san francisco
% tamanho da radial 25
% folga nas extremidades das radiais 25 pixels

As Figuras.~\ref{evidencias_sf_hh_hv_vv}\subref{evidencias_sf_hh_hv_vv:a}, \ref{evidencias_sf_hh_hv_vv}\subref{evidencias_sf_hh_hv_vv:b}, e~\ref{evidencias_sf_hh_hv_vv}\subref{evidencias_sf_hh_hv_vv:c} mostram, respectivamente, as evidências de bordas no canais $\text{hh}$, $\text{hv}$ e $\text{vv}$ obtidos pelo método estimativa de máxima verossimilhança MLE. Para a ROI estabelecemos 25 radiais com comprimento de 120 pixeis. Utilizamos em cada radial uma folga nas extremidades de  25 pixeis, a folga é usada pois constatamos uma forte oscilação da função de máxima verossimilhança nos pixeis de extremidades. Escolhemos esse valor empiricamente e pode variar de acordo com a região, canal e figura. Na ROI os 25 pixeis escolhidos em cada extremidade foram suficientes para contornar o problema  da oscilação.
\begin{figure}[H]
	\centering
     \subfloat[Canal $\text{hh}$ \label{evidencias_sf_hh_hv_vv:a}]{%
       \includegraphics[width=0.32\linewidth]{evid_real_sf_1_param_L_mu_25_pixel_r1_crop}
     }
     \subfloat[Canal $\text{hv}$ \label{evidencias_sf_hh_hv_vv:b}]{%
       \includegraphics[width=0.32\linewidth]{evid_real_sf_2_param_L_mu_25_pixel_r1_crop}
     }
     \subfloat[Canal $\text{vv}$ \label{evidencias_sf_hh_hv_vv:c}]{%
       \includegraphics[width=0.333\linewidth]{evid_real_sf_3_param_L_mu_25_pixel_r1_crop}
     }
     \caption{Evidências de bordas para os três canais de intensidades na imagem de são Francisco}
     \label{evidencias_sf_hh_hv_vv} 
\end{figure}

Figuras~\ref{fusion_sf_met}\subref{fusion_sf_met:a}, \subref{fusion_sf_met:b}, \subref{fusion_sf_met:c}, \subref{fusion_sf_met:d}, \subref{fusion_sf_met:e}, and~\subref{fusion_sf_met:f} mostram os resultados numéricos para os métodos de fusão para evidência de bordas propostos. 
%%%% sf  pixel
\begin{figure}[htb]
	\centering
     \subfloat[Fusão por média\label{fusion_sf_met:a}]{%
       \includegraphics[width=0.3\linewidth]{sf_fus_media_param_L_mu_25_pixel_crop}
     }
     \subfloat[Fusão DWT\label{fusion_sf_met:b}]{%
       \includegraphics[width=0.3\linewidth]{sf_fus_dwt_param_L_mu_25_pixel_crop}
     }\\
     \subfloat[Fusão PCA\label{fusion_sf_met:c}]{%
       %\includegraphics[width=0.2\textwidth]{example-image-a}
       \includegraphics[width=0.3\linewidth]{sf_fus_pca_param_L_mu_25_pixel_crop}       
     }
     \subfloat[Fusão ROC\label{fusion_sf_met:d}]{%
       \includegraphics[width=0.3\linewidth]{sf_fus_roc_param_L_mu_25_pixel_crop}
     }\\
     \subfloat[Fusão MR-SWT\label{fusion_sf_met:e}]{%
       \includegraphics[width=0.3\linewidth]{sf_fus_swt_param_L_mu_25_pixel_crop}
     }
     \subfloat[Fusão MR-SVD\label{fusion_sf_met:f}]{%
       \includegraphics[width=0.3\linewidth]{sf_fus_svd_param_L_mu_25_pixel_crop}
     }
     \caption{Resultado da aplicação dos seis método de fusão para a imagem de São Francisco}
     %\caption{Results of applying the six fusion methods to San Francisco}
     \label{fusion_sf_met}
\end{figure}

%% Teste com a região III Flevoland
%%%  25 radials lenght 120 - com folga 14

Uma segunda região na imagem de Flevoland é capturado para os testes e chamaremos de região de interesse II (FLEV-ROI-II). A imagem com as radiais é mostrada na figura~\ref{roi_gt_roi_ii}\subref{flevoland_radial_25_crop_roi_ii}. A figura~\ref{roi_gt_roi_ii}\subref{gt_flevoland_crop_roi_ii} mostra as bordas em pixeis vermelhos que usaremos de referência a qual chamaremos de \textit{Ground Truth} (GT).
\begin{figure}[hbt]
   \centering
     \subfloat[Imagem, ROI, e radiais. \label{flevoland_radial_25_crop_roi_ii}]{%  
       \includegraphics[width=0.216\textwidth]{flevoland_r3_radial_crop}}
     \subfloat[Ground reference\label{gt_flevoland_crop_roi_ii}]{%
       \includegraphics[width=0.216\textwidth]{gt_flevoland_r3_crop}
     }
    \caption{Decomposição de Pauli para a FLEV-ROI-II da imagem de Flevoland, e a \textit{ground truth}}
    \label{roi_gt_roi_ii}
\end{figure}

As Figuras.~\ref{evidencias_flev_roi_ii_hh_hv_vv}\subref{evidencias_flev_roi_ii_hh_hv_vv:a}, \ref{evidencias_flev_roi_ii_hh_hv_vv}\subref{evidencias_flev_roi_ii_hh_hv_vv:b}, e~\ref{evidencias_flev_roi_ii_hh_hv_vv}\subref{evidencias_flev_roi_ii_hh_hv_vv:c} mostram, respectivamente, as evidências de bordas no canais $\text{hh}$, $\text{hv}$ e $\text{vv}$ obtidas pelo método estimativa de máxima verossimilhança MLE. Para a FLEV-ROI-II estabelecemos 25 radiais com comprimento de 120 pixeis. Utilizamos em cada radial uma folga nas extremidades de  14 pixeis, a folga é usada pois constatamos uma forte oscilação da função de máxima verossimilhança nos pixeis de extremidade. Escolhemos esse valor empiricamente e pode variar de acordo com a região, canal e figura. Na FLEV-ROI-II os 14 pixeis escolhidos em cada extremidade foram suficientes para contornar o problema da oscilação.

   \begin{figure}[hbt]
	\centering
     \subfloat[Canal $\text{hh}$ \label{evidencias_flev_roi_ii_hh_hv_vv:a}]{%
       \includegraphics[width=0.32\linewidth]{evid_real_flev_hh_param_L_mu_14_pixel_r3_crop}
     }
     \subfloat[Canal $\text{hv}$ \label{evidencias_flev_roi_ii_hh_hv_vv:b}]{%
       \includegraphics[width=0.32\linewidth]{evid_real_flev_hv_param_L_mu_14_pixel_r3_crop}
     }
     \subfloat[Canal $\text{vv}$ \label{evidencias_flev_roi_ii_hh_hv_vv:c}]{%
       \includegraphics[width=0.32\linewidth]{evid_real_flev_hv_param_L_mu_14_pixel_r3_crop}
     }
     \caption{Evidências de bordas para os três canais de intensidades para FLEV-ROI-II na imagem de Flevoland com folga de 14 pixel}
     \label{evidencias_flev_roi_ii_hh_hv_vv} 
   \end{figure}

Figuras~\ref{fusion_flev_roi_ii_met}\subref{fusion_flev_roi_ii_met:a}, \subref{fusion_flev_roi_ii_met:b}, \subref{fusion_flev_roi_ii_met:c}, \subref{fusion_flev_roi_ii_met:d}, \subref{fusion_flev_roi_ii_met:e}, and~\subref{fusion_flev_roi_ii_met:f} mostram os resultados numéricos para os métodos com fusão de evidência de bordas propostos.
\begin{figure}[hbt]
	\centering
     \subfloat[Fusão por média \label{fusion_flev_roi_ii_met:a}]{%
       \includegraphics[width=0.30\linewidth]{flev_r3_fus_media_param_L_mu_14_pixel_crop}
     }
     \subfloat[Fusão DWT\label{fusion_flev_roi_ii_met:b}]{%
       \includegraphics[width=0.30\linewidth]{flev_r3_fus_dwt_param_L_mu_14_pixel_crop}
     }\\
     \subfloat[Fusão PCA\label{fusion_flev_roi_ii_met:c}]{%
       \includegraphics[width=0.30\linewidth]{flev_r3_fus_pca_param_L_mu_14_pixel_crop}       
     }
     \subfloat[Fusão ROC\label{fusion_flev_roi_ii_met:d}]{%
       \includegraphics[width=0.30\linewidth]{flev_r3_fus_roc_param_L_mu_14_pixel_crop}
     }\\
     \subfloat[Fusão MR-SWT\label{fusion_flev_roi_ii_met:e}]{%
       \includegraphics[width=0.30\linewidth]{flev_r3_fus_swt_param_L_mu_14_pixel_crop}
     }
     \subfloat[Fusão MR-SVD\label{fusion_flev_roi_ii_met:f}]{%
       \includegraphics[width=0.30\linewidth]{flev_r3_fus_svd_param_L_mu_14_pixel_crop}
     }
     \caption{Resultado da aplicação dos seis métodos de fusão para a FLEV-ROI-II de flevoland com 14 pixeis de folga}
     \label{fusion_flev_roi_ii_met}
\end{figure}
\clearpage
%% Teste com a região III Flevoland
%%%  25 radials lenght 120 - com folga 25
As Figuras.~\ref{evidencias_flev_roi_ii_25_pixel_hh_hv_vv}\subref{evidencias_flev_roi_ii_25_pixel_hh_hv_vv:a}, \ref{evidencias_flev_roi_ii_25_pixel_hh_hv_vv}\subref{evidencias_flev_roi_ii_25_pixel_hh_hv_vv:b}, e~\ref{evidencias_flev_roi_ii_25_pixel_hh_hv_vv}\subref{evidencias_flev_roi_ii_25_pixel_hh_hv_vv:c} mostram, respectivamente, as evidências de bordas no canais $\text{hh}$, $\text{hv}$ e $\text{vv}$ obtidas pelo método estimativa de máxima verossimilhança MLE. Para a FLEV-ROI-II estabelecemos 25 radiais com comprimento de 120 pixeis. Utilizamos em cada radial uma folga nas extremidades de  25 pixeis, a folga é usada pois constatamos uma forte oscilação da função de máxima verossimilhança nos pixeis de extremidades. Escolhemos esse valor empiricamente e pode variar de acordo com a região, canal e figura. Na FLEV-ROI-II os 25 pixeis escolhidos em cada extremidade foram suficientes para contornar o problema da oscilação.  
   \begin{figure}[hbt]
	\centering
     \subfloat[Canal $\text{hh}$ \label{evidencias_flev_roi_ii_25_pixel_hh_hv_vv:a}]{%
       \includegraphics[width=0.32\linewidth]{evid_real_flev_hh_param_L_mu_25_pixel_r3_crop}
     }
     \subfloat[Canal $\text{hv}$ \label{evidencias_flev_roi_ii_25_pixel_hh_hv_vv:b}]{%
       \includegraphics[width=0.32\linewidth]{evid_real_flev_hv_param_L_mu_25_pixel_r3_crop}
     }
     \subfloat[Canal $\text{vv}$ \label{evidencias_flev_roi_ii_25_pixel_hh_hv_vv:c}]{%
       \includegraphics[width=0.32\linewidth]{evid_real_flev_vv_param_L_mu_25_pixel_r3_crop}
     }
          \caption{Evidências de bordas para os três canais de intensidades para FLEV-ROI-II na imagem de Flevoland com folga de 25 pixel}
     \label{evidencias_flev_roi_ii_25_pixel_hh_hv_vv} 
   \end{figure}

Figuras~\ref{fusion_flev_roi_ii_25_pixel_met}\subref{fusion_flev_roi_ii_25_pixel_met:a}, \subref{fusion_flev_roi_ii_25_pixel_met:b}, \subref{fusion_flev_roi_ii_25_pixel_met:c}, \subref{fusion_flev_roi_ii_25_pixel_met:d}, \subref{fusion_flev_roi_ii_25_pixel_met:e}, and~\subref{fusion_flev_roi_ii_25_pixel_met:f} mostram os resultados numéricos para os métodos de fusão para evidência de bordas propostos.

\begin{figure}[hbt]
	\centering
     \subfloat[Fusão por média\label{fusion_flev_roi_ii_25_pixel_met:a}]{%
       %\includegraphics[width=0.2\textwidth]{example-image-a}
       \includegraphics[width=0.32\linewidth]{flev_r3_fus_media_param_L_mu_25_pixel_crop}
     }
     \subfloat[DWT fusion\label{fusion_flev_roi_ii_25_pixel_met:b}]{%
       \includegraphics[width=0.32\linewidth]{flev_r3_fus_dwt_param_L_mu_25_pixel_crop}
     }\\
     \subfloat[Fusão PCA \label{fusion_flev_roi_ii_25_pixel_met:c}]{%
       %\includegraphics[width=0.2\textwidth]{example-image-a}
       \includegraphics[width=0.32\linewidth]{flev_r3_fus_pca_param_L_mu_25_pixel_crop}       
     }
     \subfloat[Fusão ROC\label{fusion_flev_roi_ii_25_pixel_met:d}]{%
       \includegraphics[width=0.32\linewidth]{flev_r3_fus_roc_param_L_mu_25_pixel_crop}
     }\\
     \subfloat[Fusão MR-SWT \label{fusion_flev_roi_ii_25_pixel_met:e}]{%
       \includegraphics[width=0.32\linewidth]{flev_r3_fus_swt_param_L_mu_25_pixel_crop}
     }
     \subfloat[Fusão MR-SVD \label{fusion_flev_roi_ii_25_pixel_met:f}]{%
       \includegraphics[width=0.32\linewidth]{flev_r3_fus_svd_param_L_mu_25_pixel_crop}
     }
     \caption{Resultado da aplicação dos seis métodos de fusão para a FLEV-ROI-II de flevoland com 25 pixeis de folga}
     \label{fusion_flev_roi_ii_25_pixel_met}
\end{figure}
%%%%%%%%%%%%%%%%%%%%%%%%%%%%%%%%%%%%%%%%%%%%%%%%%%%
%% Metricas
%%%%%%%%%%%%%%%%%%%%%%%%%%%%%%%%%%%%%%%%%%%%%%%%%%%
\subsection{Mais pdfs}
\begin{figure}[hbt]
	\centering
    \subfloat[Canal $\text{hh}$ \label{evidencias_hh_hv_vv:a}]{%
    	\includegraphics[width=0.32\linewidth]{flevoland_prod_mag_hh_hv_evid_param_L_rho_14_pixel_crop}
     	}
    \subfloat[Canal $\text{hv}$ 25 pixeis\label{evidencias_hh_hv_vv:b}]{%
       	\includegraphics[width=0.32\linewidth]{flevoland_prod_mag_hh_vv_evid_param_L_rho_25_pixel_crop}
     	}
    \subfloat[Canal $\text{vv}$ \label{evidencias_hh_hv_vv:c}]{%
       	\includegraphics[width=0.32\linewidth]{flevoland_prod_mag_hv_vv_evid_param_L_rho_14_pixel_crop}
     	}
    \caption{Evidências de bordas para os três canais de intensidades na imagem de Flevoland}
    \label{evidencias_hh_hv_vv} 
\end{figure}

\begin{figure}[hbt]
	\centering
    \subfloat[Canal $\text{hh}$ \label{evidencias_hh_hv_vv:a}]{%
    	\includegraphics[width=0.32\linewidth]{flevoland_razao_hh_hv_evid_param_rho_tau_14_pixel_crop}
     	}
    \subfloat[Canal $\text{hv}$ 25 pixeis\label{evidencias_hh_hv_vv:b}]{%
       	\includegraphics[width=0.32\linewidth]{flevoland_razao_hh_vv_evid_param_rho_tau_14_pixel_crop}
     	}
    \subfloat[Canal $\text{vv}$ \label{evidencias_hh_hv_vv:c}]{%
       	\includegraphics[width=0.32\linewidth]{flevoland_razao_hv_vv_evid_param_rho_tau_14_pixel_crop}
     	}
    \caption{Evidências de bordas para os três canais razão de intensidades na imagem de Flevoland}
    \label{evidencias_hh_hv_vv} 
\end{figure}
%
\begin{figure}[hbt]
	\centering
	\vspace{-2.5cm}
	\includegraphics[width=.5\linewidth]{flevoland_span_evid_mu_media_14_pixel_crop}%
	\caption{Span}
\label{san_francisco}
\end{figure}

\begin{figure}[hbt]
	\centering
	\includegraphics[width=.60\linewidth]{metricas_evid_flevoland_port}%
	\caption{Métricas para a detecção de evidências em imagem flevoland}
\label{metrica_sim_port}
\end{figure}

\begin{figure}[hbt]
	\centering
     \subfloat[Fusão por média\label{fusion_flev_roi_ii_25_pixel_met:a}]{%
       %\includegraphics[width=0.2\textwidth]{example-image-a}
       \includegraphics[width=0.32\linewidth]{flev_7_canais_fus_media_14_pixel_crop}
     }
     \subfloat[DWT fusion\label{fusion_flev_roi_ii_25_pixel_met:b}]{%
       \includegraphics[width=0.32\linewidth]{flev_7_canais_fus_dwt_14_pixel_crop}
     }\\
     \subfloat[Fusão PCA \label{fusion_flev_roi_ii_25_pixel_met:c}]{%
       %\includegraphics[width=0.2\textwidth]{example-image-a}
       \includegraphics[width=0.32\linewidth]{flev_7_canais_fus_pca_14_pixel_crop}       
     }
     \subfloat[Fusão ROC\label{fusion_flev_roi_ii_25_pixel_met:d}]{%
       \includegraphics[width=0.32\linewidth]{flev_7_canais_fus_roc_14_pixel_crop}
     }\\
     \subfloat[Fusão MR-SWT \label{fusion_flev_roi_ii_25_pixel_met:e}]{%
       \includegraphics[width=0.32\linewidth]{flev_7_canais_fus_swt_14_pixel_crop}
     }
     \subfloat[Fusão MR-SVD \label{fusion_flev_roi_ii_25_pixel_met:f}]{%
       \includegraphics[width=0.32\linewidth]{flev_7_canais_fus_svd_14_pixel_crop}
     }
     \caption{Resultado da aplicação dos seis métodos de fusão para os 7 canais de flevoland com 14 pixeis de folga}
     \label{fusion_flev_roi_ii_25_pixel_met}
\end{figure}


\begin{table}[hbt]
	\centering
	\caption{Processing times (fusion method).}\label{metrica_de_tempo}
	\begin{tabular}{@{}lrrrrrr@{}} \toprule
		Method     & Aver.      &   PCA       &  MR-DWT   & MR-SWT     &  ROC       &  MR-SVD \\ \midrule
		Time (s)   & 0.01797480 & 0.054634950 & 0.06949015& 0.17152435 &  2.1889033 & 1.14976175  \\
		Rel. time  & 1.00       &             &           &      &        &   \\ \bottomrule
	\end{tabular}
\end{table}



\begin{figure}[hbt]
	\centering
	\includegraphics[width=.60\linewidth]{metricas_fusao_7_canais_flevoland_14_pixel_port}%
	\caption{Métricas para fusão de evidências de bordas em imagem flevoland}
\label{metrica_sim_port}
\end{figure}

\clearpage
\subsection{Probabilidade erro de detecção}

O erro é medido calculando a menor distância euclidiana do pixel de referência em cada radial na imagem Ground Truth (GT) para todos os pixeis detectado na mesma radial. Armazenamos as menores distâncias em um vetor de frequência e calcular o vetor $H(k)$ denotando o número de replicações para qual o vetor de frequência é menor que um número de pixel fixado $k_s$. A estimativa de probabilidade é encontrada por $f(k)={H(k)}/{n_r}$, onde  $n_r$ é o número de radiais, e o índice $k$ varia entre $1$ e $k_s$ definido. O algoritmo pode ser descrito em \citet{fbgm} e definimos $k_s=10$ para encontrar todas as probabilidades para o erro.     

A figura~\ref{probability_edge_detc_flev_roi_i} mostra o erro para a FLEV-ROI-I com o número de radiais igual a 100.
\begin{figure}[hbt]
\centering
	\includegraphics[width=.8\linewidth]{metricas_6_fusao_flevoland_port}
	\caption{Probabilidade de detecção dos métodos de fusão para a FLEV-ROI-I}
\label{probability_edge_detc_flev_roi_i}
\end{figure}

A figura~\ref{probability_edge_detc_flev_roi_ii} mostra o erro para a FLEV-ROI-I com o número de radiais igual a 25.
\begin{figure}[H]
\centering
	\includegraphics[width=.8\linewidth]{metricas_6_fusao_flevoland_r3_port}
    \caption{Probabilidade de detecção dos métodos de fusão para a FLEV-ROI-II}	 
\label{probability_edge_detc_flev_roi_ii}
\end{figure}

A figura~\ref{probability_edge_detc_sf_roi_i} mostra o erro para a SF-ROI-II com o número de radiais igual a 25.
\begin{figure}[hbt]
\centering
	\includegraphics[width=.8\linewidth]{metricas_6_fusao_sf_r3_port}
	 \caption{Probabilidade de detecção dos métodos de fusão para a SF-ROI-I}	  
\label{probability_edge_detc_sf_roi_i}
\end{figure}


%\subsection{Imagem da Baía de São Francisco}

%\begin{figure}[hbt]
%	\centering
%     \subfloat[Canal $\text{hh}$ \label{fig:evid_bordas_l:1a}]{%
%       \includegraphics[width=0.32\linewidth]{grafico_l_gamf_razao_hh_hv_param_rho_tau}}
%     \subfloat[Canal $\text{hv}$ \label{fig:evid_bordas_l:1b}]{%
%       \includegraphics[width=0.32\linewidth]{grafico_l_gamf_razao_hh_vv_param_rho_tau}}
%     \subfloat[Canal $\text{vv}$ \label{fig:evid_bordas_l:1c}]{%
%       \includegraphics[width=0.32\linewidth]{grafico_l_gamf_razao_hv_vv_param_rho_tau}}
%     \caption{Funções log-verossimilhanças para a radial 150}
%     \label{fig:evid_bordas_l}
%   \end{figure}	
%
%
%\begin{figure*}[hbt]
%	\centering
%     \subfloat[Evidências no canal $\text{hh}$  \label{evidencias_hh_hv_vv_gamf_mu_estimado:a}]{%
%       \includegraphics[width=0.35\linewidth]{im_sim_gamf_hh_hv_param_tau_rho_14_pixel}
%     }
%     \subfloat[Evidências no canal $\text{hv}$ \label{evidencias_hh_hv_vv_gamf_mu_estimado:b}]{%
%       \includegraphics[width=0.35\linewidth]{im_sim_gamf_hh_vv_param_tau_rho_14_pixel}
%     }      
%     \subfloat[Evidências no canal $\text{vv}$ \label{evidencias_hh_hv_vv_gamf_mu_estimado:c}]{%
%       \includegraphics[width=0.35\linewidth]{im_sim_gamf_hv_vv_param_tau_rho_14_pixel}
%     }
%    \caption{Evidências de bordas para os três canais de intensidade com $\mu$ estimado.}
%     \label{evidencias_hh_hv_vv_gamf} 
%   \end{figure*}    
%    
%    

As Figuras.~\ref{evidencias_r3_hh_hv_vv}\subref{evidencias_r3_hh_hv_vv:a}, \ref{evidencias_r3_hh_hv_vv}\subref{evidencias_r3_hh_hv_vv:b}, e~\ref{evidencias_r3_hh_hv_vv:a}\subref{evidencias_r3_hh_hv_vv:c} mostram, respectivamente, as evidências de bordas no canais hh, hv e vv obtidas pelo método estimativa de máxima verossimilhança MLE. Para a FLEV-ROI-III-Teste 1 estabelecemos 100 radiais e variação de comprimento de raios diferentes, para as radiais 1 até a 64 usamos 80 pixeis de comprimento, para as radiais 65 até 80 usamos 60 pixeis. Utilizamos em cada radial uma folga nas extremidades de  14 pixeis, a folga é usada pois constatamos uma forte oscilação da função de máxima verossimilhança nos pixeis de extremidades. Escolhemos esse valor empiricamente e pode variar de acordo com a região, canal e figura. Na FLEV-ROI-III-Teste 1 os 14 pixeis escolhidos em cada extremidade não foram suficientes para contornar o problema da oscilação, ocorrendo o aparecimento de muitos \textit{outliers} na imagens, principalmente no canal vv. 
%%% AAB inserido 
% Figura da região 2 com 80 pixel para as radiais 1 até 64 e 60 pixel para as radiais 65 até 100.
   \begin{figure}[hbt]
	\centering
     \subfloat[Channel $\text{hh}$ \label{evidencias_r3_hh_hv_vv:a}]{%
       \includegraphics[width=0.32\linewidth]{flev_r2_hh_evid_L_mu_14_80_60strip_pixel_crop}
     }
     \subfloat[Channel $\text{hv}$ \label{evidencias_r3_hh_hv_vv:b}]{%
       \includegraphics[width=0.32\linewidth]{flev_r2_hv_evid_L_mu_14_80_60strip_pixel_crop}
     }
     \subfloat[Channel $\text{vv}$ \label{evidencias_r3_hh_hv_vv:c}]{%
       \includegraphics[width=0.32\linewidth]{flev_r2_vv_evid_L_mu_14_80_60strip_pixel_crop}
     }
     \caption{Evidências de bordas para os três canais de intensidades para FLEV-ROI-III-teste 1 na imagem de Flevoland com folga de 14 pixel}
      \label{evidencias_r3_hh_hv_vv} 
   \end{figure}

%%% AAB inserido 
% Figura da região 2 com 60 pixel para todas radiais menos na faixa de radiais 51 até 64 (80 pixel).
As Figuras.~\subref{evidencias_r3_2_hh_hv_vv:a}, \ref{evidencias_r3_2_hh_hv_vv}\subref{evidencias_r3_2_hh_hv_vv:b}, e~\ref{evidencias_r3_2_hh_hv_vv}\subref{evidencias_r3_2_hh_hv_vv:c} mostram, respectivamente, as evidências de bordas no canais hh, hv e vv obtidas pelo método estimativa de máxima verossimilhança MLE. Para a FLEV-ROI-III- teste 2 estabelecemos 100 radiais e variação de comprimento de raios diferentes, para as radiais 51 até a 64 usamos 80 pixeis de comprimento, para as outras radiais usamos 60 pixeis. Utilizamos em cada radial uma folga nas extremidades de  14 pixeis, a folga é usada pois constatamos uma forte oscilação da função de máxima verossimilhança nos pixeis de extremidades. Escolhemos esse valor empiricamente e pode variar de acordo com a região, canal e figura. Na FLEV-ROI-III- teste 2 os 14 pixeis escolhidos em cada extremidade não foram suficientes para contornar o problema da oscilação, ocorrendo o aparecimento de muitos \textit{outliers} na imagens, principalmente no canal vv. Na Figuras.~\ref{evidencias_r3_2_hh_hv_vv} diminuímos o tamanho do pixel usado para representar a evidências de borda.
   \begin{figure}[hbt]
	\centering
     \subfloat[Channel $\text{hh}$ \label{evidencias_r3_2_hh_hv_vv:a}]{%
       \includegraphics[width=0.32\linewidth]{flev_r2_hh_evid_L_mu_14_60_80strip_pixel_crop}
     }
     \subfloat[Channel $\text{hv}$ \label{evidencias_r3_2_hh_hv_vv:b}]{%
       \includegraphics[width=0.32\linewidth]{flev_r2_hv_evid_L_mu_14_60_80strip_pixel_crop}
     }
     \subfloat[Channel $\text{vv}$ \label{evidencias_r3_2_hh_hv_vv:c}]{%
       \includegraphics[width=0.32\linewidth]{flev_r2_vv_evid_L_mu_14_60_80strip_pixel_crop}
     }
     \caption{Evidências de bordas para os três canais de intensidades para FLEV-ROI-III-teste 2 na imagem de Flevoland com folga de 14 pixel}
     \label{evidencias_r3_2_hh_hv_vv} 
   \end{figure}

%%% AAB inserido 
% Figura para a fusão da região 2 com 80 pixel para as radiais 1 até 64 e 60 pixel para as radiais 65 até 100.
Figuras~\ref{fusion_r3_met}\subref{fusion_r3_met:a}, \subref{fusion_r3_met:b}, \subref{fusion_r3_met:c}, \subref{fusion_r3_met:d}, \subref{fusion_r3_met:e}, and~\subref{fusion_r3_met:f} mostram os resultados numéricos para os métodos de fusão para evidência de bordas propostos.
\begin{figure}[hbt]
	\centering
     \subfloat[Fusão por média\label{fusion_r3_met:a}]{%
       %\includegraphics[width=0.2\textwidth]{example-image-a}
       \includegraphics[width=0.32\linewidth]{flev_r2_media_L_mu_14_80_60strip_pixel_crop}
     }
     \subfloat[Fusão DWT\label{fusion_r3_met:b}]{%
       \includegraphics[width=0.32\linewidth]{flev_r2_dwt_L_mu_14_80_60strip_pixel_crop}
     }\\
     \subfloat[Fusão PCA \label{fusion_r3_met:c}]{%
       %\includegraphics[width=0.2\textwidth]{example-image-a}
       \includegraphics[width=0.32\linewidth]{flev_r2_pca_L_mu_14_80_60strip_pixel_crop}       
     }
     \subfloat[Fusão ROC\label{fusion_r3_met:d}]{%
       \includegraphics[width=0.32\linewidth]{flev_r2_roc_L_mu_14_80_60strip_pixel_crop}
     }\\
     \subfloat[Fusão MR-SWT\label{fusion_r3_met:e}]{%
       \includegraphics[width=0.32\linewidth]{flev_r2_swt_L_mu_14_80_60strip_pixel_crop}
     }
     \subfloat[Fusão MR-SVD\label{fusion_r3_met:f}]{%
       \includegraphics[width=0.32\linewidth]{flev_r2_svd_L_mu_14_80_60strip_pixel_crop}
     }
     \caption{Resultado da aplicação dos seis métodos de fusão para a FLEV-ROI-III-teste 1 de flevoland com 14 pixeis de folga}
     \label{fusion_r3_met}
\end{figure}

%%% AAB inserido 
% Figura da região san francisco
% tamanho da radial 40
% folga nas extremidades das radiais 8 pixels
\pagebreak
As Figuras.~\ref{evidencias_r2_sf_hh_hv_vv}\subref{evidencias_r2_sf_hh_hv_vv:a}, \ref{evidencias_r2_sf_hh_hv_vv}\subref{evidencias_r2_sf_hh_hv_vv:b}, e~\ref{evidencias_r2_sf_hh_hv_vv}\subref{evidencias_r2_sf_hh_hv_vv:c} mostram, respectivamente, as evidências de bordas no canais $\text{hh}$, $\text{hv}$ e $\text{vv}$ obtidas pelo método estimativa de máxima verossimilhança MLE. Para a SF-ROI-II estabelecemos 100 radiais com comprimento de 40 pixeis. Utilizamos em cada radial uma folga nas extremidades de  8 pixeis, a folga é usada pois constatamos uma forte oscilação da função de máxima verossimilhança nos pixeis de extremidades. Escolhemos esse valor empiricamente e pode variar de acordo com a região, canal e figura. Na SF-ROI-II os 8 pixeis escolhidos em cada extremidade não foram suficientes para contornar o problema da oscilação. 
   \begin{figure}[htb]
	\centering
     \subfloat[Channel $\text{hh}$ \label{evidencias_r2_sf_hh_hv_vv:a}]{%
       \includegraphics[width=0.32\linewidth]{sf_hh_evid_param_L_mu_8_pixel_crop}
     }
     \subfloat[Channel $\text{hv}$ \label{evidencias_r2_sf_hh_hv_vv:b}]{%
       \includegraphics[width=0.32\linewidth]{sf_hv_evid_param_L_mu_8_pixel_crop}
     }
     \subfloat[Channel $\text{vv}$ \label{evidencias_r2_sf_hh_hv_vv:c}]{%
       \includegraphics[width=0.32\linewidth]{sf_vv_evid_param_L_mu_8_pixel_crop}
     }
     
     \caption{Evidências de bordas para os três canais de intensidades para SF-ROI-II com folga de 8 pixel}
     \label{evidencias_r2_sf_hh_hv_vv} 
   \end{figure}


Figuras~\ref{fusion_r2_sf_met}\subref{fusion_r2_sf_met:a}, \subref{fusion_r2_sf_met:b}, \subref{fusion_r2_sf_met:c}, \subref{fusion_r2_sf_met:d}, \subref{fusion_r2_sf_met:e}, and~\subref{fusion_r2_sf_met:f} mostram os resultados numéricos para os métodos de fusão para evidência de bordas propostos.

%%%% sf 40 pixel
\begin{figure}[hbt]
	\centering
     \subfloat[Fusão por média\label{fusion_r2_sf_met:a}]{%
       %\includegraphics[width=0.2\textwidth]{example-image-a}
       \includegraphics[width=0.32\linewidth]{sf_fus_media_param_L_mu_8_pixel_crop}
     }
     \subfloat[Fusão DWT\label{fusion_r2_sf_met:b}]{%
       \includegraphics[width=0.32\linewidth]{sf_fus_dwt_param_L_mu_8_pixel_crop}
     }\\
     \subfloat[Fusão PCA \label{fusion_r2_sf_met:c}]{%
       %\includegraphics[width=0.2\textwidth]{example-image-a}
       \includegraphics[width=0.32\linewidth]{sf_fus_pca_param_L_mu_8_pixel_crop}       
     }
     \subfloat[Fusão ROC\label{fusion_r2_sf_met:d}]{%
       \includegraphics[width=0.32\linewidth]{sf_fus_roc_param_L_mu_8_pixel_crop}
     }\\
     \subfloat[Fusão MR-SWT\label{fusion_r2_sf_met:e}]{%
       \includegraphics[width=0.32\linewidth]{sf_fus_swt_param_L_mu_8_pixel_crop}
     }
     \subfloat[Fusão MR-SVD\label{fusion_r2_sf_met:f}]{%
       \includegraphics[width=0.32\linewidth]{sf_fus_svd_param_L_mu_8_pixel_crop}
     }
     \caption{Resultado da aplicação dos seis métodos de fusão para a SF-ROI-II com 8 pixeis de folga}
     \label{fusion_r2_sf_met}
\end{figure}%   
%The error is measured simulating $400$ independent images and finding $\widehat\jmath$ in a line fixed.
%By construction, the vertical line $200$ is considered as the real edge in each replication, so the error for this replication is the absolute value of the difference between this point and the estimated value, and it is computed by $E(r) = |200 - \widehat{\jmath}(r)|$, $1\leq r \leq 400$.
%
%Relative frequencies to estimate the probability of having an error smaller than a number of pixels is used. 
%Denoting $H(k)$ the number of replications for which the error is less than $k$ pixels, an estimate of this probability is $f(k)=\frac{H(k)}{400}$. 
%In the tests performed in this section, $k$ varies between $1$ and $10$. 
%The algorithm is described in Ref.~\cite{fbgm}.
%Fig.~\ref{probability_edge_detc} shows these probabilities as computed in each channel $I_\text{hh}$, $I_\text{hv}$ and $I_{vv}$ of the image.
%
%A figura \eqref{fig:pdf_mag_prod} mostra a função densidade magnitude do produto com a variação das visadas $L=2,3,4$, e $8$, 
%\begin{figure}[hbt]
%\centering
%\includegraphics[width=4.0in]{dist_interferograma_multi_visadas.pdf}
%	\caption{Distribuição magnitude do produto múltiplas visadas.}
%\label{fig:pdf_mag_prod}
%\end{figure}


%
%As figuras \eqref{fig:prod_mag_l_50_r_35} até \eqref{fig:prod_mag_l_350_r_35} mostram a função de log-verossimilhança da pdf magnitude de produtos \eqref{eq:eq_log_vero_mag_prod_red} aplicada para a amostra de duas folhas simulada. No processo fixamos arbitrariamente a linha 35 da amostra e variamos o número de pixel entre 50, 150, 250 e 250. Desta maneira construímos duas funções $\ell{j}$, uma para cada lado da amostra.
% \begin{figure*}[hbt]
%	\centering
%     \subfloat[Pixel variando de 1 até 50 na linha 35.  \label{fig:prod_mag_l_rho_1_50}]{%
%       \includegraphics[width=0.50\linewidth]{fig_pdf_mag_prod_r_35_1_to_50}
%     }
%     \subfloat[Pixel variando de 51 até 400 na linha 35. \label{fig:prod_mag_l_rho_51_400}]{%
%       \includegraphics[width=0.50\linewidth]{fig_pdf_mag_prod_r_35_51_to_400}
%     }
%     \caption{Funções de máxima verossimilhança produto de magnitude com pixel fixo 50.}
%     \label{fig:prod_mag_l_50_r_35} 
%   \end{figure*}
%   
%   \begin{figure*}[hbt]
%	\centering
%     \subfloat[Pixel variando de 1 até 150 na linha 35.  \label{fig:prod_mag_l_rho_1_150}]{%
%       \includegraphics[width=0.50\linewidth]{fig_pdf_mag_prod_r_35_1_to_150}
%     }
%     \subfloat[Pixel variando de 151 até 400 na linha 35. \label{fig:prod_mag_l_rho_151_400}]{%
%       \includegraphics[width=0.50\linewidth]{fig_pdf_mag_prod_r_35_151_to_400}
%     }
%     \caption{Funções de máxima verossimilhança produto de magnitude com pixel fixo 150.}
%     \label{fig:prod_mag_l_150_r_35} 
%   \end{figure*}
%
%\begin{figure*}[hbt]
%	\centering
%     \subfloat[Pixel variando de 1 até 250 na linha 35.  \label{fig:prod_mag_l_rho_1_250}]{%
%       \includegraphics[width=0.50\linewidth]{fig_pdf_mag_prod_r_35_1_to_250}
%     }
%     \subfloat[Pixel variando de 251 até 400 na linha 35. \label{fig:prod_mag_l_rho_251_400}]{%
%       \includegraphics[width=0.50\linewidth]{fig_pdf_mag_prod_r_35_251_to_400}
%     }
%     \caption{Funções de máxima verossimilhança produto de magnitude com pixel fixo 250.}
%     \label{fig:prod_mag_l_250_r_35} 
%   \end{figure*}
%   \begin{figure*}[hbt]
%	\centering
%     \subfloat[Pixel variando de 1 até 350 na linha 35.  \label{fig:prod_mag_l_rho_1_350}]{%
%       \includegraphics[width=0.50\linewidth]{fig_pdf_mag_prod_r_35_1_to_350}
%     }
%     \subfloat[Pixel variando de 351 até 400 na linha 35. \label{fig:prod_mag_l_rho_351_400}]{%
%       \includegraphics[width=0.50\linewidth]{fig_pdf_mag_prod_r_35_351_to_400}
%     }
%     \caption{Funções de máxima verossimilhança produto de magnitude com pixel fixo 350.}
%     \label{fig:prod_mag_l_350_r_35} 
%   \end{figure*}

%As figuras \eqref{fig:prod_mag_l_25_r_50_hh}, \eqref{fig:prod_mag_l_25_r_50_hv} e \eqref{fig:prod_mag_l_25_r_50_vv} mostram a função de log-verossimilhança da pdf magnitude de produtos \eqref{eq:eq_log_vero_mag_prod_red} aplicada na ROI da imagem de flevoland. No processo fixamos arbitrariamente a radial 50 da amostra e fixamos o número de pixel 50. Desta maneira construímos duas funções $\ell{j}$ em cada lado da radial. Os gráficos das funções foram gerados para os canais de intensidades.

%Na figura \eqref{fig:prod_mag_l_rho_1_25_hh} podemos identificar o problema da função ser plana dificultando muito o processo de encontrar o ponto de máximo gerando oscilação na função $\ell(j)$. Na \eqref{fig:prod_mag_l_rho_26_120_hh} ocorre o problema das funções de bessel serem infinitas quando seu argumento assume valores grande, também dificultando o cálculo do valor máximo.  

%\begin{figure*}[hbt]
%	\centering
%     \subfloat[Pixel variando de 1 até 25 na radial 50 no canal (hh).  \label{fig:prod_mag_l_rho_1_25_hh}]{%
%       \includegraphics[width=0.50\linewidth]{fig_pdf_mag_prod_r_50_1_to_25_flev}
%     }
%     \subfloat[Pixel variando de 26 até 120 na radial 50 no canal (hh). \label{fig:prod_mag_l_rho_26_120_hh}]{%
%       \includegraphics[width=0.50\linewidth]{fig_pdf_mag_prod_r_50_26_to_120_flev}
%     }
%     \caption{Funções de máxima verossimilhança produto de magnitude com pixel fixo 25 no canal (hh).}
%     \label{fig:prod_mag_l_25_r_50_hh} 
%   \end{figure*}
   
%Nas figuras de \eqref{fig:prod_mag_l_25_r_50_hv} destacamos o problema da função ser plana. Assim como em no gráfico da função \eqref{fig:prod_mag_l_rho_1_25_hh} 
%\begin{figure*}[hbt]
%	\centering
%     \subfloat[Pixel variando de 1 até 25 na radial 50 no canal (hv).  \label{fig:prod_mag_l_rho_1_25_hv}]{%
%       \includegraphics[width=0.50\linewidth]{fig_pdf_mag_prod_r_50_1_to_25_flev_hv}
%     }
%     \subfloat[Pixel variando de 26 até 120 na radial 50 no canal (hv). \label{fig:prod_mag_l_rho_26_120_hv}]{%
%       \includegraphics[width=0.50\linewidth]{fig_pdf_mag_prod_r_50_26_to_120_flev_hv}
%     }
%     \caption{Funções de máxima verossimilhança produto de magnitude com pixel fixo 25 no canal (hv).}
%     \label{fig:prod_mag_l_25_r_50_hv} 
%   \end{figure*}
%
%Nas figuras de \eqref{fig:prod_mag_l_25_r_50_vv} destacamos o problema da função ser plana. Assim como em no gráfico da função \eqref{fig:prod_mag_l_rho_1_25_hh} 
%
%\begin{figure*}[hbt]
%	\centering
%     \subfloat[Pixel variando de 1 até 25 na radial 50 no canal (vv).  \label{fig:prod_mag_l_rho_1_25_vv}]{%
%       \includegraphics[width=0.50\linewidth]{fig_pdf_mag_prod_r_50_1_to_25_flev_vv}
%     }
%     \subfloat[Pixel variando de 26 até 120 na radial 50 no canal (vv). \label{fig:prod_mag_l_rho_26_120_vv}]{%
%       \includegraphics[width=0.50\linewidth]{fig_pdf_mag_prod_r_50_26_to_120_flev_vv}
%     }
%     \caption{Funções de máxima verossimilhança produto de magnitude com pixel fixo 25 no canal (vv).}
%     \label{fig:prod_mag_l_25_r_50_vv} 
%   \end{figure*}


%\subsubsection{Aplicação na imagem simulada com duas amostras}


%As figuras \eqref{fig:razao_tau_rho_1_50} até \eqref{fig:razao_tau_rho_51_400} mostram a função de log-verossimilhança da pdf razão de intensidades \eqref{eq:pdf_razao_intensidades_tau_w} aplicada na imagem de duas amostras simulada. No processo fixamos arbitrariamente a linha 80 da amostra e variamos o número de pixel entre 50, 150, 250 e 250. Desta maneira construímos duas funções $\ell{j}$, uma para cada lado da amostra. Nessas figuras fixamos o L = 4 e os canais (hh) e (vv).
% \begin{figure*}[hbt]
%	\centering
%     \subfloat[Pixel variando de 1 até 50 na linha 80.  \label{fig:razao_tau_rho_1_50}]{%
%       \includegraphics[width=0.50\linewidth]{fig_pdf_razao_r_80_1_to_50}
%     }
%     \subfloat[Pixel variando de 51 até 400 na linha 80. \label{fig:razao_tau_rho_51_400}]{%
%       \includegraphics[width=0.50\linewidth]{fig_pdf_razao_r_80_50_to_400}
%     }
%     \caption{Funções de máxima verossimilhança razão de intensidades com pixel fixo 50.}
%     \label{fig:razao_l_50_r_80} 
%   \end{figure*}
%\begin{figure*}[hbt]
%	\centering
%     \subfloat[Pixel variando de 1 até 150 na linha 80.  \label{fig:razao_tau_rho_1_150}]{%
%       \includegraphics[width=0.50\linewidth]{fig_pdf_razao_r_80_1_to_150}
%     }
%     \subfloat[Pixel variando de 151 até 400 na linha 80. \label{fig:razao_tau_rho_151_400}]{%
%       \includegraphics[width=0.50\linewidth]{fig_pdf_razao_r_80_150_to_400}
%     }
%     \caption{Funções de máxima verossimilhança razão de intensidades com pixel fixo 150.}
%     \label{fig:razao_l_150_r_80} 
%   \end{figure*}   
%   
%   \begin{figure*}[hbt]
%	\centering
%     \subfloat[Pixel variando de 1 até 250 na linha 80.  \label{fig:razao_tau_rho_1_250}]{%
%       \includegraphics[width=0.50\linewidth]{fig_pdf_razao_r_80_1_to_250}
%     }
%     \subfloat[Pixel variando de 251 até 400 na linha 80. \label{fig:razao_tau_rho_251_400}]{%
%       \includegraphics[width=0.50\linewidth]{fig_pdf_razao_r_80_250_to_400}
%     }
%     \caption{Funções de máxima verossimilhança razão de intensidades com pixel fixo 250.}
%     \label{fig:razao_l_250_r_80} 
%   \end{figure*}   
%   
%   \begin{figure*}[hbt]
%	\centering
%     \subfloat[Pixel variando de 1 até 350 na linha 80.  \label{fig:razao_tau_rho_1_350}]{%
%       \includegraphics[width=0.50\linewidth]{fig_pdf_razao_r_80_1_to_350}
%     }
%     \subfloat[Pixel variando de 351 até 400 na linha 80. \label{fig:razao_tau_rho_351_400}]{%
%       \includegraphics[width=0.50\linewidth]{fig_pdf_razao_r_80_350_to_400}
%     }
%     \caption{Funções de máxima verossimilhança razão de intensidades com pixel fixo 350.}
%     \label{fig:razao_l_250_r_80} 
%   \end{figure*}   
   

%O método da máxima verossimilhança \eqref{eq:TotalLogLikelihood} foi aplicado na imagem simulada com duas amostras, e as evidências de bordas estão mostradas na figura. \textcolor{red}{Base de dados gamf}
% \begin{figure*}[hbt]
%	\centering
%     \subfloat[Evidências no canal $\text{hh}$ \label{evidencias_hh_hv_vv_gamf:a}]{%
%       \includegraphics[width=0.5\linewidth]{im_sim_gamf_hh_hv_param_tau_rho_14_pixel}
%     }
%     \subfloat[xxxxxxxxxxxxxxxxxx $\text{hv}$ \label{evidencias_hh_hv_vv_gamf:b}]{%
%       \includegraphics[width=0.5\linewidth]{im_sim_gamf_hh_vv_param_tau_rho_14_pixel}
%     }      
%   %  \subfloat[Evidências no canal $\text{vv}$ \label{evidencias_hh_hv_vv_gamf:c}]{%
%    %   \includegraphics[width=0.5\linewidth]{im_sim_gamf_hh_evid_param_L_mu_14_pixel}
%    % }
%    \caption{Evidências de bordas para os três canais de intensidade}
%     \label{evidencias_hh_hv_vv_gamf} 
%   \end{figure*}
%   
%   \begin{figure*}[hbt]
%	\centering
%     \subfloat[Evidências no canal $\text{vv}$ \label{evidencias_hh_hv_vv_gamf:c}]{%
%       \includegraphics[width=0.5\linewidth]{im_sim_gamf_hv_vv_param_tau_rho_14_pixel}
%     }
%    \caption{Evidências de bordas para os três canais de intensidade}
%     \label{evidencias_hh_hv_vv_gamf} 
%   \end{figure*}   
   

%\section{Resultados numéricos para o método MLE aplicado em cada distribuíção}

%Usamos uma imagem AIRSAR PolSAR de Flevoland, banda L, de $750\times 1024$ pixels para os testes.  Fig.~\ref{flevoland_radial_4look} mostra o ROI, com as linhas radiais onde as bordas são detectadas. Fig.~\ref{flevoland_flevoland} mostra a referência do solo em vermelho.  
%  \begin{figure}[hbt]
%   \centering
%     \subfloat[Imagem, Região de Interesse (ROI), and radiais. \label{flevoland_radial_4look}]{%
%%       \includegraphics[viewport= 0 50 500 550, clip=true, width=0.23\textwidth]{flevoland_radial_4_look_black_crop}}      
%       \includegraphics[width=0.53\textwidth]{flevoland_radial_4_look_black_crop}}
%     \subfloat[Ground reference\label{gt_flevoland}]{%
%       \includegraphics[width=0.5\textwidth]{gt_flevoland_crop}
%     }
%    \caption{Decomposição de Pauli para a imagem de Flevoland, região de interesse, e referência \textit{ground}}
%    \label{roi_gt}
%\end{figure}

%\subsection{Método da verossimilhança aplicado na pdf univariada $\Gamma$.}
%Resolvendo o problema,
%$$
%\widehat{\jmath}= \arg\max\limits_{j\in [\min_s,N-\min_s]}\ell(j;\widehat{\rho}_I, \widehat{L}_I,\widehat{\rho}_E, \widehat{L}_E),
%$$

%Figs.~\ref{evidencias_hh_hv_vv}\subref{evidencias_hh_hv_vv:a},~\ref{evidencias_hh_hv_vv}\subref{evidencias_hh_hv_vv:b}, e~\ref{evidencias_hh_hv_vv:a}\subref{evidencias_hh_hv_vv:c}, mostram, respectivamente, as evidências de borda nos canais $\text{hh}$, $\text{hv}$ e $\text{vv}$ como obtidos pela MLE.

%Vale notar que a GenSA identificou com precisão o valor máximo de $\eqref{eq:TotalLogLikelihood}$, mesmo na presença de múltiplos máximos locais. 
%Uma avaliação visual leva à conclusão de que os melhores resultados são fornecidos por $\text{hv}$, embora com alguns pontos longe da borda real.

% \begin{figure*}[hbt]
%	\centering
%     \subfloat[Evidências no canal $\text{hh}$ \label{evidencias_hh_hv_vv:a}]{%
%       \includegraphics[width=0.32\linewidth]{flevoland_hh_evid_param_L_mu_14_pixel_crop}
%     }
%     \subfloat[Evidências no canal $\text{hv}$ \label{evidencias_hh_hv_vv:b}]{%
%       \includegraphics[width=0.32\linewidth]{flevoland_hv_evid_param_L_mu_14_pixel_crop}
%     }
%     \subfloat[Evidências no canal $\text{vv}$ \label{evidencias_hh_hv_vv:c}]{%
%       \includegraphics[width=0.32\linewidth]{flevoland_vv_evid_param_L_mu_14_pixel_crop}
%     }
%     \caption{Evidências de bordas para os três canais de intensidade}
%     \label{evidencias_hh_hv_vv} 
%   \end{figure*}

%\subsection{Método da verossimilhança aplicado na pdf magnitude do produto.}


% \begin{figure*}[hbt]
%	\centering
%     \subfloat[Evidências no canal $\text{hh}$ \label{evidencias_hh_hv_vv:a}]{%
%       \includegraphics[width=0.32\linewidth]{}
%     }
%     \subfloat[Evidências no canal $\text{hv}$ \label{evidencias_hh_hv_vv:b}]{%
%       \includegraphics[width=0.32\linewidth]{flevoland_hv_evid_param_L_mu_14_pixel_crop}
%     }
%     \subfloat[Evidências no canal $\text{vv}$ \label{evidencias_hh_hv_vv:c}]{%
%       \includegraphics[width=0.32\linewidth]{flevoland_vv_evid_param_L_mu_14_pixel_crop}
%     }
%     \caption{Evidências de bordas para os três canais de intensidade}
%     \label{evidencias_hh_hv_vv} 
%   \end{figure*}


%\begin{figure}[hbt]
%\minipage{0.5\textwidth}
%  \includegraphics[width=\linewidth]{funv_max_ver_j_10_flev_razao.pdf}
%  	\caption{$\sigma= 12.3426$.}\label{cap_acf_fig04}
%\endminipage\hfill
%\minipage{0.5\textwidth}
%  \includegraphics[width=\linewidth]{funv_max_ver_j_20_flev_razao.pdf}
%		\caption{$\sigma=2.1029 $.}\label{cap_acf_fig05}
%\endminipage\hfill
%\centering
%\minipage{0.5\textwidth}
%  \includegraphics[width=\linewidth]{funv_max_ver_j_30_flev_razao.pdf}
%  	\caption{$\sigma=1.4999 $.}\label{cap_acf_fig04}
%\endminipage\hfill
%\minipage{0.5\textwidth}
%  \includegraphics[width=\linewidth]{funv_max_ver_j_40_flev_razao.pdf}
%		\caption{$\sigma=12.6414 $.}\label{cap_acf_fig05}
%\endminipage\hfill
%\end{figure}
%\begin{figure}[hbt]
%\minipage{0.5\textwidth}
%  \includegraphics[width=\linewidth]{funv_max_ver_j_50_flev_razao.pdf}
%  	\caption{$\sigma=10.4523$.}\label{cap_acf_fig04}
%\endminipage\hfill
%\minipage{0.5\textwidth}
%  \includegraphics[width=\linewidth]{funv_max_ver_j_60_flev_razao.pdf}
%		\caption{$\sigma= 14.2156$.}\label{cap_acf_fig05}
%\endminipage\hfill
%\centering
%\minipage{0.5\textwidth}
%  \includegraphics[width=\linewidth]{funv_max_ver_j_70_flev_razao.pdf}
%  	\caption{$\sigma=9.8405 $.}\label{cap_acf_fig04}
%\endminipage\hfill
%\minipage{0.5\textwidth}
%  \includegraphics[width=\linewidth]{funv_max_ver_j_80_flev_razao.pdf}
%		\caption{$\sigma=13.1298 $.}\label{cap_acf_fig05}
%\endminipage\hfill
%\end{figure}
%\subsection{Distribuição univariada da magnitude do produto}
%A magnitude do produto $\mathbf{S}_i$ e $\mathbf{S}_j$ é uma importante medida para as imagem SAR polarimétrica. Definimos a magnitude normalizada por 
%
%\begin{equation}
%	\xi = \frac{\left|\frac{1}{L} \sum_{k=1}^L\mathbf{S}_i(k)\mathbf{S}_j^H(k) \right|}{\sqrt{E[|\mathbf{S}_i|^2]E[|\mathbf{S}_i|^2]}}=\frac{g}{h}.
%\end{equation}
%onde é definido por $g=|\mathbf{S}_i\mathbf{S}_j^H|$ e $h=\sqrt{E[|\mathbf{S}_i|^2]E[|\mathbf{S}_i|^2]}$.
%\begin{equation}
%\begin{array}{ccc}
%	f(\xi)&=&\frac{4L^{L+1}\xi^L}{\Gamma(L)(1-|\rho|^2)}I_0\left(\frac{2|\rho|L\xi}{1-|\rho|^2}\right)K_{L-1}\left(\frac{2L\xi}{1-|\rho|^2}\right).
%		\end{array}
%\end{equation}
%\begin{equation}
%\begin{array}{ccc}
%	\ln f(\xi)&=&\ln\left(\frac{4L^{L+1}\xi^L}{\Gamma(L)(1-|\rho|^2)}I_0\left(\frac{2|\rho|L\xi}{1-|\rho|^2}\right)K_{L-1}\left(\frac{2L\xi}{1-|\rho|^2}\right)\right).
%		\end{array}
%\end{equation}
%\begin{equation}
%\begin{array}{ccc}
%	\ln f(\xi)&=&\ln\left(\frac{4L^{L+1}\xi^L}{\Gamma(L)(1-|\rho|^2)}\right)+\ln I_0\left(\frac{2|\rho|L\xi}{1-|\rho|^2}\right)+ \ln K_{L-1}\left(\frac{2L\xi}{1-|\rho|^2}\right).
%		\end{array}
%\end{equation}
%
%\begin{equation}
%\begin{array}{ccc}
%	\ln f(\xi)&=&\ln (4L^{L+1}\xi^L)-\ln(\Gamma(L)(1-|\rho|^2))+\ln I_0\left(\frac{2|\rho|L\xi}{1-|\rho|^2}\right)+ \ln K_{L-1}\left(\frac{2L\xi}{1-|\rho|^2}\right).
%		\end{array}
%\end{equation}
%
%\begin{equation}
%\begin{array}{ccc}
%	\ln f(\xi)&=&\ln (4)+\ln L^{L+1}+\ln \xi^L-\ln\Gamma(L)-\ln(1-|\rho|^2)+\ln I_0\left(\frac{2|\rho|L\xi}{1-|\rho|^2}\right)+ \ln K_{L-1}\left(\frac{2L\xi}{1-|\rho|^2}\right).
%		\end{array}
%\end{equation}
%
%\begin{equation}
%\begin{array}{ccc}
%	\ln f(\xi)&=&\ln (4)+(L+1)\ln L+L\ln \xi-\ln\Gamma(L)-\ln(1-|\rho|^2)+\ln I_0\left(\frac{2|\rho|L\xi}{1-|\rho|^2}\right)+ \ln K_{L-1}\left(\frac{2L\xi}{1-|\rho|^2}\right).
%		\end{array}
%\end{equation}
%
%\begin{figure}[hbt]
%\minipage{0.5\textwidth}
%  \includegraphics[width=\linewidth]{funv_max_ver_j_10_flev_produto.pdf}
%  	\caption{$\sigma= 0.0001241$.}\label{cap_acf_fig04}
%\endminipage\hfill
%\minipage{0.5\textwidth}
%  \includegraphics[width=\linewidth]{funv_max_ver_j_20_flev_produto.pdf}
%		\caption{$\sigma= 0.0021969$.}\label{cap_acf_fig05}
%\endminipage\hfill
%\centering
%\minipage{0.5\textwidth}
%  \includegraphics[width=\linewidth]{funv_max_ver_j_30_flev_produto.pdf}
%  	\caption{$\sigma=0.0047520 $.}\label{cap_acf_fig04}
%\endminipage\hfill
%\minipage{0.5\textwidth}
%  \includegraphics[width=\linewidth]{funv_max_ver_j_40_flev_produto.pdf}
%		\caption{$\sigma= 0.0123943$.}\label{cap_acf_fig05}
%\endminipage\hfill
%\end{figure}
%\begin{figure}[hbt]
%\minipage{0.5\textwidth}
%  \includegraphics[width=\linewidth]{funv_max_ver_j_50_flev_produto.pdf}
%  	\caption{$\sigma= 0.0002715$.}\label{cap_acf_fig04}
%\endminipage\hfill
%\minipage{0.5\textwidth}
%  \includegraphics[width=\linewidth]{funv_max_ver_j_60_flev_produto.pdf}
%		\caption{$\sigma= 0.0001922$.}\label{cap_acf_fig05}
%\endminipage\hfill
%\centering
%\minipage{0.5\textwidth}
%  \includegraphics[width=\linewidth]{funv_max_ver_j_70_flev_produto.pdf}
%  	\caption{$\sigma= 0.0004329$.}\label{cap_acf_fig04}
%\endminipage\hfill
%\minipage{0.5\textwidth}
%  \includegraphics[width=\linewidth]{funv_max_ver_j_80_flev_produto.pdf}
%		\caption{$\sigma= 0.0002790$.}\label{cap_acf_fig05}
%\endminipage\hfill
%\end{figure}


%\subsection{Distribuição bivariada produto de intensidades - Lee } 
%
%O $PDF$ conjunto retorna de dois canais correlacionados dos radares polarimétricos e interferométricos são importantes. As $PDF's$ conjuntas conduzem a derivação da intensidade e amplitude razão $PDF's$. Da equação (\ref{eqn42}) temos que as intensidades {\it multilook} sejam 
%
%\begin{equation}\label{eqn59}
%\begin{array}{ccccc}
%	R_1&=&\frac{1}{n}\sum_{k=1}^{n}|S_i(k)|^2&=&\frac{B_1C_{11}}{n}\\
%	R_2&=&\frac{1}{n}\sum_{k=1}^{n}|S_j(k)|^2&=&\frac{B_2C_{22}}{n}\\
%\end{array}
%\end{equation}
%
%Integrando a equação (\ref{eqn52}) em relação a $\eta$ e $\psi$. A $PDF$ é
%
%\begin{equation}\label{eqn60}
%	p(B_1,B_2)=\frac{\left(B_1B_2\right)^{\frac{n-1}{2}}\exp\left(-\frac{B_1+B_2}{1-|\rho_c|^2}\right)}{\Gamma(n)(1-|\rho_c|^2)|\rho_c|^{n-1}}I_{n-1}\left(2\sqrt{B_1B_2}\frac{|\rho_c|}{1-|\rho_c|^2}\right)
%\end{equation}

%Sendo
%\begin{equation}\label{eqn61}
%	I_{\mu}(Z)=\frac{(\frac{z}{2})^{\mu}}{\Gamma(\mu+1)} F_{1}^{0}[-;\mu+1;\frac{z^2}{4}]
%\end{equation}
%
%\begin{equation}\label{eqn62}
%	p(B_1,B_2)=\frac{n^{n+1}\left(R_1R_2\right)^{\frac{n-1}{2}}\exp\left(-\frac{n(\frac{R_1}{C_{11}}+\frac{R_2}{C_{22}})}{1-|\rho_c|^2}\right)}{(C_{11}C_{22})^{\frac{n+1}{2}}\Gamma(n)(1-|\rho_c|^2)|\rho_c|^{n-1}}I_{n-1}\left(2n\sqrt{\frac{R_1R_2}{C_{11}C_{22}}}\frac{|\rho_c|}{1-|\rho_c|^2}\right)
%\end{equation}
%\subsection{Distribuição $\Gamma$ trivariada - Hagedorn }
%\begin{equation}\label{eqn62}
%\begin{array}{ccc}
%	p(I_1,I_2,I_3)&=& \frac{\exp(-\frac{1}{2}(a_1I_1+b_1I_2+c_1I_3))}{8(n-1)|C|^{\frac{n}{2}}(d_1d_2d_3)^{n-1}}\sum_{k=n-1}^{\infty}k(-1)^{k-n+1}C_{k-n+1}^{n-1}(cos(\gamma))\\
%	&&I_k(d_1\sqrt{I_1I_2})I_k(d_2\sqrt{I_2I_3})I_k(d_3\sqrt{I_1I_3})
%\end{array}
%\end{equation}
%
%
%
%Para cada $i$:
%  
%Estimar $(\mu_i, L_i )$ por $(\hat{\mu}_i, \hat{L}_i)(Z_I)$ em uma primeira metade da faixa de dados.
%
%Estimar $(\mu_i, L_i )$ por $(\hat{\mu}_i, \hat{L}_i)(Z_E)$ em uma segunda metade da faixa de dados.
%
%Usando o estimador de máxima verossimilhança, 
%\begin{equation}\label{cap_acf_16}
%    (\hat{\mu}_i, \hat{L}_i)(Z_{\bigodot})= \text{arg}\,\max\limits_{(\mu, L)\in \mathbb{R}^{+}\times \mathbb{R}^{+}}%\ell(\mu,L;Z_i).\\
%\end{equation}
%Assim
%\begin{equation}\label{cap_acf_16}
%\begin{array}{ccc}
% \ell(\mu, L)&=&\ln\left(\prod_{k=1}^{n}f_{Z_{i}}(Z_{k};\mu,L)\right)\\
%  \ell(\mu, L)&=&\sum_{k=1}^{n}\ln\left(f_{Z_{i}}(Z_{k};\mu,L)\right)
% \end{array}
% \end{equation}
%Temos duas amostras
%\begin{equation}\label{cap_acf_16}
% \begin{array}{lll}
%\ell(\mu_I, L_I,\mu_E, L_E, j)&=&\sum_{k=1}^{j}     \left[   L_I\ln L_I +(L_I   - 1) \ln Z_{i}-L_I \ln \mu_I-\ln \Gamma(L_i) -%\frac{L_I}{\mu_I} Z_i \right]\\
%                                               &+&\sum_{k=j+1}^{N}\left[   L_E\ln L_E +(L_E - 1) \ln Z_{i}-L_E \ln \mu_E-\ln \Gamma(L_E) -\frac{L_E}{\mu_E} Z_i \right]\\
%\ell(\mu_I, L_I,\mu_E, L_E, j)&=&  L_I\ln L_I \sum_{k=1}^{j} 1 +(L_I   - 1) \sum_{k=1}^{j}  \ln Z_{i}-L_I \ln \mu_I\sum_{k=1}^{j} 1-\ln \Gamma(L_i) \sum_{k=1}^{j} 1  -\frac{L_I}{\mu_I} \sum_{k=1}^{j}   Z_i \\
%                                               &+&  L_E\ln L_E \sum_{k=j+1}^{N}1+(L_E - 1) \sum_{k=j+1}^{N}\ln Z_{i}- \ln %\mu_E\sum_{k=j+1}^{N}1-\ln \Gamma(L_E)\sum_{k=j+1}^{N} 1-\frac{L_E}{\mu_E} \sum_{k=j+1}^{N}Z_i \\
%\ell(\mu_I, L_I,\mu_E, L_E, j)&=&  L_I\ln L_I j-L_I \ln \mu_I j-\ln \Gamma(L_i) j \\
%&+& (L_I  - 1) \sum_{k=1}^{j}  \ln Z_{i}  -\frac{L_I}{\mu_I} \sum_{k=1}^{j}   Z_i \\
%                                               &+&  L_E\ln L_E (N-j)-L_E \ln \mu_E(N-j)-\ln \Gamma(L_E)(N-j)- \\
%                                               &+& (L_E - 1) \sum_{k=j+1}^{N}\ln Z_{i}-\frac{L_E}{\mu_E} \sum_{k=j+1}^{N}Z_i \\
%                                                \end{array}
% \end{equation}
%
%
%\section{Imagens PolSAR reais}
%\begin{figure}[hbt]
%\centering
%\includegraphics[width=4.0in]{grafico_pdf_lee_1994_razao_amplitude.pdf}
%	\caption{Distribuição razão de amplitudes {\it L- visadas}.}
%\label{fig2}
%\end{figure}
%\begin{figure}[hbt]
%\centering
%	\includegraphics[width=4.0in]{sf_amostras_b_r_y.pdf}
%	\vspace{-2.5cm}
%	\caption{Regiões de interesses (ROIs).}
%\label{fig2}
%\end{figure}
% A tabela mostra os coeficientes de correlação para as regiões destacadas na figura. Os coeficientes correlacionam os canais $(hh-hv)$, $(hh-vv)$ e $(vv-hv)$ respectivamente para o mar ( ROI azul), floresta (ROI vermelho) e zona urbana (ROI amarelo).
%\begin{table}[hbt]
%	\centering
%	\caption{Coeficientes de correlação.}\label{cap_acf_tab04}
%\begin{tabular}{@{}lccc@{}} \toprule
%	Coeficiente de correlação & Mar  & Floresta & Zona urbana \\ \midrule
%	$(hh-hv)$ & 0.5548 & 0.7024 &  0.7177 \\ 
%	$(hh-vv)$ & 0.8743 & 0.6633 &  0.6483\\ 
%	$(vv-hv)$ & 0.5128 & 0.6065 &  0.6175\\ \bottomrule
%\end{tabular}
%\end{table}
%
%A figura acima mostra a baía de San Francisco ($450 \times 600$) com três regiões de interesses destacadas oceâno, floresta e zona urbana respectivamente nas cores azul, vermelho e amarelo. As ROI's têm dimensão ($50 \times 50$) adquirindo dados nos três canais de intensidade. O histograma e as pdf's teóricas são mostradas na figura abaixo. No cálculo das pdf's foi usado a equação 
%\begin{equation}\label{cap_acf_23}
%	f_{Z_{i}}(Z_{i};\frac{L}{\sigma_{i}^2},L)=\frac{L^{L}Z_{i}^{L-1}}{\sigma_{i}^{2L}\Gamma(L)} \exp(-L\frac{Z_{i}}{\sigma_{i}^2}), \\
%\end{equation}
%sendo $L=\{2,3,4\}$ e $\sigma_{i}^2$ a média de todos as entradas das respectivas regiões de interesses conforme \cite{nhfc}.
%
%\begin{figure}[hbt]
%\centering
%	\includegraphics[width=4.0in]{graf_pdf_roi_mar_hh.pdf}
%	\caption{Regiões de interesses (ROIs).}
%\label{fig2}
%\end{figure}
%\textcolor{red}{OBS: Tem algo errado no gráfico, provavelmente a estimativa de $\sigma_{i}$.}
% ****************************************************************

\chapter{Discussões, conclusões e contribuíções}\label{cap_conclusao}
Esta tese estudou e comparou seis métodos de fusão provenientes da detecção das evidências de bordas, nos canais de intensidade hh, hv e vv de imagens PolSAR múltiplas visadas. O método consiste em detectar evidências de bordas, isto é, pontos de transição em uma faixa de dados, a mais fina possível, idealmente com largura de um píxel, fornecendo dados que cobrem duas regiões usando o método MLE sob a distribuição de Wishart. Os métodos em questão são: média simples (MS), transformada \textit{wavelet} discreta multi-resolução (MR-DWT), análise de componente principal (PCA), estatísticas ROC, transformada \textit{wavelet} estacionária multi-resolução (MR-SWT), e um método de multi-resolução baseado na decomposição de valores singulares (MR-SVD). 

Uma etapa crucial para o desenvolvimento desta pesquisa foi a validação dos algoritmos usados em imagens simuladas, viabilizando assim, a realização dos testes em imagens adquiridas de sensores PolSAR, dando condições para que as etapas subsequentes pudessem ser realizadas. 

Na imagem simulada podemos notar que os métodos de fusão melhoram o desempenho em detectar bordas precisas com relação aos métodos de detectar as evidências de bordas para cada canal. Podemos confirmar assim o melhor desempenho dos métodos de fusão para imagens simuladas.

Prosseguindo a pesquisa, os dados fornecidos pelas linhas radiais das imagens foram selecionados e posteriormente foi aplicado o método MLE. Nesse processo ao ser usada a função de máxima verossimilhança total foram constatadas oscilações nas extremidades. Para contornar esse problema foram definidos, conforme a região de interesse, valores de folgas empiricamente. Essa alternativa mostrou-se suficiente para resolver o problema.      

Outro problema apresentado na aplicação da função de máxima verossimilhança total, se refere a sua característica de não ser diferenciável, fato que dificulta o uso de métodos de otimização que calculam a derivada da função. O problema foi resolvido usando o método GenSA.

Com os testes realizados, aplicando os métodos propostos nesta tese, foram encontradas bordas nas imagens de banda L, do sensor aerotransportado AIRSAR sobre Flevoland e São Francisco, comprovando a viabilidade dos métodos de fusão para a tarefa de detectar bordas.

Sobre os campos agrícolas de Flevoland, a melhor evidência de borda foi observada no canal hv. O canal hh forneceu as melhores estimativas de bordas entre as áreas urbanas e as áreas, de mar e vegetação, de São Francisco. Essa diversidade de conteúdo de informação justifica a necessidade de fusão das evidências de bordas.

Aplicamos métodos de fusão para reunir as evidências de bordas, obtidas nos três canais de intensidade. Ao comparar o desempenho dos métodos, os melhores resultados foram produzidos pelo método PCA e pelo método de múltipla resolução com decomposição em valores singulares, MR-SVD. Essa análise leva em conta o custo computacional em termos de tempo de processamento e a presença de \textit{outliers}. 

O método E-ROC mostrou desempenho inferior aos outros métodos, entretanto, seu desempenho pode melhorar a medida que outros canais são inseridos para a detecção de evidências de bordas.

Os resultados foram avaliados quantitativamente, verificando a proximidade dos pontos fundidos com a borda real e pela presença de \textit{outliers}. Embora a fusão por média e a fusão com PCA sejam semelhantes, com relação à probabilidade de detectar corretamente a borda, o última fornece uma ponderação mais eficaz das evidências de bordas. A partir desse fato, a fusão com PCA é capaz de descartar completamente as evidências enganosas, enquanto a fusão por média não.

É destacado as seguintes contribuição desta tese:
\begin{enumerate}[label=(\roman*)]
\item \label{item:contr_i} Propor a fusão de evidências de bordas, obtidas em cada canal, aprimorando o método de detecção de bordas;
\item \label{item:contr_ii} Aplicar e comparar os métodos de fusão: MS, MR--DWT, PCA, E--ROC, MR--SWT, e MR--SVD;
\item \label{item:contr_iii} Verificação da viabilidade do uso do método GenSA no método MLE com o algoritmo Gambini, aplicado aos canais de intensidades de uma imagem PolSAR, com o objetivo de detectar evidências de bordas, como proposto e analisado em \citet{gmbf, gmbf_sc, fbgm, nhfc};
\item \label{item:contr_iv} Realizar medidas de qualidade para a detecção em cada canal, e na fusão, usando uma métrica baseada na probabilidade de detecção de bordas, comparando com a imagem definida como referência (\textit{Ground Reference}).\index{Ground Truth}      
\item \label{item:contr_v} Identificação de duas características que dificultam a detecção de evidências de bordas: oscilação nas extremidades e não diferenciabilidade em vários pontos do domínio das funções de máxima verossimilhança total. 
Para contornar o problema propomos, respectivamente, as seguintes soluções: definir empiricamente constantes de folga nas extremidades, usar o método GenSA para encontrar o ponto de máximo da função.
\end{enumerate}

No decorrer da pesquisa alguns caminhos foram se apresentando como possíveis trabalhos futuros:
\begin{enumerate}[label=(\roman*)]
\item \label{item:pesq_fut_i} Aumentar o número de canais ou funções de distribuição de densidade para encontrar as evidências de bordas. Isso é possível, pois dados totalmente polarimétricos são mais ricos do que os canais de intensidades;
\item \label{item:pesq_fut_ii} Propor novas técnicas de fusão para evidências de bordas;
\item \label{item:pesq_fut_iii} Melhorar as medidas para aproveitamento ou descarte de canais no método de fusão; 
\item \label{item:pesq_fut_iv} Verificar os métodos de fusão inserindo textura nos modelos;
\item \label{item:pesq_fut_v} Classificar as regiões da imagem PolSAR, e usar as ideias propostas para refinar a detecção de bordas; 
\item \label{item:pesq_fut_vi} Pós-processamento, tanto para os métodos de detecção de evidências de bordas parciais, como também para os métodos de fusão das evidências de bordas.
\end{enumerate}
% cabeçalho para os apêndices
%\renewcommand{\chaptermark}[1]{\markboth{\MakeUppercase{\appendixname\ \thechapter}} {\MakeUppercase{#1}} }
%\fancyhead[RE,LO]{}
%\appendix
%\include{ape-conjuntos}      % associado ao arquivo: 'ape-conjuntos.tex'
% ---------------------------------------------------------------------------- %
% Bibliografia
\backmatter \singlespacing   % espaçamento simples
\bibliographystyle{plainnat-ime} % citação bibliográfica textual
\bibliography{../bibliografia}  % associado ao arquivo: 'bibliografia.bib'
% ---------------------------------------------------------------------------- %
% Índice remissivo
%\printindex   % imprime o índice remissivo no documento 

\end{document}
