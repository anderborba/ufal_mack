\documentclass[table,aspectratio=169]{beamer}

\usefonttheme{professionalfonts}
\renewcommand*\familydefault{\sfdefault}
\usepackage{sansmathfonts}

\usetheme[progressbar=frametitle]{metropolis}
\usepackage{appendixnumberbeamer}

\usepackage{booktabs}
\usepackage[scale=2]{ccicons}

\usepackage{pgfplots}
\usepgfplotslibrary{dateplot}

\usepackage{xspace}
\newcommand{\themename}{\textbf{\textsc{metropolis}}\xspace}
\usepackage{bibentry}       % AAB
\usepackage{amsmath, bm}    % AAB
\usepackage{tikz}           % AAB    
\usetikzlibrary{shapes,arrows,shadows}  % AAB inserido
\usepackage[caption=false,font=normalsize,labelfont=sf,textfont=sf]{subfig}
\usepackage{float}
\usepackage[linesnumbered,ruled,vlined]{algorithm2e}
\usepackage{natbib}

\DeclareMathOperator{\traco}{tr} %AAB
\graphicspath{{../Dissertacao/figuras/}%
{/Users/acfrery/Documents/Transparencias/Figuras/}
}

\definecolor{anti-flashwhite}{rgb}{0.95, 0.95, 0.96}
\definecolor{byzantine}{rgb}{0.74, 0.2, 0.64}
\definecolor{VUWgreen}{rgb}{0, .196, .136}

\setbeamercolor{background canvas}{bg=VUWgreen}
\setbeamercolor{normal text}{fg=anti-flashwhite, bg=VUWgreen}

\setbeamercovered{dynamic}
\setbeamertemplate{footline}[frame number]
\setbeamertemplate{navigation symbols}{}

\setcounter{tocdepth}{1}
{\AtBeginSection[]
	{
		\begin{frame}<beamer>
			\frametitle{Outline}
			\tableofcontents
		\end{frame}
}}

\renewcommand{\newblock}{}


\title{Edge Detection in PolSAR Imagery with Statistical Tools}
%\subtitle{A modern beamer theme}
\date{14 October 2021}
\author{Alejandro C.\ Frery} 
\institute{Victoria University of Wellington}

\begin{document}

\maketitle

\section{Introduction}

\begin{frame}{PolSAR Images}
PolSAR images are an essential source of information in Remote Sensing.

They are formed with the echoes that the scene returns when illuminated with microwaves.

Main characteristics:
\begin{itemize}
	\item The sensor is active.
	\item Microwaves are little affected by adverse atmospheric conditions.
	\item The signal may penetrate soil and canopies.
	\item The return brings textural and dielectric information.
	\item The technology has evolved to provide spatial resolutions of the order of centimetres.
	\item Their statistical properties are challenging.
\end{itemize}
\end{frame}

\begin{frame}{PolSAR Images}
	\begin{figure}[hbt]
	\centering
     \subfloat[Flevoland\label{evidencias_hh_hv_vv:a}]{%
       \includegraphics[width=0.35\linewidth]{flevoland_4_look}
     }\quad
     \subfloat[San Francisco Bay\label{evidencias_hh_hv_vv:b}]{%
       \includegraphics[width=0.35\linewidth]{san_francisco_2020}}
     \caption{PolSAR images}
     \label{fig12}
   \end{figure}	
\end{frame}

\begin{frame}[fragile]{Edges}
	Our visual system has evolved to extract edges and to use them to build higher-level cognitive representations.

	Edges are one of the most important \textit{hidden} features in any image processing pipeline.
	
	Edges are boundaries between regions with intra-class similarities and inter-class differences.
	
	The same image may present different edges depending on the definition of the feature under consideration, e.g., color, contrast, or texture.
\end{frame}

\begin{frame}{Rationale}
Our approach:
\begin{itemize}
	\item \alert{exploits} the multiplicity of information PolSAR images provide (one statistical model for each \alert{marginal description}),
	\item \alert{estimates} the location of an edge on each feature using an \alert{efficient algorithm},
	\item \alert{fuses} the evidences to provide a single estimate of the edge location.
\end{itemize} 
\end{frame}

\section{Edge Detection}

\begin{frame}{Edge detection}
	\begin{columns}
		\begin{column}{5cm}
			\includegraphics[width=\linewidth]{EdgeDetectionScheme}
		\end{column}
		\begin{column}{9cm}
			\begin{alertblock}{General procedure:}
				\begin{itemize}
					\item identify the centroid of a region of interest (ROI) in an automatic, semi-automatic or manual manner;
					\item cast rays from the centroid to the outside of the area;
					\item collect data around the rays, ideally one pixel width;
					\item detect points in the data strips which provide evidence of changes in their statistical properties, i.e., a transition point.
				\end{itemize}
			\end{alertblock}
		\end{column}
	\end{columns}
\end{frame}

\begin{frame}{Edge Detection}
	\begin{columns}
		{\begin{column}{10cm}
				\begin{algorithm}[H]
					\DontPrintSemicolon
					\KwData{Strip $
						z_1, z_2, \dots, z_{k-1}, z_k, 
						z_{k+1},  \dots, z_{n-1}, z_n
						$ in which we suspect there is coordinate $k^*$ where properties change.}
					\For{each $k$}{
						Estimate $
						\underbrace{z_1, z_2, \dots, z_{k-1}, z_k}_{\widehat{\theta}_{1}(k)}, 
						\underbrace{z_{k+1}, z_{k+2} \dots, z_{n-1}, z_n}_{\widehat{\theta}_{2}(k)}
						$
						
						Compare $\mathcal D\big(\widehat{\theta}_{1}(k)\big)$ and $\mathcal D\big(\widehat{\theta}_{2}(k)\big)$
						
						Store the measurement of the difference in $v[k]$
					}
					
					Find the maximum evidence in $v$, say $\widehat{k^*}$
					
					Return $\widehat{k^*}$
					\caption{Gambini Algorithm}
				\end{algorithm}
		\end{column}}
		{\begin{column}{3cm}
				\includegraphics[width=\linewidth]{GambiniTango}
		\end{column}}
	\end{columns}
\end{frame}

\begin{frame}{Edge detection}
	\centering
	\includegraphics<1-1>[width=.7\linewidth]{Intensity}
	\includegraphics<2-2>[width=.7\linewidth]{Parameter}
	\includegraphics<3-3>[width=.7\linewidth]{Distance}
\end{frame}

\begin{frame}{Example}
	\centering
	\includegraphics[width=.7\linewidth]{flevoland_radial_25_point_hh_crop}
\end{frame}

\begin{frame}{Question}
	We have several ways of measuring differences between samples, provided the models cited above, among them:
	\begin{itemize}
		\item joint likelihood equation,
		\item stochastic distances (Kullback-Leibler, Hellinger, $\chi^2$, etc.)
		\item difference of entropies,
		\item geodesic distances.
	\end{itemize}
	
	\begin{alertblock}{Edge estimation}
		Which of those edge estimators has the best cost-effective performance (bias, mean squared error)?
	\end{alertblock}
\end{frame}

\section{The Models}

\begin{frame}{Statistical Modeling}
The observation on each single-frequency fully PolSAR pixel is a Hermitian positive definite matrix with distribution characterized by the following probability density function:
\begin{equation}
    f_{\mathbf{Z}}(\mathbf{Z};\Sigma,L)=\frac{L^{mL}|\mathbf{Z}|^{L-m}}{|\Sigma_{s}|^{L}\Gamma_m(L)} \exp\big(-L\traco(\Sigma^{-1}\mathbf{Z})\big),
    \label{eq_04},
\end{equation} 
where
 $\traco(\cdot)$ is the trace operator, $\Gamma_m(L)$ is a multivariate Gamma function
\begin{equation*}
	\Gamma_m(L)=\pi^{\frac{1}{2}m(m-1)} \prod_{i=0}^{m-1}\Gamma(L-i),
\end{equation*}
$\Gamma(\cdot)$ is the Gamma function, and
$m=3$.

We denote this situation $\mathbf{Z}\sim \mathcal W(\Sigma, L)$.
This is the Multilook Complex Scaled Wishart distribution.
The parameters that index this law are the covariance matrix $\Sigma$ and the number of looks $L$.
\end{frame}

\begin{frame}{Statistical Modeling}
More explicitly:
$$
\mathbf{Z} = 
\begin{pmatrix}
I_{\text{HH}}	& \text{Cov}(\text{HH},\text{HV})	& \text{Cov}(\text{HH},\text{VV})\\
	*			& I_{\text{HV}}		& \text{Cov}(\text{HV},\text{VV})\\
	*			&		*			& I_{\text{VV}}
\end{pmatrix}
\sim
\mathcal W(\Sigma, L),
$$
where $I_{ii}$ is the intensity from channel $i$,
$\text{Cov}(i,j)$ is the complex covariance between channels $i$ and $j$, 
and $*$ denotes the conjugate.
\end{frame}

\begin{frame}{Marginal Distributions}
The $\mathcal W(\Sigma, L)$ model describes the complete matrix, but it also induces the following marginal distributions:
\begin{enumerate}
	\uncover<1->{\item Each of the three intensities $I_{ii}$ follows a Gamma distribution with shape parameter $L$ and a different mean.}
	\uncover<2->{\item Each of the three pairs $(I_{ii},I_{jj})$ follows a bivariate Gamma distribution.}
	\uncover<3->{\item The triplet $(I_{\text{HH}},I_{\text{HV}},I_{\text{VV}})$ follows a trivariate Gamma distribution.}
	\uncover<4->{\item Each of the six ratios $I_{ii}/I_{jj}$ follows a transformed Fisher-Snedekor distribution.}
	\uncover<5->{\item Each of the three phase differences $\phi_{i,j}$ follows a distribution with support in $(-\pi,\pi]$.}
	\uncover<6-6>{\item[] And there are other possibilities!}
\end{enumerate}
\uncover<7->{We have, at least, $3+ 3+ 1+ 6+ 3=16$ single potential sources of information.}
\end{frame}

\begin{frame}{Question}
In principle, we have all the information coded in the full covariance matrix, but it may be expensive to obtain, store and process.

\begin{alertblock}{Information content}
	Which of those partial versions of the full covariance matrix carries the most information in a cost-effective scenario?
\end{alertblock}
\end{frame}



\section{Fusion of evidences}
%
%\begin{frame}[fragile]{Evidence Fusion}
%\begin{alertblock}{Average Fusion}
%\pgfdeclarelayer{background}
%\pgfdeclarelayer{foreground}
%\pgfsetlayers{background,main,foreground}
%%
%\pgfdeclarelayer{background}
%\pgfdeclarelayer{foreground}
%\pgfsetlayers{background,main,foreground}
%\tikzstyle{sensor}=[draw, fill=blue!20, text width=5em, 
%    text centered, minimum height=2.5em,drop shadow]
%\tikzstyle{ann} = [above, text width=5em, text centered]
%\tikzstyle{wa} = [sensor, text width=10em, fill=red!20, 
%    minimum height=6em, rounded corners, drop shadow]
%\tikzstyle{sc} = [sensor, text width=13em, fill=red!20, 
%    minimum height=10em, rounded corners, drop shadow]
%\def\blockdist{2.3}
%\def\edgedist{2.5}
%	\begin{figure}[htb!]
%\centering
%\begin{tikzpicture}
%	\node (wa) [wa]  {$IF=\frac{1}{nc}\sum_{i=1}^{nc}IE_i$};
%	\path (wa.west)+(-3.2,1.5) node (e1) [sensor] {$IE_1$};
%    \path (wa.west)+(-3.2,0.5) node (e2)[sensor] {$IE_2$};
%    \path (wa.west)+(-3.2,-1.0) node (dots)[ann] {$\vdots$}; 
%    \path (wa.west)+(-3.2,-2.0) node (e3)[sensor] {$IE_{nc}$};    
%%
%    \path [draw, ->] (e1.east) -- node [above] {} 
%        (wa.160) ;
%    \path [draw, ->] (e2.east) -- node [above] {} 
%        (wa.180);
%    \path [draw, ->] (e3.east) -- node [above] {} 
%        (wa.200);
%%  
%%    \begin{pgfonlayer}{background}
%%        \path (e1.west |- e1.north)+(-0.5, 0.5) node (a) {};
%%        \path (wa.south -| wa.east)+(+1.0,-2.0) node (b) {};
%%       %   
%%        \path[fill=yellow!20,rounded corners, draw=black!50, dashed]
%%            (a) rectangle (b);           
%%       %     
%%    \end{pgfonlayer}   
%\end{tikzpicture}
%	\caption{Average Fusion.}
%\label{fig5}
%\end{figure}
%\end{alertblock}
%\end{frame}
%
%
%
%%\begin{frame}[fragile]{PolSAR Image}
%%\begin{alertblock}{Average Fusion}
%%\begin{itemize}
%%	\item 
%%	\begin{equation}
%%	IF(x,y)=\frac{1}{nc}\sum_{i=1}^{nc}IE_i(x,y),
%%\end{equation} 	
%%\end{itemize}
%%\end{alertblock}
%%\end{frame}
%
%\begin{frame}[fragile]{Evidence Fusion}
%\begin{alertblock}{PCA Fusion}
%\pgfdeclarelayer{background}
%\pgfdeclarelayer{foreground}
%\pgfsetlayers{background,main,foreground}
%\tikzstyle{sensor}=[draw, fill=blue!20, text width=5em, 
%    text centered, minimum height=2.5em,drop shadow]
%\tikzstyle{ann} = [above, text width=5em, text centered]
%\tikzstyle{wa} = [sensor, text width=7em, fill=red!20, 
%    minimum height=3em, rounded corners, drop shadow]
%\tikzstyle{sc} = [sensor, text width=10em, fill=red!20, 
%    minimum height=7em, rounded corners, drop shadow]
%\def\blockdist{2.3}
%\def\edgedist{2.5}
%	\begin{figure}[htb!]
%\begin{tikzpicture}
%	\path (wa.west)+(-2.0,0.0) node (pcanode) [wa] {$\text{PCA}$};
%	\path (wa.west)+(-6.2,1.5) node (e1) [sensor] {$IE_1$};
%    \path (wa.west)+(-6.2,0.5) node (e2)[sensor] {$IE_2$};
%    \path (wa.west)+(-6.2,-1.0) node (dots)[ann] {$\vdots$}; 
%    \path (wa.west)+(-6.2,-2.0) node (e3)[sensor] {$IE_N$};    
%    \path (wa.west)+(2.0,0.0) node (pcanodefus) [sc] {$V_m=\max{V(i)}$
%                                                      \\$p=V_m(i)/||V_m||$
%                                                      \\$IF=\sum_{i=1}^{nc}p_iIE_i$};
%    \path [draw, ->] (e1.east) -- node [above] {} 
%        (pcanode.160) ;
%    \path [draw, ->] (e2.east) -- node [above] {} 
%        (pcanode.180);
%    \path [draw, ->] (e3.east) -- node [above] {} 
%        (pcanode.200);
%        %
%    \path [draw, ->] (pcanode.east) -- node [above] {} 
%        (pcanodefus.180) ;
%%  
%%    \begin{pgfonlayer}{background}
%%        \path (e1.west |- e1.north)+(-0.5,0.3) node (a) {};
%%        \path (wa.south -| wa.east)+(+0.5,-0.3) node (b) {};
%%        \path (m3.east |- m3.east)+(+0.5,-0.75) node (c) {};
%       %   
%%        \path[fill=yellow!20,rounded corners, draw=black!50, dashed]
%%            (a) rectangle (c);           
%%       %     
%%    \end{pgfonlayer}
%   
%\end{tikzpicture}
%	\caption{PCA Fusion.}
%\label{fig6}
%\end{figure}
%\end{alertblock}
%\end{frame}
%
%\begin{frame}[fragile]{Evidence Fusion}
%\begin{alertblock}{Stationary wavelet transform -- SWT Fusion} 
%\pgfdeclarelayer{background}
%\pgfdeclarelayer{foreground}
%\pgfsetlayers{background,main,foreground}
%\tikzstyle{sensor}=[draw, fill=blue!20, text width=5em, 
%    text centered, minimum height=2.5em,drop shadow]
%\tikzstyle{ann} = [above, text width=5em, text centered]
%\tikzstyle{wa} = [sensor, text width=7em, fill=red!20, 
%    minimum height=3em, rounded corners, drop shadow]
%\tikzstyle{sc} = [sensor, text width=10em, fill=red!20, 
%    minimum height=7em, rounded corners, drop shadow]
%\def\blockdist{2.3}
%\def\edgedist{2.5}
%	\begin{figure}[htb!]
%\begin{tikzpicture}
%	\path (wa.west)+(-3.0,1.5) node (swtnode1) [sensor] {$\text{Coef SWT}_1$};
%	\path (wa.west)+(-3.0,0.5) node (swtnode2) [sensor] {$\text{Coef SWT}_2$};
%	\path (wa.west)+(-3.0,-1.0) node (dots)[ann] {$\vdots$}; 
%    \path (wa.west)+(-3.0,-2.0) node (swtnode3)[sensor] {$\text{Coef SWT}_N$};  
%	
%	
%	\path (wa.west)+(-6.2,1.5) node (e1) [sensor] {$IE_1$};
%    \path (wa.west)+(-6.2,0.5) node (e2)[sensor] {$IE_2$};
%    \path (wa.west)+(-6.2,-1.0) node (dots)[ann] {$\vdots$}; 
%    \path (wa.west)+(-6.2,-2.0) node (e3)[sensor] {$IE_N$};    
%    \path (wa.west)+(1.0,1.0) node (swtnodefus) [wa] {Fused wavalets\\
%                                                       coefficient};
%                                                       
%    \path (wa.west)+(1.0,-2.5) node (imagefus) [wa] {Image fusion};
%    \path [draw, ->] (e1.east) -- node [above] {W} 
%        (swtnode1.180) ;
%    \path [draw, ->] (e2.east) -- node [above] {W} 
%        (swtnode2.180);
%    \path [draw, ->] (e3.east) -- node [above] {W} 
%        (swtnode3.180);
%%
%    \path [draw, ->] (swtnode1.east) -- node [above] {} 
%        (swtnodefus.160) ;
%    \path [draw, ->] (swtnode2.east) -- node [above] {} 
%        (swtnodefus.180);
%    \path [draw, ->] (swtnode3.east) -- node [above] {} 
%        (swtnodefus.200);      
%    \path [draw, ->] (swtnodefus.south) -- node [right] {$W^{-1}$}      
%        (imagefus.north);        
%        
%%        %
%%    \path [draw, ->] (pcanode.east) -- node [above] {} 
%%        (pcanodefus.180) ;
%%  
%%    \begin{pgfonlayer}{background}
%%        \path (e1.west |- e1.north)+(-0.5,0.3) node (a) {};
%%        \path (wa.south -| wa.east)+(+0.5,-0.3) node (b) {};
%%        \path (m3.east |- m3.east)+(+0.5,-0.75) node (c) {};
%       %   
%%        \path[fill=yellow!20,rounded corners, draw=black!50, dashed]
%%            (a) rectangle (c);           
%%       %     
%%    \end{pgfonlayer}
%   
%\end{tikzpicture}
%	\caption{SWT Fusion.}
%\label{fig7}
%\end{figure}
%\begin{itemize}
%\vspace{-0.8cm}
%\item $W$ is wavelet transformed.
%\end{itemize}
%\end{alertblock}
%\end{frame}
%
%
%\begin{frame}[fragile]{Conclusion}
%\begin{alertblock}{Discrete wavelet transform -- DWT Fusion}
%\pgfdeclarelayer{background}
%\pgfdeclarelayer{foreground}
%\pgfsetlayers{background,main,foreground}
%\tikzstyle{sensor}=[draw, fill=blue!20, text width=5em, 
%    text centered, minimum height=2.5em,drop shadow]
%\tikzstyle{ann} = [above, text width=5em, text centered]
%\tikzstyle{wa} = [sensor, text width=7em, fill=red!20, 
%    minimum height=3em, rounded corners, drop shadow]
%\tikzstyle{sc} = [sensor, text width=10em, fill=red!20, 
%    minimum height=7em, rounded corners, drop shadow]
%\def\blockdist{2.3}
%\def\edgedist{2.5}
%	\begin{figure}[htb!]
%\begin{tikzpicture}
%	\path (wa.west)+(-3.0,1.5) node (swtnode1) [sensor] {$\text{Coef DWT}_1$};
%	\path (wa.west)+(-3.0,0.5) node (swtnode2) [sensor] {$\text{Coef DWT}_2$};
%	\path (wa.west)+(-3.0,-1.0) node (dots)[ann] {$\vdots$}; 
%    \path (wa.west)+(-3.0,-2.0) node (swtnode3)[sensor] {$\text{Coef DWT}_N$};  
%	
%	
%	\path (wa.west)+(-6.2,1.5) node (e1) [sensor] {$IE_1$};
%    \path (wa.west)+(-6.2,0.5) node (e2)[sensor] {$IE_2$};
%    \path (wa.west)+(-6.2,-1.0) node (dots)[ann] {$\vdots$}; 
%    \path (wa.west)+(-6.2,-2.0) node (e3)[sensor] {$IE_N$};    
%    \path (wa.west)+(1.0,1.0) node (swtnodefus) [wa] {Fused wavalets\\
%                                                       coefficient};
%                                                       
%    \path (wa.west)+(1.0,-2.5) node (imagefus) [wa] {Image fusion};
%    \path [draw, ->] (e1.east) -- node [above] {W} 
%        (swtnode1.180) ;
%    \path [draw, ->] (e2.east) -- node [above] {W} 
%        (swtnode2.180);
%    \path [draw, ->] (e3.east) -- node [above] {W} 
%        (swtnode3.180);
%%
%    \path [draw, ->] (swtnode1.east) -- node [above] {} 
%        (swtnodefus.160) ;
%    \path [draw, ->] (swtnode2.east) -- node [above] {} 
%        (swtnodefus.180);
%    \path [draw, ->] (swtnode3.east) -- node [above] {} 
%        (swtnodefus.200);      
%    \path [draw, ->] (swtnodefus.south) -- node [right] {$W^{-1}$}      
%        (imagefus.north);        
%        
%%        %
%%    \path [draw, ->] (pcanode.east) -- node [above] {} 
%%        (pcanodefus.180) ;
%%  
%%    \begin{pgfonlayer}{background}
%%        \path (e1.west |- e1.north)+(-0.5,0.3) node (a) {};
%%        \path (wa.south -| wa.east)+(+0.5,-0.3) node (b) {};
%%        \path (m3.east |- m3.east)+(+0.5,-0.75) node (c) {};
%       %   
%%        \path[fill=yellow!20,rounded corners, draw=black!50, dashed]
%%            (a) rectangle (c);           
%%       %     
%%    \end{pgfonlayer}
%   
%\end{tikzpicture}
%	\caption{DWT Fusion.}
%\label{fig7}
%\end{figure}
%\end{alertblock}
%\end{frame}
%
%\begin{frame}[fragile]{Evidence Fusion}
%\begin{alertblock}{ROC statistics Fusion}
%\begin{itemize}
%\item Part I
%\pgfdeclarelayer{background}
%\pgfdeclarelayer{foreground}
%\pgfsetlayers{background,main,foreground}
%\tikzstyle{sensor}=[draw, fill=blue!20, text width=5em, 
%    text centered, minimum height=2.5em,drop shadow]
%\tikzstyle{ann} = [above, text width=5em, text centered]
%\tikzstyle{wa} = [sensor, text width=7em, fill=red!20, 
%    minimum height=5em, rounded corners, drop shadow]
%\tikzstyle{sc} = [sensor, text width=13em, fill=red!20, 
%    minimum height=10em, rounded corners, drop shadow]
%\def\blockdist{2.3}
%\def\edgedist{2.5}
%	\begin{figure}[htb!]
%\begin{tikzpicture}
%\path (wa.west)+(-3.0,0.0) node (pcanode) [wa] {$V=\sum_{i=1}^{N}IE_i$};
%	\path (wa.west)+(-7.2,1.5) node (e1) [sensor] {$IE_1$};
%    \path (wa.west)+(-7.2,0.5) node (e2)[sensor] {$IE_2$};
%    \path (wa.west)+(-7.2,-1.0) node (dots)[ann] {$\vdots$}; 
%    \path (wa.west)+(-7.2,-2.0) node (e3)[sensor] {$IE_N$};    
%    %\path (wa.west)+(2.0,0.0) node (pcanodefus) [sc] {$V_m=\max{V(i)}$
%    %                                                  \\$p=V_m(i)/||V_m||$
%    %                                                  \\$IF=\sum_{i=1}^{nc}p_iIE_i$};
%    \path [draw, ->] (e1.east) -- node [above] {} 
%        (pcanode.160) ;
%    \path [draw, ->] (e2.east) -- node [above] {} 
%        (pcanode.180);
%    \path [draw, ->] (e3.east) -- node [above] {} 
%        (pcanode.200);
%        %
%	%\node (wa) [wa]  {$V=\sum_{i=1}^{N}IE_i$};
%	%\path (wa.west)+(-3.2,1.5) node (e1) [sensor] {$IE_1$};
%    %\path (wa.west)+(-3.2,0.5) node (e2)[sensor] {$IE_2$};
%    %\path (wa.west)+(-3.2,-1.0) node (dots)[ann] {$\vdots$}; 
%    %\path (wa.west)+(-3.2,-2.0) node (e3)[sensor] {$IE_N$};    
%%%   
%    \path (pcanode.east)+(3.2,1.5) node (m1) [sensor] {$M_1$};
%    \path (pcanode.east)+(3.2,0.5) node (m2) [sensor] {$M_2$};
%    \path (pcanode.east)+(3.2,-1.0) node (dots)[ann] {$\vdots$}; 
%    \path (pcanode.east)+(3.2,-2.0) node (m3) [sensor] {$M_N$};
%%%
%    %\path [draw, ->] (e1.east) -- node [above] {} 
%    %    (wa.160) ;
%    %\path [draw, ->] (e2.east) -- node [above] {} 
%    %    (wa.180);
%    %\path [draw, ->] (e3.east) -- node [above] {} 
%    %    (wa.200);
%	\path [draw, ->] (pcanode.east) -- node [above] {\tiny{$CT_1$}} 
%        (m1.west);
%	\path [draw, ->] (pcanode.east) -- node [above] {\tiny{$CT_2$}} 
%        (m2.west);
%	\path [draw, ->] (pcanode.east) -- node [right] {\tiny{$CT_N$}} 
%        (m3.west);
%%               
%%%    \path (wa.south) +(0,-\blockdist) node (asrs) {Estrutura geral da fusão de evidência proposta};
%%  
%%    \begin{pgfonlayer}{background}
%%        \path (e1.west |- e1.north)+(-0.5,0.3) node (a) {};
%%        \path (wa.south -| wa.east)+(+0.5,-0.3) node (b) {};
%%        \path (m3.east |- m3.east)+(+0.5,-0.75) node (c) {};
%       %   
%%        \path[fill=yellow!20,rounded corners, draw=black!50, dashed]
%%            (a) rectangle (c);           
%%       %     
%%    \end{pgfonlayer}
%   
%\end{tikzpicture}
%	\caption{Fusion based in ROC statistics - Part I.}
%\label{fig8}
%\end{figure}
%\item $CT_i$ is a threshold.
%\end{itemize}
%\end{alertblock}
%\end{frame}
%
%
%\begin{frame}[fragile]{Evidence Fusion}
%\begin{alertblock}{ROC statistics Fusion}
%\begin{itemize}
%\item Part II - for each $M_j$
%\tikzstyle{sensor}=[draw, fill=blue!20, text width=2.5em, 
%    text centered, minimum height=2em,drop shadow]
%\tikzstyle{ann} = [above, text width=5em, text centered]
%\tikzstyle{wa} = [sensor, text width=2em, fill=red!20, 
%    minimum height=2em, rounded corners, drop shadow]
%\tikzstyle{wa1} = [sensor, text width=2em, fill=red!20, 
%    minimum height=2em, rounded corners, drop shadow]
%\begin{figure}[hbt]
%\begin{tikzpicture}
%\node[wa] (wa) at (0.0,0.0) {$M_j$};
%\node[wa1] (wa1) at (4.0,0.0) {$\overline{TP}_j$};
%
%    \path (wa.west)+(2.5,1.5) node (e1_1) [sensor] {$TP_1$};
%    \path (wa.west)+(2.5,0.5) node (e2_1)[sensor] {$TP_2$};
%    \path (wa.west)+(2.5,-1.0) node (dots)[ann] {$\vdots$}; 
%    \path (wa.west)+(2.5,-2.0) node (e3_1)[sensor] {$TP_N$};    
%%
%	\path [draw, ->] (wa.east) -- node [left] {\tiny{$\overline{\cap E_1}$}} 
%        (e1_1.180) ;
%	\path [draw, ->] (wa.east) -- node [below] {\tiny{$\overline{\cap E_2}$}} 
%        (e2_1.180);
%	\path [draw, ->] (wa.east) -- node [right] {\tiny{$\overline{\cap E_3}$}} 
%        (e3_1.180);
%	\path [draw, ->] (e1_1.east) -- node [right] {\tiny{$+$}} 
%        (wa1.160);
%	\path [draw, ->] (e2_1.east) -- node [above] {\tiny{$+$}} 
%        (wa1.180);
%	\path [draw, ->] (e3_1.east) -- node [right] {\tiny{$+$}} 
%        (wa1.200);
%  
% %   \begin{pgfonlayer}{background}
% %       \path (wa_1.west |- wa_1.north)+(5.25,1.75) node (a) {};
% %       \path (e1_1.south -| e1_1.north)+(-2.75,-3.75) node (b) {};
% %       %\path (wa1.east |- wa1.east)+(+4.0,-0.5) node (c) {};
% %      %   
% %       \path[fill=yellow!20,rounded corners, draw=black!50, dashed]
% %           (a) rectangle (b);           
% %      %     
% %   \end{pgfonlayer}
%    
%\end{tikzpicture}
%\caption{ROC Fusion for each $j$. It is true to $\overline{TN}_j$,$\overline{FP}_j$ and, $\overline{FN}_j$. }
%\label{fig9}
%\end{figure}
%\item To generate the confusion matrix, and calculate the ROC statistics.
%\end{itemize}
%\end{alertblock}
%\end{frame}

\begin{frame}{Strategies for fusion}
	For each ray
\begin{description}
	\item[Input:] \mbox{}
		\begin{itemize}
			\item As many points as $\text{feature}\times\text{algorithm}$ are available.
			\item \alert{A measure of their quality}
		\end{itemize}
	\item[Output:] \mbox{}
		\begin{itemize}
			\item A single edge estimate
			\item \alert{A measure of confidence}
		\end{itemize}
\end{description}

Currently available: Mean, PCA, ROC, SVD, and two wavelets-based techniques.
\end{frame}

\begin{frame}{Image and ROI}
\centering
	\includegraphics[width=.7\linewidth]{flevoland_radial_4_look_black}
\end{frame}


\begin{frame}{Evidences in Intensity channels}
	\centering
     \includegraphics[width=0.46\textwidth]{flevoland_100_point_hh_crop}    \quad
     \includegraphics[width=0.46\linewidth]{flevoland_100_point_hv_crop}
     \quad
     \includegraphics[width=0.46\linewidth]{flevoland_100_point_vv_crop}
\end{frame}

\begin{frame}{Fusion}
		\begin{figure}[hbt]
			\centering
			\subfloat[Average\label{fusion_met:a}]{%
				%\includegraphics[width=0.18\textwidth]{example-image-a}
				\includegraphics[width=0.18\linewidth]{flevoland_fus_media_param_L_mu_14_pixel_crop}
			}
			\subfloat[MR-DWT\label{fusion_met:b}]{%
				\includegraphics[width=0.18\linewidth]{flevoland_fus_dwt_param_L_mu_14_pixel_crop}
			}
			\subfloat[PCA \label{fusion_met:c}]{%
				%\includegraphics[width=0.18\textwidth]{example-image-a}
				\includegraphics[width=0.18\linewidth]{flevoland_fus_pca_param_L_mu_14_pixel_crop}       
			}\\
			\subfloat[ROC\label{fusion_met:d}]{%
				\includegraphics[width=0.18\linewidth]{flevoland_fus_roc_param_L_mu_14_pixel_crop}
			}
			\subfloat[MR-SWT\label{fusion_met:e}]{%
				\includegraphics[width=0.18\linewidth]{flevoland_fus_swt_param_L_mu_14_pixel_crop}
			}
			\subfloat[MR-SVD\label{fusion_met:f}]{%
				\includegraphics[width=0.18\linewidth]{flevoland_fus_svd_param_L_mu_14_pixel_crop}
			}
			\caption{}
			\label{fusion_met}
		\end{figure}

\end{frame}

\begin{frame}{Image}
		\begin{figure}[hbt]
			\centering
			\subfloat[Image and rays. \label{flevoland_radial_25}]{%  
				\includegraphics[width=.48\linewidth]{flevoland_r3_radial_crop}}
			\subfloat[Ground reference\label{gt_flevoland_r3_crop}]{%
				\includegraphics[width=.48\linewidth]{gt_flevoland_r3_crop}
			}
			\caption{Flevoland image in Pauli decomposition, and ground reference}
			\label{roi_gt_2}
		\end{figure}
\end{frame}

\begin{frame}{Evidences}
	\begin{figure}[hbt]
	\centering
	\subfloat[Channel $\text{hh}$ \label{evidencias_flev_hh_hv_vv:a}]{%
		\includegraphics[width=0.32\linewidth]{evid_real_flev_hh_param_L_mu_25_pixel_r3_crop}
	}
	\subfloat[Channel $\text{hv}$ \label{evidencias_flev_hh_hv_vv:b}]{%
		\includegraphics[width=0.32\linewidth]{evid_real_flev_hv_param_L_mu_25_pixel_r3_crop}
	}
	\subfloat[Channel $\text{vv}$ \label{evidencias_flev_hh_hv_vv:c}]{%
		\includegraphics[width=0.32\linewidth]{evid_real_flev_vv_param_L_mu_25_pixel_r3_crop}
	}
	\caption{Edges evidences from the three intensity channels, Flevoland image}
	\label{evidencias_flev_hh_hv_vv} 
\end{figure}
\end{frame}

\begin{frame}{Fusion}
		\begin{figure}[hbt]
			\centering
			%	\subfloat[Average fusion\label{fusion_flev_met:a}]{%
			%		%\includegraphics[width=0.2\textwidth]{example-image-a}
			%		\includegraphics[width=0.32\linewidth]{flev_r3_fus_media_param_L_mu_25_pixel_crop}
			%	}
			%	\subfloat[DWT fusion\label{fusion_flev_met:b}]{%
			%		\includegraphics[width=0.32\linewidth]{flev_r3_fus_dwt_param_L_mu_25_pixel_crop}
			%	}
			\subfloat[PCA fusion \label{fusion_flev_met:c}]{%
				%\includegraphics[width=0.2\textwidth]{example-image-a}
				\includegraphics[width=0.48\linewidth]{flev_r3_fus_pca_param_L_mu_25_pixel_crop}       
			}
			%	\subfloat[ROC fusion\label{fusion_flev_met:d}]{%
			%		\includegraphics[width=0.32\linewidth]{flev_r3_fus_roc_param_L_mu_25_pixel_crop}
			%	}
			%	\subfloat[MR-SWT fusion\label{fusion_flev_met:e}]{%
			%		\includegraphics[width=0.3\linewidth]{flev_r3_fus_swt_param_L_mu_25_pixel_crop}
			%	}
			\subfloat[MR-SVD fusion\label{fusion_flev_met:f}]{%
				\includegraphics[width=0.48\linewidth]{flev_r3_fus_svd_param_L_mu_25_pixel_crop}
			}
			\caption{Two best fusion results in the Flevoland image}
			\label{fusion_flev_met}
		\end{figure}
\end{frame}



%\begin{frame}[fragile]{Results}
%\begin{alertblock}{Results}
%	\begin{figure}[hbt]
%	\centering
%	\subfloat[Channel $\text{hh}$ \label{evidencias_sf_hh_hv_vv:a}]{%
%		\includegraphics[width=0.32\linewidth]{evid_real_sf_1_param_L_mu_25_pixel_r1_crop}
%	}
%	\subfloat[Channel $\text{hv}$ \label{evidencias_sf_hh_hv_vv:b}]{%
%		\includegraphics[width=0.32\linewidth]{evid_real_sf_2_param_L_mu_25_pixel_r1_crop}
%	}
%	\subfloat[Channel $\text{vv}$ \label{evidencias_sf_hh_hv_vv:c}]{%
%		\includegraphics[width=0.333\linewidth]{evid_real_sf_3_param_L_mu_25_pixel_r1_crop}
%	}
%	\caption{Edges evidences from the three intensity channels to San Francisco}
%	\label{evidencias_sf_hh_hv_vv} 
%\end{figure}
%\end{alertblock}
%\end{frame}
%
%
%




%\begin{frame}[fragile]{Results - Fusion}
%\begin{alertblock}{Results}
%	\begin{figure}[hbt]
%	\centering
%%	\subfloat[Average fusion\label{fusion_sf_met:a}]{%
%%		%\includegraphics[width=0.2\textwidth]{example-image-a}
%%		\includegraphics[width=0.32\linewidth]{sf_fus_media_param_L_mu_25_pixel_crop}
%%	}
%%	\subfloat[DWT fusion\label{fusion_sf_met:b}]{%
%%		\includegraphics[width=0.32\linewidth]{sf_fus_dwt_param_L_mu_25_pixel_crop}
%%	}
%	\subfloat[PCA fusion \label{fusion_sf_met:c}]{%
%		%\includegraphics[width=0.2\textwidth]{example-image-a}
%		\includegraphics[width=0.48\linewidth]{sf_fus_pca_param_L_mu_25_pixel_crop}       
%	}
%%	\subfloat[ROC fusion\label{fusion_sf_met:d}]{%
%%		\includegraphics[width=0.32\linewidth]{sf_fus_roc_param_L_mu_25_pixel_crop}
%%	}
%%	\subfloat[MR-SWT fusion\label{fusion_sf_met:e}]{%
%%		\includegraphics[width=0.32\linewidth]{sf_fus_swt_param_L_mu_25_pixel_crop}
%%	}
%	\subfloat[MR-SVD fusion\label{fusion_sf_met:f}]{%
%		\includegraphics[width=0.48\linewidth]{sf_fus_svd_param_L_mu_25_pixel_crop}
%	}
%	\caption{Two best fusion results in the San Francisco image}
%	\label{fusion_sf_met}
%\end{figure}
%
%\end{alertblock}
%\end{frame}

\section{Assessment}


\begin{frame}{Assessment}
\begin{figure}[hbt]
	\centering
	\includegraphics[width=.6\linewidth]{metricas_6_fusao_flevoland}
	\caption{Probability of detecting the edge by the fusion methods.}
	\label{probability_edge_detc}
\end{figure}
\end{frame}

\begin{frame}{Assessment}
\begin{alertblock}{Run time}	
\begin{table}[hbt]
	\centering
	\caption{Processing times (fusion method).}\label{metrica_de_tempo}
	\begin{tabular}{@{}lrrrrrr@{}} \toprule
		Method       & Aver.   &   PCA      &  MR-DWT  & MR-SWT &  ROC  &  MR-SVD \\ \midrule
		Time (s)      & 0.01      & 0.02       &  0.08 & 0.18      &  0.40       & 1.11  \\
		Rel. time     & 1.00      & 2.19       &  9.25 & 21.05     &  46.59      & 129.57  \\ \bottomrule
	\end{tabular}
\end{table}
\end{alertblock}
\end{frame}

\begin{frame}{Question}
We are able to
\begin{itemize}[<+->]
	\item \alert{estimate} the position of the edge with data of type $d\in\{I_1, I_2, I_3, (I_1, I_2), \dots\}$ and algorithm of type $a\in\{\mathcal L, d^\phi_h, d_{\text{G}}, \dots\}$;
	\item \alert{fuse} the edge evidences with $f\in\{F_1, F_2, \dots\}$.
\end{itemize}
\onslide<+->{Can we obtain measures of the quality of the estimators to improve the fusion?}
\end{frame}

\section{Conclusions}

\begin{frame}{Conclusions}
\begin{itemize}
	\item Evidences from intensity channels are complementary and, therefore, relevant and suitable for fusion
	\item SVD fusion and PCA fusion have better performance taking into account detection accuracy, run time and, robustness.
\end{itemize}

\begin{alertblock}{Promising directions}	
\begin{itemize}
	\item Increase the number of channels to improve the fusion
	\item Investigate new fusion methods
	\item Use other distances
	\item Enhance with measures of performance
\end{itemize}
\end{alertblock}
\end{frame}

\begin{frame}[allowframebreaks]
\nocite{FusionofEvidencesinIntensitiesChannelsforEdgeDetectioninPolSARImages,Gambini:StatisticsComputing,NonparametricEdgeDetectionSpeckledImagery,EdgeDetectionDistancesEntropiesJSTARS,OpticalBasedSAREdgeDetectionKNOWSys,GeodesicDistanceGI0JSTARS,DistanceBasedEdgeDetectiononSyntheticApertureRadarImagery}	
\bibliographystyle{agsm}
\bibliography{acfrery}
\end{frame}

\begin{frame}
Thanks for your attention!

https://people.wgtn.ac.nz/alejandro.frery\\
alejandro.frery@vuw.ac.nz
\end{frame}

\end{document}
