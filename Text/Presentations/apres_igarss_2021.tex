\documentclass[10pt]{beamer}

%\usetheme{metropolis}
%\usetheme{AnnArbor}
\usetheme[progressbar=frametitle]{metropolis}
%\usecolortheme{beaver}
\usepackage{appendixnumberbeamer}

\usepackage{booktabs}
\usepackage[scale=2]{ccicons}

\usepackage{pgfplots}
\usepgfplotslibrary{dateplot}

%\usepackage[brazil]{babel}
\usepackage{xspace}
\usepackage{algorithm2e} % AAB inserido
\newcommand{\themename}{\textbf{\textsc{metropolis}}\xspace}
%\usepackage[brazil]{babel}  % AAB
\usepackage{bibentry}       % AAB
\usepackage{amsmath, bm}    % AAB
\usepackage{tikz}           % AAB
\usetikzlibrary{chains,fit,shapes} % AAB 
\usepackage[detect-weight=true, binary-units=true]{siunitx}
%\usepackage{siunitx}        % AAB inserido    
\usetikzlibrary{shapes,arrows,shadows}  % AAB inserido
%\usepackage[utf8]{inputenc}                  % AAB inserido
                                              % AAB inserido
\usepackage[caption=false,font=normalsize,labelfont=sf,textfont=sf]{subfig}
%\usepackage[caption=false,font=footnotesize]{subfig}
%
%\addto\captionsportuguese{
%\renewcommand{\figurename}{Fig.}
%\renewcommand{\tablename}{Tab.}
%}% AAB Inserido
\DeclareMathOperator{\traco}{tr} %AAB
\graphicspath{{../Dissertacao/figuras/}}        % AAB - caminho das figuras
%\graphicspath{{../Images/PDF/}}                 % AAB - caminho das figuras (recomendável) 


\title{Quantifying the influence of intensity channels from PolSAR images for edge detection on information fusion}
\subtitle{IGARSS -- 2021}
\date{}
\author{Dr. Anderson Adaime de Borba - IBMEC -- SP -- BR\\
        Dr. Mauricio Marengoni -- UFMG -- BR\\
        Dr. Alejandro Frery - Victoria University of Wellington -- NZ} 
\institute{SP-JUL-2021}
\begin{document}
\maketitle
%
\begin{frame}[fragile]{Inter-institutional cooperation projects}
\centering{
\includegraphics[width=6cm]{logo_uni_wellington}\\
\includegraphics[width=6cm]{logo_ufmg}\\
\includegraphics[width=5cm]{Marca_Ibmec}\\}
\end{frame}
%
%\begin{frame}{Índice}
%  \setbeamertemplate{section in toc}[sections numbered]
%  \tableofcontents[hideallsubsections]
%\end{frame}
%	
%\begin{frame}[fragile]{Database (PolSAR Image)}
%\begin{alertblock}{PolSAR Data}
%\begin{table}[hbt]
%\scriptsize
%	\centering
%	\caption{Information from the PolSAR system.}
%\begin{tabular}{@{}lccc@{}} \toprule
%	Polarization & hh  & hv & vv \\ \midrule
%	hh & $\sigma_\text{hh}$ & $\Re\big(\text{Cov}(\text{hh}, \text{hv})\big) + \Im\big(\text{Cov}(\text{hh}, \text{hv})\big)\hat{\jmath}$  & $\Re\big(\text{Cov}(\text{hh}, \text{vv})\big) + \Im\big(\text{Cov}(\text{hh}, \text{vv})\big)\hat{\jmath}$\\ 
%	hv &- &$\sigma_\text{hv}$ & $ \Re\big(\text{Cov}(\text{hv}, \text{vv})\big)+ \Im\big(\text{Cov}(\text{hv}, \text{vv})\big)\hat{\jmath}$\\ 
%	vv &- & -&$\sigma_\text{vv}$ \\ \bottomrule 
%\end{tabular}\label{tab:sistema_polsar}
%\end{table}
%\begin{table}[hbt]	
%	\centering
%	\caption{PolSAR intensity channels.}
%\begin{tabular}{@{}lcc@{}} \toprule
%	 $\text{C}_1$ &$\text{C}_2$&$\text{C}_3$ \\ \midrule
%	$\sigma_\text{hh}$&$\sigma_\text{hv}$&$\sigma_\text{vv}$\\ \bottomrule
%\end{tabular}\label{tab:canais}
%\end{table}
%\end{alertblock}
%\end{frame}
%
%\begin{frame}[fragile]{Image channels.}
%   \begin{figure}[hbt]
%\minipage{0.35\textwidth}
%  \includegraphics[width=\linewidth]{sf_hh.pdf}
%\endminipage
%\minipage{0.35\textwidth}
%	\includegraphics[width=\linewidth]{sf_vh.pdf}
%\endminipage
%\centering
%\minipage{0.35\textwidth}
%	\includegraphics[width=\linewidth]{sf_vv.pdf}
%\endminipage
%	\caption{PolSAR images with polarization hh, hv e vv.}\label{fig:sf_hh_hv_vv}
%\end{figure}
%\end{frame}
%
\begin{frame}[fragile]{Remote Sensing of Natural Hazards in Latin America II}
\begin{alertblock}{Application:}
\begin{description}
\item[-] Edge detection to estimate area with collapse of debris 
\item[-] Edge detection to estimate rapid flooding.  
\end{description}
\end{alertblock}
\begin{alertblock}{PolSAR images apply in:}
\begin{description}
\item[-] At night, 
\item[-] Under severe weather conditions.
\end{description}
\end{alertblock}
\begin{alertblock}{Objective of the work:}
\begin{description}
\item[-] Quantifying the influence of intensity channels from PolSAR images for edge detection on information fusion.
\end{description}
\end{alertblock}
\end{frame}
%
%
%\section{Modelagem Estatística}
\begin{frame}[fragile]{Statistical Modeling}
\begin{alertblock}{Statistical modeling for PolSAR data (1 - Look)}
\begin{itemize}
\item $\mathbf{S}$ is the complex scattering matrix :
\begin{equation}
\mathbf{S} = \left[
\begin{array}{cc}
	S_\text{hh}   & S_\text{hv}   \\
	S_\text{vv}   & S_\text{vv}   
\end{array}
\right].
\end{equation}\label{eq_01}
\item And, if the medium of propagation of waves is reciprocal ($S_\text{hv}=S_\text{vh}$) results in a vector, 
$$\mathbf{s}=[S_\text{hh},S_\text{hv},S_{\text{vv}}]^T.$$
\end{itemize}
\end{alertblock}
\end{frame}
%
\begin{frame}[fragile]{Statistical Modeling}
\begin{alertblock}{Statistical modeling for PolSAR data}
\begin{itemize}
\item The estimated sample covariance matrix:
\begin{equation}
    \mathbf{Z}=\frac{1}{\text{L}}\sum_{\ell=1}^{\text{L}} {\mathbf{s}_\ell}{\mathbf{s}_\ell}^\text{H},
    \label{eq_03}
\end{equation}
\begin{description}
      \item[-] Where,
      \item[-] $\mathbf{s}_\ell$  is an one look vector, $\ell = 1, \dots, \text{L}$ ;
      \item[-] L independent samples distributed as $\mathbf{s}$ . 
      \item[-] $\text{H}$ denotes the conjugate complex number, 
\end{description}
\end{itemize}
\end{alertblock}
\begin{alertblock}{Multilook -- (L  Looks) }
\begin{description}
\item[-] Multi-look is a good tool to mitigate of speckle noise.
\end{description}
\end{alertblock}
\end{frame}
%
\begin{frame}[fragile]{Statistical Modeling}
\begin{alertblock}{The marginal distribution for intensity channel -- Multi-look}
\begin{description}
\item
\begin{equation}
	f_{Z}(z;\mu,\text{L})=\frac{\text{L}^\text{L}}{\Gamma(\text{L})\mu^{\text{L}}} z^{\text{L}-1} \exp\left\{-\frac{\text{L}}{\mu}z\right\}, 
\label{pdf_gauss_univ}
\end{equation}
\item where, $\mu>0$ and $\text{L}>0$ are parameters.
\item And, we apply natural logarithm in both sides of the equation \eqref{pdf_gauss_univ}, thus, we will have the equation:
\begin{equation}\label{func_log_univ_gaussiana}
	\ln f_{Z}(z;\mu,\text{L})=\text{L}\ln\frac{\text{L}}{\mu}-\ln\Gamma(\text{L})+(\text{L}-1)\ln z - \frac{\text{L}}{\mu} z.
\end{equation}
\end{description} 
\end{alertblock}
\end{frame}
%
\begin{frame}[fragile]{Statistical Modeling}
\begin{alertblock}{MLE -- Maximum Likelihood Estimator.}
\begin{description}
\item[-] Let a PolSAR image sample $\bm z = (z_1,\dots,z_n)$.  
\item[-] The log-likelihood is defined by,
\begin{equation}
    \mathcal{L}(\bm z;\mu, \text{L})=n\left[\text{L}\ln\frac{\text{L}}{\mu}-\ln\Gamma(\text{L})\right]+\text{L}\sum_{k=1}^{n}\ln z_k -\frac{\text{L}}{\mu}\sum_{k=1}^{n} z_k.
\end{equation}
\item[] \textcolor{red}{There is an observation.}
\item[-] \textcolor{red}{Is it a flat function? Yes!!!!! }
\item[-] \textcolor{red}{BFGS optimization avoid this problem.}
\end{description}
\end{alertblock}
\end{frame}
%
\begin{frame}[fragile]{Statistical Modeling}
\begin{alertblock}{MLE -- Maximum Likelihood Estimator.}
\begin{description}
\item[-] Splitting the sample
$$
\bm z = (\underbrace{z_1,z_2,\dots,z_j}_{\bm z_\text{I}}, 
\underbrace{z_{j+1}, z_{j+2},\dots,z_n}_{\bm z_\text{E}}),
$$ 
\item[-] There are two models $$\bm Z_\text{I} \sim \Gamma(\mu_\text{I},\text{L}_\text{I}),$$ and, $$\bm Z_\text{E} \sim \Gamma(\mu_\text{E},\text{L}_\text{E}).$$
\item[-] To estimate the parameters we use the BFGS method
\end{description}
\end{alertblock}
\end{frame}
%
\begin{frame}[fragile]{Statistical Modeling}
\begin{alertblock}{MLE -- The total log-likelihood is defined at pixel $j$ by,}
\begin{equation}
\begin{split}
\mathcal{L}(j&;\widehat{\mu}_I, \widehat{\text{L}}_I,\widehat{\mu}_E, \widehat{\text{L}}_E)=\\
&j \big[\widehat{\text{L}}_\text{I}\ln (\widehat{\text{L}}_\text{I} / \widehat{\mu}_\text{I}) - \ln \Gamma(\widehat{\text{L}}_\text{I})\big]
+\widehat{\text{L}}_\text{I} \sum_{k=1}^{j}\ln z_k -\frac{\widehat{\text{L}}_\text{I}}{\widehat{\mu}_\text{I}}\sum_{k=1}^{j} z_k +\\
&(n-j) \big[\widehat{\text{L}}_\text{E}\ln (\widehat{\text{L}}_\text{E} / \widehat{\mu}_\text{E}) - \ln \Gamma(\widehat{\text{L}}_\text{E})\big]\\
&+\widehat{\text{L}}_\text{E} \sum_{k=j+1}^{n}\ln z_k - \frac{\widehat{\text{L}}_\text{E}}{\widehat{\mu}_\text{E}}\sum_{k=j+1}^{n} z_k,
\end{split}
\end{equation}
$$
\widehat{\jmath}= \arg\max\limits_{j\in [\min_s,N-\min_s]}\mathcal{L}(j;\widehat{\mu}_I, \widehat{\text{L}}_I,\widehat{\mu}_E, \widehat{\text{L}}_E),
$$
\begin{description}
\item[-] \textcolor{red}{Is it a non-differentiable function? Yes! The GenSA method avoid this problem.}
\item[-] \textcolor{red}{Are there oscillations in the extremities of the function? Yes! Set a margin avoid this problem.}
\end{description}
\end{alertblock}
\end{frame}
%
\begin{frame}[fragile]{Evidence edge detection: Gambini Algorithm}
\begin{algorithm}[H]
\SetAlgoLined
\For{Channel $1\leq c\leq n_c$}{
	\For{Radial}{
		$\bm z = (z_1,z_2,\dots,z_n)\leftarrow$ data collected around the radial\;
		\For{$\min_s\leq j\leq n-\min_s$}{\nllabel{Line:InitFor}
			Splitting the sample as $\bm z_{\text{I}}=(z_{\min_s},\dots,z_j)$ e 
			$\bm z_{\text{E}}=(z_{j+1},\dots,z_{n-\min_s})$\;
			Compute $\big(\widehat{\mu}_\text{I}, \widehat{\text{L}}_\text{I}\big)$ com $\bm z_{\text{I}}$, e $\big(\widehat{\mu}_\text{E}, \widehat{\text{L}}_\text{E}\big)$ com $\bm z_{\text{E}}$\;
			Compute the total log-likelihood at $j$ with $\mathcal L\big(j;\widehat{\mu}_I, \widehat{\text{L}}_I,\widehat{\mu}_E, \widehat{\text{L}}_E\big)$\;
		}
		$\widehat\jmath\leftarrow$ The value of $j$ which maximizes the total log-likelihood function\;
		\Return $(\widehat x, \widehat y)$,  the coordinates of each $\widehat\jmath$\;
	}
\Return The binary image $\widehat{\bm\jmath}_c$ with $1$ at every  $(\widehat x, \widehat y)$, and $0$ otherwise.
}
\end{algorithm}
\end{frame}
\begin{frame}[fragile]{The Flevoland Image}
\begin{alertblock}{ROI -- Region of interest}
	\begin{figure}[hbt]
\centering
	\includegraphics[width=.5\linewidth]{flevoland_radial_4_look_black}
	\caption{Region of interest (ROI) in the image of Flevoland.}
\label{fig10}
\end{figure}
\end{alertblock}
\end{frame}

\begin{frame}[fragile]{Gambini Algorithm Results}
\begin{alertblock}{FLEV-ROI-I}
\begin{figure}[hbt!]
	\centering
    \subfloat[Channel $\text{hh}$ \label{evidencias_hh_hv_vv:a}]{%
    	\includegraphics[width=0.32\linewidth]{flevoland_hh_evid_param_L_mu_14_pixel_crop}
     	}
    \subfloat[Channel $\text{hv}$ \label{evidencias_hh_hv_vv:b}]{%
       	\includegraphics[width=0.32\linewidth]{flevoland_hv_evid_param_L_mu_14_pixel_crop}
     	}
    \subfloat[Channel $\text{vv}$ \label{evidencias_hh_hv_vv:c}]{%
       	\includegraphics[width=0.32\linewidth]{flevoland_vv_evid_param_L_mu_14_pixel_crop}
     	}
    \caption{Edge evidence to FLEV-ROI-I}
    \label{evidencias_hh_hv_vv} 
\end{figure}	
\end{alertblock}
- \textcolor{red}{Now, we are going to fusion of the evidences}
\end{frame}
%
%\section{Edge detection}
%\begin{frame}[fragile]{Detecção de bordas}
%\begin{alertblock}{Ideia}
%O seguinte procedimento é proposto para detectar evidências de bordas no canais $\text{hh}$, $\text{hv}$ e $\text{vv}$:
%\begin{itemize}
%    \item Identificar o centroide da região de interesse, de forma automática, semi-automática ou manual;
%	\item Contruir radiais do centroide para fora da região, e; 
%	\item Coletar dados usando o algoritmo \textit{Bresenham's midpoint line algorithm};
%	\item Detectar pontos que fornecem evidências de mudanças de propriedades estatísticas nas radiais da região (Evidências de bordas) 
%	\item Usar o método GenSA - \textit{Generalized Simulated Annealing} para encontrar o máximo da função proveniente do método MLE; 
%\end{itemize}
%\end{alertblock}
%\end{frame}
%
%\begin{frame}[fragile]{Evidence edge detection: Gambini Algorithm}
%\begin{algorithm}[H]
%\SetAlgoLined
%\For{Channel $1\leq c\leq n_c$}{
%	\For{Radial}{
%		$\bm z = (z_1,z_2,\dots,z_n)\leftarrow$ data collected around the radial\;
%		\For{$\min_s\leq j\leq n-\min_s$}{\nllabel{Line:InitFor}
%			Splitting the sample as $\bm z_{\text{I}}=(z_{\min_s},\dots,z_j)$ e 
%			$\bm z_{\text{E}}=(z_{j+1},\dots,z_{n-\min_s})$\;
%			Compute $\big(\widehat{\mu}_\text{I}, \widehat{\text{L}}_\text{I}\big)$ com $\bm z_{\text{I}}$, e $\big(\widehat{\mu}_\text{E}, \widehat{\text{L}}_\text{E}\big)$ com $\bm z_{\text{E}}$\;
%			Compute the total log-likelihood at $j$ with $\mathcal L\big(j;\widehat{\mu}_I, \widehat{\text{L}}_I,\widehat{\mu}_E, \widehat{\text{L}}_E\big)$\;
%		}
%		$\widehat\jmath\leftarrow$ The value of $j$ which maximizes the total log-likelihood function\;
%		\Return $(\widehat x, \widehat y)$,  the coordinates of each $\widehat\jmath$\;
%	}
%\Return The binary image $\widehat{\bm\jmath}_c$ with $1$ at every  $(\widehat x, \widehat y)$, and $0$ otherwise.
%}
%\end{algorithm}
%\end{frame}
%
%
\pgfdeclarelayer{background}
\pgfdeclarelayer{foreground}
\pgfsetlayers{background,main,foreground}
\tikzstyle{sensor}=[draw, fill=blue!20, text width=3.5em, 
text centered, minimum height=2.5em,drop shadow]
\tikzstyle{sensor1}=[draw, fill=green!20, text width=3.5em, 
text centered, minimum height=2.5em,drop shadow]
\tikzstyle{ann} = [above, text width=5em, text centered]
\tikzstyle{wa} = [sensor, text width=5em, fill=red!50, 
minimum height=3em, rounded corners, drop shadow]
\tikzstyle{sc} = [sensor, text width=10em, fill=red!20, 
minimum height=7em, rounded corners, drop shadow]
%\def\blockdist{0.3}
%\def\edgedist{5.5}
%\begin{frame}[fragile]{General Fusion}
%\begin{alertblock}{Fusion scheme}
%\begin{figure}[hbt]
%	\centering
%	\resizebox{.8\columnwidth}{!}{%
%		\begin{tikzpicture}
%			\node (wa)[wa]  {Image Fusion};
%			\path (wa.west)+(-3.5,1.5) node (swtnode1) [sensor1] {$\widehat{\bm\jmath}_1$};
%			\path (wa.west)+(-3.5,0.5) node (swtnode2) [sensor1] {$\widehat{\bm\jmath}_2$};
%			\path (wa.west)+(-3.5,-1.0) node (dots)[ann] {$\vdots$}; 
%			\path (wa.west)+(-3.5,-2.0) node (swtnode3)[sensor1] {$\widehat{\bm\jmath}_{n_c}$};  
%			%
%			\path (wa.west)+(-6.2,1.5) node (e1) [sensor] {$\text{I}_1$};
%			\path (wa.west)+(-6.2,0.5) node (e2)[sensor] {$\text{I}_2$};
%			\path (wa.west)+(-6.2,-2.0) node (e3)[sensor] {$\text{I}_{n_c}$};    
%			\path [draw, ->] (e1.east) -- node [above] {GA} 
%			(swtnode1.180) ;
%			\path [draw, ->] (e2.east) -- node [above] {GA} 
%			(swtnode2.180);
%			\path [draw, ->] (e3.east) -- node [above] {GA} 
%			(swtnode3.180);
%			%%
%			\path [draw, ->] (swtnode1.east) -- node [above] {} 
%			(wa.160) ;
%			\path [draw, ->] (swtnode2.east) -- node [above] {} 
%			(wa.180) ;
%			\path [draw, ->] (swtnode3.east) -- node [above] {} 
%			(wa.200) ;        
%		\end{tikzpicture}
%	}
%	\caption{Fusion Scheme}
%	\label{fig:fusion_scheme}
%\end{figure}
%\end{alertblock}
%\end{frame}
%
\begin{frame}[fragile]{General Idea}
\begin{alertblock}{Fusion scheme}
%\begin{itemize}
%item Parte II - Para cada $\bm{\widehat\imath}_t$
\pgfdeclarelayer{background}
\pgfdeclarelayer{foreground}
\pgfsetlayers{background,main,foreground}
%
\pgfdeclarelayer{background}
\pgfdeclarelayer{foreground}
\pgfsetlayers{background,main,foreground}
\tikzstyle{sensor}=[draw, fill=blue!20, text width=2.5em, 
    text centered, minimum height=2em,drop shadow]
\tikzstyle{ann} = [above, text width=5em, text centered]
\tikzstyle{wa} = [sensor, text width=2em, fill=red!20, 
    minimum height=2em, rounded corners, drop shadow]
\tikzstyle{waimage} = [sensor, text width=4em, fill=red!20, 
    minimum height=2em, rounded corners, drop shadow]
    \tikzstyle{waimage1} = [sensor, text width=3.75em, fill=red!20, 
    minimum height=2em, rounded corners, drop shadow]
\tikzstyle{wa1} = [sensor, text width=2em, fill=red!20, 
    minimum height=2em, rounded corners, drop shadow]
\tikzstyle{wa2}=[draw, fill=blue!20, text width=2.5em, 
    text centered, minimum height=3em,drop shadow]    
\def\blockdist{2.3}
\def\edgedist{2.5}
\begin{figure}[hbt]
\begin{tikzpicture}
\node[waimage] (waimage1) at (-7.0,0.0) {Image};
\node[waimage] (waimage2) at (-5.3,0.0) {ROI};
\node[wa2] (waimage3) at (-5.3,3.5) {GR};
\node[waimage] (waimage4) at (-3.5,0.0) {Radials and center};
\node[waimage1] (wa1) at (0.5,2.0) {Fusion 1};
\node[wa2] (waimage5) at (2.0,3.5) {Error};
\path (waimage4.west)+(4.7,1.0) node (dots)[ann] {$\vdots$};
\path (waimage4.west)+(4.7,0.5) node (dots)[ann] {$\vdots$};
\path (waimage4.west)+(4.7,0.0) node (dots)[ann] {$\vdots$};
\path (waimage4.west)+(4.7,-0.5) node (dots)[ann] {$\vdots$};
\path (waimage4.west)+(4.7,-1.0) node (dots)[ann] {$\vdots$};
\path (waimage4.west)+(4.7,-1.5) node (dots)[ann] {$\vdots$};
\node[waimage1] (wa6) at (0.5,-2.0) {Fusion N};
%
\path [draw, ->] (waimage1.east) -- node [left] {} 
        (waimage2) ;
\path [draw, ->] (waimage2.east) -- node [left] {} 
        (waimage4) ;
\path [draw, ->] (waimage2.north) -- node [left] {} 
        (waimage3) ;
\path [draw, ->] (waimage3.east) -- node [left] {} 
        (waimage5) ;        
%
    \path (waimage4.west)+(2.5,1.5) node (e1_1) [sensor] {Ch. hh};
    \path (waimage4.west)+(2.5,0.0) node (e2_1)[sensor] {Ch. hv}; 
    \path (waimage4.west)+(2.5,-1.5) node (e3_1)[sensor] {Ch. vv};    
%
	\path [draw, ->] (waimage4.east) -- node [left] {} 
        (e1_1.180) ;
	\path [draw, ->] (waimage4.east) -- node [below] {} 
        (e2_1.180);
	\path [draw, ->] (waimage4.east) -- node [right] {} 
        (e3_1.180);
	\path [draw, ->] (e1_1.east) -- node [right] {} 
        (wa1.160);
	\path [draw, ->] (e2_1.east) -- node [above] {} 
        (wa1.180);
	\path [draw, ->] (e3_1.east) -- node [right] {} 
        (wa1.200);
    \path [draw, ->] (e1_1.east) -- node [right] {} 
        (wa6.160);
	\path [draw, ->] (e2_1.east) -- node [above] {} 
        (wa6.180);
	\path [draw, ->] (e3_1.east) -- node [right] {} 
        (wa6.200);
%
\path [draw, ->] (wa1.east) -- node [left] {} 
        (waimage5) ;
\path [draw, ->] (wa6.east) -- node [left] {} 
        (waimage5) ;
\end{tikzpicture}
\caption{Fusion Scheme}
\label{fig9}
\end{figure}
\end{alertblock}
\end{frame}

%\section{Fusion methods to edge detection}
\begin{frame}[fragile]{Fusion methods to edge detection}
\begin{alertblock}{Simple average Fusion -- MS}
\pgfdeclarelayer{background}
\pgfdeclarelayer{foreground}
\pgfsetlayers{background,main,foreground}
%
\pgfdeclarelayer{background}
\pgfdeclarelayer{foreground}
\pgfsetlayers{background,main,foreground}
\tikzstyle{sensor}=[draw, fill=blue!20, text width=5em, 
    text centered, minimum height=2.5em,drop shadow]
\tikzstyle{ann} = [above, text width=5em, text centered]
\tikzstyle{wa} = [sensor, text width=15em, fill=red!20, 
    minimum height=6em, rounded corners, drop shadow]
\tikzstyle{sc} = [sensor, text width=13em, fill=red!20, 
    minimum height=10em, rounded corners, drop shadow]
\def\blockdist{2.3}
\def\edgedist{2.5}
	\begin{figure}[htb!]
\centering
\begin{tikzpicture}
	\node (wa) [wa]  {$\bm I_\text{F}(x,y)=(n_c)^{-1}\sum_{c=1}^{n_c} \widehat{\bm\jmath}_c(x,y)$};
	\path (wa.west)+(-3.2,1.5) node (e1) [sensor] {$\widehat{\bm\jmath}_1(x,y)$};
    \path (wa.west)+(-3.2,0.5) node (e2)[sensor] {$\widehat{\bm\jmath}_2(x,y)$};
    \path (wa.west)+(-3.2,-1.0) node (dots)[ann] {$\vdots$}; 
    \path (wa.west)+(-3.2,-2.0) node (e3)[sensor] {$\widehat{\bm\jmath}_{n_c}(x,y)$};    
%
    \path [draw, ->] (e1.east) -- node [above] {} 
        (wa.160) ;
    \path [draw, ->] (e2.east) -- node [above] {} 
        (wa.180);
    \path [draw, ->] (e3.east) -- node [above] {} 
        (wa.200);   
\end{tikzpicture}
	\caption{Simple average Fusion.}
\label{fig:cap_fusao_media_simples}
\end{figure}
\end{alertblock}
\end{frame}
%
\begin{frame}[fragile]{Fusion methods to edge detection}
\begin{alertblock}{Discrete wavelet multi-resolution fusion -- MR-DWT}
\pgfdeclarelayer{background}
\pgfdeclarelayer{foreground}
\pgfsetlayers{background,main,foreground}
\tikzstyle{sensor}=[draw, fill=blue!20, text width=5em, 
    text centered, minimum height=2.5em,drop shadow]
\tikzstyle{ann} = [above, text width=5em, text centered]
\tikzstyle{wa} = [sensor, text width=7em, fill=red!20, 
    minimum height=3em, rounded corners, drop shadow]
\tikzstyle{sc} = [sensor, text width=10em, fill=red!20, 
    minimum height=7em, rounded corners, drop shadow]
\def\blockdist{2.3}
\def\edgedist{2.5}
	\begin{figure}[htb!]
\begin{tikzpicture}
	\path (wa.west)+(-3.0,1.5) node (swtnode1) [sensor] {$\text{Coef DWT}_1$};
	\path (wa.west)+(-3.0,0.5) node (swtnode2) [sensor] {$\text{Coef DWT}_2$};
	\path (wa.west)+(-3.0,-1.0) node (dots)[ann] {$\vdots$}; 
    \path (wa.west)+(-3.0,-2.0) node (swtnode3)[sensor] {$\text{Coef DWT}_{n_c}$};  
	
	
	\path (wa.west)+(-6.2,1.5) node (e1) [sensor] {$\widehat{\bm\jmath}_1(x,y)$};
    \path (wa.west)+(-6.2,0.5) node (e2)[sensor] {$\widehat{\bm\jmath}_2(x,y)$};
    \path (wa.west)+(-6.2,-1.0) node (dots)[ann] {$\vdots$}; 
    \path (wa.west)+(-6.2,-2.0) node (e3)[sensor] {$\widehat{\bm\jmath}_{n_c}(x,y)$};    
    \path (wa.west)+(1.0,1.0) node (swtnodefus) [wa] {Coeficients wavelets\\
                                                       Fusion};
                                                       
    \path (wa.west)+(1.0,-2.5) node (imagefus) [wa] {Images Fusion};
    \path [draw, ->] (e1.east) -- node [above] {W} 
        (swtnode1.180) ;
    \path [draw, ->] (e2.east) -- node [above] {W} 
        (swtnode2.180);
    \path [draw, ->] (e3.east) -- node [above] {W} 
        (swtnode3.180);
%
    \path [draw, ->] (swtnode1.east) -- node [above] {} 
        (swtnodefus.160) ;
    \path [draw, ->] (swtnode2.east) -- node [above] {} 
        (swtnodefus.180);
    \path [draw, ->] (swtnode3.east) -- node [above] {} 
        (swtnodefus.200);      
    \path [draw, ->] (swtnodefus.south) -- node [right] {$W^{-1}$}      
        (imagefus.north);           
\end{tikzpicture}
	\caption{MR--DWT Fusion}
\label{fig7}
\end{figure}
\begin{itemize}
\vspace{-0.8cm}
\item $W$ is a wavelet transform.
\end{itemize}
\end{alertblock}
\end{frame}

\begin{frame}[fragile]{Fusion methods to edge detection}
\begin{alertblock}{Stationary wavelet multi-resolution fusion -- MR-SWT} 
\pgfdeclarelayer{background}
\pgfdeclarelayer{foreground}
\pgfsetlayers{background,main,foreground}
\tikzstyle{sensor}=[draw, fill=blue!20, text width=5em, 
    text centered, minimum height=2.5em,drop shadow]
\tikzstyle{ann} = [above, text width=5em, text centered]
\tikzstyle{wa} = [sensor, text width=7em, fill=red!20, 
    minimum height=3em, rounded corners, drop shadow]
\tikzstyle{sc} = [sensor, text width=10em, fill=red!20, 
    minimum height=7em, rounded corners, drop shadow]
\def\blockdist{2.3}
\def\edgedist{2.5}
	\begin{figure}[htb!]
\begin{tikzpicture}
	\path (wa.west)+(-3.0,1.5) node (swtnode1) [sensor] {$\text{Coef SWT}_1$};
	\path (wa.west)+(-3.0,0.5) node (swtnode2) [sensor] {$\text{Coef SWT}_2$};
	\path (wa.west)+(-3.0,-1.0) node (dots)[ann] {$\vdots$}; 
    \path (wa.west)+(-3.0,-2.0) node (swtnode3)[sensor] {$\text{Coef SWT}_{n_c}$};  
	
	
	\path (wa.west)+(-6.2,1.5) node (e1) [sensor] {$\widehat{\bm\jmath}_1(x,y)$};
    \path (wa.west)+(-6.2,0.5) node (e2)[sensor] {$\widehat{\bm\jmath}_2(x,y)$};
    \path (wa.west)+(-6.2,-1.0) node (dots)[ann] {$\vdots$}; 
    \path (wa.west)+(-6.2,-2.0) node (e3)[sensor] {$\widehat{\bm\jmath}_{n_c}(x,y)$};    
    \path (wa.west)+(1.0,1.0) node (swtnodefus) [wa] {Coeficients wavelets \\
                                                       Fusion};
                                                       
    \path (wa.west)+(1.0,-2.5) node (imagefus) [wa] {Images Fusion};
    \path [draw, ->] (e1.east) -- node [above] {W} 
        (swtnode1.180) ;
    \path [draw, ->] (e2.east) -- node [above] {W} 
        (swtnode2.180);
    \path [draw, ->] (e3.east) -- node [above] {W} 
        (swtnode3.180);
%
    \path [draw, ->] (swtnode1.east) -- node [above] {} 
        (swtnodefus.160) ;
    \path [draw, ->] (swtnode2.east) -- node [above] {} 
        (swtnodefus.180);
    \path [draw, ->] (swtnode3.east) -- node [above] {} 
        (swtnodefus.200);      
    \path [draw, ->] (swtnodefus.south) -- node [right] {$W^{-1}$}      
        (imagefus.north);               
\end{tikzpicture}
	\caption{MR--SWT fusion.}
\label{fig7}
\end{figure}
\begin{itemize}
\vspace{-0.8cm}
\item $W$ is a wavelet transform.
\end{itemize}
\end{alertblock}
\end{frame}
%
\begin{frame}[fragile]{Fusion methods to edge detection}
\begin{alertblock}{Principal Component Analysis -- PCA Fusion}
\pgfdeclarelayer{background}
\pgfdeclarelayer{foreground}
\pgfsetlayers{background,main,foreground}
\tikzstyle{sensor}=[draw, fill=blue!20, text width=5em, 
    text centered, minimum height=2.5em,drop shadow]
\tikzstyle{ann} = [above, text width=5em, text centered]
\tikzstyle{wa} = [sensor, text width=5em, fill=red!20, 
    minimum height=3em, rounded corners, drop shadow]
\tikzstyle{sc} = [sensor, text width=10em, fill=red!20, 
    minimum height=5em, rounded corners, drop shadow]
\def\blockdist{2.3}
\def\edgedist{2.5}
	\begin{figure}[htb!]
\begin{tikzpicture}
	\path (wa.west)+(-2.0,0.0) node (pcanode) [wa] {$\text{PCA}$};
	\path (wa.west)+(-6.2,1.5) node (e1) [sensor] {$\widehat{\bm\jmath}_1(x,y)$};
    \path (wa.west)+(-6.2,0.5) node (e2)[sensor] {$\widehat{\bm\jmath}_2(x,y)$};
    \path (wa.west)+(-6.2,-1.0) node (dots)[ann] {$\vdots$}; 
    \path (wa.west)+(-6.2,-2.0) node (e3)[sensor] {$\widehat{\bm\jmath}_{n_c}(x,y)$};    
    \path (wa.west)+(2.0,0.0) node (pcanodefus) [sc] {$\bm I_\text{F}(x,y)=\sum_{c=1}^{n_c} P(c)\bm{\widehat\jmath}_c(x,y)$};
    \path [draw, ->] (e1.east) -- node [above] {} 
        (pcanode.160) ;
    \path [draw, ->] (e2.east) -- node [above] {} 
        (pcanode.180);
    \path [draw, ->] (e3.east) -- node [above] {} 
        (pcanode.200);
        %
    \path [draw, ->] (pcanode.east) -- node [above] {} 
        (pcanodefus.180) ;  
\end{tikzpicture}
	\caption{PCA Fusion.}
\label{fig:cap_fusao_pca}
\end{figure}
\end{alertblock}
\end{frame}
%
\begin{frame}[fragile]{Fusion methods to edge detection}
\begin{alertblock}{Receiver Operating Characteristic fusion - ROC Fusion}
\begin{itemize}
\item Parte I
\pgfdeclarelayer{background}
\pgfdeclarelayer{foreground}
\pgfsetlayers{background,main,foreground}
\tikzstyle{sensor}=[draw, fill=blue!20, text width=5em, 
    text centered, minimum height=2.5em,drop shadow]
\tikzstyle{ann} = [above, text width=5em, text centered]
\tikzstyle{wa} = [sensor, text width=7em, fill=red!20, 
    minimum height=5em, rounded corners, drop shadow]
\tikzstyle{sc} = [sensor, text width=13em, fill=red!20, 
    minimum height=10em, rounded corners, drop shadow]
\def\blockdist{2.3}
\def\edgedist{2.5}
	\begin{figure}[htb!]
\begin{tikzpicture}
\path (wa.west)+(-3.0,0.0) node (pcanode) [wa] {$V=\sum_{c=1}^{n_c}\bm{\widehat\jmath}_c$};
	\path (wa.west)+(-7.2,1.5) node (e1) [sensor] {$\bm{\widehat\jmath}_1$};
    \path (wa.west)+(-7.2,0.5) node (e2)[sensor] {$\bm{\widehat\jmath}_2$};
    \path (wa.west)+(-7.2,-1.0) node (dots)[ann] {$\vdots$}; 
    \path (wa.west)+(-7.2,-2.0) node (e3)[sensor] {$\bm{\widehat\jmath}_{n_c}$};    
    %\path (wa.west)+(2.0,0.0) node (pcanodefus) [sc] {$V_m=\max{V(i)}$
    %                                                  \\$p=V_m(i)/||V_m||$
    %                                                  \\$IF=\sum_{i=1}^{nc}p_iIE_i$};
    \path [draw, ->] (e1.east) -- node [above] {} 
        (pcanode.160) ;
    \path [draw, ->] (e2.east) -- node [above] {} 
        (pcanode.180);
    \path [draw, ->] (e3.east) -- node [above] {} 
        (pcanode.200);
        %
	%\node (wa) [wa]  {$V=\sum_{i=1}^{N}IE_i$};
	%\path (wa.west)+(-3.2,1.5) node (e1) [sensor] {$IE_1$};
    %\path (wa.west)+(-3.2,0.5) node (e2)[sensor] {$IE_2$};
    %\path (wa.west)+(-3.2,-1.0) node (dots)[ann] {$\vdots$}; 
    %\path (wa.west)+(-3.2,-2.0) node (e3)[sensor] {$IE_N$};    
%%   
    \path (pcanode.east)+(3.2,1.5) node (m1) [sensor] {$\bm{\widehat\imath}_1$};
    \path (pcanode.east)+(3.2,0.5) node (m2) [sensor] {$\bm{\widehat\imath}_2$};
    \path (pcanode.east)+(3.2,-1.0) node (dots)[ann] {$\vdots$}; 
    \path (pcanode.east)+(3.2,-2.0) node (m3) [sensor] {$\bm{\widehat\imath}_{n_c}$};
%%
    %\path [draw, ->] (e1.east) -- node [above] {} 
    %    (wa.160) ;
    %\path [draw, ->] (e2.east) -- node [above] {} 
    %    (wa.180);
    %\path [draw, ->] (e3.east) -- node [above] {} 
    %    (wa.200);
	\path [draw, ->] (pcanode.east) -- node [above] {\tiny{$CT_1$}} 
        (m1.west);
	\path [draw, ->] (pcanode.east) -- node [above] {\tiny{$CT_2$}} 
        (m2.west);
	\path [draw, ->] (pcanode.east) -- node [right] {\tiny{$CT_{n_c}$}} 
        (m3.west);
%               
%%    \path (wa.south) +(0,-\blockdist) node (asrs) {Estrutura geral da fusão de evidência proposta};
%  
%    \begin{pgfonlayer}{background}
%        \path (e1.west |- e1.north)+(-0.5,0.3) node (a) {};
%        \path (wa.south -| wa.east)+(+0.5,-0.3) node (b) {};
%        \path (m3.east |- m3.east)+(+0.5,-0.75) node (c) {};
       %   
%        \path[fill=yellow!20,rounded corners, draw=black!50, dashed]
%            (a) rectangle (c);           
%       %     
%    \end{pgfonlayer}
   
\end{tikzpicture}
	\caption{Fusion based in ROC statistics -- Part I.}
\label{fig8}
\end{figure}
\item $CT_c$ are thresholds.
\end{itemize}
\end{alertblock}
\end{frame}
%
\begin{frame}[fragile]{Fusion methods to edge detection}
\begin{alertblock}{ROC fusion}
\begin{itemize}
\item Part II - For each $\bm{\widehat\imath}_t$
\tikzstyle{sensor}=[draw, fill=blue!20, text width=2.5em, 
    text centered, minimum height=2em,drop shadow]
\tikzstyle{ann} = [above, text width=5em, text centered]
\tikzstyle{wa} = [sensor, text width=2em, fill=red!20, 
    minimum height=2em, rounded corners, drop shadow]
\tikzstyle{wa1} = [sensor, text width=2em, fill=red!20, 
    minimum height=2em, rounded corners, drop shadow]
\begin{figure}[hbt]
\begin{tikzpicture}
\node[wa] (wa) at (0.0,0.0) {$\bm{\widehat\imath}_t$};
\node[wa1] (wa1) at (4.0,0.0) {$\overline{TP}_t$};

    \path (wa.west)+(2.5,1.5) node (e1_1) [sensor] {$TP_1$};
    \path (wa.west)+(2.5,0.5) node (e2_1)[sensor] {$TP_2$};
    \path (wa.west)+(2.5,-1.0) node (dots)[ann] {$\vdots$}; 
    \path (wa.west)+(2.5,-2.0) node (e3_1)[sensor] {$TP_{n_c}$};    
%
	\path [draw, ->] (wa.east) -- node [left] {\tiny{$\overline{\cap \bm{\widehat\jmath}_1}$}} 
        (e1_1.180) ;
	\path [draw, ->] (wa.east) -- node [below] {\tiny{$\overline{\cap \bm{\widehat\jmath}_2}$}} 
        (e2_1.180);
	\path [draw, ->] (wa.east) -- node [right] {\tiny{$\overline{\cap \bm{\widehat\jmath}_{n_c}}$}} 
        (e3_1.180);
	\path [draw, ->] (e1_1.east) -- node [right] {\tiny{$+$}} 
        (wa1.160);
	\path [draw, ->] (e2_1.east) -- node [above] {\tiny{$+$}} 
        (wa1.180);
	\path [draw, ->] (e3_1.east) -- node [right] {\tiny{$+$}} 
        (wa1.200);
\end{tikzpicture}
\caption{ROC fusion for each $t$. Similar to $\overline{TN}_t$,$\overline{FP}_t$ and, $\overline{FN}_t$. }
\label{fig9}
\end{figure}
\item  This generates the confusion matrix to calculate the ROC statistic.
\end{itemize}
\end{alertblock}
\end{frame}
\begin{frame}[fragile]{Fusion methods to edge detection}
\begin{alertblock}{Singular Values Decomposition fusion -- MR-SVD}
\pgfdeclarelayer{background}
\pgfdeclarelayer{foreground}
\pgfsetlayers{background,main,foreground}
\tikzstyle{sensor}=[draw, fill=blue!20, text width=5em, 
    text centered, minimum height=2.5em,drop shadow]
\tikzstyle{ann} = [above, text width=5em, text centered]
\tikzstyle{wa} = [sensor, text width=7em, fill=red!20, 
    minimum height=3em, rounded corners, drop shadow]
\tikzstyle{sc} = [sensor, text width=10em, fill=red!20, 
    minimum height=7em, rounded corners, drop shadow]
\def\blockdist{2.3}
\def\edgedist{2.5}
	\begin{figure}[htb!]
\begin{tikzpicture}
	\path (wa.west)+(-3.0,1.5) node (swtnode1) [sensor] {$\text{Coef SVD}_1$};
	\path (wa.west)+(-3.0,0.5) node (swtnode2) [sensor] {$\text{Coef SVD}_2$};
	\path (wa.west)+(-3.0,-1.0) node (dots)[ann] {$\vdots$}; 
    \path (wa.west)+(-3.0,-2.0) node (swtnode3)[sensor] {$\text{Coef SVD}_{n_c}$};  
	
	
	\path (wa.west)+(-6.2,1.5) node (e1) [sensor] {$\widehat{\bm\jmath}_1(x,y)$};
    \path (wa.west)+(-6.2,0.5) node (e2)[sensor] {$\widehat{\bm\jmath}_2(x,y)$};
    \path (wa.west)+(-6.2,-1.0) node (dots)[ann] {$\vdots$}; 
    \path (wa.west)+(-6.2,-2.0) node (e3)[sensor] {$\widehat{\bm\jmath}_{n_c}(x,y)$};    
    \path (wa.west)+(1.0,1.0) node (swtnodefus) [wa] {Singular values\\
                                                       Fusion};
                                                       
    \path (wa.west)+(1.0,-2.5) node (imagefus) [wa] {Images Fusion};
    \path [draw, ->] (e1.east) -- node [above] {W} 
        (swtnode1.180) ;
    \path [draw, ->] (e2.east) -- node [above] {W} 
        (swtnode2.180);
    \path [draw, ->] (e3.east) -- node [above] {W} 
        (swtnode3.180);
%
    \path [draw, ->] (swtnode1.east) -- node [above] {} 
        (swtnodefus.160) ;
    \path [draw, ->] (swtnode2.east) -- node [above] {} 
        (swtnodefus.180);
    \path [draw, ->] (swtnode3.east) -- node [above] {} 
        (swtnodefus.200);      
    \path [draw, ->] (swtnodefus.south) -- node [right] {$W^{-1}$}      
        (imagefus.north);           
\end{tikzpicture}
	\caption{MR--SVD fusion}
\label{fig7}
\end{figure}
\begin{itemize}
\vspace{-0.8cm}
\item $W$ is a SVD decomposition.
\end{itemize}
\end{alertblock}
\end{frame}

%\begin{frame}[fragile]{Evidence edge detection}
%\begin{alertblock}{Example with 25 radials in the ROI of the Flevoland image} 
%\begin{figure}[hbt]
%\centering
%	\includegraphics[width=.7\linewidth]{flevoland_radial_25_point_hh_crop}
%	\caption{Detection of evidence of edges in the channel $\text{hh}$.}
%\label{fig1}
%\end{figure}
%\end{alertblock}
%\end{frame}

%
\begin{frame}[fragile]{Evidence Fusion Results}
\begin{alertblock}{FLEV-ROI-I Fusion}
\begin{figure}[hbt!]
	\centering
     \subfloat[Average\label{fusion_met:a}]{%
       \includegraphics[width=0.23\linewidth]{flevoland_fus_media_param_L_mu_14_pixel_crop}
     }
     \subfloat[MR-DWT\label{fusion_met:b}]{%
       \includegraphics[width=0.23\linewidth]{flevoland_fus_dwt_param_L_mu_14_pixel_crop}
     }
     \subfloat[PCA \label{fusion_met:c}]{%
       %\includegraphics[width=0.2\textwidth]{example-image-a}
       \includegraphics[width=0.23\linewidth]{flevoland_fus_pca_param_L_mu_14_pixel_crop}       
     }\\
     \subfloat[E-ROC\label{fusion_met:d}]{%
       \includegraphics[width=0.23\linewidth]{flevoland_fus_roc_param_L_mu_14_pixel_crop}
     }
     \subfloat[MR-SWT \label{fusion_met:e}]{%
       \includegraphics[width=0.23\linewidth]{flevoland_fus_swt_param_L_mu_14_pixel_crop}
     }
     \subfloat[MR-SVD\label{fusion_met:f}]{%
       \includegraphics[width=0.23\linewidth]{flevoland_fus_svd_param_L_mu_14_pixel_crop}
     }
     \caption{Fusion methods to FLEV-ROI-I}
     \label{fusion_met}
\end{figure}
\end{alertblock}
\end{frame}
%
%
\begin{frame}[fragile]{Metrics}
\begin{alertblock}{Metrics that are used}
\begin{itemize}
\item[-] \textcolor{red}{MAC -- Accuracy}
\begin{align*}
	\text{Mac} &= \frac{\text{TP}+\text{TN}}{\text{TP}+\text{TN}+\text{FP}+\text{FN}},
\end{align*}
\item[-] \textcolor{red}{MFE -- $F1\text{-score}$}
\begin{align*}
	F1\text{-score} &= \frac{2\text{TP}}{2\text{TP}+\text{FP}+\text{FN}},
\end{align*}
\item[-] \textcolor{red}{nMCC- Matthews correlation coefficient normalized}
\begin{align*}
	\text{nMcc} &=\frac{\frac{\text{TP}\cdot\text{TN}-\text{FP}\cdot\text{FN}}{\sqrt{\text{(TP +FP)}\cdot\text{(TP +FN)}\cdot\text{(TN +FP)}\text{(TN +FN)}}}+1}{2}.
\end{align*}
\end{itemize}
\end{alertblock}
\end{frame}
%
\begin{frame}[fragile]{Results}
\begin{alertblock}{MAC Metric}
\begin{figure}[hbt!]
	\centering
    \subfloat[MAC \label{relative_matrics:a}]{%
    	\includegraphics[width=0.5\linewidth]{metrica_igarss_2021_mac_y_fixo}
     	}
   % \subfloat[MFE \label{relative_matrics:b}]{%
   %    	\includegraphics[width=0.32\linewidth]{metrica_igarss_2021_mfe_y_fixo}
   %  	}
   % \subfloat[nMCC \label{relative_matrics:c}]{%
   %    	\includegraphics[width=0.32\linewidth]{metrica_igarss_2021_mcc_y_fixo}
    % 	}
    %\caption{}
    \label{Fig:measures} 
\end{figure}	
\textcolor{red}{MAC does not provide information on the intensity channels.}
\end{alertblock}
\end{frame}
%
\begin{frame}[fragile]{Results}
\begin{alertblock}{MFE Metric}
\begin{figure}[hbt!]
	\centering
    %\subfloat[MAC \label{relative_matrics:a}]{%
    %	\includegraphics[width=0.32\linewidth]{metrica_igarss_2021_mac_y_fixo}
    % 	}
    \subfloat[MFE \label{relative_matrics:b}]{%
       	\includegraphics[width=0.5\linewidth]{metrica_igarss_2021_mfe_y_fixo}
     	}
    %\subfloat[nMCC \label{relative_matrics:c}]{%
    %   	\includegraphics[width=0.32\linewidth]{metrica_igarss_2021_mcc_y_fixo}
    % 	}
    %\caption{The figures show the measures of quality applied in the MLE methods and the fusion methods}
    \label{Fig:measures} 
\end{figure}	
\textcolor{red}{MFE provides information on the intensity channels, and to the fusion.}
\end{alertblock}
\end{frame}
%
\begin{frame}[fragile]{Results}
\begin{alertblock}{nMCC Metrics}
\begin{figure}[hbt!]
	\centering
    %\subfloat[MAC \label{relative_matrics:a}]{%
    %	\includegraphics[width=0.32\linewidth]{metrica_igarss_2021_mac_y_fixo}
    % 	}
    %\subfloat[MFE \label{relative_matrics:b}]{%
    %   	\includegraphics[width=0.32\linewidth]{metrica_igarss_2021_mfe_y_fixo}
    % 	}
    \subfloat[nMCC \label{relative_matrics:c}]{%
       	\includegraphics[width=0.5\linewidth]{metrica_igarss_2021_mcc_y_fixo}
     	}
    %\caption{The figures show the measures of quality applied in the MLE methods and the fusion methods}
    \label{Fig:measures} 
\end{figure}	
\textcolor{red}{nMCC provides information on the intensity channels, and to the fusion.}
\end{alertblock}
\end{frame}
%
\begin{frame}[fragile]{Conclusions, and  discussions}
\begin{alertblock}{Conclusions, and discussions}
\begin{itemize}
    \item MAC does not provide information on the intensity channels or in the information fusion process (unbalanced datasets).
    \item Mfe and nMCC metrics allow for an ordering of the intensity channels and decide the channels to be used in the fusion process.
    \item The metrics Mfe and nMCC provide information about removing some channel of the fusion, for example, the vv channel.
    \item More and better metrics are necessary.
    \item It is a work in progress.
\end{itemize}
\end{alertblock}	
\end{frame}
%
\begin{frame}[fragile]{Reproducibility and replicability}
\begin{alertblock}{Platforms, and computational resources}
\begin{itemize}
\item[-] R Language.
\item[-] Matlab Language.
\item[-] Computer Intel\copyright\ Core i7-9750HQ CPU \SI{2.6}{\giga\hertz}  with \SI{16}{\giga\byte} of the RAM memory.
\end{itemize}
\end{alertblock}
\begin{alertblock}{Reproducibility and replicability}
\begin{itemize}
\item[-] \url{https://github.com/anderborba/igarss_2021_bmf}.
\end{itemize}
\end{alertblock}
\end{frame}
%
%\setbeamercolor{palette primary}{fg=white, bg=red!80!black}
\begin{frame}[standout]
  Thanks to everyone!!!!
\begin{itemize}
\item emails: anderborba@gmail.com
\item anderson.aborba@professores.ibmec.edu.br
\item Linkedin: https://www.linkedin.com/in/anderson-borba-4469653a/
\item Orcid: https://orcid.org/0000-0001-8479-9128 
\end{itemize}
\end{frame}
%\begin{frame}[allowframebreaks]
%\bibliographystyle{IEEEtran}
%\bibliography{../bibliografia}
%\end{frame}
\end{document}
